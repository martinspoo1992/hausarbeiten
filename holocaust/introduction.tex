\section{Einleitung} 
In der folgenden Seminararbeit zum Thema \glqq Holocaust als Unterrichtsthema\grqq{} soll ein Einblick in den Holocaust anhand des Buches \glqq Thema Holocaust – Ein Buch für die Schule\grqq{} von Ido Abram und Matthias Heyl (\cite{IA96}) gegeben werden.
Dabei wird sich die Rezeption des Buchinhaltes auf die Seiten 61-71 zu den Grundüberlegungen und Fragen zur pädagogischen Orientierung sowie die Seiten 118-126 zum Umgang mit dem Holocaust im Vorschulalter und in der Grundschule beziehen.

Im Alltag wird der Lehrer immer wieder mit dem Holocaust konfrontiert.
Er ist ein wichtiger Teil der deutschen Geschichte und prägt unser Leben bis heute.
Wenn sich in Deutschland Menschen maskieren, mit Fackeln durch die Straßen ziehen und an bekannte Persönlichkeiten der nationalsozialistischen Diktatur \glqq erinnern\grqq{}, so ruft das bei den meisten Menschen unangenehme Erinnerungen an vergangene Zeiten hervor.
Von Schülern hört man oft Sätze wie z.B. \glqq Ich kann es nicht mehr hören\grqq{} oder \glqq Das ist lange her.
Ich trage keine Schuld an der Vergangenheit.\grqq{}
Sicherlich mögen diese Einschätzungen nicht gänzlich falsch sein, jedoch haben wir aufgrund der Ereignisse während des 2.\,Weltkrieges eine Verantwortung, der wir uns nicht so leicht entziehen können und dürfen.

Ziel dieser Seminararbeit soll sein, einige pädagogische Ansätze zum Holocaust als Unterrichtsthema zu beleuchten und Antworten auf die Frage zu geben, wie sich die internationale pädagogische Wahrnehmung des Holocaust für den Unterricht darstellt.
Zudem soll auf die Frage eingegangen werden, welche Rolle dem Lehrer bei der Vermittlung der NS-Zeit und der Ereignisse des Holocausts zukommt.
Zuletzt soll untersucht werden, in welchem Alter der Holocaust in der Schule behandelt werden kann.
Mit besonderem Fokus soll dabei auf Kinder im Vorschulalter und in der Grundschule Bezug genommen werden.

Im weiteren Verlauf werden die gesammelten Erkenntnisse kritisch hinterfragt und zu ausgewählten Aussagen Stellung bezogen.
