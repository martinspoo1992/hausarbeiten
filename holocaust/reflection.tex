\section{Kritische Reflexion}
Die Auszüge aus „Thema Holocaust – Ein Buch für die Schule“ haben deutlich gemacht, inwiefern der Holocaust als Unterrichtsthema eine besondere Herausforderung für Lehrkräfte, Eltern und Schüler darstellt.
Heyl ist dabei in besonderer Weise auf den Begriff der Holocaust-Education eingegangen und stellte klar, dass in diesem Zusammenhang nicht nur über den Holocaust im Allgemeinen, sondern auch über die Art und Weise der Umsetzung als Unterrichtsinhalt zu reflektieren sei.
Des Weiteren stellte er die besondere Rolle der Lehrkräfte heraus, die sich nicht nur in der didaktischen Umsetzung, sondern auch im Umgang mit familiären Hintergründen widerspiegelt.
Die Fragen nach dem Grund, dem Ziel, der Art und Weise und der methodischdidaktischen Umsetzung von Carlebach, auf die er sich bezog, bilden dabei das Gerüst für einen fruchtbaren Unterricht.

Zentral war auch die Frage nach einer geeigneten Altersstufe für die Auseinandersetzung mit dem Holocaust.
Heyl stellte klar, dass diese Frage zu einer angeregten Diskussion führte.
Einerseits wurde angeführt, dass es die kindliche Entwicklung empfindlich stören könnte und man warten müsse, bis Kinder selbst über etwaige Fragen zu reflektieren im Stande seien.
Andererseits wurde angemerkt, dass es Impulse seitens der Erwachsenen geben müsse, um Kinder zum Nachdenken und -fragen anzuregen.

Ob es sinnvoll ist, bereits in der Grundschule ein solch komplexes und durch die besondere Grausamkeit der Ereignisse gekennzeichnetes Kapitel deutscher Geschichte zu behandeln, darf berechtigterweise in Frage gestellt werden, da man als Lehrperson stets die individuelle Entwicklung der Kinder im Blick haben muss.
Hierbei stellt sich zudem noch die Frage nach der Eignung für eine notwendige didaktische Reduktion und der allgemeinen Umsetzbarkeit im Unterricht der Primarstufe.
Jedoch erscheint der Ansatz, sich der kindlichen Wahrnehmung mit Hilfe von Märchen o.ä.
zu nähern, als eine schlüssige Methode.
Alles in allem bietet die Lektüre von „Thema Holocaust – Ein Buch für die Schule“ einen Einblick in das Thema Holocaust, welche Schwierigkeiten es mit sich bringt und gibt Hilfestellungen, wie ein Umgang mit der Thematik realisiert werden kann.
