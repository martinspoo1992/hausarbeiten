\section{Holocaust als Unterrichtsthema}
\subsection{Erziehung nach Auschwitz und \glqq Holocaust"=Education\grqq{}}
Wie bereits in der Einleitung zum Ausdruck gekommen ist, stößt die Beschäftigung mit den Ereignissen des Holocaust im schulischen Unterricht bei den meisten Schülern auf Ablehnung.
Dieses Phänomen beschreibt auch Matthias Heyl in seinen Überlegungen.
Er stellt dazu zunächst folgende These auf: \glqq Die Geschehnisse von 1933 bis 1945 prägen unsere Gegenwart vielleicht stärker als jede andere Periode deutscher und europäischer Geschichte zuvor.\grqq{} (\cite[S.\,61]{IA96}) Von dieser Annahme ausgehend erklärt sich die Empörung, wenn in Städten wie Rostock, Lübeck oder Solingen, die er u.a.
als Beispiele nennt, Synagogen brennen oder jüdische Friedhöfe geschändet werden.
Innerhalb vieler Familien lässt sich trotzdem der Effekt beobachten, dass die Zeit des NS-Regimes gar nicht oder kaum thematisiert wird, sodass sich eine Diskrepanz zum schulischen Bildungsauftrag ergibt.
Heyl möchte daher Ansätze entwickeln, wie im modernen Unterricht mit dem Nationalsozialismus und seinen Folgen umgegangen werden kann.

Zunächst ist von elementarer Bedeutung, dass der Begriff \glqq Holocaust\grqq{} eine klare Definition erhält.
Im engeren Sinne versteht man darunter die faktische Vernichtung der europäischen Juden.
Fasst man den Begriff jedoch weiter, so bezieht er auch den Weg dorthin, also die Verfolgung, Entrechtung, Enteignung und Deportation mit ein.
Um später den ganzen geschichtlichen Kontext verstehen zu können, soll den Schülern der Prozess bis \glqq Auschwitz\grqq{} näher gebracht werden.
Bei der Frage, wie schülergerecht mit dem Thema \glqq Holocaust\grqq{} umgegangen werden kann, bezieht sich Heyl auch auf seinen Mitautor Ido Abram, der von der sogenannten \glqq Erziehung nach Auschwitz\grqq{} spricht.
Diese beinhaltet nicht nur das Nachdenken über Erziehung nach Auschwitz im Allgemeinen, sondern auch die Betrachtung der Frage, was über den Holocaust vermittelt werden und wie dies geschehen soll.
Heyl bekräftigt, dass Lehrer in dieser Frage nur bedingt auf Schulbücher setzen können, da es zwar eine sog.
\glqq Gedenkstättenpädagogik\grqq{} gibt, diese jedoch aufgrund der geringen Bedeutung deutscher Konzentrationslager für die \glqq Endlösung\grqq{} eine untergeordnete Rolle einnimmt.
(\cite[S.\,64]{IA96})

International herrscht eine große Diskussion, wie mit dem Holocaust pädagogisch umzugehen sei, die jedoch in Deutschland weit weniger Aufmerksamkeit findet, als beispielsweise in Israel, den Niederlanden oder den USA.(\cite[S.\,64]{IA96}) Dort werden seit Jahren pädagogische Konzepte erarbeitet und im Jahre 1990 wurde in Amsterdam der erste Lehrstuhl für \glqq Holocaust-Education\grqq{}.
Dieser Begriff lässt sich mit \glqq Holocaust-Erziehung\grqq{} übersetzen, um möglichen Missverständnissen vorzubeugen, es gehe darum, Schülern zu lehren, wie man einen derartigen Völkermord durchführt.
Heyl stellt jedoch klar, dass vor allem bei den US-amerikanischen Projekten die Didaktik zugunsten der Geschichtswissenschaft, Geschichtsdidaktik und Erziehungswissenschaft in den Hintergrund tritt.
(\cite[S.\,64]{IA96})

\subsection{Fragen zur pädagogischen Orientierung}
Lehrer müssen sich bewusst werden, welche Bedeutung der Holocaust in Bezug auf unsere familiäre Herkunft und unsere gesellschaftlichen Erfahrungen hat.
Heyl bezieht sich daher auf vier Fragen der israelischen Pädagogin Mirjam Gillis-Carlebach, mit denen man sich beschäftigen sollte, bevor man mit Schülerinnen und Schülern über den Holocaust spricht.
Carlebach fragt dabei nach dem Grund, dem Ziel, dem genauen Gegenstand und der methodisch-didaktischen Umsetzung.
Die Frage nach dem Grund ist nach Heyl nicht leicht zu beantworten und doch stellt er fest: \glqq Dem Geschehen kommt eine besondere Bedeutung zu, da er ein Teil unserer Geschichte ist, und zwar ein besonderer Teil – besonders schwerwiegend, besonders belastend, auch besonders sperrig.\grqq{} (\cite[S.\,68]{IA96}) Die Fragen nach den Tätern, Mittätern und der Komplizenschaft unter den eigenen Verwandten wie Eltern und Großeltern führt daher zu Angst und Scham vor allem unter nicht-jüdischen Menschen.
Daher ergibt sich als Grund für die Behandlung im Unterricht, diese Vergangenheit als elementaren Bestandteil unserer Historie zu begreifen und den Schülern auch durch unsere persönlichen Motive verständlich zu machen.

Ziel der didaktischen Umsetzung soll nicht nur die Erziehung zu Toleranz, Humanität und Zivilcourage sein, weil dies den Holocaust zum Lernanlass machen würde, so Heyl. (\cite[S.\,68]{IA96})
Der Historiker Moshe Zimmermann begründet diese Sicht damit, dass die Ziele der Beschäftigung mit dem 2. Weltkrieg – Auseinandersetzung mit Verantwortung, Schuld und einer persönlichen Sicht - auch ohne den Holocaust denkbar wären. (\cite[S.\,69]{IA96})
Allerdings bietet der Holocaust den entscheidenen Unterschied, da es sich nicht um ein \glqq normales\grqq{} Kriegesgeschehen handelte, sondern sich jeglicher logischer Begründung für den Kriegsausgang entzog, wie Heyl weiter ausführt.
Der Erziehungswissenschaftler Micha Brumlik geht dabei noch weiter, indem er formuliert, dass \glqq Erziehung nach Auschwitz\grqq{} in erster Linie eine Erziehung, ein Lernen über Auschwitz sein und Auschwitz als Thema haben müsse. (\cite{MB95})

\subsection{Kinder im Vorschulalter und in der Grundschule}
In den meisten Fällen werden Schüler erst in der 10. Jahrgangsstufe mit dem Holocaust konfrontiert.
Hierzu beruft sich Heyl auf den Erziehungwissenschaftler Malte Dahrendorf, der die entscheidene Frage stellt, ab welchem Alter der Schüler eine Behandlung im Unterricht sinnvoll ist. (\cite[S.\,118]{IA96})
Weitere Fragen ergeben sich daraus wie \glqq Soll man Kindern vom Faschismus erzählen? Oder belastet man damit die Kinder?\grqq{}.
Heyl macht deutlich, dass sich sowohl Eltern als auch Pädagogen äußerst unsicher sind, wie mit der Thematik umzugehen ist.
Gillis-Carlebach spricht sich hierbei dafür aus, die Schülerinnen und Schüler möglichst früh, aber auch auf die Inhalte vorzubereiten, da ein kindgerechter Umgang die Basis für einen weitergehenden Unterricht über die Primarstufe hinaus legt.
Auch hieraus leiten sich Fragen nach einer altersgemäßen Darstellung, der pädagogischen Greifbarkeit für Kinder und der richtige Zeitpunkt für den Beginn mit \glqq Erziehung nach Auschwitz\grqq{} ab.

Heyl führt hierzu aus, das eine wichtige Voraussetzung für den frühen Umgang mit dem Holocaust ist, dass die Kinder in der Lage sind, zwischen Phantasie und Realität zu differenzieren.
Dies ist wichtig, um nachvollziehen zu können, dass der Holocaust aus einer Verschiebung antisemitischer Phantasien hin zur Realität entstand.
Hierzu benötigen sie einerseits das märchenhafte Erzählen, dass ihre dichotonome Sicht der Welt unterstützt, d.h.
ihr Denken in Kategorien wie Gut und Böse, etc., andererseits aber auch Eltern, die ihnen aus ihrer Vergangenheit erzählen, um in der Lage zu sein, diese Erzählungen für sie selbst verständlich zu fassen.
