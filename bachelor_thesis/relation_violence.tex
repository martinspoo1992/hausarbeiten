\section{Das Verhältnis von Sprache zu Gewalt}
Für das Verständnis von pejorativen Sprachhandlungen in Fangesängen ist die Bestimmung des Verhältnisses zwischen Sprache und Gewalt von zentraler Wichtigkeit, das heißt die Beantwortung der Frage, worin genau die Verletzung besteht, wenn man davon spricht, dass Worte "`verletzen"' können.
Zunächst wird man feststellen, dass der sprachlichen Gewalt, anders als der physischen Gewalt, die sichtbaren Folgen ihrer Ausübung zunächst fehlen.
Ein Opfer physischer Gewalt ist aufgrund seiner äußeren Erscheinung meist als solches zu erkennen, anders als ein Opfer sprachlicher Gewalt.

Steffen Herrmann und Hannes Kuch betrachten das Verhältnis von Sprache und Gewalt aus drei Perspektiven\cite[S. 13]{SKH07}.
Dabei beziehen sie sich auf die Formulierungen "`Gewalt und Sprache"', "`Gewalt der Sprache"' und "`Gewalt durch Sprache"'.
Spricht man von Gewalt und Sprache, so stehen beide Begriffe nicht mehr gegensätzlich gegenüber, sondern das Verhältnis zwischen ihnen wird betont.
Zudem komme der Sprache eine beschreibende Funktion eines für sie äußerlichen Begriffes zu.
Gewalt der Sprache hingegen würde implizieren, dass Gewalt ein fester Bestandteil der Sprache zu sei.
Gewalt durch Sprache wiederum legt den Fokus letztlich auf die Handlung des Sprechers, der Gewalt durch seinen Gebrauch der Sprache vollzieht.

Sybille Krämer stellt ihrerseits die Frage, worum es sich bei sprachlicher Gewalt im Allgemeinen handelt.
Dabei differenziert sie zunächst den Gewaltbegriff in ausgeübte Gewalt, was sie unter dem lateinischen Wort "`potestas"' fasst und verübte Gewalt, die sie dem Begriff "`violentia"' zuordnet.
Mit dieser Unterscheidung macht Krämer eine Mehrdeutigkeit des Begriffs "`Gewalt"' deutlich.
Einerseits kann damit die Amts- oder auch Staatsgewalt, also das Handeln des Staates, gemeint sein, andererseits aber auch die negativ behaftete Bedeutung der verübten Gewalt als Gewaltakt oder eben der Beleidigung.
Die sprachliche Gewalt ist folglich der "`violentia"' zuzuordnen.
Krämer sieht das Verletzende, was sprachliche Äußerungen haben können, in deren Illokution, also der sprachlichen Handlung begründet\cite[S. 35]{SK07}.
Daraus folgt für sie, dass die Sprache erst verletzend wird, wenn der Sprecher eine verletzende Handlung vollziehen möchte.
Letztlich seien verletzende Worte kein bloßer Bestandteil der Sprache, sondern eine Erscheinung, die durch den Sprachgebrauch zustande komme.

Auch König und Stathi sehen Gewalt nicht als Bestandteil des Sprachsystems an.
Für sie kommt der gewalttätige Aspekt durch "`kommuniktative Akte, die in einer konkreten Situation vollzogen werden"'\cite[S. 47]{EK10} zustande.
Wörter und Sprache könnten jedoch als mögliche Instrumente für aggressive Handlungen taugen.
Demnach tragen auch alle der Sprache zugehörigen Merkmale wie z. B. Körpersprache, Intonation, etc. zu einer solchen Wirkung bei.
Im Unterschied zu Krämer stellen König und Stathi fest, dass die Sprache weder gewaltfrei noch jegliche Sprechakte von sich aus gewalttätig sind\cite[S. 48]{EK10}.
Die Sprache beinhalte eine große Bandbreite von Wirkungen, die der Sprecher mit seinen Äußerungen hervorrufen könne.
Sie führen weiter aus, dass "`alle Kommunikationsakte, die den Erwartungen positiver und negativer Höflichkeit nicht gerecht werden, die das Gesicht (face im Sinne von Goffman) verletzen"', als unangenehm empfunden werden.
In diesem Zusammenhang werden sprachliche Aggressionen als besonders unangenehm eingestuft.
Dieses Verhältnis lässt sich an einem Beispiel veranschaulichen.
Missachtet ein Student die Distanz zwischen ihm und seinem Dozenten, in dem er ihn mit seinem Vornamen anstelle beispielsweise der Anrede "`Herr Prof. Dr."' anspricht, so findet eine Verletzung einer solchen Höflichkeitsbeziehung statt, die vonseiten des Angesprochenen in der Regel als inakzeptabel empfunden wird.
Grundgedanke jeglicher Interaktion sei die Wahrung des eigenen "`Gesichts"'\cite[S. 50]{EK10}.

König und Stathi gliedern kommunikative Gewaltakte nach bestimmten Parametern und Typen.
Sie sehen sprachliche Gewalt, die einen konkreten Adressaten hat, als stärker an im Vergleich zu solcher, die sich auf eine Gruppe und nicht auf eine bestimme Person bezieht.
Diese Aussage belegen sie anhand von Alltagsbeispielen.
Der Ausruf "`Scheibenkleister!"', der nicht eindeutig an eine Person adressiert ist, wird demnach ebenso wenig als Beleidigung angesehen wie das Singen der französischen Nationalhymne, die zwar Gewaltdarstellungen thematisiere, es sich jedoch in diesem Fall um eine "`rituelle Beleidigung"' handele\cite[S. 51]{EK10}.
Demgegenüber ständen beispielsweise sexistische Ausdrücke über Frauen oder rassistische Witze über Menschen bestimmter Nationalitäten, die in deren Beisein durch die Adressierung als verletzend eingestuft werden könnten.

Neben dieser Unterscheidung zwischen einer Adressierung stellt sich auch die Frage, inwiefern eine Aussage gegen ein Individuum oder eine Gruppe gerichtet ist, also die Frage nach der Zugehörigkeit des Adressaten.
So gebe es eine gravierende Differenz der Verletzung durch eine Zuordnung zu einer Gruppe wie z.B. "`Schwarze"' und "`Weiße"' und der Zugehörigkeit zu einem Land.
In diesem Zusammenhang können auch andere Eigenschaften wie Geschlecht, Alter oder Religion herangezogen werden.
Aus diesem Zusammenhang heraus sehen König und Stathi auch eine Diffamierung des Alters oder der Körperlichkeit als weniger schwer im Vergleich zu einer Herabsetzung aufgrund der Rasse.

Einen weiteren wichtigen Ansatzpunkt bei der Betrachtung verletzender Sprache lässt sich darin sehen, wie direkt oder indirekt eine solch geartete Ausdrucksweise verwendet wird.
So gibt es in der Sprache Flüche, Kraftausdrücke und Schimpfwörter, deren herabsetzende und negative semantische Erscheinung klar ersichtlich ist.
Dabei sollte zwischen Kraftausdrücken und Beleidigungen erneut differenziert werden, da erstgenannte meist mit der Verletzung von Tabus operieren, letztere jedoch eine Hierarchie zwischen Sprecher und Adressaten thematisieren\cite[S. 53]{EK10}.
Aber nicht nur mit diesen direkten Aussagen kann eine Verletzung stattfinden sondern auch mit Äußerungen, die zunächst keine offensichtiliche Beleidigung enthalten, jedoch in einem bestimmten Kontext als solche interpretiert werden können.
Als Beispiel nennen König und Stathi an dieser Stelle den Ausspruch "`Wir sind hier in Deutschland"', der einerseits eine Feststellung der Lokalität, andererseits für Menschen ausländischer Herkunft das beigeordnete Versagen der Zugehörigkeit zur Gesellschaft enthält.
Jedoch kann aus dieser Einteilung in direkte und indirekte sprachliche Gewalt keine Gewichtung der Verletzung vorgenommen werden.

Weitere Ansatzpunkte für die Klassifizierung von sprachlicher Gewalt sind die Beziehung zwischen Sprecher und Adressat.
Diese sei dadurch bedeutsam, dass eine Verletzung durch einen nahestehenden Menschen als weitaus schlimmer eingestuft wird, als die durch eine unbekannte Person.
Zudem sei die Symmetrie des Verhältnisses entscheidend, da durch ein asymmetrisches Machtverhältnis die Wirkung sprachlicher Gewalt verstärkt würde.
Hinzu käme die Anwesenheit eines Publikums, da durch sie die Grundeigenschaft des Menschen, ein soziales Wesen zu sein, angesprochen und in die Verletzung des eigenen Ansehens miteinbezogen würde\cite[S. 56]{EK10}.
Schlussendlich ist auch der Wahrheitsgehalt einer Äußerung ein wichtiger Indikator, da der Ausdruck einer wahrheitsgemäßen Aussage für den Adressaten unangenehmer sein kann, als die Verbreitung einer widerlegbaren Behauptung.
In diesem Zusammenhang ist dann auch die Iteration und die Art und Weise von Bedeutung\cite[S. 58]{EK10}.

Es zeigt sich, dass sprachliche Gewalt die Sprache als Medium nutzt und dadurch eine Verletzungvorgenommen wird. Der Sprache als System wohnt aber keine Gewalt inne, so dass die Sprache erst durch ihre Verwendung durch den Sprecher zu Handlungen der Gewalt fähig ist.

