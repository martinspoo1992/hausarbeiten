\section{Sprachliches Handeln – Was tun wir mit Sprache?}
Die menschliche Sprache stellt ein komplexes System dar, welches aus verschiedenen Perspektiven betrachtet werden kann.
Dabei bildet die Pragmatik neben der Semantik, der Semiotik, der Syntax, der Phonetik und weiteren Disziplinen den Bereich der Linguistik, der "`die Beziehung zwischen sprachlichen Zeichen und den Benutzern sprachlicher Zeichen untersucht."'
Die Pragmatik behandelt demnach die Beziehung zwischen der Sprache und den Nutzern dieses Systems unter Fokussierung auf deren Handlungen, die sie mithilfe sprachlicher Äußerungen vollziehen.
Hierzu haben sich im 20. Jahrhundert verschiedene Theorien und Modelle entwickelt, die das Verhältnis von Sprache und Handeln beschreiben.
Bereits der Philosoph Ludwig Wittgenstein (1889-1951) widmete der Sprache in seinen "`Philosophischen Untersuchungen"' seine Aufmerksamkeit, wenngleich er dies vornehmlich aus semantischer und nicht aus pragmatischer Perspektive tat.
Jedoch stellte er trotzdem einen Bezug zum sprachlichen Handeln her, indem er sagte, dass die Bedeutung eines Wortes seinem Gebrauch in der Sprache entspreche.
Daraus folgte demnach auch, dass Wörter, die nicht in der Sprache benutzt würden, diese Bedeutung verlieren könnten, was sich anhand des Aussterbens mancher Wörter der deutschen Sprache nachvollziehen lässt.
Zugleich kritisierte Wittgenstein die sogenannte Abbildungstheorie, nach der Wörter und Sätze Dinge in der Welt bezeichnen würden, die wirklich existieren und beschrieben werden könnten.
Dadurch werde der Bedeutung eines Wortes die Menge an Gegenständen zugeordnet, die es bezeichne.
Er verwendete hierzu das Bild von "`Namenstäfelchen"', die Wörtern angeheftet seien.
Es lässt sich folglich sagen, dass Wittgenstein einer Definition von Bedeutungen, die Wörter haben können, ohne Betrachtung der kontextuellen Verwendung, ablehnend gegenüberstand.
Darüber hinaus besteht die Bedeutung der Sprache in einem Verständnis von Aussagen.
Dieses Verstehen stellt für ihn das Gegenstück zum Meinen dar, da das Verstehen im Gegensatz zum Meinen nicht versteckt sei, sondern aufgrund der hervorgerufenen Reaktionen bestimmt sei.

Paul Grice definierte für sein Modell der konversationellen Implikaturen vier Konversationsmaximen, welche er aus dem kommunikativen Prinzip der Kooperation ableitete.
Demnach beruhe Kommunikation darauf, Verständigung unter den Teilnehmenden zu erreichen.
Das bedeutet, dass Sprechbeiträge grundsätzlich immer mit dem Gedanken geäußert werden, verständlich für das Gegenüber zu sein.
Aus diesem Grundprinzip folgten für Grice die Maximen der Quantität, Qualität, Relation und Modalität, welche Vorgaben schufen, nach denen sprachliche Handlungen eingeordnet werden können.
Demnach sei eine Aussage kooperativ, wenn sie gewisse Bedingungen erfülle.
Sie dürfe nur so lang wie nötig sein, was Grice als Maxime der Quantität definierte.
Gleichzeitig sollte sie jedoch auch so informativ wie möglich und wahrheitsgetreu sein, was er als der Maxime der Qualität bezeichnete.
Hinzu kam die Relevanz der Äußerung für den Kontext des Gesprächs, also die Maxime der Relation, sowie die Angemessenheit im jeweiligen Kontext, die Maxime der Art und Weise.
Aus dieser Definition von Maximen folgte gleichzeitig, dass sprachliches Handeln durch ein Verletzen eben dieser gestört werden kann.
Sprechen beispielsweise zwei Menschen über das Wetter und eine dritte Person beteiligt sich daran, spricht jedoch über Politik, so wird die Maxime der Relation verletzt, da die Äußerung der dritten Person nicht in den Kontext der bestehenden Konversation passt.
Nach Grice kann neben der beschriebenen unabsichtlichen Missachtung auch eine absichtliche in Form von Ironie oder auch der Lüge in der Kommunikation stattfinden.
Im Falle der Ironie findet eine Verletzung der Qualitätsmaxime statt, genauso wie bei einer Lüge.

Neben der Theorie der Konversationsmaximen von Paul Grice entwickelten sich ebenso die Theorien von John Langshaw Austin und John Searle zu zentralen Theorien der Pragmatik.
Austin unterscheidet in seiner Theorie zunächst zwei Erscheinungsformen von Sätzen: konstative und performative.
Konstative Sätze dienen nach Austin grundsätzlich dazu, Behauptungen aufzustellen, die wahr oder falsch sein können.
Demgegenüber stehen die performativen Sätze, mit denen man sprachliche Handlungen vollzieht.
Als Beispiel lässt sich der Satz "`Ich erkläre Sie zu Mann und Frau"' nennen, bei dem in diesem Fall eine Eheschließung folgt.
Austins eigentliche Sprechakttheorie basiert auf der Einteilung der Sprachhandlung in drei Sprechakte, die er als Lokution, Illokution und Perlokution bezeichnet.
Unter Lokution versteht Austin die Verwendung der Stimme oder eines anderen Mediums wie einer Tastatur, um Laute bzw. sprachliche Zeichen zu erzeugen.
Aus diesen Untereinheiten würden dann Einheiten des Sprachsystems wie Morpheme, Sätze, etc. gebildet.
Zudem hätten solche Äußerungen stets eine Verknüpfung zur Umgebung, die mit der Sprache beschrieben werden solle.
Man äußert damit eine Proposition.
Der zweite Sprechakt ist die Illokution, mit deren Hilfe die Intention ausgedrückt wird, mit der man eine Äußerung tätigt.
Möchte man sein Gegenüber warnen, grüßen oder informieren, um nur einige Möglichkeiten zur Verdeutlichung zu nennen, so handelt es sich dabei um den illokutionären Akt der Sprechhandlung.
Unter dem letzten Sprechakt, der Perlokution, versteht Austin die gewünschte Reaktion, die eine Sprechhandlung zur Folge haben soll.
Möchte man, dass der Angesprochene etwas Bestimmtes nicht tut, so warnt man ihn.
Ein Sprechakt besteht demnach aus der parallelen Abfolge dieser Teilakte.

John Searle erweitert Austins Theorie, in dem er die Definition der Lokution in zwei Teilakte unterteilt.
Daraus entstehen in seinem Modell einerseits der Äußerungsakt, der die Nutzung von Werkzeugen zur Realisierung umfasst, sowie der propositionale Akt, der die Aussage über die Welt widerspiegelt.
Der propositionale Akt besteht wiederum aus zwei Teilakten, wobei der Referenzakt die Bezugnahme auf ein Referenzobjekt bezeichnet und der Prädikationsakt auf eine Eigenschaft dieses Objektes verweist, die ihm zugeschrieben wird.
Die Akte der Illokution und der Perlokution verhalten sich analog zu Austins Theorie.
Neben dieser Erweiterung der Theorie Austins differenzierte Searle Illokutionen in Grundtypen, nach denen diese bestimmten Zwecken zugeordnet werden können.
Unter den repräsentativen Akten versteht Searle Aussagen, die Behauptungen über die Welt aufstellen, die von Austin zunächst noch in konstative und performative Sätze getrennt worden waren.
Direktive Sprechakte hingegen stellen Handlungen dar, bei denen der Hörer zu einer Handlung bewegt werden soll, die durch Wörter wie \emph{raten}, \emph{bitten} oder \emph{befehlen} ausgedrückt werden können.
Hinzu kommen kommissive Sprechakte, mit denen sich der Sprecher selbst auf etwas festlegen kann, was mit Versprechen, Androhungen oder auch Vereinbarungen zum Ausdruck gebracht werden kann.
Expressive Akte dienen nach Searle dazu, persönliche Gefühle auszudrücken und seinen eigenen Zustand zu kommunizieren.
Schlussendlich folgt der Typ der deklarativen Sprechakte, die zum Vollzug von Ritualen verwendet werden können.

Neben solchen direkten Sprechakten existieren auch indirekte Sprechakte, die durch die Eigenschaft näher bestimmt werden, dass sie nicht klar erkennen lassen, welche Funktion sie erfüllen.
Der Philologe Peter Auer nennt dazu als Beispiel "`Wenn Sie nicht $X$ tun, wird $Y$ passieren."', wobei $Y$ in diesem Beispiel als negatives Folgeereignis vorausgesetzt wird.
In diesem konkreten Beispiel kann die Funktion dieses Satzes einerseits der Rat sein, $X$ zu tun, um $Y$ zu vermeiden.
Andererseits kann er auch als Warnung verstanden werden, $X$ in keinem Fall zu tun, weil dies $Y$ zur Folge hätte.
Anhand dieses Beispiels wird die Ambiguität indirekter Sprechakte klar verständlich.

Anhand dieses Überblicks über verschiedene Theorien der pragmatischen Linguistik lässt sich bereits feststellen, dass die Sicht auf sprachliche Handlungen im 20. Jahrhundert äußerst
unterschiedlich war und es keine eindeutige Sichtweise auf sprachliches Handeln gab.
Während Wittgenstein sprachliches Handeln vornehmlich aus semantischer Perspektive zu begreifen versuchte, da für ihn die Bedeutung von Wörtern nur in einem kontextuellen Zusammenhang denkbar sei und nicht mit der bloßen Verwendung des Wortes, kamen Grice, Austin und Searle zu dem Schluss, dass sprachliches Handeln Gesetzmäßigkeiten folgen müsse, ohne die Kommunikation nicht möglich sei.
Grice sah sprachliches Handeln als Kooperation zwischen Gesprächspartnern unter Berücksichtigung von Merkmalen, die sprachliche Aussagen erfüllen müssten, während Austin und Searle den Fokus auf die Formulierung von Sprechakten legten, mit denen sie die Abläufe sprachlichen Handelns ohne den Aspekt des kooperativen Handels beschrieben.

