\section{Was unterscheidet Fußballfans von Zuschauern?}
Um sich den Fangesängen und ihrer Struktur anzunähern, sollte zunächst klar definiert werden, worum es sich bei "`Fans"' allgemein handelt.
Es lässt sich feststellen, dass sich größere Gruppen von Menschen, die ein Fußballstadion besuchen, anhand bestimmter Merkmale unterscheiden lassen.

Grob gesehen bezeichnet der englische Begriff \emph{Fan} grundsätzlich nichts anderes als einen Menschen, der eine emotionale Begeisterung für ein bestimmtes Fanobjekt – beispielsweise
Sportvereine, einzelne Sportler, aber auch Musikgruppen, etc. – entwickelt und auslebt\cite{Dud14}.
Für den Sozialwissenschaftler Christoph Bremer zeichnen sich Fans nicht nur durch ihre Begeisterung für das Fanobjekt aus, sondern auch durch ihre Identifikation, "`welche eindrucksvoll (durch Schals, Trikots, Jacken, Abzeichen usw.) nach außen getragen wird"', die aber – und das ist der entscheidende Punkt – "`von äußerst starker innerer Qualität ist."'\cite[S. 57]{CB03}
Demgegenüber fehlt dem klassischen Zuschauer diese Form des Enthusiasmus, da er im Gegensatz zum Fan nicht über eine Beziehung zu einem Objekt verfügt oder verfügen muss.

Es zeigt sich, dass der Fan mit einer anderen Motivation in ein Stadion geht, als ein gewöhnlicher Zuschauer.
Während der Fan durch sein inneres Verhältnis zu seinem Fanobjekt stets mit dem Hintergedanken, dass – an dieser Stelle - die bevorzugte Mannschaft erfolgreich sein soll, der Auseinandersetzung mit seinem Fanobjekt nachgeht, ist der Ausgang des Fußballspiels für den Zuschauer im Vergleich eher unerheblich, weil er unter Umständen lediglich ein sportlich ansehnliches Spiel sehen möchte.
Dadurch grenzt sich der Fan klar vom Zuschauer ab.
Bremer stellt weiter fest, dass der Fan unabhängig vom Aufwand, den er betreiben muss, tunlichst kein Spiel seiner Mannschaft versäumen will, da er sich im Zuge solcher Reisen in andere Städte mit anderen Anhängern seines Fanobjektes solidarisiert und ein "`Wir-Gefühl"' entsteht.
Daher sind die emotionalen Folgen des sportlichen Misserfolges für einen Fan weitaus gravierender als für den Zuschauer.
Ein weiterer Charakterzug von Fans in Abgrenzung zu Zuschauern liegt für Christoph Bremer darin, dass sie den wöchentlichen Stadionbesuch häufig fest in ihren Wochenablauf eingeplant haben\cite[S. 58]{CB03}.
Demgegenüber ist der Zuschauer unabhängig von zeitlichen Festlegungen, da er bei ausreichenden, zeitlichen Ressourcen einen Stadionbesuch planen kann und meist nicht regelmäßig im Stadion zugegen ist.
Daraus könnte sich der Schluss ziehen lassen, dass der Fan nichts anderes als ein Fanatiker sei, der seiner Leidenschaft alle anderen Lebensbereiche unterordne.
Diese Annahme ist sicherlich nicht völlig falsch, würde jedoch zu kurz greifen.

Der Fanbegriff darf nicht leichtfertig mit dem des Fanatikers assoziiert werden.
Im Englischen ist Fan auch als Abkürzung für "`fanatic"' gebräuchlich.
Im Deutschen dagegen wird klar zwischen dem Fan und dem Fanatiker unterschieden, da der Begriff Fanatiker einen Menschen bezeichnet, der "`von bestimmten Ideen, einer bestimmten Weltanschauung o.ä. so überzeugt ist, dass er sich leidenschaftlich, mit blindem Eifer (und rücksichtslos) dafür einsetzt."'\cite[S. 2]{HS08}
Im englischen Sprachgebrauch ist diese Differenzierung zwischen Fans und Fanatikern nicht gleichermaßen ausgeprägt, was zur Folge hat, dass man in diesem Fall die Bedeutung aus dem sprachlichen Kontext ableiten muss.

Es lässt sich demnach festhalten, dass der Fanbegriff in der allgemeinen Wahrnehmung vor allem über die menschlichen Emotionen und dem daraus resultierenden Verhalten der Fans definiert wird.
Jedoch lässt sich eine solche Definition auch aus einer anderen Perspektive vornehmen.
Der Sozialforscher Jochen Roose steht einer solchen Definition kritisch gegenüber und sieht Fans eher als "`Menschen, die längerfristig eine leidenschaftliche Beziehung zu einem für sie externen, öffentlichen, entweder personalen, kollektiven, gegenständlichen oder abstrakten Fanobjekt haben und in die emotionale Beziehung zu diesem Objekt Zeit und / oder Geld investieren."'\cite[S. 3]{HS08}
Auch hier ist der Beziehungsaspekt des Fan-Seins zunächst im Fokus.
Ohne diesen Aspekt ist es nicht denkbar, Fan zu sein.
Roose ergänzt diesen im Unterschied zu Bremer jedoch, indem er festhält, dass eine solche Beziehung auf Langfristigkeit angelegt ist.
Hierzu ist zu sagen, dass es sich dabei um eine Begrifflichkeit handelt, die einerseits auf einen längeren Zeitraum hindeuten lässt, er eine Angabe der Zeitdauer andererseits bewusst auslässt und so einen Spielraum der Auslegung zulässt.
Zentraler Unterschied ist, dass nach Roose das Fanobjekt etwas Externes ist, also nicht zu den persönlichen Freizeitaktivitäten gehört\cite[S.4]{HS08}.
Daraus folgert er, dass man demnach nicht Fan seiner eigenen Fußballmannschaft sein kann, in der man folglich selbst Mitglied ist, da sie nicht extern ist.
Hinzu kommt der Aspekt, dass das Fanobjekt etwas Öffentliches ist, was das Fantum gleichzeitig von zwischenmenschlichen Beziehungen unterscheide, die im privaten Umfeld stattfänden.
Abschließend enthält seine Definition auch den kommerziellen Aspekt, dass Fans finanzielle und zeitliche Investitionen vornehmen, um Fan sein zu können.
Anhand dieser Gegenüberstellung lässt sich sagen, dass der Zustand, Fan eines Vereins zu sein, offensichtlich mehr beinhaltet, als lediglich den regelmäßigen Besuch im Stadion.

Die Tatsache, dass Fußballfans in einem affektiven Spannungsverhältnis zu ihrem Verein stehen, das sich sowohl in frenetischer Unterstützung als auch in großer Unzufriedenheit über sportliche Leistungen widerspiegeln kann, führt dazu, dass Fangruppierungen in Konflikt mit staatlichen Organen kommen können.
So ist das Verhältnis zwischen den Ordnungskräften und Fans oftmals angespannt, wenn es um ausgetragene Meinungsverschiedenheiten oder Gefahrenlagen wie z.B. den Einsatz pyrotechnischer Gegenstände geht.
Die Vereine stehen bei solch gearteten Problematiken zwischen den Fronten, da sie einerseits die Fans als Einnahmequelle durch Ticketverkäufe und als Unterstützung der Mannschaft brauchen, sie aber andererseits auch als Veranstalter für die Sicherheit Sorge zu tragen haben.
Daraus folgt, dass das Verhalten von Fans nicht nur ein vereinsinternes Thema ist, sondern einige von ihnen auch von der Gesellschaft aufgrund ihres Verhaltens kritisch beäugt werden.\cite[S. 22]{EH88}
Dieses Bedürfnis nach politischer Mitbestimmung in einem Verein findet sich bei Zuschauern nicht.
