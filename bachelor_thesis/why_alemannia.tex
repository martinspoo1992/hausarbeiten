\section{Alemannia Aachen - ein besonderer Fußballverein?}
Im Vorfeld der Themenfindung ergab sich die Frage, warum der Fußballverein Alemannia Aachen als Forschungsgegenstand der vorliegenden Ausarbeitung gewählt wurde, um das sprachliche
Verhalten von Fußballfans exemplarisch darzustellen.
Es ergaben sich drei zentrale Beweggründe, die im Folgenden dargestellt werden sollen:

Zunächst lässt sich sagen, dass Alemannia Aachen in der sportlichen Landschaft des deutschen Profifußballs eine ganz besondere Rolle einnimmt.
Der Verein spielte so lange wie kein anderer deutscher Fußballverein in der 2. Fußball-Bundesliga.
Ganze 28 Spielzeiten in der zweithöchsten deutschen Spielklasse machen den Verein, der aktuell in der viertklassigen Regionalliga West vertreten ist, zu einem bedeutenden Verein des deutschen professionellen Fußballs.

Gegründet wurde der Verein am 16.12.1900 von 18 Schülern des damaligen Kaiser-Wilhelm-Gymnasiums Aachen (heute Einhard-Gymnasium) als "`Fußballclub Alemannia"'\cite{AA14}.
Der Verein entwickelte sich, anders als andere deutsche Fußballvereine, nicht aus einem Turnverein heraus, sondern aus einer Schülermannschaft, da die damaligen Schulen die Teilnahme von Nichtschülern untersagten.
Der Name "`1. FC Aachen"' war dabei nicht mehr zulässig, da er bereits an einen anderen Verein vergeben war, der sich wenig später auflöste.

Daher wurde der Verein "`Alemannia Aachen"' genannt, um den germanischen Ursprung des Vereins im damaligen Deutschen Reich zu repräsentieren.
Zum damaligen Zeitpunkt existierte noch keine Fußballliga in Deutschland, weshalb der Verein sein erstes Spiel am 16. Dezember 1900 gegen den belgischen FC Dolhain bestritt.
In der Spielzeit 1904/1905 nahm der Verein erstmals an einem geregelten Spielbetrieb des Fußballverbandes Rheinland-Westfalen teil.
Drei Jahre später erreichte er als "`Meister des 1. Bezirks"' den ersten Meisterschaftstitel der Vereinshistorie.
1909 gelang es der Alemannia Aachen, sich für die Westdeutsche Ligaklasse zu qualifizieren, an der sie bis 1913 teilnahm.

Während des 1. Weltkrieges konnte aufgrund der Tatsache, dass die meisten Mitglieder der Mannschaft als Soldaten eingezogen wurden, kein geregelter Spielbetrieb aufrechterhalten werden.
Ein Jahr nach Ende des Krieges konnte der Spielbetrieb wiederaufgenommen werden und der Verein fusionierte mit dem Aachener Turnverein von 1847 zum Aachener Turn- und Sportverein von 1847, der die Sportarten Turnen, Fußball und Leichtathletik betrieb.
1924 wurde diese Fusion wieder aufgelöst und der Verein erhielt seinen heutigen Namen "`Aachener Turn- und Sportverein (ATSV) Alemannia 1900 e.V."'.
Am 3. Juni 1928 wurde das Stadion "`Tivoli"' errichtet, das bis Mai 2009 die Heimspielstätte darstellte, bis es am 17.August 2009 durch einen Neubau ersetzt wurde, der bis heute als Spielstätte dient.

Nach der Machtergreifung der Nationalsozialisten änderten sich die politischen Verhältnisse in Deutschland, so dass der langjährige Vorsitzende des Vereins, Karl Moll, von seinen Aufgaben entbunden und durch den parteitreuen Dr. Peter Müller ersetzt wurde.
Im August 1933 wurden alle Sportvereine aufgerufen, jüdische Mitglieder unverzüglich auszuschließen.
Nach dem 2. Weltkrieg wurde der sportliche Wettbewerb im Fußball professionalisiert und ein Vertragsspielersystem eingeführt.
Infolgedessen waren die Vereine verpflichtet, Steuern abzuführen, was steigende Kosten zur Folge hatte und Alemannia Aachen erst kurz vor dem Saisonende den Klassenerhalt sichern konnte.

1963 bewarb sich Alemannia Aachen vergeblich um die Aufnahme in die neu gegründete Bundesliga, nachdem man zuvor seit Gründer der Oberliga dieser ununterbrochen angehört hatte.
Dies stellte in der Geschichte des Vereins einen Tiefpunkt dar und führte zu einem angespannten Verhältnis zum Lokalrivalen, dem 1. FC Köln.
1967 gelang als Meister der Regionalliga West und dem anschließenden Erfolg in der Aufstiegsrunde der Aufstieg in die Bundesliga.
Ein Jahr später wurde Alemannia Aachen deutscher Vizemeister.
Nach dem Abstieg 1970 aus der Bundesliga qualifizierte sich der Verein im Jahre 1980 für die neu geschaffene 2. Bundesliga, in der er 10 Jahre vertreten war.
1999 konnte Alemannia Aachen nach 9 Jahren Abstinenz in die zweithöchste deutsche Spielklasse zurückkehren und schaffte 2003 mit dem Einzug in das Finale des DFB-Pokals einen der größten Erfolge der Vereinsgeschichte, indem sie sich für den UEFA-Pokal qualifizierte.
2006 gelang nach 36 Jahren der erneute Aufstieg in die Fußball-Bundesliga, welche allerdings nach einer Saison der Zugehörigkeit verlassen werden musste.
2009 wurde schließlich das letzte Spiel im über 80 Jahre alten "`Tivoli"' bestritten.

Neben der traditionsreichen Historie des Vereins war auch das Umfeld des Vereins ein wichtiger Faktor, der zu der Entscheidung führte, Fans von Alemannia Aachen auf ihr sprachliches Handeln im Stadion hin zu untersuchen.
Obwohl der Verein, wie bereits erwähnt, seit der Saison 2013/2014 in der viertklassigen Regionalliga West spielt, besitzt er mit dem "`Tivoli"' deutschlandweit eines der modernsten Stadien in dieser Spielklasse.
Es verfügt über 32960 Plätze, die sich auf 11681 Stehplätze, 19345 Sitzplätze, 1348 Business-Seats, 28 Logen à 12 Plätzen, 100 Plätze für beeinträchtigte Besucher sowie auf 110 Presseplätze verteilt\cite{AA15}.

Zudem wurde das Stadion als eines der ersten mit der sogenannten "`Torlinientechnologie"' ausgestattet, die bereits bei der Fußballweltmeisterschaft 2014 in Brasilien erfolgreich eingesetzt wurde.
Abseits dieser – für die herrschenden sportlichen Bedingungen – überdurchschnittlichen Infrastruktur ist der Verein ganz besonders tief mit den Symbolen und Traditionen der Stadt Aachen verbunden.
So führt der Verein den Adler und die Farben schwarz und gelb in seinem Vereinsemblem und es existieren Fangruppierungen, die einen Bezug zu Karl dem Großen herstellen, der während seiner Herrschaft im 8. Jahrhundert n. Chr. eine Residenz in Aachen besaß und dessen Einfluss auf die Stadt bis heute nachwirkt.

Abschließend stellte sich die Frage, in welchem Verein die Anfertigung einer Tonaufnahme von Fangesängen möglich war.
An dieser Stelle wäre prinzipiell jeder deutsche Fußballverein in Frage gekommen, allerdings ergab sich durch persönliche Kontakte die Möglichkeit, frühzeitig vor Beginn des Spiels das Stadion betreten und die nötigen Vorbereitungen treffen zu können.
Hinzu kam die geographische Nähe des Standortes, was letztendlich zur Wahl des Tivoli-Stadions zur Untersuchung von pejorativen Sprachhandlungen bei Fangesängen führte.
