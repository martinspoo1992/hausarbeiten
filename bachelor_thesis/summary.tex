\section{Resümee}
Zusammenfassend lässt sich sagen, dass sich Fans und Zuschauer deutlich voneinander unterscheiden.
Das betrifft besonders die Motivation, mit welcher sie ein Stadion besuchen.
Fans zeichnen sich einerseits durch ihr Verhalten, das deutlich durch die emotionale Bindung zu ihrem Bezugsobjekt geprägt ist und großen Einfluss auf den Alltag hat, andererseits aber auch durch das Verlangen nach Einflussnahme in besonderem Maße aus.
Außerdem zeigt sich bei ihnen ein Zusammengehörigkeitsgefühl durch eine gemeinsame Leidenschaft.
Dieses "`Wir-Gefühl"' wird folglich auch in den Fangesängen greifbar.
Zuschauer hingegen legen den Schwerpunkt auf den Fußball als Sport und betrachten diesen ohne solche Leidenschaft.

Der Einfluss bestimmter Faktoren auf die Entstehung und die linguistische Beschaffenheit von Fangesängen konnte anhand der analysierten Transkriptionen klar dargelegt werden.
Einflüsse wie das Fanverhalten bei der Fußball-Welt\-meis\-ter\-schaft 1962 in Chile oder die populäre Musik fanden sich auch in Fangesängen von Alemannia Aachen wieder.
Es zeigte sich ergänzend dazu, wie eng die Entwicklung deutscher Fangesänge mit der Ultra-Bewegung in Italien und der Fankultur in England verknüpft ist.
Aus den gesellschaftlichen Entwicklungen in Verbindung mit Beeinflussungen Englands und Italiens entstand letztlich die Fankultur in Deutschland und damit die Fangesänge.

In der Betrachtung des Verhältnisses von Sprache zu Gewalt, das sich für die Untersuchung von Fangesängen als ein zentrales Merkmal herauskristallisiert hat, ergab sich, dass die Sprache selbst keine direkte Gewalt enthält, sondern sich diese aus dem Anwendungskontext ihrer Nutzer ergibt.
Daher ließ sich das Verhältnis nicht als "`Gewalt \emph{und} Sprache"' sondern eher als "`Gewalt \emph{durch} Sprache"' begreifen.

Eingangs stellte sich die Frage, und und inwiefern Fans mit Fangesängen sprachliche Handlungen vollziehen.
Es hat sich gezeigt, dass Fans oftmals wenige Worte oder gar Ausrufe genügen, um pejorative Sprachhandlungen zu erzeugen, mit Hilfe derer sie die unterschiedlichsten Absichten verfolgen.
So hat sich in der Analyse der Transkriptionen verschiedener Fangesänge von Fans des Vereins Alemannia Aachen ergeben, dass diese Handlungen ein Spektrum von Verhöhnungen und Schmähungen bis hin zu ritualisierter Gewalt abdecken können.
Es wurde demzufolge deutlich, dass Fangesänge in diesem Zusammenhang nicht bloß von Unmutsbekundungen geprägt sind, sondern auch bestimmte Eigenschaften besitzen, die ihre jeweilige Funktion unterstützen.

So konnten sich Merkmale wie Adressierung, bestimmte Formulierungen abseits des konkret auf Fußball bezogenen Sprachgebrauchs, aber auch diverse Ausdrucksformen, die sich in der Art und Weise des Ausdrucks unterschieden, nachgewiesen werden.
Die Sprechakttheorie von John Searle ließ sich auf Fangesänge in besonderer Weise anwenden.
Es zeigte sich, dass Fangesänge sprachliche Handlungen in einer anderen Ausprägungsform darstellen.
Dabei handelte es sich teilweise um die Bezugnahme auf den Sport, aber vor allem um seine pejorativen Bezügen.
Diese kamen durch Ausdrücke der Umgangssprache und teils schwere Schimpfwörter zustande.
Zusätzlich konnte anhand eines Beispiels ritualisierte Gewalt nachgewiesen werden.
Auch die emotionale Bindung zwischen den Fans und ihrem Bezugsobjekt, dem Verein, konnte anhand der untersuchten Gesänge belegt werden.

Inhaltlich betrachtet konnte festgestellt werden, dass Fangesänge oftmals sprachliche Gewalt enthalten können, die in unterschiedlicher Schwere verletzende Wirkung entwickeln kann.
So ließen sich eher schwächere und besonders starke Formen sprachlicher Gewalt belegen.
Während es sich bei Gesängen wie "`Ohne Schiri habt ihr keine Chance"' eher um schwächere Formen handelte, kamen bei dem Gesang "`Im Wagen vor mir"' klare Gewaltakte zur Sprache.
Zudem wurde anschaulich, wie groß die Vielfalt der inhaltlichen Tiefe ist.
So konnten pejorative Sprachhandlungen mit wenigen Worten oder Zeilen, aber auch mit ganzen Handlungssträngen vollzogen werden.

Ausblickend kann man feststellen, dass vorliegende wissenschaftliche Erkenntnisse verifiziert werden konnten.
So boten diese musikalische, sprachliche und funktionale Perspektiven der Betrachtung.
In besonderem Maße sei hier das "`Fan-Abitur"' von Höfer genannt, das eine veranschaulichte Darstellung der Erscheinungsformen von Fangesängen liefert.
Der Fokus der Ausarbeitung lag auf der sprachlichen Ebene, was zur Folge hatte, dass gesellschaftliche und soziale Aspekte der Betrachtung von Fangesängen in den Hintergrund traten.
Bei einer Untersuchung von Fangesängen aus dieser Sicht könnte es unter Umständen bedeutsam sein, die Herkunft der Fans aus den verschiedenen Gesellschaftsschichten in den Blick zu nehmen und worin sich Fangesänge verschiedener Vereine in Bezug auf ihre Beeinflussung durch die soziale Herkunft ihrer Fangruppen unterscheiden.

Es lässt sich final konstatieren, dass Fangesänge mehr ausmacht als Rufen oder Singen.
Sie drücken die Gefühle und das besondere Verhältnis der Fans zum Fußball und zu ihrem Verein aus und können demnach als äußerst vielschichtiges Phänomen angesehen werden.
