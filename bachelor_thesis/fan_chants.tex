\section{Was sind Fangesänge? - Entstehung und Geschichte}
Worum es sich bei Fangesängen handelt und durch welche Eigenschaften sie charakterisiert werden, soll im folgenden Abschnitt Bestandteil des Blickfeldes sein.
Fangesänge stellen ein geschichtlich gesehen junges, gesellschaftliches Phänomen dar, das in seiner Geschichte einer anderen Chronologie folgte, als dies der Fußballsport selbst tat.

Reinhard Kopiez und Guido Brink beschreiben in ihrem Werk eine Einteilung von Fangesängen in verschiedene Kategorien, die K.J. Höfer in seiner Hausarbeit "`Musik als Stimulans. Studien zum musikalischen Verhalten von Fußballfans."'\cite[S. 15]{RK98} formuliert hat.
Darin wird zwischen vier unterschiedlichen Ausprägungen differenziert, die pyramidenförmig angeordnet wurden.
Den Sockel dieser Pyramide bilden dabei die sogenannten Primärreaktionen.
Dabei kann es sich um Rufen, Pfeifen oder auch den Einsatz von Lärminstrumenten handeln.
Sie bilden die häufigste und einfachste Ausprägungsform von Fangesängen.

Die zweite Stufe bildet das rhythmische Klatschen, welches die Vorstufe der eigentlichen Gesänge bildet.
Dieses Klatschen kann in Verbindung mit gesprochenen Rufen gebracht werden.\cite[S. 196]{AB08}
Das rhythmische Klatschen stellt zudem die erste Verbindung von Fanverhalten und Musik her.
Auf der dritten Ebene finden sich Kurzgesänge, welche durch eingängige Texte und Melodien, die eine bestimmte Zeit lang wiederholt werden, auszeichnen.

Schließlich stehen auf der obersten Stufe dieses Schemas letztlich die Lieder, welche den Fans nach Georg Brunner sowohl kognitive als auch gesangliche Fähigkeiten abverlangen.
Dabei bezieht er sich auf das Repertoire, dass sich Fangruppen aneignen, welches sowohl in Bezug auf den Text als auch auf die Melodien als heterogen bezeichnet werden kann.
So würden in Fangesängen ganz unterschiedlichen Musikgenres verwendet werden.\cite[S. 197]{AB08}
Diese Heterogenität begründet sich dadurch, dass "`jeder Mensch eine private Ästhetik besitzt, die sich aus einer Vielzahl von verschiedenen, musikalischen Genres und Praxen bildet – ein und dieselbe Person kann Mitglied im Karnevalsverein, Fußballfan, Kirchgänger sein, sein Kind beim Martinszug begleiten oder an Schützenfesten teilnehmen -, [hier] gibt es offensichtlich eine relativ große Schnittmenge."'\cite[S.197]{AB08}
Es zeigt sich, dass es nicht "`die"' Fangesänge gibt, sondern diese immer abhängig vom Hintergrund derer sind, die sie formulieren und letztendlich auch singen.

Fangesänge lassen sich jedoch auch nach ihrer Funktion gliedern. Khodadadi und Gründel unterteilen sie in acht Klassen, denen andersartige Eigenschaften zugeordnet werden.\cite[S. 10]{FK14}
Die erste Klasse bilden die unterstützenden Fangesänge oder auch Vereinslieder.
Sie dienen der grundlegendsten Funktion der Fangesänge im Allgemeinen, der Unterstützung der eigenen Mannschaft.
Eine den unterstützenden Fangesängen ähnliche Klasse bilden die solidarischen Fangesänge, die sich allerdings dadurch unterscheiden, dass sie das Zusammengehörigkeitsgefühl der Fans und der Mannschaft unterstreichen.
Sie gehen über das Unterstützen der Mannschaft hinaus und verbinden Mannschaft und Anhänger.

Dem solidarischen Fangesang steht der fordernde entgegen, bei dem von der Mannschaft eine sportliche Reaktion auf das aktuelle, sportliche Geschehen verlangt wird.
Sie sprengen die Einheit von Fans und Mannschaft, indem sie den Fokus auf einer Art "`Belohnung"' der Fans durch die Mannschaft legen.
Euphorische Fangesänge treten vor allem auf, wenn die favorisierte Mannschaft erfolgreich ist.
Sie bilden eine weitere Klasse von Fangesängen nach Khodadadi und Gründel.
Gesänge wie "`Berlin, Berlin, wir fahren nach Berlin"' in Bezug auf das Erreichen des DFB-Pokalfinales, das jährlich im Berliner Olympiastadion ausgetragen wird, sind Ausdruck der Gefühle, die Fans mit diesen Gesängen ausdrücken.
Diese euphorische Stimmung kann sogar durch huldigende Gesänge gesteigert werden, mit denen Fans ihre Verehrung gegenüber bestimmten Spielern aufgrund erbrachter Leistungen entgegenbringen.
Neben diesen beschriebenen positiven Ausdrucksformen, denen sich Fans bedienen, existieren auch negativ behaftete Gesänge, mit denen sie auf Spielereignisse reagieren können.
Darunter fallen die ausgrenzenden, diffamierenden und irreführenden Gesänge.
Bei den ausgrenzenden Gesängen richtet sich die Kritik der Fans besonders auf die eigene Mannschaft oder den bzw. die Trainer.
Mithilfe dieser Gesänge bringen die Anhänger ihren Unmut über sportliche Entscheidungen oder im Management des Vereins zum Ausdruck.
Verstärkt werden diese Gesänge, wenn die gegnerische Mannschaft zum Ziel wird und die Gesänge diese diffamieren sollen.
Die letzte Klasse bilden die irreführenden Gesänge, welche vor allem die Funktion haben, die gegnerische Mannschaft bzw. gegnerische Fans zu verunsichern oder zu stören.
Als Beispiel für diese Kategorie von Fangesängen lässt sich an dieser Stelle der Ausruf "`Bayern, wir hören nix"' nennen, wobei sich "`Bayern"' durch andere Vereinsnamen ersetzen lässt und auf die Anhänger des genannten Vereins abzielt und diese darauf "`hinweisen"' soll, dass ihre Gesänge kaum zu hören sind.

Betrachtet man Fangesänge aus linguistischer Perspektive, so ermöglichen diese die verschiedensten sprachlichen Handlungsprozesse.
Einerseits kann mit ihnen die eigene Mannschaft unterstützt werden.
Diese Unterstützung kann auch auf einzelne Spieler oder andere beteiligte Personen bezogen werden.
Andererseits können Fangesänge auch dazu dienen, den Gegner oder Entscheidungsträger wie z.B. den Schiedsrichter zu diffamieren und damit zu beeinflussen.
Darauf soll im Verlauf der Ausarbeitung noch weiter eingegangen werden.
Angesichts dessen lässt sich sagen, dass Fangesänge durchaus in der Lage sein können, Einfluss auf das Spielgeschehen zu nehmen.

Betrachtet man die Entwicklung der Fangesänge, wie sie sich in heutiger Form in Fußballstadien darstellen, wird nicht nur die Verknüpfung Englands mit dem Fußballsport selbst, sondern auch mit dem Fußballfangesang in den 1950er Jahren, deutlich.
Nach dem Pokalspiel zwischen dem Blackpool FC und dem Bolton Wanderers FC am 2. Mai 1953 rief der Kapitän der siegreichen Mannschaft den Fans im Stadion "`Hip, Hip"'-Rufe zu, welche diese mit "`Hurra"' beantworteten.
Dieses Ereignis, welches als Ehrbekundung für die englische Königin interpretiert wurde, stellte die erste aktive, sprachliche Handlung von Fußballfans im Stadion dar.
Jedoch handelte es sich dabei noch nicht um einen Fangesang.

Morris sieht für die Entwicklung von Fangesängen drei Quellen als bedeutsam an.
Zum einen erwähnt er die englische Tradition, vor jedem Finalspiel das Lied "`Abide with me"' zu singen, welches eigentlich ein religiöses Lied darstellt.
Desweiteren stellt für ihn der Einfluss Südamerikas einen wichtigen Einflussfaktor in Bezug auf die Entwicklung von Fangesängen dar.
Demnach könne die Begeisterung der Fans durchaus aggressive Verhaltensweisen wie das Abfeuern von Revolvern beinhalten.
Außerdem hätten sich in den 50er und 60er Jahren die Reisemöglichkeiten der Fans durch die Erschließung des Luftverkehrs verbessert, was zur Folge hatte, dass es zu einem internationalen Austausch der Fankulturen kam.
Einen entscheidenden historischen Zeitpunkt in der Entwicklung der Fangesänge sieht Morris zudem in der Fußball-Weltmeisterschaft 1962 in Chile, in deren Folge englische Fußballfans das bereits beschriebene rhythmische Klatschen in ihren Stadien übernommen hätten.
Letztlich gilt die berühmte Stehplatztribüne "`Kop"' im Anfield-Stadion des FC Liverpool als Ursprung des Fangesangs, da die Fans dieses Vereins erstmals die südamerikanischen Traditionen der Fans übernahmen.
Es lässt sich demnach sagen, dass der FC Liverpool der erste Fußballverein ist, bei dem die Fans im Stadion Gesänge anstimmten.
Die dritte entscheidende Quelle stellt für Morris die englische Popmusik dar, welche sich in den 60er Jahren durch Bands wie die Beatles auszeichnete, die ebenfalls aus Liverpool stammten.
Zudem übernahmen die Fans Lieder, die zu diesem Zeitpunkt kommerziell erfolgreich waren in ihr Repertoire, wie die bekannte Hymne "`You'll never walk alone"' von Gerry and the Pacemakers, die zunächst ihren Weg in andere englische Stadien fand und bis heute auch in vielen deutschen Stadien vor den Spielen gesungen wird.
Sie hat sich infolgedessen zu einer "`Hymne der Fußballfans"' entwickelt.

Doch auch die Entwicklungen in Folge der Hillsborough-Katastrophe, bei der 1989 bei einem Stadionunglück 96 Menschen ums Leben kamen und die zur Folge hatte, dass in englischen Stadien Stehplätze verboten wurden, hat eine große Bedeutung für die Entwicklung von Fangesängen.
Daraus resultierte eine Veränderung des Fanverhaltens in englischen Stadien, so dass sich die Fangesänge vom Stadion in ortsansäßige Pubs verlagerten.
Demgegenüber stehen deutsche Fußballstadien, in denen Stehplätze auch weiterhin zulässig sind.
Es erscheint daher nachvollziehbar, welche Bedeutung Stehplatztribünen für die Entwicklung und Entstehung von Fangesängen hatten und bis heute haben.
Die Entwicklung des Gesangs in deutschen Fußballarenen hängt neben dem Einfluss der englischen Fankultur eng mit der sogenannten Ultra-Bewegung zusammen.
Dabei handelt es sich um engagierte Gruppen von Fußballfans, die ihren Verein organisiert unterstützen wollen.
Diese Bewegung entwickelte sich aus einer politisch links einzustufenden Jugend- und Fankultur, welche durch Studenten- und Arbeiterproteste angestoßen wurde.
Dabei bemängelten die Anhänger besonders bestehende Ungleichheiten in der italienischen Gesellschaft und verbreiteten ihre Botschaften vor allem mithilfe von Bannern, die sie in Fankurven von Fußballstadien präsentierten.
In Italien wurde durch die Zuhilfenahme von Megafonen, Doppelhaltern, Rauchkörpern und bengalischen Fackeln der Grundstein für die heutige Fankultur gelegt.
Aus dieser Bewegung heraus folgte darüberhinaus, dass die Fankurven nicht mehr von sich aus begannen, Fangesänge zu singen, sondern von einem "`Capo"', einer Art Vorsänger, koordiniert wurden, was von einigen Fans kritisch beurteilt wurde.
Die erste Gruppierung "`Fossa dei Leoni"', die zum AC Mailand gehört, gründete sich im Jahr 1968.
In den 1970er Jahren entstanden neben den bereits existierenden linken Gruppierungen solche, die sich offen faschistischem und rassistischem Gedankengut zuwandten.
Zudem entstanden Ultragruppen rivalisierender Vereine, was in der Summe zu gewalttätigen Auseinandersetzung aufgrund der politischen oder sportlichen Zugehörigkeit führte.

Die Entwicklung einer sogenannten Fanidentität setzte in Deutschland nach dem Ende des 2. Weltkrieges ein.
Nach Sommerey und Gabler weist diese Parallelen zu den englischen Verhältnissen auf.
So habe die Einführung des Acht-Stunden-Arbeitstages, die Entkopplung traditioneller kultureller Manifestationen und die Verdrängung bürgerlicher Vereine durch Arbeitervereine zu dieser Entwicklung beigetragen.
Anders als in England jedoch, war die Beziehung zwischen Sportlern und Anhängern vor der Professionalisierung des Fußballsports deutlich enger, sodass sich Fans mit ihnen identifizieren konnten, was sich durch den Bau von Stadien und die Vergrößerung der zwischenmenschlichen Entfernung von Spielern zu Fans änderte.
Daraus resultierte "`ein Verhältnis voller emotionaler Spannung, wobei Verehrung und Verachtung nahe beieinander liegen."'
Fußballspieler wurden nun vonseiten der Fans dadurch charakterisiert, dass sie nur solange bei einem Verein aktiv waren, wie es ihnen durch sportliche Erfolge attraktiv genug war.
Im Zuge von baulichen Veränderungen in den Stadien zur Fußball-Weltmeisterschaft 1974 in Deutschland verorteten sich die jugendlichen Fans in den für sie finanziell erschwinglichen Stehkurven, wo sich letztlich nach englischem Vorbild faneigene Traditionen entwickelten.
