\section{Sprachliche Analyse von Fangesängen}
Nachdem die theoretische Grundlage, was sprachliches Handleln ausmacht, gelegt wurde, soll der Fokus nun auf der Analyse von Fangesängen in Bezug auf die Anwendbarkeit der Sprechakttheorie von John Searle gerichtet werden.
Dazu ist es von zentraler Bedeutung, zunächst darauf einzugehen, wie die untersuchten Transkripte gewonnen wurden.

Die vorliegenden Transkripte stammen von einer Aufnahme vom 06. Dezember 2014 aus dem Spiel "`Alemannia Aachen - SG Wattenscheid 09"'.
Das Spiel fand im Zusammenhang mit dem 18. Spieltag der Saison 2014/15 der Regionalliga West von 14 Uhr bis 15:50 Uhr im Aachener "`Tivoli"'-Stadion statt.
Diese Aufnahme wurde in Auszügen mithilfe des Gesprächsanalytischen Transkriptionsverfahrens (GAT) verschriftlicht.
Zudem wurden Aufnahmen von Fangesängen von Alemannia Aachen transkribiert, die aus einem Internetforum stammen, dass sich mit Fangesängen auseinander setzt.

Am Rande sei die sportliche Ausgangslage erwähnt, die Alemannia Aachen auf dem zweiten Tabellenplatz und die gegnerische Mannschaft SG Wattenscheid 09 auf dem 14. Tabellenplatz zeigte.
Nach Ende des Spiels, das Aachen mit 3:0 gewinnen konnte, befand sich Aachen auf dem 1. Platz und Wattenscheid auf dem 16. Platz, der in dieser Ligenstufe einen Abstiegsrang in die fünftklassige Oberliga darstellt.
Zudem handelte es sich bei dem Spiel um das letzte Heimspiel im Kalenderjahr 2014 und dem letzten Spiel vor dem Aufeinandertreffen mit Rot-Weiß Essen, das sowohl durch die vereinsseitige Rivalität als auch durch die sportliche Brisanz, da es Essen gelingen konnte, die "`Herbstmeisterschaft"' der Regionalliga West zu erringen, ein besonderes Fußballspiel darstellt.
Die vorliegende Tonaufnahme beinhaltet Fangesänge und Reaktionen der Fans von der 48. Spielminute bis zum Ende des Spiels.
Für die sprachliche Analyse pejorativer Sprachhandlungen bei Fangesängen wurde die zweite Halbzeit des beschriebenen Fußballspiels herangezogen.

Bei dem ersten, zu untersuchenden Transkript handelt es sich um die Bekanntmachung des Spielstandes, nachdem Aachen in der 64. Spielminute das 2:0 erzielen konnte.
Es entsteht dabei eine Konversation zwischen dem Stadionsprecher und den Anhängern.
Dabei teilt der Stadionsprecher zunächst mit, in welcher Spielminute ein Tor erzielt wurde und für welche Mannschaft, welche im vorliegenden Beispiel die Heimmannschaft darstellt.
Er ergänzt diese Information, indem er darauf hinweist, dass der nachfolgende Spieler bereits in der ersten Halbzeit des Spiels ein Tor erzielen konnte.
Danach teilt er den Anwesenden mit, welche Rückennummer, an der Schiedsrichter und Zuschauer jeden Spieler der Mannschaften erkennen können, derjenige trägt und nennt seinen Vornamen "`Fabian"'.
Die Fans antworten daraufhin, indem sie dessen Nachnamen "`Graudenz"' rufen.
Der Stadionsprecher wiederum geht zur Verkündung des Spielstandes über: "`Damit der neue Spielstand: Die Alemannia"', woraufhin die Fans mit einem gemeinsamen "`Zwei"' reagieren.
Dieser Vorgang wiederholt sich in gleicher Weise mit dem Gastverein – hier Wattenscheid – und die Fans rufen "`Null"'.
Anschließend bedankt sich der Stadionsprecher bei den Fans mit dem Ausruf "`Danke, danke!"' und die Fans erwidern "`Bitte, bitte!"'.

Zunächst lässt sich festhalten, dass sich dieser Fangesang nach den Stufen zum Fan-Abitur zu den Primärreaktionen zählen lässt.
Die Fans antworten lediglich mit Ein-Wort-Konstruktionen bzw. mit dem Ausruf "`Bitte, bitte!"' mit einer Zwei-Wort-Konstruktion.
Weiter lässt sich sagen, dass der vorliegende Fangesang nach Khodadadi und Gründel verschiedene Funktionen erfüllt.
Zunächst lässt er sich als euphorisch einstufen, da die Fans den Erfolg der eigenen Mannschaft unterstreichen.
Parallel dazu grenzen sich die Fans damit jedoch auch von den Anhängern der gegnerischen Mannschaft ab, indem sie das Zahlwort "`Null"' für den Spielstand des Gegners besonders betonen.

Überträgt man den Inhalt dieser Fanreaktionen auf die Sprechakttheorie von John Searle, so kann man sehen, dass sich der Äußerungsakt der Fans lediglich auf die Wörter "`zwei"', "`null"' und "`bitte"' beschränkt, wobei Letzteres wiederholt wird.
Es lässt sich dabei in Bezug auf die Äußerung der Fans darauf verweisen, dass diese Wörter der deutschen Sprache verwenden, jene jedoch nicht in einem vollständigen Satz vorgebracht werden.
Die Fans beziehen sich auf propositionaler Ebene einerseits klar auf den Spielstand des laufenden Spiels, andererseits aber auch auf den beteiligten Spieler.
Dabei weisen sie jenen Referenzobjekten jedoch keine Eigenschaft im eigentlichen Sinne zu, sondern tun dies lediglich indirekt dadurch, dass sie den Erfolg des Spielers der eigenen Mannschaft betonen.
Zudem stellt der gegnerische Verein ein weiteres Referenzobjekt dar, dem die Eigenschaft der Erfolglosigkeit zugeschrieben wird, was sich anhand der Betonung der Äußerung ebenfalls zeigt.

Der illokutionäre Akt dieser Sprachhandlung besteht einerseits darin, den aktuellen Spielstand mitzuteilen.
An dieser Stelle sei jedoch darauf hingewiesen, dass sich dies nur auf die Zahl der erzielten Tore der eigenen Mannschaft bezieht.
So hätten die Fans bei einem Spielstand von 2:1 auf die Frage des Stadionsprechers, wieviele Tore der Gast erzielen konnte, ebenfalls mit "`Null"' geantwortet.
Neben dieser Mitteilung beglückwünschen die Fans den Spieler, indem sie dessen Nachnamen rufen.
Hinzu kommt die Ebene der Behauptung in Bezug auf den Spielstand des Gegners.
Schlussendlich wird durch die Sprachhandlung der Fans die Perlokution der Verunsicherung der gegnerischen Fans vollzogen, welche auf diese Handlung reagieren müssen.
Darüber hinaus lässt sich die Verspottung als mögliche Absicht des Fangesanges nennen, die sich aus der besonderen Betonung des Zahlenwortes "`null"' ergibt.

Auf den Inhalt dieses Fangesangs lassen sich einige der Typen und Parameter für sprachliche Gewalt anwenden.
Zunächst lässt sich sagen, dass die vorliegende Äußerung eine klare Adressierung in Form des Gastvereins SG Wattenscheid 09 vornimmt.
Folglich ist die Äußerung mit einem Bezug zu einer Gruppe, den Fans der SG Wattenscheid 09, versehen.
Zudem lässt sich konstatieren, dass Fangesänge grundsätzlich öffentlich sind und dieser Parameter als grundsätzlich erfüllt angesehen werden kann.
Darüber hinaus findet sich in der vorliegenden Äußerung, wie bereits erwähnt, der Typus der Behauptung, der durch die Aussage über Zahl der erzielten Tore des Gegners deutlich wird.

Auch der Fanausruf "`Spitzenreiter, Spitzenreiter, hey hey!"' lässt sich im Kontext der sportlichen Ausgangslage auf seinen pejorativen Aspekt hin untersuchen.
Wie im vorangegangenen Beispiel bereits beschrieben, findet sich auch hier inhaltlich zunächst ein Fangesang, der lediglich auf der Kommunizierung von Euphorie bezogen ist.
Die Fans gehen mit dem Wort "`Spitzenreiter"' darauf ein, dass die von ihnen unterstützte Mannschaft an der ersten Stelle im Wettbewerb platziert ist.
Diese Botschaft verstärken sie zusätzlich durch die Betonung der ersten beiden Silben des Wortes "`Spitzenreiter"'.
Auf eine Wiederholung dieses Wortes folgt der zweimalige Ruf "`hey"', der zur rhythmischen Untermalung dient.
Bei diesem Gesang handelt es sich um gesprochene Rufe, die von ihrer inhaltlichen Qualität her gesehen über den Primärreaktionen des ersten Transkriptes anzusiedeln sind, auch wenn sich die Äußerungen der Fans wiederum auf wenige Worte beschränken, da sie erneut auf syntaktisch vollständige Sätze verzichten.
Im Unterschied zum ersten Beispiel wird jedoch das Rufen noch durch die rhythmische Komponente ergänzt.

Es lässt sich auch in diesem Fangesang zunächst eine unterstützende Funktion ausfindig machen.
Doch wird dieser Gesang an dieser Stelle auch indirekt ausgrenzend eingesetzt, da davon ausgegangen werden muss, dass lediglich ein Verein zu diesem Zeitpunkt an führender Position steht und dadurch die Abgrenzung von der gegnerischen Mannschaft hergestellt wird.
Wie vorhin reduzieren die Fans ihre Äußerung auf wenige Worte und verzichten dabei auf die Konstruktion von Sätzen, jedoch verwenden sie aus semantischer Perspektive mit dem Wort "`Spitzenreiter"' einen Ausdruck, der lediglich in Bezug auf sportliche Wettkämpfe gebräuchlich ist, wodurch semantisch gesehen eine Bezugnahme auf den Kontext des Sports hergestellt wird.

Sie nehmen in ihrem Gesang Bezug auf die eigene Mannschaft und machen sie dadurch zum Bezugsobjekt der Äußerung.
Diesem wird die Eigenschaft, erfolgreich und der aktuell sportlich stärkste Verein der Liga zu sein, zugeschrieben.

Die Fans äußern auch mit diesem Fangesang eine Mitteilung nach Searle, da sie eine Behauptung aufstellen, dass Aachen an erster Stelle steht, die in diesem Fall auch zutreffend ist.
Gleichzeitig behaupten sie, dass der Gegner der eigenen Mannschaft unterlegen ist, da diese bei der an erster Stelle gelisteten Mannschaft spielt.
Somit vollziehen die Fans mit diesem Gesang einen perlokutionären Akt der Provokation der Gastmannschaft, indem sie sie auf die Tabelle aufmerksam machen, die deren Verein als deutlich schlechter gelistet widerspiegelt.

Die pejorativen Sprechhandlungen von Fußballfans lassen sich jedoch in besonderer Weise an den folgenden Beispielen illustrieren:
Zwischen Minute 26:17 und 26:27 blicken die Fans bereits auf das bevorstehende, erste Heimspiel des Kalenderjahres 2015 voraus, in dem sie sich auf Rot-Weiss Essen beziehen.
Dabei erwähnen sie zunächst lediglich den Vereinsnamen und ergänzen diesen anschließend durch den vulgärsprachlichen Zusatz "`ficken und vergessen"'.
Den Inhalt dieses Ausrufes verstärken sie mit der Betonung der Silbe "`fick"' und der Silbe "`ges"' noch darüber hinaus.
Dabei verwenden sie den Ausdruck "`ficken"', der sonst einen derb umgangssprachlichen Begriff für "`Geschlechtsverkehr ausüben"' darstellt und das Verb "`vergessen"', dass als Reim zum Stadtnamen Essen, aber auch zur inhaltlichen Untermalung der Botschaft dient.
Die Fans nutzen die Fäkalsprache inhaltlich, um darauf aufmerksam zu machen, dass die von ihnen bevorzugte Mannschaft gegen Essen sportlich erfolgreich abschneiden wird.
Man kann also von einer Bedeutungsverschiebung des Wortes "`ficken"' hin zu einer Bedeutung des "`Besiegens"' auf sportlicher Ebene sprechen.
Bei dem Ausdruck "`vergessen"' ist diese semantische Bedeutungsverschiebung jedoch nicht in dieser Form gegeben.
Er verweist darauf, dass man sich nach dem Spiel gegen Essen nicht mehr mit diesem als sportlicher Konkurrenz auseinandersetzen müsse.
Gleichzeitig dient der erzielte Reim zwischen "'Essen"' und "`vergessen"' der Bestätigung und besonderen Betonung des Inhaltes.

Der Ausruf der Fans lässt sich als Kurzgesang charakterisieren, der in seinem Aufbau aus der Wiederholung des Vereinsnamens und des anschließenden Zusatzes besteht.
Diese Einheit wird im zugrunde liegenden Transkript mehrmals wiederholt.
Es wird eine diffamierende Beziehung zwischen dem Vereinsnamen "`Rot-Weiss Essen"' und den Verbformen "`ficken"' und "`vergessen"' hergestellt, die trotz ihres sportlichen Bezuges jene Diffamierung enthalten.
Der Äußerungsakt dieses Fangesangs zeichnet sich wie bei den zuvor skizzierten Fangesängen durch eine Aufzählung aus, die aus dem Namen des Vereines, auf den Bezug genommen wird, sowie den bereits erwähnten Verbformen besteht, die mihilfe der Konjunktion "`und"' verbunden werden.
Des Weiteren wird der Verein Rot-Weiss Essen als Referenzobjekt genutzt, dem menschliche Züge zugeschrieben werden, was an der Wahl der Verben, die sich gewöhnlich klar auf Handlungen zwischen Menschen beziehen, klar wird.
Das konkret sprachliche Handeln der Fans mit diesem Ausruf lässt sich anhand der Sprechaktklassifikationen von Searle verschiedenen Klassen zuordnen.
Einerseits handelt es sich dabei um eine direktive Illokution, da die Fans mit Verweis auf das bevorstehende Spiel gegen Essen die Spieler der eigenen Mannschaft verpflichten, ihr Bestes für den sportlichen Erfolg zu geben.
Andererseits kann dieser Sprechakt auch der kommissiven Kategorie zugeschrieben werden, da er eine Form von Drohung gegenüber Rot-Weiss Essen enthält.
Der perlokutionäre Akt dieser Sprechhandlung liegt daher darin, dass die Anhänger sowohl die eigenen Spieler, aber durchaus auch den zukünftigen Kontrahenten darüber in Kenntnis setzen wollen, dass dieses Sportereignis in naher Zukunft bevorsteht.
Gleichzeitig zeigt sich an diesem Ausruf jedoch auch eine Form der Motivierung der Spieler, die die Fans zum Ziel haben, da sie bekunden, welche Bedeutung ein erfolgreiches Ergebnis für sie nicht nur rein sportlich, sondern auch symbolisch hat, da es sich bei dem angesprochenen Verein um den größten Rivalen Alemannia Aachens handelt.

Der Kurzgesang "`Rot-Weiss Essen, ficken und vergessen"' enthält deutliche Merkmale sprachlicher Gewalt.
So lässt sich erkennen, dass er eine prägnante Adressierung an den Verein Rot-Weiss Essen enthält, der namentlich genannt wird, sowie sich an eine angesprochene Gruppe richtet.
Im Unterschied zu dem ersten Transkript zeigt sich anhand dieses Beispiels jedoch erstmals eine klare Form der Diffamierung, welche durch die aggressive Wortwahl transparent wird.
Auch die Tatsache, dass jener Fangesang sich durch eine erhöhte Lautstärke auszeichnet, potenziert die Wirkung, die der Kurzgesang erzielen soll.

Die besondere Beziehung zwischen Alemannia Aachen und Rot-Weiss Essen zeigt sich gegen Ende der Aufnahme erneut.
Auch hier stimmen sie einen Fangesang an, der sich speziell gegen Essen richtet.
Dabei bedienen sie sich der Melodie des Liedes "`Im Wagen vor mir"' von Henry Valentino und Ursula Peysang aus dem Jahre 1977.
Anhand dessen wird der Bezug von populärer Musik zu Fangesängen und deren Herkunft erkennbar.
Während das originale Lied von einem Mann erzählt, der hinter einem jungen Mädchen fährt, dessen Aufmerksamkeit er anstrebt, wurde der Text von den Fans verändert.

Der Fangesang berichtet, wie ein Fan hinter einem Anhänger Rot-Weiss Essens – an dieser Stelle wird die Abkürzung RWE genutzt – fährt, womit auch auf das ursprüngliche Lied angespielt wird.
Dieser wird konkret als "`RWE-Schwein"' bezeichnet.
Im Folgenden werden gewalttätige Übergriffe gegen diesen skizziert, indem es in dem Ruf heißt:
"`Ich hau ihm eine Flasche über'n Kopf.
Und während er verblutet, schlag ich ihm die Zähne aus."'
Dem folgt gegen Ende der Diebstahl des Fanschals, den dieser bei sich hat, um sich als Fan von Rot-Weiss Essen kenntlich zu machen.
Hier wird der Bezug zu den Fanutensilien hergestellt und auf die Rivalität der Vereine hingewiesen, da der Fan seines identitätsstiftenden Symbols beraubt wird.

Bei diesem Fangesang handelt es sich um ein klassisches Fanlied, dass sich in besonderer Form dadurch von den zuvor behandelten und transkribierten Fangesängen unterscheidet, da es auf einer festen Melodie basiert.
Dieser Bezug zu alltäglicher Musik bildet daher den größten Differenzierungspunkt zwischen Liedern und Kurzgesängen gegenüber den gesprochenen Rufen und den Primärreaktionen, wie es Hofer in seinem Modell des "`Fan-Abiturs"' beschreibt\cite[S. 15]{RK98}.
Darüber hinaus lässt sich über das Lied sagen, dass es sich um einen herabsetzenden Gesang handelt, der im Unterschied zu den anderen Beispielen einen Aufruf zur Gewalt enthält bzw. Gewaltanwendung zumindest nicht grundsätzlich ablehnt.
Dieser Aufruf kann jedoch auch als bloßer Ausdruck der Ablehnung gesehen werden, sodass sich keine eindeutige Aussage über die inhaltliche Tragweite treffen lässt.

Auffällig ist bei der Betrachtung dieser Äußerung, dass erstmals eine Verwendung von Sätzen stattfindet, wodurch in Relation zu den zuvor betrachteten Gesangsbeispielen eine größere, inhaltliche Tiefe erzeugt wird.
So wird als Referenzobjekt ein Bild der Anhänger von Rot-Weiss Essen verwendet, dem durch die pejorative Bezeichnung als "`RWE-Schwein"' ausschließlich negative und verachtenswerte Eigenschaften zugeschrieben werden\cite{Dud15}.
Zudem wird hier das Tierwort "`Schwein"' ausschließlich in pejorativer Semantik gebraucht.

Die Fans, die diesen Gesang verbalisieren, zeigen durch ihr sprachliches Handeln eine kommissive Form der Illokution.
Sie drohen den Anhängern Essens indirekt damit, sie mit Flaschen zu schlagen, jegliche Hilfe zu unterlassen und sie ihrer Fanutensilien zu berauben.
Damit tun sie eine Absicht kund, allerdings lässt sich nicht eindeutig sagen, ob es sich dabei tatsächlich um die Absicht der Gewaltausübung oder aber eine Provokation handelt.
Gleichzeitig behaupten sie mit dem Ausdruck "`RWE-Schwein"', dass es sich bei Anhängern von Rot-Weiss Essen um besonderes dumme und dreckige Menschen handelt, da dies die Eigenschaften sind, die fälschlicherweise Schweinen oft zugeschrieben werden.
Man kann also davon sprechen, dass dieser Fangesang die Perlokution der Beleidigung, Kränkung und Provokation enthält, indem behauptet wird, dass Essens Fans wenig intelligent und ungepflegt seien, sowie ihnen körperliche Gewalt angedroht wird.
Es handelt sich demnach um eine extreme Form pejorativer Sprachhandlungen.

Die Typen und Parameter kommunikativer Gewaltakte von König und Stathi\cite[S. 50]{EK10} lassen sich auch hier teilweise anwenden.
Aus dem Gesang wird deutlich, dass er eine klare Adressierung an Fans von Rot-Weiss Essen enthält.
Gleichzeitig ist er an die Zugehörigkeit zu einer Gruppe – in diesem Fall der Gruppe von RWE-Fans – gerichtet und weist eine Diskriminierung auf, die aus der Eigenschaft abzuleiten ist, dass diese Fans des Gegners sind.
Auch lässt sich sagen, dass es sich um eine Behauptung handelt, die nicht den Anspruch hat, Wahrheit zu kommunizieren.

Bisher hat sich anhand der Analyse von Fangesängen aus dem Spiel gegen die SG Wattenscheid 09 gezeigt, dass die Fans in ihren Gesängen offensichtlich lediglich die Anhänger gegnerischer Vereine und die Vereine selbst zu Referenzobjekten machen.
Sei es die Verkündung des Spielstandes, bei der neben der Euphorie des erzielten Tores auch die Schmähung des Gegners ein Ziel ist oder aber die Titulierung gegnerischer Fans als "`Schweine"'.
Jedoch ist aus der Forschung über das Fanverhalten bereits deutlich geworden, dass Fans durchaus auch andere Personen und Personengruppen als Anlass für Fangesänge nehmen.
Dies sollen im Folgenden zwei Beispiele veranschaulichen, die von einem Internetforum stammen, das Tonaufnahmen von Gesängen verschiedener Vereine aus Fußball, Basketball, Handball und Eishockey sammelt und archiviert\cite{BT14}.

Neben der zuvor skizzierten Rivalität zu Rot-Weiss Essen, wird anhand des ersten Beispiels ein vergleichbares Verhältnis zu Arminia Bielefeld beschrieben.
Zu Beginn wird auf Ostwestfalen, einen Teil des Landesteiles Westfalen Bezug genommen, das neben dem Rheinland für die Namensgebung des Bundeslandes Nordrhein-Westfalen verantwortlich ist.
Anschließend folgt eine Ergänzung mit der Bezeichnung "`Idioten"', mit der auf die Einwohner Ostwestfalens verwiesen wird.

Abschließend äußern die Fans mit dem Ausruf "`Scheiß Arminia Bielefeld"' ihre Abneigung gegenüber diesem direkt angesprochenen Fußballverein.
Dabei wird durch die verlängerte Wiedergabe der Silbe "`mi"' eine Verschärfung und Betonung der Ablehung erreicht.
Im Unterschied zu den vorherigen Fangesängen, bei denen sich der Inhalt lediglich auf den Verein an sich bezog, wird anhand dieses Gesangs zunächst ein ganzer Landesteil miteinbezogen.

Mit der Verbindung zwischen dem Ausdruck "`Idiot"' und dem Begriff "`Ostwestfalen"', der in diesem Kontext nicht geographisch sondern kulturell-ethnisch die Bevölkerungsgruppe bezeichnet, werden alle Einwohner Ostwestfalens indirekt mit den Eigenschaften verbunden, die die Bedeutung des Wortes "`Idiot"' nahelegt.
Diese Titulierung wird letztlich dann mit der Erwähnung des Vereinsnamens "`Arminia Bielefeld"' präzisiert, da nun die Stadt Bielefeld und der ortsansässige Verein zum Objekt des Gesanges werden.
Während zuvor behauptet wurde, dass alle Ostwestfalen Idioten seien, wird diese Aussage letztlich auf Fans von Arminia Bielefeld übertragen.

Ordnet man den Fangesang "`Ostwestfalen, Idioten, Scheiß Arminia Bielefeld"' in das Modell des Fan-Abiturs ein, so stellt man fest, dass sich dieser den Kurzgesängen zuordnen lässt.
Die Fans bedienen sich dabei der Melodie des Liedes "`Oh my darling Clementine"' von Percy Montrose aus dem Jahre 1884 und verändern seinen Text entsprechend.
Dabei wird eine bestimmte Textzeile mehrfach wiederholt.
Nach Khodadadi und Gründel lässt sich dieser Gesang bei Betrachtung seiner Funktion den diffamierenden Gesängen zuordnen, was anhand der Verwendung der Wörter "`Idiot"' oder "`Scheiß"' deutlich wird.
Die Fans bringen ihre Äußerung in Form einer Aufzählung zum Ausdruck, bei der sie jedoch keinen syntaktisch vollständigen Satz formulieren, da ein Verb, das die Funktion eines Prädikates erfüllen könnte, gänzlich fehlt.

Wie bereits aus inhaltlicher Perspektive klar wurde, bedienen sich die Fans verschiedener Referenzobjekte.
Zunächst gehen sie dabei auf die Ostwestfalen ein, denen sie Eigenschaften wie Dummheit oder Unwissenheit zuschreiben, indem sie die Bezeichnung Idiot verwenden.
In der zweiten Textzeile wird dann Arminia Bielefeld zum Referenzobjekt, das durch den vorweg genommenen Ausdruck "`Scheiß"', einer verkürzten Form des Fäkalbegriffes "`Scheiße"', näher beschrieben wird.
Der Ausdruck, der in anderem Kontext entweder für die Bezeichnung menschlichen Kotes oder besonders schlechter Dinge verwendet wird\cite{Dud14c}, drückt hier die Ablehnung der Fans aus.
Die sprachliche Handlung der Fans findet vor allem repräsentativ statt.
Sie stellen zunächst die Behauptung auf, dass Ostwestfalen Idioten seien und verallgemeinern damit eine solche Aussage.
Anschließend drücken sie durch ihre Ablehnung eine expressive Haltung aus, die sich in Form der Zeile "`Scheiß Arminia Bielefeld"' äußert.
Damit vollziehen sie einerseits die Illokution der Behauptung, andererseits jene der Kränkung.
Perlokutionär handelt es sich damit um eine Beleidigung und Provokation, die erst allgemein und dann konkret vorgebracht wird.

Die Besonderheit dieser sprachlichen Äußerung besteht darin, dass die Fans zunächst eine allgemeine Provokation vornehmen, die dann verdeutlicht wird.
Der Ausdruck "`Scheiß Arminia Bielefeld"' stellt dabei gleichzeitig einen Rückbezug dar, da nun die Fans Bielefelds, dadurch, dass sie Ostwestfalen sind, als Idioten betrachtet werden und zudem der von ihnen unterstützte Verein durch die Bedeutung an sich herabgesetzt wird.
Es findet sich demnach eine Pluralität in der Bedeutung und Funktion dieses Ausrufs.
Der Fangesang ist klar an Ostwestfalen und im Folgenden auch an Arminia Bielefeld adressiert, wobei im Unterschied zu den zuvor festgestellten Adressaten die Gruppe der Ostwestfalen eine deutlich größere darstellt, die in die Diffamierung miteinbezogen wird.
Es findet durch die Bezeichnung "`Idiot"' eine Diffamierung statt, die durch den Bezug auf die kognitive Leistungsfähigkeit, die Ostwestfalen und konkret Bielefeldern unterstellt wird, als durchaus stark eingestuft werden kann.

Bei dem letzten, zu untersuchenden Fangesang ist das Handeln des Schiedsrichters Anlass für die Fans, sich zu äußern.
In ihrem Gesang gehen sie darauf ein, dass der sportliche Kontrahent ohne die "`Hilfe"' des Schiedsrichters chancenlos gegen die eigene Mannschaft sei.
So wird dem Schiedsrichter unterstellt, durch seine Entscheidungen Einfluss auf das Spiel zugunsten der gegnerischen Mannschaft zu nehmen.
Schlussendlich wird die eigene Mannschaft durch diesen Ausspruch von jeglicher Verantwortung für die sportlichen Ereignisse genommen.
Es zeigt sich folglich ein Perspektivwandel der Fans, die nicht die von ihnen unterstützte Mannschaft für einen eventuellen Misserfolg verantwortlich machen, sondern andere Personen.

Es handelt sich bei dem vorliegenden Gesang um einen Kurzgesang, bei dem die Melodie des bekannten Schlagers "`Einer geht noch"' verwendet wurde, dessen ursprünglicher Interpret unbekannt ist.
Zudem wird ebenfalls eine Textzeile mehrfach wiederholt, mit der die Fans eine ausgrenzende Haltung einnehmen.
Diese äußert sich in deutscher Umgangssprache, was anhand der Verwendung der verkürzten Form "`Schiri"' für "`Schiedsrichter"' kenntlich gemacht wird.
Verglichen mit dem zuvor behandelten Fangesang bedienen sich die Anhänger an dieser Stelle eines phonologisch und grammatikalisch gesehen vollständigen Satzes in deutscher Sprache, da diese Äußerung mit dem Pronomen "`ihr"' ein Subjekt, dem Hilfsverb "`haben"' ein Prädikat und mit dem Nomen "`Schiri"' ein Objekt besitzt.

Die Äußerung besitzt zwei zentrale Referenzobjekte.
Einerseits beziehen sich die Fans auf den Schiedsrichter, dem die Eigenschaft zugeordnet wird, nicht neutral zu sein und die gegnerische Mannschaft zu übervorteilen.
Andererseits wird durch das Pronomen "`ihr"' auch die gegnerische Mannschaft zum Referenzobjekt gemacht, der das Merkmal der Annahme eines unfairen Vorteils durch den Schiedsrichter, zugeteilt wird.
Dadurch zeichnet sich auch dieser Fangesang durch eine Pluralität der Referenzobjekte aus.

Diese führt sich auch in Zusammenhang zu den Illokutionen dieser Sprachhandlung fort, die an dieser Stelle repräsentativ, direktiv und expressiv sind.
Einerseits stellen die Fans mithilfe dieses Gesanges zwei Behauptungen auf.
Die erste bezieht sich auf den Schiedsrichter, dem unterstellt wird, dass er seiner Aufgabe, einen fairen und gerechten Ablauf des Spiels zu gewährleisten, nicht ordnungsgemäß nachkommt.
Andererseits wird der anderen Mannschaft vorgeworfen, aus dieser angeblichen Benachteiligung der eigenen Mannschaft vonseiten des Schiedsrichters Kapital schlagen zu wollen, da sie nicht durch sportliche Leistung gewinnen könne.
Zugleich findet sich in dieser Äußerung auch eine Aufforderung an den Schiedsrichter, zukünftig neutral, bzw. mehr im Sinne der eigenen Mannschaft - zu entscheiden.
Schlussendlich ist die Äußerung expressiv, da die Fans mit ihr ihr Klagen über die bestehende Situation kundtun und diese zu ändern versuchen.
Die Perlokution dieses Fangesangs kann final dahingehend verstanden werden, dass die Fans mit ihrer Äußerung einerseits den Schiedsrichter in seiner Entscheidungsfindung zu beeinflussen versuchen und andererseits die gegnerische Mannschaft zur Fairness zu mahnen, indem diese auf Fehlentscheidungen reagiert.

Die Adressierung dieses Fangesanges liegt in der Bezugnahme auf den Schiedsrichter, welcher direkt mit dem Wort "`Schiri"' angesprochen wird.
Dabei handelt sich logischerweise um eine Äußerung, die anders als die vorherigen, an eine bestimmte Person gerichtet sind.
Es zeigt sich auch anhand dieses Gesanges erstmals der Einfluss eines Machtverhältnisses, da der Schiedsrichter als Instanz, der den Ablauf des Fußballspiels regeln soll und daher in einer Hierarchie deutlich über den Fans anzusiedeln wäre, zum Objekt des Gesanges gemacht wird.
Die Fans versuchen dadurch eine Form der Machtausübung auf den Schiedsrichter zu vollziehen, die ihnen grundsätzlich nicht gegeben ist.
Auch wird die subjektive Meinung der Fans durch den Gesang deutlich zum Ausdruck gebracht.
