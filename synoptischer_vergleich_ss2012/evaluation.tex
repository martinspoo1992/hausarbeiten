\section{Auswertung}

Im folgenden Abschnitt sollen die inhaltlichen Ergebnisse des synoptischen Vergleichs ausgewertet werden. Dabei wird zunächst auf die Unterschiede und dann auf die Gemeinsamkeiten eingegangen.
Auf den ersten Blick erkennt man in der tabellarischen Darstellung schon, dass die Stelle des Matthäusevangeliums die längste der drei vorliegenden darstellt. Dies lässt sich unter anderem durch die ausgeschmückte Beschreibung erklären, die  unmittelbar auf den Tod Jesu folgt. Ein großes Erdbeben erschüttert das Land, die Felsen bersten und die Heiligen entsteigen ihrer Gräber. Da sich diese Passage in den Texten von Lukas und Markus gar nicht findet, liegt die Vermutung nahe, dass dieser Teil der Erzählung aus dem Sondergut des Matthäus stammt und hier zeigt sich, dass Matthäus den Tod Jesu als etwas apokalyptisches interpretiert. Ein weiterer Grund für diese Beschreibung könnte darin liegen, dass Matthäus die Tatsache hervorheben möchte, dass Jesus Gottes Sohn ist und sich für die Menschen geopfert hat. Als Symbol dient Matthäus hier besonders das Bersten der Felsen und die Auferstehung der Heiligen. Ein weiteres Beispiel für die Verwendung von Sondergut ist die, dass Lukas schreibt: \glqq Und die ganze Volksmenge, die zu diesem Schauspiel mitgekommen war, schlug sich beim Anblick dessen, was geschehen war, an die Brust und kehrte zurück.\grqq\ (Lk 23,48) Diese Stelle zeigt, dass Lukas herausstellt, wie sich das Volk verhält. Daher schwingt in dieser Textpassage ein mahnender Unterton, weil er beschreibt, wie das Volk buchstäblich umkehrt.

Auffällig ist auch, dass der Hauptmann bei Matthäus den Tod Jesu nicht alleine zur Kenntnis nimmt, sondern, dass von mehreren weiteren die Rede ist, die bei ihm sind. Diese Beschreibung soll vermutlich die Authentizität der Handlung unterstreichen, indem weitere Beobachter genannt werden. In der unterschiedlichen Formulierung, ob Jesus ein \glqq Gerechter\grqq\ (Lk 23,47) oder \glqq Gottes Sohn \grqq (Mk 15,39 und Mt 27,54) war, zeigt sich, dass Markus und Matthäus im Gegensatz zu Lukas nur das Geschehen an sich beobachten, wohingegen Lukas interpretiert. Lukas will mit dieser Formulierung vermutlich seine heidenchristlichen Leser ansprechen.
Bei allen Unterschieden die sich in der Darstellung des Todes Jesu ergeben, gibt es auch Gemeinsamkeiten, die dafür sprechen, dass das Markusevangelium als Vorlage diente und die ursprünglichste Fassung darstellt. Grundsätzlich gemein ist allen Texten, dass der Vorhang des Tempels zerreißt, was vom Leser als schreckliches Ereignis aber auch als Symbol der Befreiung der Menschen durch Jesus gedeutet werden kann. In \cite{AH04} wird die These vertreten, dass der Vorhang, der im jüdischen Tempel das Heilige vom Rest trennt und in dessen Raum nur der Hohepriester einmal jährlich eintreten darf, für die Distanz zwischen Gott und den Menschen steht. Durch den Riss mitten hindurch, wird deutlich, dass durch Jesu Tod am Kreuz Gott den Menschen nah ist.

Insgesamt gesehen kann man feststellen, dass ein Grundgerüst der Handlung vorliegt, welches aus dem Reißen des Vorhangs im Tempel, dem Handeln des Hauptmanns und der individuellen Beschreibung der Anwesenden besteht. Gefüllt wird dieses Gerüst letztlich vom Sondergut des Matthäus bzw. Lukas, die damit eine Akzentuierung der Handlung vornehmen. Zur anfangs gestellten Forschungsfrage, inwiefern sich anhand der Textpassage Mk 15,38-41 die Zwei-Quellen-Theorie belegen lässt, kann man sagen, dass sich einige inhaltliche und syntaktische Unterschiede wie Gemeinsamkeiten auf diese Weise veranschaulichen lassen.


