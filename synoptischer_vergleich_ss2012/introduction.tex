\section{Einleitung}zur 
In der folgenden Hausarbeit soll eine synoptische Analyse der biblischen Textstelle Mk 15,38-41 durchgeführt werden. Dabei liegt das besondere Augenmerk auf der Untersuchung der Evangelien auf inhaltliche und sprachliche Differenzen. Um einen Einstieg zu ermöglichen, will ich die Methode des synoptischen Vergleiches kurz erläutern.

Beim synoptischen Vergleich handelt es sich um eine Untersuchungsmethode, bei der vergleichbare Textpassagen der synoptischen Evangelien tabellarisch nebeneinander gestellt werden. Textliche Unterschiede werden bis hinunter zur Wortebene gegenübergestellt. In dieser Hausarbeit wird es sich dabei um die zu Beginn genannte Passage des Markusevangeliums sowie Lk 23,45-49 und Mt 27,51-56 handeln.

Der synoptische Vergleich bietet uns die Möglichkeit, eine Textstelle aus den synoptischen Evangelien nach Markus, Lukas und Matthäus näher zu untersuchen. Nach der Zweiquellentheorie geht man davon aus, dass das Markusevangelium als erstes entstand und als Vorlage für Matthäus und Lukas diente. Diese haben sich wiederum zusätzlich noch einer Logienquelle Q und ihres jeweiligen Sondergutes bedient.

Als Forschungsfrage leitet sich daher ab: \glqq Inwiefern kann man die Zweiquellentheorie anhand der Textpassage Mk 15,38-41 bestätigen?\grqq\
Die Wahl des Themas der Passion für einen synoptischen Vergleich ergab sich daraus, dass mir die Passion bereits aus dem Religionsunterricht bekannt ist und sie zu einem der Themen der Bibel gehört, die mich am meisten interessieren. Da die Größe der Textstelle auf vier Verse vorgegeben war, musste eine Stelle gewählt werden, die inhaltlich und sprachlich genug Untersuchungspunkte bietet. Des Weiteren lässt sich anhand der Leidensgeschichte Jesu gut darstellen, worauf die Evangelisten den inhaltlichen Schwerpunkt in ihren Evangelien legten, um ihre jeweiligen Zielgruppen anzusprechen.
