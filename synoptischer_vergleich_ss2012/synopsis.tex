\section{Synopse und inhaltliche Analyse}
Zunächst sollen die drei Textstellen, die aus \cite{CP75} entnommen wurden, in der Synopse gegenübergestellt werden. Hierzu werden die Unterschiede im Text farblich gekennzeichnet, sodass auf den ersten Blick ersichtlich ist, welche Textstellen Gemeinsamkeiten aufweisen und welche sich zwischen den drei Evangelien unterscheiden.

Textstellen die in allen drei Evangelien identisch enthalten sind, werden in der Synopse blau hinterlegt markiert. Solche, die sich als Zusammenführung des ursprünglichen Markustextes mit dem Sondergut des jeweiligen Evangelisten erwiesen haben und die somit nur zwischen Markus und Matthäus bzw. Lukas identisch sind, werden braun (identisch zwischen Mk und Mt) bzw. orange (identisch zwischen Mk und Lk) markiert.
Textfragmente, die rein aus der Logienquelle Jesu stammen und die folglich Markus unbekannt waren, sind rot gekennzeichnet.

Der übrige Text entstammt den jeweiligen Sondergütern der beiden Evangelisten Matthäus und Lukas oder wurde von dem Ursprungstext des Markus nicht übernommen. Dieser Text ist grün markiert.


\definecolor{orange}{cmyk}{0, 0.6, 1.0, 0}
\definecolor{green}{cmyk}{0.7, 0, 0.9, 0}
\definecolor{blue}{cmyk}{1, 0.3, 0, 0.1}
\definecolor{red}{cmyk}{0, 1, 1, 0.2}
\definecolor{brown}{cmyk}{0.2, 0.7, 1, 0.2}

\begin{center}
\begin{longtable}{p{4cm}|p{4cm}|p{4cm}}
\hline
\textbf{Mt\,27,51-56} & \textbf{Mk\,15,38-41} & \textbf{Lk\,23,45-49} \\
\hline

\endhead

\sethlcolor{brown}\hl{Und} \sethlcolor{green}\hl{siehe}, \sethlcolor{blue}\hl{der Vorhang im Tempel} \sethlcolor{brown}\hl{zerri{\ss} von oben bis unten in zwei St"ucke}

\sethlcolor{green}\hl{und die Erde erbebte, und die Felsen zerrissen, und die Gr"ufte "offneten sich, und viele Leiber der entschlafenen Heiligen wurden auferweckt; und sie kamen nach seiner Auferweckung aus den Gr"uften hervor, gingen in die heilige Stadt und erschienen vielen.}

&

\sethlcolor{brown}\hl{Und} \sethlcolor{blue}\hl{der Vorhang im Tempel} \sethlcolor{brown}\hl{zerri{\ss} in zwei St"ucke von oben bis unten.}

&

\sethlcolor{green}\hl{indem die Sonne ihren Schein verlor;} \sethlcolor{blue}\hl{der Vorhang im Tempel} \sethlcolor{green}\hl{ri{\ss} mitten entzwei.}

\\

\sethlcolor{blue}\hl{Als aber der Hauptmann} \sethlcolor{green}\hl{und die, welche mit ihm Jesus bewachten, das Erdbeben sahen und was da geschah, f"urchteten sie sich sehr und sagten:}

&

\sethlcolor{blue}\hl{Als aber der Hauptmann,} \sethlcolor{green}\hl{der ihm gegen"uber in der N"ahe stand,} \sethlcolor{orange}\hl{sah,} \sethlcolor{green}\hl{da{\ss} er auf diese Weise verschieden war,}
\sethlcolor{orange}\hl{sprach er:}

&

\sethlcolor{blue}\hl{Als aber der Hauptmann} \sethlcolor{orange}\hl{sah,} \sethlcolor{green}\hl{was geschehen war, pries} \sethlcolor{orange}\hl{er} \sethlcolor{green}\hl{Gott und} \sethlcolor{orange}\hl{sprach:}

\\

\sethlcolor{blue}\hl{Dieser} \sethlcolor{brown}\hl{war in Wahrheit Gottes Sohn.}

&

\sethlcolor{blue}\hl{Dieser} \sethlcolor{orange}\hl{Mensch} \sethlcolor{brown}\hl{war in Wahrheit Gottes Sohn.}

&

\sethlcolor{blue}\hl{Dieser} \sethlcolor{orange}\hl{Mensch} \sethlcolor{green}\hl{war wirklich ein Gerechter. Und die ganze Volksmenge, die zu diesem Schauspiel mitgekommen war, schlug sich beim Anblick dessen, was geschehen war, an die Brust und kehrte zur"uck.}

\\

\sethlcolor{blue}\hl{Es} \sethlcolor{brown}\hl{sahen} \sethlcolor{blue}\hl{aber} \sethlcolor{green}\hl{dort viele} \sethlcolor{blue}\hl{Frauen} \sethlcolor{brown}\hl{von ferne zu,} \sethlcolor{blue}\hl{die} \sethlcolor{green}\hl{Jesus} \sethlcolor{red}\hl{von} \sethlcolor{blue}\hl{Galil"aa} \sethlcolor{red}\hl{her} \sethlcolor{green}\hl{gefolgt} \sethlcolor{red}\hl{waren,} \sethlcolor{green}\hl{um ihm zu dienen; und unter diesen waren} \sethlcolor{brown}\hl{Maria aus Magdala und Maria, die Mutter} \sethlcolor{green}\hl{des} \sethlcolor{brown}\hl{Jakobus} \sethlcolor{green}\hl{und} \sethlcolor{brown}\hl{Joses,} \sethlcolor{green}\hl{und die Mutter der S"ohne des Zebed"aus.}

&

\sethlcolor{blue}\hl{Es} \sethlcolor{brown}\hl{sahen} \sethlcolor{blue}\hl{aber} \sethlcolor{green}\hl{auch} \sethlcolor{blue}\hl{Frauen} \sethlcolor{brown}\hl{von ferne zu, unter} \sethlcolor{green}\hl{ihnen} \sethlcolor{brown}\hl{Maria aus Magdala und Maria, die Mutter} \sethlcolor{green}\hl{von} \sethlcolor{brown}\hl{Jakobus} \sethlcolor{green}\hl{dem J"ungern und von} \sethlcolor{brown}\hl{Joses,} \sethlcolor{green}\hl{und von Salome,} \sethlcolor{blue}\hl{die} \sethlcolor{orange}\hl{ihm,} \sethlcolor{green}\hl{als er in} \sethlcolor{blue}\hl{Galil"aa} \sethlcolor{green}\hl{war, folgten und dienten, und viele andre, die mit ihm nach Jerusalem hinaufgezogen waren.}

&

\sethlcolor{blue}\hl{Es} \sethlcolor{green}\hl{standen} \sethlcolor{blue}\hl{aber} \sethlcolor{green}\hl{alle seine Bekannten von ferne und die} \sethlcolor{blue}\hl{Frauen, die} \sethlcolor{orange}\hl{ihm} \sethlcolor{red}\hl{von} \sethlcolor{blue}\hl{Galil"aa} \sethlcolor{red}\hl{her} \sethlcolor{green}\hl{nachgefolgt} \sethlcolor{red}\hl{waren,} \sethlcolor{green}\hl{und sahen dies.}

\\

\caption{Synopse der untersuchten Textstellen\label{tab:synopse}, entnommen aus \cite{CP75}}
\end{longtable}
\end{center}

Inhaltlich ist zu den vorliegenden Textstellen zu sagen, dass sie alle die Folgen und Reaktionen des Kreuzestodes Jesu beschreiben. Nachdem Jesus seinen Geist ausgehaucht hat, zerreißt bei allen synoptischen Evangelien der Vorhang des Tempels. Die Beschreibung des Tempelvorhangs unterscheidet sich nur innerhalb der Satzstellung zwischen Markus und Matthäus. Lukas verkürzt die Beschreibung \glqq von oben bis unten \grqq  und ersetzt sie durch \glqq entzwei\grqq.

Bei Matthäus folgt darauf die Beschreibung des Erdbebens, dass die Felsen spaltet. Darüber hinaus kommen die verstorbenen Heiligen aus ihren Gräbern hervor und erscheinen den Menschen in der Stadt. Bei Markus und Lukas fehlt diese Passage.

Übereinstimmend setzen alle drei Texte wieder bei der Beschreibung des Hauptmanns ein, der den Tod Jesu am Kreuz kommentiert. Der Satzanfang \glqq Als aber der Hauptmann \grqq ist bei allen sowohl syntaktisch als auch sprachlich identisch. Matthäus beschreibt als Einziger, dass der Hauptmann nicht allein das Kreuz bewacht und dass alle Anwesenden von dem Erdbeben verängstigt sind. Desweiteren gibt es bei dem, was der Hauptmann zum Geschehen sagt, jedoch große Unterschiede im Sinn. Bei Matthäus heißt es in Mt 27,54: \glqq Dieser war in Wahrheit Gottes Sohn.\grqq Der identische Satz findet sich auch bei Mk 15,39. Lukas schreibt in Lk 23,47 lediglich: \glqq Dieser Mensch war wirklich ein Gerechter.\grqq

Die Erzählung endet mit der Beschreibung der anwesenden Personen, bei der sich auch wieder große Unterschiede in den einzelnen Evangelien ergeben. Während die Nennung derjenigen, die der Kreuzigung beiwohnten, bei Matthäus und Markus sehr ausführlich ist und diese beim Namen genannt werden (Maria, die Mutter Jesu,  Maria aus Magdala, etc.), spart Lukas diese Informationen aus und bezeichnet sie lediglich allgemein als die \glqq Bekannten \grqq Jesu.  Die detaillierte Nennung der Namen derer, die dem Geschehen beiwohnten, mag ein Hinweis darauf sein, dass die Evangelisten davon ausgehen, dass ihren Adressaten die genannten Personen noch persönlich bzw. vom Hörensagen bekannt waren.
