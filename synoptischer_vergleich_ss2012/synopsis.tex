\definecolor{brown}{cmyk}{0, 0.6, 1.0, 0}

\section{Synopse und inhaltliche Analyse}
\begin{center}
\begin{longtable}{p{4cm}|p{4cm}|p{4cm}}
\hline
\textbf{Mt\,27,51-56} & \textbf{Mk\,15,38-41} & \textbf{Lk\,23,45-49} \\
\hline

\endhead

\colorbox{brown}{Und siehe}, der Vorhang im Tempel zerriß von oben bis unten in zwei Stücke, und die Erde erbebte, und die Felsen zerrissen, und die Grüfte öffneten sich, und viele Leiber der entschlafenen Heiligen wurden auferweckt; und sie kamen nach seiner Auferweckung aus den Grüften hervor, gingen in die heilige Stadt und erschienen vielen.

&

Und der Vorhang im Tempel zerriß in zwei Stücke von oben bis unten.

&

indem die Sonne ihren Schein verlor; der Vorhang im Tempel riß mitten entzwei.

\\

Als aber der Hauptmann und die, welche mit ihm Jesus bewachten, das Erdbeben sahen und was da geschah, fürchteten sie sich sehr und sagten:

&

Als aber der Hauptmann, der ihm gegenüber in der Nähe stand, sah, daß er auf diese Weise verschieden war,
sprach er:

&

Als aber der Hauptmann sah, was geschehen war, pries er Gott und sprach:

\\

Dieser war in Wahrheit Gottes Sohn.

&

Dieser Mensch war in Wahrheit Gottes Sohn.

&

Dieser Mensch war wirklich ein Gerechter. Und die ganze Volksmenge, die zu diesem Schauspiel mitgekommen war, schlug sich beim Anblick dessen, was geschehen war, an die Brust und kehrte zurück.

\\

Es sahen aber dort viele Frauen von ferne zu, die Jesus von Galiläa her gefolgt waren, um ihm zu dienen; und unter diesen waren Maria aus Magdala und Maria, die Mutter des Jakobus und Joses, und die Mutter der Söhne des Zebedäus.

&

Es sahen aber auch Frauen von ferne zu, unter ihnen Maria aus Magdala und Maria, die Mutter von Jakobus dem Jüngern und von Joses, und von Salome, die ihm, als er in Galiläa war, folgten und dienten, und viele andre, die mit ihm nach Jerusalem hinaufgezogen waren.

&

Es standen aber alle seine Bekannten von ferne und die Frauen, die ihm von Galiläa her nachgefolgt waren, und sahen dies.

\\

\caption{Synopse der untersuchten Textstellen\label{tab:synopse}}
\end{longtable}
\end{center}

Inhaltlich ist zu den vorliegenden Textstellen zu sagen, dass sie alle die Folgen und Reaktionen des Kreuzestodes Jesu beschreiben.  Nachdem Jesus seinen Geist ausgehaucht hat, zerreißt bei allen synoptischen Evangelien der Vorhang des Tempels. Die Beschreibung des Tempelvorhangs unterscheidet sich nur innerhalb der Satzstellung zwischen Markus und Matthäus. Lukas spart sich die Beschreibung „von oben bis unten“ und ersetzt sie durch „entzwei“.

Bei Matthäus folgt darauf die ausgeschmückte Beschreibung des Erdbebens, dass die Felsen spaltet. Darüber hinaus kommen die verstorbenen Heiligen aus ihren Gräbern hervor und erscheinen vielen Menschen in der Stadt. Bei Markus und Lukas fehlt die Passage.

Übereinstimmend setzen alle drei Texte wieder bei der Beschreibung des Hauptmanns ein, der den Tod Jesu am Kreuz kommentiert. Der Satzanfang „Als aber der Hauptmann“ ist bei allen sowohl syntaktisch als auch sprachlich identisch. Matthäus beschreibt als Einziger, dass der Hauptmann nicht allein das Kreuz bewacht und dass alle Anwesenden von dem Erdbeben verängstigt sind. Desweiteren gibt es bei dem, was der Hauptmann zum Geschehen sagt, jedoch große Unterschiede im Sinn. Bei Matthäus heißt es in Mt 27,54: „ Dieser war in Wahrheit Gottes Sohn.“ Der identische Satz findet sich auch bei Mk 15,39. Lukas schreibt in Lk 23,47 lediglich: „Dieser Mensch war wirklich ein Gerechter.“ Markus und Matthäus beobachten im Gegensatz zu Lukas nur das Geschehen an sich, wohingegen Lukas interpretiert.

Die Erzählung endet mit der Beschreibung der anwesenden Personen, bei der sich auch wieder große Unterschiede in den einzelnen Evangelien ergeben. Während die Nennung derjenigen, die der Kreuzigung beiwohnten, bei Matthäus und Markus sehr ausführlich ist und diese beim Namen genannt werden (Maria, die Mutter Jesu,  Maria aus Magdala, etc.), spart Lukas diese Informationen aus und bezeichnet sie lediglich allgemein als die „Bekannten“ Jesu.  Die detaillierte Nennung der Namen derer, die dem Geschehen beiwohnten, mag ein Hinweis darauf sein, dass die Evangelisten davon ausgehen, dass ihren Adressaten die genannten Personen noch persönlich bzw. vom Hörensagen bekannt waren.
