\section{Ergebnisse und Ausblick}

Die inhaltliche Analyse und die Interpretation haben gezeigt, dass die Kinder des 4. Schuljahres der Grundschule Koblenz-Kesselheim über eine differenzierte Glücksvorstellung verfügen. 
Es konnte herausgearbeitet werden, dass die Schülerinnen und Schüler vor allem folgende Aspekte des Glücks als wichtig für ihr Verständnis ansehen:

Wiederholt findet sich im Gespräch mit den Kindern vor allem die Perspektive, in der das Glück auf der Beziehungsebene zwischen Menschen betrachtet wird. 
Die Kinder sehen das Glück in geschlossenen Freundschaften, in der Beziehung zur Familie oder aber auch in gemeinsamen Aktivitäten. 
Offensichtlich ist den Kindern besonders wichtig, zum Ausdruck zu bringen, dass sie ein besonderes Bedürfnis nach sozialer Sicherheit, Aufmerksamkeit und Wertschätzung haben. 
Dieses Merkmal von Glück, dass die Kinder aus ihrer Sicht entfalten, findet sich auch in der pädagogischen Literatur wieder. 
Münch und Wyrobnik beziehen sich auf eine ähnliche Erhebung, bei der eine Frankfurter Grundschulklasse mit 22 Kindern -- 11 Jungen und 11 Mädchen -- befragt wurde, was sie unter Glück verstehen. 
Auch sie beziehen sich auf den Beziehungsaspekt des gemeinsamen Spielens und Zusammenseins mit Freunden\cite[S.\,62]{JM11}.

An dieser Stelle zeigt sich auch, wie wichtig für Grundschulkinder der Kontakt zu Gleichaltrigen und der gegenseitige Austausch ist. 
Auch der gemeinsam verbrachten Zeit und den dabei gesammelten Erfahrungen kommt daher eine besondere Bedeutung zu. 
Dabei deuten die Schülerinnen und Schüler ebenfalls an, dass für sie Glück ohne das Zutun anderer Menschen nur schwer vorstellbar ist. 
Zwar sei der Mensch in begrenztem Maße und in einem gewissen zeitlichen Rahmen in der Lage, auch allein glücklich zu sein. 
Dieses Glück wird jedoch von den Kindern nicht in der selben Qualität begriffen, wie das durch Mitmenschen geschaffene Glück. 
Die individuellen Bedürfnisse des Menschen nach Heimat, Versorgung mit Lebensmitteln und mit der Geborgenheit der Familie wird von den Kindern als elementar angesehen. 
Daher ziehen sie daraus auch die Konsequenz, dass Glück für arme Menschen oder solche, die ohne familiäre Bindungen leben müssen, nur schwer zu denken ist. 
Auf diese Weise drücken die Kinder aus, wie wichtig ihnen vor allem ein sorgenfreies Leben, die Versorgung durch die Familie und Freundschaften sind. 

Zudem haben die Schülerinnen und Schüler vereinzelt auch das Bedürfnis nach Anerkennung ihrer Leistungen. 
Durch Aysches Bekenntnis am Anfang des Gespräches, dass sie Glück in guten Noten erkennt,  lässt sich der Schluss ziehen, dass die Kinder auch das Leistungsbewusstsein in ihr Glücksverständnis übernommen haben. 
Evident erscheint daher, dass die Kinder im Grundschulalter bereits unter einem signifikanten Leistungsdruck im Vorfeld des Übergangs zur weiterführenden Schule zu stehen scheinen. 

Ein weiterer wichtiger Punkt, den die Schülerinnen und Schüler erwähnen, ist das Motiv des Zufalls. 
Kinder verbinden mit dem Zufall nach Münch vor allem Glück, \glqq das ihnen ohne eigenes Zutun zufällig in den Schoß fällt.\grqq{} \cite[S.\,62]{JM11}
Im Gegensatz zu Münchs Ergebnissen nennen die Kinder in Kesselheim dabei jedoch nicht den Fund von Geld oder anderen Gegenständen, sondern beschreiben das Glück vor allem in Form der Ereigniskette, die dann zu einem -- für die Kinder -- positiven Abschluss kommt. 
Dadurch drücken sie Wünsche und Hoffnungen aus, die sie mit dem Glück verbinden, wenn diese in Erfüllung gehen. 

Den Zufall verorten die Kinder allerdings auch in vermiedenen, negativen Konsequenzen von Ereignisketten. 
Konkret benannten die Kinder eine mögliche Verletzung, die sich trotz eines Sturzes von einem Baum nicht einstellte. 
Daran lässt sich erkennen, dass auch Grundschulkinder bereits Erfahrungen mit dem Glück im Unglück gemacht haben und dieses ebenfalls Eingang in ihr Glücksverständnis gefunden hat.
Auch das Gefühl, dass die Schülerinnen und Schüler in Glückssituationen erleben, spielt eine wichtige Rolle. 
Die emotionale Komponente des Glücks wird von den Kindern vor allem mit positiver Mimik wie dem Lachen und einer Gefühlslage des Herzens verstanden. 
Dazu wird ausdrücklich das Lachen, welches durch Glück ausgelöst werde, von dem ohne solchen Kontext auftretenden differenziert. 
Damit zeigen die Schüler, dass sie in der Lage sind, emotionale Regungen im Kontext einer Handlung zu sehen und zu unterscheiden. 

Innerhalb des Gespräches ergaben sich durch die Analyse und Interpretationen auch Hinweise darauf, dass die Schülerinnen und Schüler in ihren Sichtweisen auf den Glücksbegriff antike und neuzeitliche Positionen aufgriffen. 
So konnte festgestellt werden, dass in den Glücksvorstellungen der Kinder bereits das Bedürfnis nach der Vermeidung von Leid vorhanden ist, dass in der Philosophie vor allem durch Epikur, Aristippos und John Stuart Mill vertreten wurde. 
Hinzu kommt, dass sich in den Beiträgen der Kinder nachweisen lassen konnte, dass es offensichtlich unterschiedliche Arten von Glück gibt und dass das Handeln der Schülerinnen und Schüler in unterschiedlichem Maße zu diesem Glück beitragen kann. 
Daher lässt sich festhalten, dass sich die philosophischen Vorstellungen der Kinder zum Teil den utilitaristischen Strömungen zuordnen lassen. 

Darüber hinaus lassen sich jedoch noch weitere philosophische Überschneidungen zwischen den Gedanken der Kinder und den philosophischen Positionen finden. 
So sind die Kinder zwar teilweise der Meinung, dass das Glück auch von den Mitmenschen mitbestimmt wird und nicht nur an die eigene Person gebunden ist. 
Jedoch kann man sagen, dass die Kinder nicht allgemein davon ausgehen, dass ein Mensch, wie Aristoteles es sagt, per se ein unglückliches Leben führen muss, weil ihm bestimmte Güter fehlen. 
Dies machen die Kinder daran deutlich, dass sie nicht kategorisch ausschließen, dass auch ärmere Mensch glücklich werden können.
Gleichwohl lässt sich festhalten, dass das Glück des Menschen als oberstes Lebensziel auch die Ansichten der Schülerinnen und Schüler mitbestimmt. 

Das philosophische Gespräch mit der Klasse hat deutlich gemacht, dass die Kinder eine Vorstellung davon haben, wie sich Glück anfühlen kann. 
Diese Verknüpfung von Emotion und Glück steht im eindeutigen Widerspruch zur Stoa und auch zu den Lehren Immanuel Kants, welche das Glück im Kontext einer gottgegebenen Ordnung betrachten.
Damit bringen die Schülerinnen und Schüler innerhalb des Gespräches  eine klare Gegenposition gegenüber einer philosophischen Lehrmeinung  zum Ausdruck. 
Vor allem die Position Kants, dass Glück lediglich durch ein rational erklärbares Handlungsmuster und nicht durch Gefühle gesteuert sei, ist eine Sicht der Dinge, die die Schülerinnen und Schüler nicht teilen. 
Dabei ließ sich aber auch ausmachen, dass die Kinder dieses Glücksgefühl augenscheinlich nicht genauer beschreiben konnten.
So gelang es lediglich, das Gefühl, das Glück auslöst, in einem gewissen Rahmen eingrenzen zu können.
Dies zeigt an dieser Stelle aber auch, dass sich die Kinder mit der Beschreibung der Gefühle, die sie mit Glück verbinden, schwer taten.

Weiterhin ist auch auffällig, dass sich die Positionen der Kirchenvertreter Augustinus und Luther sowie von Nietzsche in den Aussagen der Grundschüler nicht in der Häufigkeit erkennen ließen, wie etwa die Theorien der Antike oder der Aufklärung. 
Dies erscheint plausibel aufgrund der Tatsache, dass diese Vertreter ihr Glücksverständnis sehr spezifisch entwickelten und wenig Spielraum für Interpretation und Einordnung lassen. 

Denn sowohl Augustinus als auch Luther beziehen ihr Glücksverständnis auf die Rolle des allmächtigen Gottes, der durch seine Unabhängigkeit Ausgangspunkt für das Glück sein müsse. 
Darin spiegelt sich ein Weltbild wieder, dass sich so bei Kindern höchstwahrscheinlich nicht wiederfinden wird. 
Zudem fällt bezüglich Nietzsches Glücksbild auf, dass auch er, der sein Bild der drei Säulen vor allem aus psychologischer Perspektive  aufbaute, in der Betrachtung der Parallelen und Unterschiede zwischen den Positionen der Kinder und der Philosophie kaum eine Rolle spielte. 
Auch darin findet sich kein Denken wieder, das mit einer kindlichen Sicht auf das Glück zu vereinen zu sein scheint. 
Diesen drei Vertretern gemeinsam ist folglich auch, dass sie -- im Gegensatz zu den vorhin erläuterten Philosophen -- ein sehr eng gefasstes Glücksverständnis vertreten und daher eine Bezugnahme auf das Denken von Kindern kaum möglich ist. 

Es lässt sich resümieren, dass die zu Beginn gestellte Forschungsfrage, welche Vorstellungen Grundschulkinder von Glück haben, exemplarisch anhand der Untersuchung eines vierten Schuljahres beantwortet werden konnte. 
Darüber hinaus konnten auch die Vorstellungen dieser Schülerinnen und Schüler selbst herausgearbeitet werden. 

Neben den philosophischen Inhalten lassen sich im analysierten Gespräch die zentralen Gedanken des Rahmenplans Sachunterricht für die Grundschulen in Rheinland-Pfalz erkennen.
So bekamen die Schülerinnen und Schüler im Gespräch über das Glück die Möglichkeit, sich mit einem Thema zu befassen, von dem jeder Mensch ein Bild hat.
So waren die Kinder selbstbewusst, ihre Ideen und Gedankengänge offen zu diskutieren.
Dies lässt sich unter der personalen Kompetenz fassen.
Des Weiteren konnten sie an den Ansichten ihrer Mitschüler partizipieren (sozial), schulten die Kompetenz, eigene Erfahrungen kundzutun (methodisch) und setzten sich fachlich mit einem für sie greifbaren philosophischen Problem auseinander (fachlich).
Hinzu kommt, dass die gewählte Methodik des philosophischen Gespräches eindeutig in der Lage war, dem Forschungs- und Wissensdrang der Kinder in der Grundschule gerecht zu werden.
Andersherum ausgedrückt wäre kein solches Gespräch zustande gekommen, wenn die Schülerinnen und Schüler kein Interesse an einer Diskussion über Glück gehabt hätten.

Auch die perspektivenübergreifenden Denk-, Arbeits- und Handlungsweisen, wie sie der Perspektivrahmen Sachunterricht beschreibt, finden sich im untersuchten Gespräch wieder.
So lässt sich festhalten, dass die Schülerinnen und Schüler im Gespräch Eindrücke anderer Wertvorstellungen und Wissen der Mitschüler im Kontext des Erkennen/Verstehen sammeln konnten.
Zusätzlich mussten die Kinder ihre Positionen in der Diskussion argumentativ unterstützen und gegen Kritik verteidigen.
Eine weitere Denkweise des Perspektivrahmens, die durch das Gespräch aufgezeigt werden konnte, ist die des Evaluierens und Reflektierens. 
Die Kinder reflektierten ihre Erkenntnisse und bezogen zum Teil auch Stellung zu den Vermutungen anderer, denen sie Alternativen gegenüberstellten.
Zur Reflexionskompetenz passt an dieser Stelle auch die Kompetenz des Kommunizierens.
Die Schülerinnen und Schüler stellten ihre Meinung begründet dar und diskutieren miteinander über das Thema \glqq Glück\grqq{}.
Es zeigt sich, dass sowohl die philosophischen wie auch die pädagogischen Aspekte des Philosophierens anhand des Gesprächs mit den Kindern exemplarisch gezeigt werden konnten.

Die in dieser Masterarbeit behandelte Untersuchung stößt jedoch auch an Grenzen. 
So konnte im Rahmen der Erarbeitung lediglich eine Klasse untersucht werden. 
Um mögliche Unterschiede zwischen den Klassenstufen in den Blick zu nehmen, würde es sich daher auch anbieten, verschiedene Klassenstufen einer Schule zu einer Thematik zu befragen und diese Ergebnisse in Beziehung zueinander zu setzen.
Für den wissenschaftlichen Diskurs zu Glücksvorstellungen von Grundschulkindern können neben der an dieser Stelle gewählten Methodik auch andere Verfahren zum Einsatz kommen. 
So könnten philosophische Gespräche über Glück auch aus partizipatorischer Sicht betrachtet werden. 
Dabei würde  eine Partizipationsanalyse zum Einsatz kommen, die untersucht, inwiefern die Schülerinnen und Schüler sich aufeinander beziehen und wie sich ihr Glücksverständnis im Laufe des Gespräches weiterentwickelt. 
Diese Fragen konnten in dieser wissenschaftlichen Auseinandersetzung nicht behandelt werden, da der Fokus dieser Arbeit auf der inhaltlichen Komponente und nicht etwa auf der partizipatorischen des Gespräches lag. 

Zudem könnte auch interessant sein, welche Faktoren für die Entwicklung der Glücksvorstellungen der Schülerinnen und Schüler von Bedeutung sind. 
Entscheidende Fragen in diesem Kontext könnten sein, ob das soziale Umfeld Einfluss auf die Sichtweisen der Kinder nimmt. 
Dazu könnten Untersuchungen aus Schulen, deren Schüler eher aus sozial schwächerem Milieu kommen, mit denen von Schulen aus bürgerlichem und aus wohlhabenderen Milieus verglichen werden, um zu ergründen, ob und inwiefern der soziale Hintergrund der Kinder Einfluss auf ihr philosophisches Denken hat.
Zusätzlich dazu ergibt sich auch die Frage, ob man an Großstadtschulen andere Ergebnisse erzielen wird, als in ländlich geprägteren Gegenden.

Insgesamt kann man sagen, dass im Rahmen dieser Arbeit gezeigt werden konnte, dass nicht nur Erwachsene sondern auch Grundschüler bereits über ein differenziertes Glücksverständnis verfügen.
Die Untersuchung stützt auch die Definition des Begriffs \glqq Glück\grqq{}, die zu Beginn dargelegt wurde, dass es sich beim Glücksbegriff nicht um einen definierbaren und klar zu fassenden Untersuchungsgegestand handelt.
Vielmehr unterscheidet sich die Einordnung dessen, was ein Mensch als Glück sehen wird, von Mensch zu Mensch.
Und so man kann sagen, dass man nicht nur von zehn Erwachsenen zehn verschiedene Antworten auf die Frage, was Glück ist, bekommen wird, sondern auch von zehn Grundschulkindern.