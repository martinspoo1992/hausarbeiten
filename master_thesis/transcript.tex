\section{Transkript}

\begin{lstlisting}[language={}]
S1:   was is denn für euch glück; (7.0)
S3:   wenn ich traurig bin un wenn man mich tröstet dann  
        s das Glück (unverständlich, 7 sek)
S2:   nimm ruhig das nächste kind dran; (5.0)   
S3:   für mich is jetzt (unverständlich) glück, weil wir losen
        immer und dann wurd ich gezogen also das war für mich glück. 
S1:   mhh ok, (6.0)
M1:   also is mir noch nich passiert aber wenn jetz zum beispiel 
        jemand irgendwo runterfällt und äh un irgendwo festhalten
         kann und dann nich ganz nach unten fällt das is auch glück
S1:   mhh (4.0)
D2:   (unverständlich, 4 sek) aus drei meter höhe oder zwei meter 
        höhe runterfällst ja zwei meter höhe un dann äh hinfällt
        auf sein arm vielleicht hat der dann noch glück un sich nicht
        das er nichts gebrochen hat.
S1:   genau; glück im unglück (5.0)
A1:   wenn ich gute noten habe
S1:   wenn du gute noten hast (erfreut) ja das kann auch glück sein 
       (6.0)
S3:   ähm wenn man für jemanden sorgt helfen wenn er grade 
        schwierigkeiten hat
S1:   mhh, (4.0)
S3:   es gibt ja unterschiede einmal pech un glück manchmal hat man 
        pech weil wenn man jetz hingeht un ähm (3.0) ahh ich weiß jetz 
        nich wenn man jetz hingeht un irgendwas macht un dann auf 
        einmal aus versehen wegwirft obwohl das für jemanden is hat 
        man auch pech.
S1:   mhh,
S3:   weil ähm danach kann man auch also das is halt eben anderster 
        als (unverständlich, 3 sek) weil is glück is was halt wenn was 
        richtig passiert weißte? (unverständlich, 5 sek) pech is jetz 
        was wenn was schiefläuft. (5.0)
M1:   also ich wollte noch sagen das mit dem pech gibts ja auch jetz 
        zum beispiel jetzt in der klasse wenn ma jetz irgend irgendwas 
        überhaupt gar nich weiß oder nich kapiert dass der lehrer 
        einen dann genau dann auch annimmt also sagt sag mir mal was 
        das ergibt das is dann auch irgenwie doof. (6.0)
S1:   nimmste das nächste kind dran,
D2:   und ähm pech is zum beispiel auch wenn ich groß äh wenn ich   
        etwas sehr sehr wichtiges hab zum beispiel sehr sehr viel geld 
        gekostet hat un ich dann einfach verliere (4.0) ja? 
S3:   oder wenn man zu einem flugzeug geht un dann äh kommt man zu 
        spät das geht auch. dann hat man wirklich pech (4.0)
S2:   okay, dann stelle ich euch jetzt mal noch eine
        frage. ähm (.) was braucht man denn, um glücklich 
        zu sein? oder zU:m glück (.), was denkt 
        ihr?(10.0)
S3:   äh, ich zum beispiel bin jetzt glücklich, dass 
        ich meinen hund hab und meine familie. (-) weil 
        (.)((lacht)) un meine freunde/ weil das ist ja 
        so, wenn man jetzt was (-), wenn man jetzt eine 
        mutter hat und dann, ähm, nicht sie gut versteht 
        mit der, ist das ja auch blöd (.), weil dann hat
        man fast nie was. weil manchmal geht man dann 
        hoch ins zimmer und dann will man nicht mitessen, 
        ver, verhungert man dann fast. ((lacht))
M1:   ähm (.), U:nd (.) man braucht ein dAch überm 
        kO:pf, e:ssen und eine familie. (damit/dass) man 
        sich wohl fühlt, gebO:rgen. (.) weil man fühlt 
        sich ja nicht gU:t, wenn man kein essen hat und 
        ((unverständlich, ca. 1. Sekunde)) wenn man für
        alles betteln muss.
D2:   und, ähm, (.) wenn man überhaupt, fam, ähm,
        familie hat und freunde. weil (.) es gibt auch 
        leute, die, ähm, keine freunde haben und keine 
        familie. (-) zum beispiel kinder im kinderheim.
        die haben auch keine <<leise> familie> ich gebe
        das wort weiter an (--) Stefanie.
S3:   A:lso, jetzt (.) wenn man hingeht und dann sagt,
        äh, ich mag meine mutter nicht oder so und über 
        die lästert, oder über die freundin, obwohl man 
        mit der spIElt, dann ist man komisch und das die 
        andere/ also dass, dass niemand von der erfährt. 
        weil das macht man irgendwie nicht. das ist dann
        irgendwie (.) auch pech. für die freundin, die 
        dann über die lästert. dann hat die nämlich
        keinen mehr.
S2:   okay, das wäre für dich dann der gegensatz, das 
        wäre dann pech?
S3:   hm=hm. (3.0) hm (-) darius.
D2:   wie jetzt/ zum thema freunde, wenn man zum 
        beispiel was wichtiges hat und äh/ und das dann 
        einem sagt. und der das dann sofort der ganzen 
        schule verpetzt, ähm (.) dann (.) und bricht man
        ja auch den kontakt ab und dann/ jaja, ich mach
        das nie wieder und dann machen die das zehn 
        minuten später wieder. ((mehrere SuS lachen))
        kann doch auch sein, dann verrät man ein 
        geheimnis und dann sagen die das wieder.
M1:   ja aber auch wenn einen frEU:nd trifft/ wenn man
        (.) jetzt/ ich hab mal/ mit meinem bruder waren 
        wir mal auf so nem advents/ auf einem wochenE:nde
        ((unverständlich ca. 1 Sekunde)) und da hatten
        wir sO ein oder zwei freunde gefunden, aber wir 
        wussten nicht, ob wir die jemals wiedersehen, 
        weil (.) wir hatten die zum ersten mal da 
        gesehen. (2.0) Stefanie.
S3:   das ist jetzt fast genauso wie bei dir. wir waren 
        auf einer feier (.), also einer 
        überraschungsfeier und da war ein kind, dass ich 
        dann kennengelernt hab und so, aber da wusste ich 
        nicht, ob wir auch uns/ wiedersehen. 
S2:   hm=hm, weil ihr jetzt so viel über freunde 
        sprecht- (.) denkt ihr denn, dass man alleine
        glücklich sein kann? oder braucht man dazu 
        andere, um glücklich zu sein? (8.0)
M1:   ((flüstert) man braucht andere (unverständlich, ca. 15 
        sekunden)
S2:   nimm das nächste kind dran; (6.0)
S3:   ich finde man kann das auch alleine machen, (---) weil wenn 
        man jetz hingeht un irgendwas schaffen will wie so ein 
        beispiel wie bei sport (unverständlich, 2.0) schwer irgendwas 
        und dann doch schafft dann macht macht das nämlich auch 
        glücklich (.) also man braucht jetz nich unbedingt auch andere 
        (4.0)
M1:   ich denke man kann auch alleine glücklich sein              
        aber bestimmt wenn man noch einen zum reden hat is
        man noch glücklicher, (6.0)
D1:   man braucht freunde weil (unverständlich, 2.0) hat man      
        ja keine freunde un dann steht man da alleine in der
        ecke. (3.0)
D2:   man braucht eigentlich braucht man die dann unbedingt 
        braucht man die jetz auch nicht weil zum beispiel
        wenn jetz irgendeiner alleine zuhause is son paar stunden dann 
        braucht der auch keine freunde zuhause da kann man auch 
        alleine glücklich sein(.) wenn man ein paar stunden alleine 
        zuhause is (.) man braucht nich unbedingt sehr viele es gibt 
        ja auch freunde die hält man als gut un dann ähm danach ähm wo 
        ähm ähm nutzen die dich dann nur aus,
S2:   mhh,
D2:   dann machen die alles dann irgendwie denken die nur an sich 
        ähm sagten dir dass sie vor dir stehen und dann lachen die 
        dich einige von denen aus (5.0)
S3:   es gibt ja auch manche kinder die (--) die dann so auf sich 
        bezogen sind also die dann fast gar keinen mögen un dann 
        wundern die sich dann manchmal wenn die keine freunde haben so  
        in der art; (2.0)ähh es gibt zum beispiel jetz so ein mädchen 
        das nur an sich denkt un dann immer petzt oder so un dann auf 
        einmal hingeht äh un dann fragt die dich wollen wir spielen 
        obwohl die eigentlich gesagt hätte dass man gelästert hätte 
        oder so dann hätte man über die dann hätte man eigentlich nein 
        sagen müssen? (.) wir hatten letztens auch mit euch die  
        geschichte mit diesem (--)irgendwie anders un da war ja auch 
        so freunde die die ham den dann nicht auf den einen reagiert  
        (2.0) weil das is dann voll komisch für den anderen der das 
        dann macht, (4.0)
M1:   ja weil bei irgendwie anders man kann finde ich ganz alleine 
        alleine au nich glücklich sein du hast ja gesagt dass
        beim sportunterricht da warst du aber nicht
        du hattest deine freunde dabei gehabt die dir vielleicht mut 
        gemacht haben oder der lehrer dir das beigebracht hat also 
        hast du das au nicht ganz alleine gemacht (3.0) und man is ja 
        au nich fröhlicher wenn man ganz alleine is weil wenn du jetz 
        jetz zum beispiel über nen bock springen musst un keiner is 
        dabei un dus geschafft hast (unverständlich, 2 sekunden) weil 
        du kannst deine du kannst das du kannst wie du fröhlich bist 
        deine fröhlichheit mit gar nich mit andern teilen oder wenn 
        man jetzt irgendwas cooles im fernsehen gesehen hat un niemand 
        is da dem du das erzählen kannst mir passiert das manchmal 
        ich seh was cooles un dann sag ich das genau guck mal  
        (unverständlich, ca.5 sekunden)
S2:   okay wenn wir jetz so viel über glück gesprochen haben äh wie  
        fühlt sich glück denn für euch an woher weißt woher wisst ihr 
        denn dass ihr glücklich seid, (12.0)
D2:   das fühlt man schon das fühlt man im herzen wenn man sehr viel 
        lacht un dein herz äh wenn man herzhaft lacht wenn man nicht 
        glücklich ist dann lacht man nur mhh wenn man halt glücklich 
        ist un freut dann macht man das halt glücklicher man fühlt das 
        halt im herzen das kommt dann schon mal raus, (2.0)
M1:   man is aber auch glücklich wenn man sich jetz zum beispiel zum 
        GAnz lange ein kuscheltier wünscht und das dann auch bekommt 
        also jetz nikolaus mal als beispiel; (5.0)
D2:   zum beispiel ICH aber das war jetz eh nich bekommen ich wollte  
        nämlich unbedingt nen hund ((Kichern der Mitschüler)) 
        (unverständlich,ca. 11 sekunden)
S3:   für mich also für mich is des jetz wenn man glück hat wenn man  
        jetz mhh irgendjemand hat eben gesagt ja ähm das das nich so 
        schön wär wenn man jetz arm is oder so für mich is jetz glück 
        wenn man in unter einem festen dach un so lebt weil dann wird 
        man nicht wenns regnet nass oder man kriegt dann auch essen un 
        so (--)das für mich jetz glück halt; (3.0)
S2:   mhh du hast eben gesagt wenn man arm is sind arme menschen 
        dann nie glücklich, (14.0)
D1:   ja äh die sin manchmal glücklich aber manchmal auch nich;  nur 
        (---) weil manche leben ja auch auf der straße: un dann haben 
        die halt manchmal betten oder zwei cent oder so (9.0)
S3:   ich glaube eher so halb wei::l <<lacht>> dann sin sie manchmal 
        hungrig ham aber nichts dann kommt ein mensch ein guter mensch 
        der ihnen dann was gibt weil jetzt wenn da so liegt un ähm was 
        zu essen gibt kann der aber manchmal auch hingehen un nen euro 
        geben un sich dafür was kaufen weil dann kann der das auch 
        essen geht ja auch un wenns dann nu:r zwanzig cent oder so  
        kostet hat er immer noch bisschen geld was der dann zurück  
        kriegt; (5.0)
D2:   es gibt auch ähm leute die sich als obdachloser ähm 
        so verhalten un dann sagen ich bin obdachlos dann
        aber in wirklichkeit macht mans nur für ne geldquelle (--) un 
        es gibt auch leute zum beispiel die haben so ne hilfs die 
        haben dann dann hilft man automatisch man will dem eigentlich 
        gar nicht helfen aber dein herz sagt dir hilf dem hilf dem 
        hilf dem dann hilft man dem obwohl der vielleicht gar keiner 
        obdachloser ist un gar nich arm is (.) arme menschen können 
        auch glücklich sein vielleicht wenn man nur ein wenn man nur 
        paar cent reinwirft dann freuen die sich freuen die 
        obdachlosen sich auch (3.0) zum beispiel hier in kesselheim is 
        auch einer ähm un da ham wir ähm letztes mal ham wir dem was 
        zu essen gegeben da hat der sich gefreut; (---)
S3:   wenn man ein paar cent kriegt und in den einkaufsladen geht 
        aber es gibt da nichts un da an der kasse jemand is der dann 
        wirklich auch zu steht kann man auch hingehen un dafür das 
        bezahlen dann bekommt man das trotzdem was man wollte aber 
        nicht wenn man jetzt (--) teure sachen kauft wie zum beispiel 
        was für hundertfünfzig (6.0)
S1:   ja jetz ham wir ja schon ganz viel über glück gehört 
        (--) un äh bestimmt hat jeder von euch einen glücksbringer      
        zuhause  (---) ihr kennt bestimmt alle glücksbringer; 
        glaubt ihr denn an glücksbringer? dass die euch wirklich glück 
        bringen (6.0)
D2:   doch eigentlich schon
S1:   ja?
D2:   weil letztes mal hab ich ähm ganz speziell an den glücksding  
        bringer gedacht dass es an diesem tag NICHt regnen wird un es 
        hat dann wirklich nich geregnet; ((Gelächter der Klasse, ca. 2 
        sekunden) gestern war ein wer wird millionär dran da war echt 
        einer der hat glück der hat da zehn stunden rumgesessen wie 
        ähm bei wieviel steinchen hatn wür zauberwürfel da hat der 
        zehn stunden nachgerechnet da hat der ersmal gesagt hat der 
        die antwort gesagt un das war dann richtig un dann wurd der 
        millionär ((Gelächter der Klasse, ca. 2 sekunden)) so ungefähr 
        halbe stunde ähh de die der daran gesessen hat dann hat nur an  
        den händen geguckt eins zwei drei eins zwei drei 
        (unverständlich, ca 2 sekunden)sechsunzwanzig richtig 
        (Gelächter der Klasse, ca. 2 sekunden)(unverständlich, ca. 6 
        sekunden) weil ichs dir grad gesagt hab; stefanie,
S3:   (---) heute hatt ich ja glück weil gestern hatte ich                
        den glücksBRINGER? einen bekommen vorher hatt ich
        noch nie einen (---) ((seufzt))
D2:   ich hab immer noch keinen
S3:   un dann ähh hab ich mir gewünscht dass ich heute drankomme mit  
        dem ziehen un das is auch passiert (--)(unverständlich, 6 
        sekunden) aysche?
A1:   einmal als wir in urlaub gefahren sind hat meine freundin auch 
        ähm selber einen glücksbringer gebastelt (-) der hat so eine 
        schöne form gehabt so aus filz un dann hat sie selbst en text 
        geschrieben das dann in folie eingepackt un mir gegeben;
S1:   mhh, (5.0) 
A1:   hat mir auch glück gebracht;
S1:   nimmste das nächste kind dran, (3.0)
M1:   ich hab auch mal aus wasserburg hat ich mal so nen schneemann 
        gemacht hatte dann ne schneeschaufel draufgemacht un dann war 
        der mir nicht weiß genug weil der war für meine eltern für 
        nikolaus hat ich wollt ich son stern und kerzen basteln un 
        dann als die adventsfeier am sonntag vorbei war hatten wir hab 
        ich so alle meine geschenke verteilt un die hab ich so 
        eingepackt (unverständlich, ca. 2 sekunden) vier mal ich so hä  
        wann hab ich denn das dahin getan un dann war da der 
        schneemann drin meine mama hat da noch so ein herz drauf 
        gemalt hier so nen schwarzen hut auf und (unverständlich, ca. 
        1 sekunde) und ne nase un seitdem hab ich nen glücksbringer 
        (.) einmal hat der mir so sogar geholf geholfen bei ner note, 
        (.) ja; (8.0)
D2:   menschen können auch glücksbringer sein (.) der mario 
        war mal glücksbringer,
M1:   ja, bei der fussball wm
D2:   bei welcher bei der wm da hat da hat er ers nich mitgeguckt da 
        war kein tor wo der mitgeguckt hat isn tor passiert; 
        (Gelächter der Klasse, ca. 2 sekunden) (4.0) aber ähm was für 
        ich hab also zum beispiel glücksbringer hab ich ähm eine münze  
        aus ähm kroatien (5.0) stefanie,
S3:   da wir jetz grad über glücksbringer reden (.) celine eine  
        freundin hat mir einen traumfänger gegeben wenn ich jetz so    
        schlimme träume hatte die ganze zeit hab ich den geholt un hab 
        auf einmal gut geträumt (4.0) die machen irgend son zeug rein 
        das weiß ich (3.0) ((Gelächter der Klasse))
S1:   genau ihr habt jetzt schon gesehen glücksbringer können (--) 
       kann so ziemlich alles sein alles was woran ihr glaubt wenn 
       ihr was habt was euch glück bringt dann kann das euer 
       glücksbringer sein (-- )-) das heißt man kann quasi nich sagen 
       (-) n kleeblatt bringt jedem glück wenn man nicht dran glaubt 
        oder? ja?
S3:   ja manchmal da bringt man sich auch selber glück; (---)
S1:   mhh wie meinst du das genau? (3.0)
S3:   ich meinte dass wenn man zum beispiel (--) fahrradtour oder 
        irgendwas machen will un der vater sagt nein wir machen das    
        heut nicht weil weil es schlechtes wet äh wetter is (2.0) ja 
        (10.0)
S1:   mhh,
M1:   versteh ich jetz nich;
S3:   wenn man jetz den vater dazu überredet;
D2:   versteh nur bahnhof; (6.0)
S1:   ja ich glaub das musst du nochmal erklären; (2.0)
S3:   also wenn man eine fahrradtour macht
S1:   mhh,
S3:   machen will un dann ähm un dann bringt man den vater        
        dazu fahrrad zu fahrn
S1:   AHH das heißt für dich is das dann glück wenn du ne 
        fahrradtour trotzdem machen kannst ah ok; (4.0)
D2:   das war letztes mal bei uns so als ich dann schon wieder an 
        meinen glücksbringer geda glücksbringer gedacht un da ham wir 
        also das das wir auf jeden fall dieses spiel noch gewinnen un 
        da ham wir das spiel echt fünf null gewonnen ((Gelächter der 
        Klasse) (11.0)
S3:   ähh wie die aysche vorhin gesagt hat ähm weil jetz hat hab ich  
        ja auch en glücksbringer schon gestern auf da wollt ich 
        unbedingt n kinofilm gucken un da mein papa hat mir das zwar 
        versprochen aber dann hatten wir keine zeit mehr weil ich muss 
        ja immer um halb acht schlafen gehen (--- -) und ähm dann hat 
        hab ich bin ich heute früh aufgewacht mit meim papa der hat 
        mich auch hingefahrn also hierhin da hab ich ihn gefragt ob 
        wir das machen können heute da hat er einfach ja gesagt (.) 
        das war für mich glück; (3.0)
M1:   einmal als ich beim BA:sketballtraining war ham wir dann so 
        gespielt (unverständlich, ca. 1,5 sekunden) un da hab ich 
        gehofft dass ich nich so schnell ähh auskomme un dann bin ich 
        doch noch bis ins viertelfinale gekommen (15.0)
S2:   ok dann mal wieder vielen dank dass ihr so gut mitgemacht habt 
        wie beim letzten mal habt ihr echt gut teilgenommen und (.) 
        dann kommen(.) noch einmal
S1:   noch einmal kommen wir 
S2:   einmal kommen wer noch mit nem andern thema also vielen dank
\end{lstlisting}