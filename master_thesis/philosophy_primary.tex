\section{Philosophieren mit Kindern}

Das Philosophieren ist mittlerweile vor allem an den weiterführenden Schulen im deutschen Bildungssystem angekommen.
 In vielen Fällen wird es sogar als eigenständiges Unterrichtsfach angeboten. 
 Doch auch in der Grundschule gewinnt das Philosophieren mit Kindern stetig an Bedeutung, weshalb es von zentraler Bedeutung ist, zu klären, inwiefern es auch im Sachunterricht der 
 Grundschule Sinn macht, den Schülerinnen und Schülern ein solches Unterrichtsangebot zu machen. 
 
Zunächst soll der Fokus jedoch auf der Betrachtung des Philosophiebegriffs liegen und darauf, was man letztlich tut, wenn man philosophiert.



\subsection{Was versteht man unter Philosophie und Philosophieren?}

Der Begriff \glqq Philosophie\grqq{} setzt sich aus den beiden Ausdrücken \glqq philos\grqq{}, welcher Liebe bedeutet und \glqq sophia\grqq{}, was mit Weisheit übersetzt werden kann, zusammen\cite[S.\,8]{BB10}. 
An anderer Stelle, wie in  \cite{GT16}, wird der Begriff \glqq philos\grqq{} hingegen als \glqq Freund\grqq{} übersetzt. 
Von der Wortherkunft lässt sich daher sagen, dass sich die Philosophie als \glqq Liebe zur Weisheit\grqq{} oder \glqq Freundschaft zur Weisheit\grqq{} übersetzen lässt. 

Aufgabe der Philosophen als Philosophie betreibende Menschen und der Philosophie allgemein ist es, \glqq über wichtige Lebensfragen\grqq{} des Menschen und seiner Umwelt nachzudenken\cite[S.\,8]{BB10}. 

Brüning beschreibt, dass sich zwei zentrale Formen der Philosophie entwickelt haben. 
Die esoterische Philosophie beruht darauf, dass seit den Anfängen der griechischen Philosophietradition ein weitreichendes Wissen über die Welt und ihr Dasein gesammelt und zusammengetragen wurde. 
Dieses Wissen wurde dann an philosophischen Hochschulen und Universitäten gelehrt und in eine akademische Disziplin überführt, welche sich folglich Philosophie nennt. 
Darüber hinaus wird in der esoterischen Philosophie angenommen, dass lediglich diejenigen Menschen auch Philosophen sein können, die sich beruflich und mit der Absicht der Forschung mit den Fragen über die Welt beschäftigen.

Demgegenüber steht die exoterische Philosophie, welche ein entgegengesetztes Bild der Philosophie als Disziplin als gegeben ansieht. 
Demnach sind im Wesentlichen alle Menschen in der Lage, sich über die zentralen philosophischen Themen Gedanken zu machen. 
Dabei ist es unerheblich, ob dies aus persönlicher Neugierde oder aus dem Bedürfnis nach Forschungszuwachs geschieht. 
Ferner ist für die exoterische Philosophie auch nicht von Bedeutung, ob das Nachdenken über Philosophisches in einer regelmäßigen Auseinandersetzung stattfindet oder ob der Anspruch des Philosophen erhoben wird, haltbare Theorien zu entwickeln\cite[S.\,8]{BB10}.

Dr. Holm Bräuer, der an der TU Dresden im Institut für Philosophie tätig ist, bezeichnet die Philosophie als wissenschaftliche Disziplin, welche im Gegensatz zu anderen Einzelwissenschaften vor allem gemeingültige Fragen behandelt.
 Dabei sieht auch er vor allem die Frage nach dem Wesen der Welt, aber auch nach dem Sinn des Lebens oder den Ursachen für \glqq Vergangenes und Zukünftiges von religiösen oder mythischen Lehren und Legenden oder von Weltanschauungen\grqq{}\cite{PL16} als zentrale Fragen der Philosophie an.

 Ziel der Philosophie ist es dabei aus seiner Sicht, auf diese allgemein gefassten Fragen der Menschheit Erklärungen zu finden, welche auf der Basis von rationalen Zusammenhängen zustande kommen.
  Bräuer bezieht sich insbesondere auf die Grundfragen Immanuel Kants, der die philosophischen Probleme mit vier Fragen erstmals resümierte: \glqq Was ist der Mensch?\grqq{}, \glqq Was kann ich wissen?\grqq{}, \glqq Was soll ich tun?\grqq{}, \glqq Was darf ich hoffen?\grqq{}.  
  
Im Laufe der Geschichte hat sich die Philosophie durch die verschiedenartigen Fragen, mit denen sie sich beschäftigt, in Teildisziplinen untergliedert, die sich unter den Bereichen der Metaphysik, der Physik und der Ethik fassen lassen. 
Über diesen drei Teildisziplinen stehen einerseits die theoretische Philosophie, die sich vor allem auf den Erkenntnisgewinn und das Verstehen der Welt konzentriert, andererseits die praktische Philosophie, die die Aufmerksamkeit vor allem auf das Handeln des Menschen aus philosophischer Sichtweise richtet. 
Die Teilbereiche der Metaphysik und der Physik lassen sich der theoretischen Philosophie, die Ethik und dabei vor allem die Staatsphilosophie und die Rechtsphilosophie der praktischen Philosophie zuordnen. 

Die Tätigkeit des Philosophierens sieht Barbara Brüning vor allem durch einen Aufbau gekennzeichnet, der sich anhand von zentralen Merkmalen manifestiert. 
Am Beginn eines jeden Philosophierens sieht sie zunächst das Staunen, das dazu führt, das etwas zuvor als trivial Angesehenes von nun an kritisch hinterfragt wird und nach Begründungen für Existierendes gesucht wird. 
Aus diesem Staunen entspringt in einem weiteren Schritt das Fragen. 
Der Mensch möchte erfahren, warum sich Dinge verhalten, wie sie es tun und gehen damit \glqq erste Schritte auf dem Weg zu neuen Erkenntnissen\grqq{}\cite[S.\,10]{BB10}. 
Interessant ist an dieser Stelle auch die Erkenntnis John Lockes, auf den sich Brüning bezieht, der feststellte, dass bereits Kinder Fragen nach dem Wesen der Welt stellen würden. 
Daher empfiehlt Locke den Eltern grundsätzlich, den Drang der Kinder nach Wissen zu erhalten und durch Antworten, welche neue Fragen hervorrufen, weiterzuführen.

Die Fragen die sich der Mensch stellt, möchte er in der Folge selbstverständlich auch beantwortet wissen. 
Daher liegt der nachfolgende Schritt im Nachdenken, welches sowohl im Nachdenken des Einzelnen, aber auch im Diskurs mit Anderen vollzogen werden kann. 

Das Nachdenken vollzieht sich nach Brüning in einem Dreischritt. 
Zunächst müssen im philosophischen Diskurs Begriffe ge-- bzw. erklärt werden, um sich dem zentralen Gegenstand der jeweiligen Frage nähern zu können. 
Tauscht man sich beispielshalber über die Frage aus, ob es ein Leben nach dem Tod gibt, so muss zunächst geklärt werden, was das Leben eines Lebewesens ausmacht und was der Tod bedeutet. 
Somit wird zunächst eine Standortbestimmung der Begrifflichkeiten vorgenommen. 
Anschließend werden Gründe dafür gesucht, die eine Position bejahen oder verneinen können und darüber geurteilt, welche Argumente am stichhaltigsten sind, um eine Sicht zu untermauern. 
Im sokratischen Gespräch können solche Begründungen dann kontrovers diskutiert und dabei bestätigt oder widerlegt werden. 
Dabei muss das Gespräch keine feste Lösung ergeben, sondern es können auch sich ausschließende Antworten Seite an Seite stehen bleiben\cite[S.\,11]{BB10}.

Das Philosophieren kann und möchte trotz rational begründeter Annahmen nicht den Anspruch erheben, vollständig beweisbare Antworten zu liefern. 
Das hat zur Folge, dass jede philosophische Antwort Zweifler finden wird, die Gegenargumente heranziehen werden. 
Daher gehört zum Philosophieren auch die Erkenntnis, zu wissen, dass solche Antworten stets nur einen Teil der Realität abbilden und daher nur provisorischen Charakter haben können. 
Dazu gehört auch, dass bei jeder Diskussion auch die gänzlich gegensätzliche Ansicht als denkbar anerkannt werden muss. 
Die Erkenntnis darüber und die Zweifel an der Antwort auf eine Frage bringen es mit sich, dass die Gedanken des Zweifels weitergedacht werden müssen. 
Dadurch wird der Prozess des Überlegens am Leben erhalten und das Weiterdenken, wie Brüning es nennt, kommt zustande. 
Den Abschluss des Weiterdenkens bringt dann die Einsicht, \glqq dass der Prozess des Nachdenkens über ein philosophisches Problem unabgeschlossen bleibt\grqq{}\cite[S.\,13]{BB10} und es eine \glqq richtige\grqq{} Antwort nicht geben kann.

Am Schluss des Philosophierens wird das eigene Denken infrage gestellt, angezweifelt und unter Umständen auch korrigiert. 
Wird die gleiche Thematik erneut diskutiert, ist es demnach möglich, dass eine zuvor als korrekt angesehene Antwort inzwischen überarbeitet oder gänzlich neu gefasst wurde. 
Daher sieht es Barbara Brüning als besonders zentral an, im Philosophieren mit Kindern in der Grundschule die Antworten der Kinder stets dahingehend zu hinterfragen, ob auch die andere Sichtweise für möglich erachtet wird.


\newpage
\subsection{Eigenschaften philosophischer Unterrichtsgespräche}

Allgemein versteht man unter Unterrichtsgesprächen Gespräche, die im schulischen Kontext -- vor allem im Unterricht zwischen Lehrern und Schülern -- geführt werden. 
Im Gegensatz zu Alltagsgesprächen zwischen Fremden oder der Unterhaltung im familiären Rahmen sind diese Gespräche \glqq institutionell geprägt\grqq{}\cite[S.\,28]{HB15}. 
Daher folgen sie auch schulisch geprägten Normen und die verwendete Sprache ist eine auf diesen Kontext zugeschnittene. 
Feilke fand dazu heraus, dass \glqq die Schulsprache keine Bildungssprache sei, sondern ein Instrument der Erziehung zur Bildungssprache; denn ein kompetenter Sprachgebrauch werde orientiert an den Vorstellungen der Schule und Schulfächer aufgebaut.\grqq{}\cite[S.\,117]{HF13}

De Boer verweist auf Lüders, der davon ausgeht, dass sich das Unterrichtsgespräch in ein wiederkehrendes Muster einordnen lässt\cite[S.\,28]{HB15}:
 Zunächst geht von der Lehrperson eine \textit{initiation}, beispielsweise eine Frage, aus, auf die Schüler meistens mit einem \textit{reply} bzw. einer Antwort reagieren. 
 Schließlich wird die gegebene Antwort in der \textit{evaluation} hinterfragt und der Prozess beginnt erneut. 
 Es lässt sich daher sagen, dass in einem klassischen Unterrichtsgespräch immer eine Validierung vorgenommen wird, die sich bei Lüders in der \textit{evaluation} finden lässt. 
 Für das philosophische Unterrichtsgespräch ist jedoch zu sagen, dass es vor allem gilt, \glqq dem gemeinsamen Nachdenken Raum zu geben, unterschiedliche Denkwege auszutauschen, Fragen zu entwickeln und ein offenes Gespräch ohne Validierungszwänge zuzulassen\grqq{}\cite[S.\,159]{HD15}.
 
Das Unterrichtsgespräch befindet sich darüber hinaus in zwei zentralen Spannungsfeldern. 
Zum einen muss die Lehrkraft eine Organisation vornehmen, die für die Schüler einerseits durch den gewählten Zeitpunkt und andererseits durch Transparenz im Umgang Klarheit vermittelt, zum anderen führt eine detaillierte Planung zu repetierenden Abläufen\cite[S.\,31]{HB15}. 
Das zweite Spannungsfeld sieht die Lehrperson vor dem Problem, das Gespräch fachlich fundiert zu führen und darüber hinaus auch flexibel zu gestalten, um auf situationsbedingte Veränderungen reagieren zu können. 
Daher wird von den Lehrerinnen und Lehrern eine zunehmend selbstreflektierte Moderatorentätigkeit verlangt.

Ziel des philosophischen Gesprächs sollte es sein, einen Dialog zwischen den Schülerinnen und Schülern herzustellen. 
Martin Buber entwickelte dazu drei Merkmale, die für ihn ein dialogisches Gespräch ausmachen. 
Zum einen müssten sich die Gesprächspartner einander zuwenden, d.h. die Existenz des Anderen akzeptieren und nicht seine Gedanken und Handlungen widerspruchslos  übernehmen\cite[S.\,106]{MB15}.
Weiter erachtet es Buber für wichtig, dass die Gesprächspartner eigene Gedanken einbringen und diese unverkürzt wie uneingeschränkt zur Sprache bringen. 
Zuletzt sollte in einem Dialog auf die Selbstdarstellung verzichtet werden, da dann nicht die eigentlichen Positionen mitgeteilt werden, sondern die authentische Komponente des Gesprächs verloren geht.

Für philosophische Unterrichtsgespräche ergeben sich in Bezug auf die dialogische Ausgestaltung zusätzliche Merkmale, die nicht vernachlässigt werden dürfen. 
Grundlegend ist zu philosophischen Unterrichtsgesprächen zu sagen, dass es sich bei ihnen um sokratische Gespräche handelt. 
Diese Gesprächsform zeichnet sich vor allem dadurch aus, \glqq dass in ihm ein philosophischer Gegenstand diskutiert wird, bspw. Glück, Gerechtigkeit oder Freundschaft.\grqq{}\cite[S.\,27]{BB10}

Ein sokratisches Gespräch, wie es auch mit Schülern in der Grundschule geführt werden kann, gliedert sich in drei Phasen. 

In der ersten Phase, welche Brüning Vorbereitungsphase nennt, geht es vor allem darum, die zentrale philosophische Fragestellung zu klären. 
Dabei kann die Fragestellung sowohl die eines Kindes als auch die der Lehrperson sein. 
Daneben müssen Regeln für den Ablauf des Gespräches aufgestellt werden, welche im gesamten Gesprächsverlauf eingehalten werden müssen. 
Dazu können das aktive Zuhören, das Begründen der eigenen Meinung gehören aber auch, andere ausreden zu lassen\cite[S.\,31]{BB10}.
In der zweiten Phase kommt es zum eigentlichen philosophischen Gespräch. 
Das zuvor ausgewählte Thema wird mit den Schülerinnen und Schülern diskutiert und die Lehrerin bzw. der Lehrer übernehmen lediglich die Aufgabe des Moderators. 
Brüning weist ausdrücklich darauf hin, dass auch ein Schüler diese Tätigkeit wahrnehmen kann, wenn die Schülerinnen und Schüler bereits mit dem gemeinsamen Philosophieren vertraut sind. 
Während des Gespräches sollen auch unbekannte Begriffe geklärt und über die Thematik argumentativ gestritten werden. 
Gegen Ende des Gespräches soll sich dann ein gemeinsames Ergebnis herauskristallisieren. 
Dieses kann entweder in einer alleinigen Lösung oder aber in verschiedenen bestehen. 

An diese Phase schließt sich in einem dritten Schritt das Metagespräch an, in  dem das gemeinsame Nachdenken unterbrochen wird oder ein Anschluss an die Diskussion gefunden wird. 
Unter anderem beschreibt Brüning, dass in diesem Teil des sokratischen Gespräches die Einhaltung der Regeln erneut diskutiert und auf etwaige Verletzungen hingewiesen werden kann. 
Zudem können die Ergebnisse des Gespräches an dieser Stelle erneut zusammengefasst werden.

Während für ein Unterrichtsgespräch generell vor allem der gedankliche Lernprozess des Kollektivs im Vordergrund steht, ist dies allein für das philosophische Gespräch nicht mehr ausreichend. 
Vielmehr muss den Schülerinnen und Schülern näher gebracht werden, die Sichtweisen und Blickwinkel anderer zu verstehen und akzeptieren zu können\cite[S.\,234]{HDB15}. 
Auch muss für philosophische Unterrichtsgespräche das Rollenbild zwischen Lehrperson und den Schülerinnen und Schülern aufgebrochen werden. 
Bleibt es nämlich, wie bei dieser Gesprächsform üblich, dabei, dass ausschließlich die Lehrerin bzw. der Lehrer die Fragen stellt, werden die Schüler nicht in der Art und Weise für die Thematik zu begeistern sein, wie in dem Falle, in dem sie selbst ihre Fragen in den Raum stellen dürfen. 
Daher würde bei einer klassischen Anordnung im philosophischen Gespräch die Chance, Lernprozesse bei den Schülerinnen und Schülern zu initiieren, verpuffen.

Für de Boer wird das Philosophieren mit Kindern dann zum Unterrichtsprinzip, wenn Lehrerinnen bzw. Lehrer und Erwachsene im Allgemeinen dazu beitragen, dass das kategoriale Denken der Kinder in weitergehende und multidimensionale Denkweisen übertragen werden.


\newpage
\subsection{Konzeptionen des Philosophierens in der Grundschule: Warum sollte in der Grundschule philosophiert werden?}

In den letzten Jahren hat sich der pädagogische Blick auf das Philosophieren an Schulen kontinuierlich gewandelt. 
So wurden teilweise von Philosophen selbst Gesprächsanlässe und Konzeptionen geschaffen, die die Schülerinnen und Schüler als selbständige Menschen sehen und ihnen die Möglichkeit geben wollen, sich die Welt autonom zu erschließen\cite[S.\,617]{AN13}. 
In diesem Kontext wurde auch die Frage diskutiert, ob das Philosophieren mit Kindern als eigenes Unterrichtsfach angeboten oder in bereits vorhandenen Fachunterricht eingebettet werden soll. 

Professor Dr. Andreas Nießeler, der an der Universität Würzburg in den Bereichen Grundschulpädagogik und Theorie des Sachunterrichts, Philosophieren mit Kindern, Kulturanthropologische Theorie der Bildung und des Lernens und Bildungsphilosophie und pädagogische Anthropologie forscht und lehrt, stellt klar, dass das Ziel des Philosophierens mit Kindern aus pädagogischer Sicht vor allem darin besteht, \glqq junge Menschen mit ihren Fragen ernst zu nehmen\grqq{}\cite[S.\,617]{AN13}.
Hinzu kommt, dass die Schülerinnen und Schüler in die Lage versetzt werden sollen, ihre Deutung von der Welt ausdrücken zu können und sich eigenständig in ihrem Denken zu orientieren. 

Nießeler erwähnt dazu Herman Nohl, der in seiner 1922 veröffentlichten Schrift \glqq Philosophie in der Schule\grqq{} feststellt, dass der Philosophieunterricht \glqq anstatt als Einzelfach ein trauriges Dasein zu fristen die Gesamtheit der Fächer zu tragen habe und dessen Aufgabe eine Erziehung zur \glqq Vergeistigung unseres Seins, Denkens und Tuns durch Rückbeziehung alles Einzelnen auf Sinn und Grund, eine platonische Anleitung zur Erhebung in das höhere geistige Leben der Idee\grqq{} sei.\grqq{}\cite[S.\,618]{AN13}
Hier zeigt sich die Verortung der Philosophie in der Schule, die keinen eigenen Platz in der Fächerlandschaft der Grundschule zugewiesen bekommt und deren Ziel es ist, ein autonomes Denken unter den Schülerinnen und Schülern zu etablieren. 
Darüber hinaus spricht Andreas Nießeler die Ideen des Philosophen Leonard Nelson an, der als zentrale Unterrichtsmethode die sokratische Maieutik betrachtete. 

Diese Methode, auch Hebammenprinzip, sah nach Sokrates vor, durch gezielte Fragen das vorhandene Wissen des Befragten zutage zu fördern. 
Mithilfe dieser Maieutik wollte Nelson den Schülern die Unvollständigkeit des Wissens deutlich machen. 
Hauptsächlich ging es ihm darum, dass die Schülerinnen und Schüler nicht lernen sollten, was Philosophie ist, sondern sie sollten das Philosophieren als solches lernen und so selbst zu Philosophen werden\cite[S.\,618]{AN13}. 

Ferner bezieht sich Nießeler darauf, dass die ersten, wissenschaftlichen Untersuchungen vor allem in den USA stattfanden. 
Matthew Lipman, der 1974 das \glqq Institute for the Advancement of Philosophy for Children\grqq gründete, war zusammen mit Gareth B. Matthews einer der ersten, die ein Philosophiekonzept für die Durchführung mit Kindern entwickelten. 
Lipman war der überzeugung, dass Kinder bereits früh in der Lage sind, logisch zu denken und sah das logische Denken daher als eine Basisqualifikation an\cite[S.\,619]{AN13}. 
Matthews hingegen untersuchte vor allem philosophische Gespräche zu verschiedenen Themen wie \glqq Glück\grqq{}, \glqq Wünsche\grqq{} oder \glqq Gerechtigkeit\grqq{}. 
Er konnte durch Erinnerungsprotokolle ermitteln, dass in diesen Gesprächen bei den Kindern eine Entwicklung stattfindet und philosophische Gespräche es erlauben, die philosophischen Vorstellungen von Kindern in Relation zu bedeutenden Fragen untersuchen zu können.

Barbara Weber stellt in ihrem Beitrag \glqq Philosophieren mit Kindern: Wieso, Weshalb, Wozu?\grqq{} fest, dass das Philosophieren mit Kindern einerseits aus pädagogischer andererseits aus philosophischer Sicht in der Kritik stehe. 
Es werde argumentiert, man könne die Schülerinnen und Schüler überfordern, wenn man mit ihnen philosophische Themen erörtert, andererseits ist die Bedeutung der Philosophie in sich eine der größten Fragen, der sich die Philosophie entgegensieht. 
Weber hebt hervor, dass es Skeptiker gebe, die den Kindern nicht die Fähigkeiten zusprechen würden, sich ein umfassendes Wissen über philosophische Themen anzueignen, aber auch Befürworter, die in der Philosophie eine umfassende pädagogische Lösung sehen würden, mit deren Hilfe jegliche sozialen und inhaltlichen Kompetenzen erworben werden könnten\cite[S.\,623]{BW13}.
Jedoch sollten überhöhte Erwartungen an das Philosophieren mit Kindern vermieden werden. 

Entscheidend sei aus ihrer Sicht auch, welche Themen für das Philosophieren gewählt würden und ob diese philosophisch gesichert seien, was Ekkehard Martens 1999 bezweifelt habe\cite[S.\,624]{BW13}. 
Das Philosophieren in Unterrichtsgesprächen mit Kindern kann für Barbara Weber von großem Nutzen für die pädagogische Forschung sein. 
Nutzt man es als qualitative Methode, so könne man aus diesen Gesprächen Rückschlüsse daraus ziehen, wie Schülerinnen und Schüler über philosophische Themen denken und was sie dazu sagen.

Weber erstellt ein Schema, nach dem sich philosophische Gespräche grob gliedern lassen. 
So geht jedem philosophischen Gespräch ein Staunen und Fragen voraus, in dem die Schülerinnen und Schüler sich über eine Fragestellung klar werden und diese hinterfragen. 
Dem schließt sich der gemeinsame Dialog mit Anderen - das Denken-Sprechen - an, der die Positionen anderer in den Blick nimmt und gegebenenfalls auch die eigenen verändern kann. 
In einem weiteren Schritt kommt es zum Werten-Handeln, das zu einer Prüfung oder sogar Veränderung der eigenen Lebenswelt führen kann. 
Eigenes Handeln und individuelle Werte werden geprüft und dies führt zu weiterem Staunen und Fragen\cite[S.\,626]{BW13}.
Barbara Weber betrachtet das gesammelte Wissen der Philosophie nicht als abgeschlossen, sondern verweist auf Pierre Hadot, der in seinem Werk \glqq Wege zur Weisheit\grqq{} darauf hinwies, dass \glqq die Philosophie ein historisches Phänomen ist, das zu einer Zeit begonnen und sich bis heute weiterentwickelt hat.\grqq{}
Diese Sicht auf die Philosophie hat zur Folge, dass diese nicht mehr bloß Modelle behandelt, sondern sich auch mit philosophischen Verhaltens- und Lebensweisen beschäftigen müsse.

Kerstin Michalik verortet das Philosophieren in einem eigenen Unterrichtsprinzip, dass die Schülerinnen und Schüler in ihren persönlichen Lernprozessen unterstützen und ihnen die Möglichkeit bieten soll, ein differenziertes Weltbild zu entwickeln\cite[S.\,635]{KM13}. 
Weiter schließt Michalik in ihrem Beitrag an die Ausführungen von Andreas Nießeler an, der die Frage nach der Philosophie als eigenständigem Unterrichtsfach aufwarf. 
Sie sieht die Philosophie als Unterrichtsprinzip in den Fachunterricht eingebettet durch philosophische Gespräche mit den Kindern. 
Es geht dabei nicht um die Einführung neuer Inhalte, wie es in den anderen Fächern des Grundschulunterrichts der Fall ist, sondern um überfachliche Zugänge zu Fragen, mit denen sich die Schülerinnen und Schüler auseinander setzen. 
Dadurch eröffnet das Philosophieren mit Kindern vollkommen neue Möglichkeiten im interaktiven Lernen. 
Die Schüler werden durch die Gestaltung des Unterrichts und die Gelegenheit, eigene Fragen in den Unterricht einzubringen, an ihrem Lernstand abgeholt und es wird eine besondere Bedeutsamkeit für sie geschaffen.

Michalik stellt klar, dass Kinderfragen im alltäglichen Unterrichtsgeschehen keine besonders große Wichtigkeit zugeschrieben wird. 
Dem kann das Philosophieren als Unterrichtsprinzip entgegenwirken, indem es diese Fragen zum Ausgangspunkt des Unterrichts macht. 
Sie erörtert, dass der klassische Unterricht vor allem durch Lehrerfragen dominiert wird, die darauf ausgerichtet sind, vorhandenes oder zu erwerbenes Wissen abzufragen und zu festigen. 
Sie haben zudem einen \glqq fixierenden und einschränkenden Charakter\grqq{}\cite[S.\,635]{KM13}. 
Sie stellt jedoch den Stellenwert von Schülerfragen in Bezug auf den Lernprozess der Schülerinnen und Schüler heraus und knüpft an die Positionen des US-amerikanischen Medienwissenschaftlers Neil Postman an, der eine Perspektivenumkehr forderte, infolge derer das Stellen von Fragen als \glqq Kunst und Wissenschaft beizubringen\grqq{}\cite[S.\,636]{KM13} sei. 
Das Stellen von Fragen hängt zusätzlich eng mit dem Bildungsbegriff zusammen. 
So sagt der Philosoph Peter Bieri, dass man sich bilde, um etwas zu werden. 
Dabei gehe die Bildung immer von einer Neugierde und damit von eigenen Fragen aus.
Michalik weist darauf hin, dass es das Ziel des Philosophierens mit Kindern sein müsse, \glqq das Fragenstellen zu ermutigen und zu fördern und den Kindern zu zeigen, dass das gemeinsame Nachdenken über schwierige Fragen interessant und lohnenswert ist.\grqq{}\cite[S.\,637]{KM13}
Sie beschreibt, dass das klassische Unterrichtsgespräch vor allem von einer Lehrerrolle dominiert wird, die über umfassendes Wissen verfügt und Fragen stellt, die das Gespräch zu vorgefertigten Antworten lenken.

Daher müsse nach den Erkenntnissen Rainer Kokemohrs eine Modalisierung und Validierung stattfinden, um das Gespräch in Gang zu halten und offen zu gestalten.
Unter Modalisierung versteht Kokemohr \glqq die Öffnung des Gespräches hin zu konkurrierenden Lesarten.\grqq{}\cite[S.\,637]{KM13}
Das bedeutet, dass den Interpretationen und Denkweisen mehr Raum zugestanden wird.
Die Validierung hingegen bezeichnet die gegensätzliche Vorgehensweise, durch die das Ende eines Gespräches herbeigeführt wird, indem dieser Spielraum immer weiter eingeengt wird.
Die Schule stehe unter einem Validierungszwang, der dazu führe, dass einige mögliche Lern- und Verstehensprozesse verloren gehen würden.
 
Das Philosophieren mit Kindern kann nach Weber dazu beitragen, den Blick auf das Wissen, dass die Schule vermittelt, zu erneuern.
Werde durch die Schule bisher überwiegend Wissen vermittelt, dass sich durch begründbare Tatsachen charakterisiere, so könne das Philosophieren den Kindern zeigen, dass nicht alles in der Welt \glqq restlos vermessen, geordnet und erklärt ist, sondern auch Raum für Staunen, Nachdenklichkeit, Weiterfragen, Forschen bietet.\grqq{}\cite[S.\,639]{KM13}
Die Schülerinnen und Schüler sollten daher auch lernen, dass das Wissen des Menschen keineswegs allumfassend oder abgeschlossen ist, sondern, dass es immer Fragen gebe, die nicht abschließend beantwortet werden könnten.
Daher sei es wichtig, den Kinder diese Erfahrung der Begrenztheit von Wissenschaft zu ermöglichen und so \glqq Grundlagen für ein reflektiertes und differenziertes Welt- und Wissenschaftsbild zu legen.\grqq{}\cite[S.\,640]{KM13}
Für Michalik kann das Philosophieren dazu dienen, dem Lernen der Kinder einen Sinn zu geben und es vom Stigma des bloßen Bewältigens von Arbeitsaufträgen zu lösen.
Sie betont, dass Wissenschaft und Philosophie keine Gegensätze seien, sondern die Auseinandersetzung mit philosophischen Fragen sei ein zentraler Bestandteil für einen Unterricht, der sich an der Wissenschaft anlehnt.
  
Abseits der pädagogischen Begründungen, auf die vorhin eingegangen wurde, verweist Michalik in ihrem Artikel zudem auf verschiedene Studien, die im anglo-amerikanischen Raum entstanden sind und den positiven Einfluss auf die kognitive und sprachliche Entwicklung der Schülerinnen und Schüler belegen. 
Sie bezieht sich dabei auf die Studien, die, angelehnt an den kinderphilosophischen Ansatz von Lipman, \glqq kritisches, logisches, kreatives Denken (thinking skills) und Argumentations- und Gesprächsfähigkeiten\grqq{}\cite[S.\,643]{KM13} als relevante Aspekte ansehen. 
Demnach wurden Entwicklungen positiver Natur im Sozial- und Gruppenverhalten sowie hinsichtlich des Selbstwertgefühls, Selbstvertrauens und Selbstbewusstseins beobachtet.
Außerdem wurde in kanadischen Studien festgestellt, dass der gegenseitige Austausch unter Gleichaltrigen das Gesprächsbewusstsein und die Gesprächsfähigkeit fördern könne. 
Ergänzend werde das Denken der Schülerinnen und Schüler durch philosophische Gespräche \glqq zunehmend komplexer und multimodal im Sinne logischen, kreativen, verantwortlichen und meta-kognitiven Denkens.\grqq{}\cite[S.\,643]{KM13}

Mecklenburg-Vorpommern ist bisher das einzige Bundesland der Bundesrepublik Deutschland, in dem das Philosophieren mit Kindern ein eigenes Schulfach darstellt.
In diesem Bundesland haben die Fächer Philosophie und Philosophieren mit Kindern den Status des Ersatzfaches, das heißt, dass sie nur dann angeboten werden können, wenn auch gleichzeitig ein Angebot auf Religionsunterricht zur Verfügung steht.
Dieser Umstand kommt zustande, da der Religionsunterricht als einziges Fach im Fächerspektrum der Primarstufe eine besondere Stellung als ordentliches Unterrichtsfach einnimmt.
Das Ziel des Philosophierens ist es nach Silke Pfeiffer in Mecklenburg-Vorpommern, \glqq Kinder und Jugendliche bei ihrer Suche nach Orientierung und Sinn zu begleiten und ihnen Denk- und Verstehensangebote in existentiellen Fragen zu unterbreiten.\grqq{}\cite[S.\,652]{SP13}
Grundlage für das Philosophieverständnis, das hierfür zugrunde gelegt wird, ist es, die Interessen, intellektuellen Fähigkeiten etc. zu berücksichtigen.
Fachlich bezieht sich der Unterricht vor allem auf die Themenbereiche \glqq Familie\grqq{}, \glqq Natur\grqq{}, \glqq Konfliktbewältigung bzw. Gut und Böse\grqq{} sowie auf \glqq Fernseh- und Computerwelten\grqq{}.
Aus einem Interview mit einer Lehrenden und zwei Schülern heraus entwickelt Silke Pfeiffer, dass der Unterricht des Philosophierens mit Kindern vor allem schüler- und problemorientiert sei.
Darüber hinaus zeichne ihn ein besonderes Verhältnis zwischen Lehrperson und den Schülerinnen und Schülern aus.

Besonders positiv sei aus der Sicht der Teilnehmer, dass die inhaltliche Breite in einer neuen Dimension gegeben sei.
Jedoch gibt es auch Nachteile, die das Philosophieren als eigenes Unterrichtsfach mit sich bringen würde.
So sei es besonders schwierig, ein befriedigendes Zeitmanagement vorzunehmen, der Austausch unter den Fachkräften fände in einem unzureichenden Rahmen statt und die Benotung gestalte sich nicht zuletzt durch die Beschaffenheit des Faches komplex.

Es lässt sich zusammenfassend sagen, dass es verschiedene Faktoren gibt, die eine Etablierung des Philosophierens im Schulunterricht der Grundschule rechtfertigen. 
Das Philosophieren bietet den Lehrerinnen und Lehrern aber auch den Schülerinnen und Schülern abseits des gewöhnlichen Fachunterrichts die Möglichkeit, über Themen ins Gespräch zu kommen, die die Schüler interessiert und die ihnen persönlich wichtig sind. 
So kann das Philosophieren in der Grundschule das Selbstbewusstsein der Kinder, ihre Fragen einzubringen, aber auch das Selbstwertgefühl steigern. 
Dies kann jedoch nur gelingen, wenn die Lehrkraft bereit ist, die eigenen Gesprächsbeiträge zu minimieren und die Schülerinnen und Schüler zu Wort kommen zu lassen.
Das Philosophieren mit Kindern kann durch die Breite der Themen, mit denen es sich auseinander setzt, dazu beitragen, den Blick der Schüler auf das Wissen dahingehend zu verändern, dass Wissen nicht mehr bloß als rezipierbar wahrgenommen wird, sondern, dass es eine individuelle Bedeutung für jeden einzelnen Schüler entwickelt.
Das Philosophieren mit Kindern kann auch dazu beitragen, den Schülern den Sinn des Lernens zu erschließen, indem die Schülerfragen zum Ausgangspunkt des Unterrichts gemacht werden.
Gleichzeitig dürfen jedoch nicht zu hohe Erwartungen an dieses Unterrichtsprinzip gestellt werden, die dieses nicht in der Lage sein kann zu erfüllen. 
Es zeigt sich jedoch auch, dass das Philosophieren als eigenständiges Unterrichtsfach Vor- und Nachteile auf sich vereint.
So sei vor allem die Problematik der Notenfindung von Bedeutung, die bei einer Einbindung des Philosophierens in den Sachunterricht nicht in der Häufigkeit auftreten wird.
Schlussendlich können beim Philosophieren mit Kindern signifikante Verbesserungen in Sozial- und Fachkompetenzerwerb der Schülerinnen und Schüler nachgewiesen werden, wodurch es für den Lernprozess der Kinder einen elementaren Baustein darstellen kann.

\newpage
\subsection{Verortung des Philosophierens mit Kindern im Teilrahmenplan Sachunterricht Grundschule des Landes Rheinland-Pfalz}

Neben der fachlich-wissenschaftlichen Ebene ist in Bezug auf die Philosophie im Sachunterricht auch der pädagogische Blickwinkel von besonderer Bedeutung, um den Nutzen des Philosophierens mit Kindern in der Grundschule einordnen zu können. 
Dazu soll nun der Teilrahmenplan Grundschule für den Sachunterricht, der im Mai 2006 vom Ministerium für Bildung, Frauen und Jugend des Landes Rheinland-Pfalz herausgegeben wurde, in den Blick genommen werden. 

Ziel der Teilrahmenpläne, die für alle an Grundschulen in Rheinland-Pfalz angebotenen Fächer (Deutsch, Mathematik, Ethik, katholische/evangelische Religion, Sport, Kunst, Musik und Fremdsprachen) erarbeitet wurden, ist es, die Qualität und die Optimierung des Unterrichts voranzutreiben. 
Zudem bietet er Lehrerinnen und Lehrern die Möglichkeit, sich in Bezug auf die Kompetenzen, welche die Schülerinnen und Schüler am Ende der Grundschulzeit erworben haben sollten, in den jeweiligen Fächern zu informieren und zu orientieren. 
Gleichzeitig zeigt er den Lehrenden auch auf, welche fachlichen und didaktischen Voraussetzungen sie mitbringen müssen.

Grundsätzlich sieht der Teilrahmenplan die Entwicklung von Sprachfähigkeiten als untrennbar verknüpft mit dem Erwerb von Erfahrungswissen, da die Auseinandersetzung mit Umwelt Sprache benötige\cite[S.\,6]{MBFJ06}.
Im Sammeln von Erfahrungen wird dabei zwischen Primärerfahrungen, die das Kind in direkter Konfrontation macht und Sekundärerfahrungen, die durch Sprache zum Ausdruck gebracht werden, differenziert. 

Diese Primär- und Sekundärerfahrungen finden sich ebenfalls im Philosophieren, in dessen Rahmen die Schülerinnen und Schüler ihre Primärerfahrungen zu philosophischen Themen kundtun und die Mitschüler auf diese Weise Sekundärerfahrungen machen. 
Zusätzlich lernen die Kinder, ihre persönlichen Erfahrungen zu prüfen, mit denen der Mitschüler zu vergleichen, zu verifizieren und gegebenenfalls auch zu falsifizieren. 
Dieser Aspekt des kritischen Umgangs mit der eigenen Erfahrung wird auch vom Teilrahmenplan als zentraler Bestandteil des Spracherwerbs angesehen.

Der Teilrahmenplan weist ausdrücklich darauf hin, dass die Schülerinnen und Schüler Erfahrungen innerhalb und außerhalb der Sphären der Schule machen. 
Dies wird unter dem Schlagwort der \glqq Handlungskompetenz\grqq{} verortet. 
Das Philosophieren in der Klassengemeinschaft bietet in diesem Rahmen einen besonderen Raum für Fragen, die sich außerhalb der gewöhnlichen inhaltlichen Linien des Fachunterrichtes der Grundschule befinden und eröffnet den Kindern so völlig neue Zugänge. 
Der Kompetenzerwerb wird dabei nochmal in vier verschiedene Arten gegliedert: personal, sozial, methodisch und fachlich. 

Die Kinder können durch das Philosophieren auf allen Ebenen profitieren. 
Sie erwerben das Selbstbewusstsein, ihre Gedanken zu philosophischen Inhalten zu kommunizieren (personal), sie partizipieren an den Ansichten der Anderen (sozial), sie lernen, ihre Erfahrungen sprachlich auszudrücken (methodisch) und sie setzen sich mit für sie bedeutsamen philosophischen Fragestellungen auf ihrem kognitiven Niveau auseinander (fachlich).

Zusätzlich zu diesen allgemeinen Aspekten formuliert der Teilrahmenplan ein Leistungsprofil für den Sachunterricht, das angeben soll, welche Leistungen die Kinder im Rahmen ihrer Fähigkeiten am Ende der Grundschulzeit erbringen können. 
Für das Führen von philosophischen Gesprächen im Sachunterricht ist dabei vor allem von Bedeutung, dass die Schülerinnen und Schüler \glqq angemessene Darstellungsformen\grqq{}\cite[S.\,8]{MBFJ06} beherrschen.
Zu diesen zählen insbesondere die Gesprächsformen, oder die verschiedenen Darstellungen sprachlicher Beiträge wie z.B. Berichten, Darstellen und Referieren.

Darüber hinaus ist es wichtig, dass sich Kinder \glqq Fragen und Problemen aus ihren natürlichen, sozialen, kulturellen, technischen und wirtschaftlichen Erfahrungsbereichen mit Neugier und Selbstvertrauen\grqq{}\cite[S.\,8]{MBFJ06} zuwenden. 
Ebenfalls für philosophische Gespräche relevante Kompetenzen liegen im \glqq Auffinden, Erklären, Darstellen und Begründen von Strategien zur Lösung von Sachfragen (Bilden von Hypothesen, überprüfen von Annahmen, Experimentieren, Schlüsse ziehen, übertragen von Ergebnissen auf analoge Sachverhalte).\grqq{}\cite[S.\,8]{MBFJ06}
Denn in diesen Gesprächen steht der interaktive Austausch im Fokus des Sachunterrichts, so dass die Schülerinnen und Schüler diese Kompetenzen in besonderer Weise schulen.

Dem Sachunterricht kommt nach Ansicht des Teilrahmenplans eine ganz besondere Rolle im Fächerspektrum der Grundschule zu. 
Er diene \glqq nicht nur dem Erwerb von Kenntnissen, Fähigkeiten und Fertigkeiten im Umgang mit der sozialen und natürlichen Umwelt, sondern schließt die Förderung der sprachlichen, ästhetischen und interkulturellen Bildung, der wertbewussten Orientierung und des Verstehens ein.\grqq{}\cite[S.\,9]{MBFJ06} 
Diesem überfachlichen Anspruch kann das Philosophieren mit Kindern im Unterricht der Grundschule Rechnung tragen, indem, wie vorhin bereits angesprochen, auch außerschulische Dimensionen, die in den Fragen der Schülerinnen und Schüler zum Ausdruck kommen, Bestandteil der Auseinandersetzung werden.

Eine Besonderheit des Sachunterrichts stellt die Tatsache dar, dass dieser sich aus verschiedenen Themengebieten zusammensetzt, die vom Teilrahmenplan fünf Erfahrungsbereichen zugeordnet werden: Natur, Gesellschaft, Technik, Raum und Zeit. 
In diesem Zusammenhang liegt nahe, dass sich das Philosophieren mit Kindern in der Grundschule vor allem den Erfahrungsbereichen \glqq Gesellschaft\grqq{} und \glqq Zeit\grqq{} des Teilrahmenplans zuordnen lässt. 

Im Rahmen des Erfahrungsbereiches \glqq Gesellschaft\grqq{} lässt sich vor allem konstatieren, dass das Philosophieren dazu beitragen kann, dass sich die Kinder mit der Andersartigkeit der Mitmenschen auseinandersetzen in Bezug auf deren Bedürfnisse, Gefühle und Sehnsüchte. 
Diese Schlüsselkompetenz nennt der Teilrahmenplan unter Punkt 1, was ebenfalls dessen Bedeutung unterstreicht. 

Darüber hinaus ist die Antizipationsfähigkeit der Schüler, die sie im Rahmen des Sachunterrichts erwerben sollen, unabdingbar in ihrem späteren Leben, damit sie in die Lage versetzt werden, das Handeln, die Ansichten aber auch die Empfindungen Anderer besser verstehen zu können. 
Beim Philosophieren kann das beispielsweise bedeuten, dass die Kinder darüber nachdenken, was für sie Glück bedeutet, was für ihre Mitschüler Glück ist und warum sie Glück für sich so definieren, wie sie es tun. 

Auch die Perspektive der Zeit kann zur Einordnung des Philosophierens in den Kontext des Sachunterrichts herangezogen werden. 
Für die Schüler ist es wichtig, zu erfahren, dass Zeit etwas ist, was dem menschlichen Leben Struktur gibt und es möglich macht, in Epochen und Zeiträumen die Entwicklung menschlichen Denkens über Jahrtausende hinweg zu untersuchen und zu durchdringen. 
Dabei sieht der Teilrahmenplan unter anderem die \glqq Beurteilung von Entscheidungen und Handlungen\grqq{}\cite[S.\,15]{MBFJ06} unter Beachtung der vorherrschenden Lebensumstände als wesentlich an, um zum Beispiel die Ansichten von Philosophen im Vergleich zur heutigen Denkweise richtig einordnen zu können. 
Auf diese Weise bekommen die Schülerinnen und Schüler einen umfassenderen Blick auf die philosophischen Hintergründe.

Letztlich sieht der Teil"-rah"-men"-plan Grund"-schu"-le den Sach"-unter"-richt in der Pflicht, 
Anreize zu schaffen, die die von Natur aus vorhandene Neugier der Kinder wecken und sie in ihrem eigenständigen Sammeln von Erfahrungen begleiten sollen. 
Dazu sollten Lernumgebungen geschaffen werden, die diese Voraussetzungen ermöglichen. 

Als eine solche, gewinnbringende Lernumgebung kann das Philosophieren mit Kindern in der Grundschule verstanden werden, das sich neben den bereits behandelten, positiven Entwicklungen des Kompetenzerwerbs und der fachlichen Breite
auch dadurch auszeichnen kann, dass es die sonst üblichen Rollen und Rituale des Fachunterrichts der Primarstufe für die Kinder nutzbringend verändern kann.

\newpage
\subsection{Bezug zum Perspektivrahmen der Gesellschaft für die Didaktik des Sachunterrichts  e.V. (GDSU)}

Der Perspektivrahmen Sachunterricht, der 2013 in einer überarbeiteten Fassung erschienen ist, sieht seine Aufgabe darin, die Didaktik des Sachunterrichts zu fördern und Disziplin des wissenschaftlichen Diskurses im Hinblick auf Forschung und Lehre zu etablieren. 
Die Gesellschaft selbst sieht sich als Bund von Angehörigen von Hochschulen, Lehrerfortbildungen, Lehrerweiterbildungen und Schulen. 
Ziel des Perspektivrahmens soll es sein, Lehrerinnen und Lehrer darin zu unterstützen, kompetenzorientierten Sachunterricht planen, ausarbeiten, untersuchen und reflektieren zu können.

Dazu liefert er ein Kompetenzmodell, dass den Lehrenden einen Überblick über inhaltliche Perspektiven, perspektivenübergreifende sowie perspektivenbezogene Handlungsweisen geben soll. 
Diese Handlungsweisen werden im Folgenden weiter aufgeschlüsselt und die perspektivenbezogenen Handlungsweisen den fünf Inhaltsperspektiven, die den Sachunterricht in einen sozialwissenschaftlichen, naturwissenschaftlichen, geographischen, historischen und technischen Bereich gliedern, zugewiesen\cite[S.\,5f]{GDS13}.
 Zudem werden konkrete Unterrichtsbeispiele vorgestellt, die jedoch in der hier vorgenommenen Einordnung des Philosophierens mit Kindern in der Grundschule keine Rolle spielen.
 
Der Perspektivrahmen Sachunterricht sieht im Fach Sachunterricht einen \glqq zentralen Beitrag zu grundlegender Bildung\grqq{}\cite[S.\,9]{GDS13}.
Dabei wird der Begriff Bildung als charakteristisch für das Wesen des Menschen aufgefasst und daraus der Anspruch für den Sachunterricht abgeleitet, die Schülerinnen und Schüler zu einem verantwortungsbewussten Verhalten in ihrer Umgebung zu erziehen. 
Die vornehmliche Aufgabe sieht er darin, die Schülerinnen und Schüler zu einem Verständnis ihres natürlichen, kulturellen, sozialen und technischen Umfeldes zu bringen, in dessen Erwerbsprozess auch die Vorerfahrungen der Kinder aus Familie, Kindertagesstätten und anderen sozialen Einrichtungen einfließen sollen und diese ein darauf aufbauendes Lernen begünstigen sollen. 
In diesem Zusammenhang sieht der Perspektivrahmen Sachunterricht eine zweifache Herausforderung in der Anschlussfähigkeit des Sachunterrichts, der einerseits die Prämissen der Wissensstände der Kinder sowie deren Interessen und Fragen im Blick haben muss. 
Andererseits darf auch die Anschlussfähigkeit an das fachliche Wissen nicht ausgeklammert werden. 

Analog zum Teilrahmenplan des Landes Rheinland-Pfalz sieht auch der Perspektivrahmen Sachunterricht die Sprache eng verknüpft mit dem Sachunterricht selbst. 
So sei die Sprache im Kontext des Sachunterrichts einerseits ein Werkzeug des Austauschs und der Konstruktion von Inhalten, andererseits entwickle sich die Sprache aber auch im Laufe der Grundschulzeit mithilfe des Sachunterrichts, da es auch eine der Aufgaben der Lehrerinnen und Lehrer sei, die Alltagssprache der Kinder durch eigene Impulse und die Struktur des Sachunterrichts zu einer Fachsprache zu führen. 
Durch den Sachunterricht werde ein wesentlicher \glqq Beitrag zur sprachlichen Bildung von Schülerinnen und Schülern, wenn (häufig sinnlich wahrnehmbare) \glqq Sachen\grqq (wie Gegenstände oder auch Prozesse) zu benennen sind\grqq{}\cite[S.\,11]{GDS13} geleistet.

Wie bereits erwähnt legt der Perspektivrahmen Sachunterricht seinen Überlegungen ein Kompetenzmodell zugrunde, das zwischen perspektivenbezogenen und perspektivenübergreifenden Handlungsweisen in Bezug auf die inhaltlichen Schwerpunkte differenziert.
 Darüber hinaus wird zwischen den Dimensionen \glqq Konzepte/Themenbereiche\grqq{} und \glqq Denk-, Arbeits- und Handlungsweisen\grqq{} unterschieden, wobei sich erstere vor allem auf den deklarativen Wissenserwerb, die zweite auf das prozedurale Wissen fokussiert. 
 Es wird ferner darauf verwiesen, dass für die Planung von Unterricht beide Dimensionen gemeinsam zu denken seien.
 Nun soll näher auf die perspektivenübergreifenden Aspekte des Kompetenzmodells eingegangen werden und aufgezeigt werden, wie sich das Philosophieren mit Kindern in dieses Raster einordnen lässt. 
 Dabei werden die im Perspektivrahmen verwendeten Überschriften zum besseren Verständnis übernommen. 
Zudem wurde die Handlungsweise des Umsetzen/Handeln für den Kontext des Philosophierens ausgeklammert, da der Fokus des Philosophierens mehr auf dem Denken als dem Handeln liegt.

\newpage

\subsection{Perspektivenübergreifende Denk, Arbeits- und Handlungsweisen: Erkennen/Verstehen}


Das Verstehen bildet für den Lernprozess und den Kompetenzerwerb der Schülerinnen und Schüler in der Grundschule eine entscheidende Grundlage. 
Bereits mehrfach wurde auf den Einfluss der Vorerfahrungen der Kinder hingewiesen, denn für das Verstehen sind diese essenziell. 
Daher muss im Sachunterricht dafür Sorge getragen werden, dass Lernumgebungen konstruiert werden, in denen diese Verstehensprozesse zustande kommen können. 
Das Philosophieren bietet den Kindern dazu eine Möglichkeit, indem es diese Erfahrungen einbindet und Diskussionen anstößt, durch die die Kinder ihre Positionen argumentativ begründen, verteidigen und unter Umständen auch überdenken müssen. 

Dabei kommt auch der Lehrperson eine wichtige Rolle zu, die durch ihr Frageverhalten den Verstehensprozess einscheidend beeinflussen kann. 
Kobarg unterscheidet verschiedene Arten von Lehrerfragen: keine, offene und geschlossene Fragen. 
Offene Fragen zeichnen sich dadurch aus, dass sie keine bestimmte Antwort verlangen, während geschlossene Fragen genau darauf abzielen. 
Darüber hinaus gibt es Niveaus, denen diese Fragen zugeordnet werden können\cite[S.\,22]{HB15} und durch die der Lern- und Verstehensprozess mitbestimmt wird.
In diesem Zusammenhang ist es für die Lehrperson wichtig, besonders sogenannte \glqq Deep-Reasoning\grqq{}-Fragen zu stellen, die auf eine länger ausgestaltete Antwort abzielen und von den Kindern autonomes Denken einfordert. 

Solche Prozesse können auch kollektiv durch Partner- und Gruppenarbeiten geschehen, in denen die Kinder sich über ihre Ansichten und Argumente austauschen und so gemeinsam philosophische Ideen entwickeln. 
Ergänzend dazu lassen sich auch \glqq komplexe, problemhaltige Anforderungen, die eine Übertragung vorhandenen Wissens in neue Kontexte erfordern\grqq{}\cite[S.\,21]{GDS13}, wie es der Perspektivrahmen Sachunterricht beschreibt, in das Philosophieren einbinden, indem etwa bereits erworbenes Wissen in anderen Zusammenhängen wie einer Geschichte angewandt und übertragen wird. 

\newpage

\subsubsection{Perspektivenübergreifende Denk, Arbeits- und Handlungsweisen: Eigenständig erarbeiten}


Der Perspektivrahmen Sachunterricht sieht die Fähigkeit der Schülerinnen und Schüler, sich selbstständig neues Wissen anzueignen, als essenziell an, um sich in einer Welt, die von einer schnell wachsenden und sich verändernden Wissenslandschaft geprägt ist, zurechtzufinden. 
Um dies zu erreichen, müssen den Schülerinnen und Schülern im Sachunterricht auch Aufgaben gestellt werden, \glqq die Lernende aus eigenem Interesse entwickeln oder die sie sich zu eigen machen.\grqq{}\cite[S.\,22]{GDS13}
Das Entfalten von Wissen aus eigenem Antrieb kann im Sachunterricht durch das Philosophieren geleistet werden. 
Dadurch, dass die Kinder die Gelegenheit haben, in diesem Rahmen ihre persönlichen Fragen zu stellen und sich darüber mit Gleichaltrigen auszutauschen, erweitern sie ihr eigenes Weltwissen und reflektieren es selbstständig im Kontext der Klassengemeinschaft. 
Hinzu kommt, dass die Schüler beim Philosophieren auch die Möglichkeit erhalten, unterschiedliche Zugangsweisen zu Wissen kennenzulernen. 
So können im Kontext philosophischer Fragestellungen neben Printmedien oder Expertenbefragungen auch moderne Medien wie das Internet genutzt werden.
Zusätzlich lernen die Schüler durch das Philosophieren im Sachunterricht, eine eigene Reflexionskompetenz auszubilden. 
So sollen sie nach Meinung des Perspektivrahmens \glqq ihre selbst gewählten Lernwege erläutern, begründen und überprüfen.\grqq{}\cite[S.\,23]{GDS13}


\subsubsection{Perspektivenübergreifende Denk, Arbeits- und Handlungsweisen: Evaluieren/ Reflektieren}


Die Reflexion und die Evaluation der eigenen Positionen sind für Schülerinnen und Schüler wichtig, um das eigene Denken und Handeln nicht nur an den eigenen Interessen und Bedürfnissen auszurichten, sondern auch die der Mitmenschen im Blick zu haben. 
In diesem Kontext kann das Philosophieren einen entscheidenen Beitrag leisten, indem die Kinder Meinungen ausdrücken und diese anschließend durch den Erwerb von philosophischen Wissen bestätigt oder widerlegt werden können. 
Hinzu kommt, dass es für die Kinder wertvoll ist, ihre eigenen Vermutungen anderen gegenüberzustellen, um so etwaige Überschneidungen oder auch Differenzen ausfindig machen zu können. 

Ein konkretes Beispiel könnte das sogenannte \glqq Heinz-Dilemma\grqq{} von Lawrence Kohlberg sein, das an die \glqq Was soll ich tun?\grqq{}-Frage von Immanuel Kant anknüpft. 
In der Geschichte wird von einem Mann erzählt, dessen Frau schwer krank ist. 
Es gibt jedoch eine Medizin, die sie retten könnte.
 Allerdings verlangt der Apotheker den zehnfachen Preis des Herstellungspreises für eine Dosis, so dass sich der Mann das Medikament auch mithilfe seiner Freunde nicht leisten kann. 
 So stellt sich die Frage, ob er das Medikament stehlen soll oder nicht. 
 Die Schülerinnen und Schüler könnten einerseits vermuten, dass es besser wäre, es zu stehlen, weil der Mann damit seine Frau retten könne.
 Andererseits würde er damit gegen das Gesetz verstoßen, den Apotheker schädigen und unter Umständen eine Gefängnisstrafe riskieren. 
 Die Kinder könnten daraufhin reflektieren, was sie anstelle des Mannes tun oder aber auch nicht tun würden und sich darüber austauschen.

Die Möglichkeiten, die dieses Beispiels deutlich macht, werden auch vom Perspektivrahmen genauer benannt.
So verweist die Gesellschaft für die Didaktik des Sachunterrichts darauf, dass folgende, sogenannte \glqq Lernmöglichkeiten\grqq{} die Kinder unterstützen können:
Das Ansprechen von \glqq Vermutungen und Vorerfahrungen vor der Erarbeitung neuen Wissens\grqq{}\cite[S.\,23]{GDS13}, welche anschließend bejaht oder verneint werden, die Schaffung von Lernumgebungen, in denen die Kinder zu eigenen Annahmen andere Optionen entdecken, um den Inhalt gänzlich zu durchdringen und Phasen des Austauschs, der Reflexion und des Nachdenkens, in denen die Kinder für sie bedeutsame Fragen besprechen können.

\subsubsection{Perspektivenübergreifende Denk, Arbeits- und Handlungsweisen: Kommunizieren/Mit anderen zusammenarbeiten}


Die Konstruktion von Wissen in der Schule ist kein Prozess, den Schülerinnen und Schüler allein bewältigen können. 
Daher liegt es nahe, dass die Kommunikation und Interaktion eine weitere, elementare Kompetenz darstellt, die die Kinder im Laufe ihrer Grundschulzeit erwerben müssen.
 Die philosophischen Unterrichtsgespräche können die Kinder dabei in ihrer Kommunikationsfähigkeit unterstützen, da sie im Rahmen dieser Unterrichtsanordnung mehr Anreize bekommen, sich zu äußern, als in den üblichen Unterrichtsgesprächen, die den schulischen Alltag seit Jahren dominieren.

Hinzu kommt, dass das Philosophieren mit Kindern ein Baustein für die Kinder sein kann, um von der Alltagssprache langfristig zu einer detaillierteren Fachsprache zu gelangen, um in der Lage zu sein, Wissen und Sachverhalte adäquat versprachlichen zu können\cite[S.\,24]{GDS13}.
Das Kommunizieren in der Gruppe unterstützt die Kinder darin, gemeinsam mit ihren Mitschülern Ideen zu konzipieren und zu optimieren. Daher nennt der Perspektivrahmen Sachunterricht dies als eine weitere Lernmöglichkeit.

\subsubsection{Perspektivenübergreifende Denk, Arbeits- und Handlungsweisen: Den Sachen interessiert begegnen}


Letztlich wird der Lernerfolg der Schülerinnen und Schüler in der Grundschule auch dadurch mitbestimmt, inwiefern dieser ihre ganz persönliche Neugier und ihre Begeisterung wecken kann. 
Daher sollte auch dem Schülerinteresse besonders Rechnung getragen werden und dieses im Sachunterricht in die Unterrichtsplanung miteinbezogen werden.
 Dadurch, dass das Philosophieren mit Kindern nicht direkt an zentrale Vorgaben durch den Lehrplan gebunden ist, ergibt sich ein großes Potential für philosophische Gespräche, da diese genau auf eine bestimmte Lerngruppe abgestimmt werden können. 
Der Perspektivrahmen Sachunterricht hat die Bedeutung des Schülerinteresses als eine eigene, perspektivenübergreifende Denk-, Arbeits- und Handlungsweise erkannt und hebt so ihren Stellenwert für einen kompetenzorientierten Sachunterricht hervor. 
Hierbei tauchen auch die bereits angesprochenen Schülerfragen als ein Ausgangspunkt des Sachunterrichts auf, die das gesteigerte Interesse der Schülerinnen und Schüler wecken können und ihnen die Bedeutsamkeit eines bestimmten Unterrichtsinhaltes greifbar machen können. 

Grundlegend sei auch der bewusste Umgang der Lehrkraft mit \glqq Rückmeldungen, die wertschätzend die Anstrengung und die geleistete Arbeit beurteilen\grqq{}\cite[S.\,25]{GDS13}, da diese ebenfalls zu einer erhöhten Interessenshaltung der Schülerinnen und Schüler beitragen können. 
Letztendlich müsse auch die Gestaltung des Unterrichts an sich durch die Auswahl der Themen, die ästhetische Aufbereitung und die Einbindung der Schülerinnen und Schüler so ausgearbeitet werden, dass die behandelten Themen \glqq von den Schülerinnen und Schülern erlebt, nachvollzogen und bearbeitet werden\grqq{}\cite[S.\,25]{GDS13}.






