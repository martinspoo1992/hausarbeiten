\RequirePackage{ifvtex}
\documentclass[12pt,headlines=2,footlines=2,a4paper,oneside,bibliography=totoc]{scrartcl}

\usepackage{ngerman}
\usepackage[utf8]{inputenc}
\usepackage[T1]{fontenc}
\usepackage{setspace}
\usepackage[left=25mm,top=25mm,right=40mm,bottom=40mm,nohead,nofoot]{geometry}
\usepackage[font=small,format=plain,labelfont=bf,up,textfont=it,up]{caption}
\usepackage[headsepline,footsepline]{scrpage2}
\usepackage{graphicx}
\usepackage{tikz}
\usepackage{hyperref}
\usepackage{float}
\usepackage{longtable}
\usepackage[numbers]{natbib}
\usepackage{soul}
\usepackage{dialogue}
\usepackage{attrib}

\setcounter{LTchunksize}{100}

\usetikzlibrary{shapes,arrows}

\pagestyle{scrheadings}
\ihead{\headmark}
\ohead{Martin Spoo}
\ifoot{Philosophische Vorstellungen von Kindern zum Thema \glqq Glück\grqq{} in der Grundschule am Beispiel der Grundschule Kesselheim}
\cfoot{}
\ofoot{\pagemark}
\addtolength{\headsep}{5mm}
\addtolength{\footskip}{5mm}
\setlength{\headheight}{1.1\baselineskip}
\setlength{\footheight}{1.5\baselineskip}




%\bibliographystyle{plainde}

\renewcommand{\figurename}{Abb.}

\begin{document}

%% Titelseite
\thispagestyle{empty}
\newgeometry{left=1.5cm,right=1.5cm,head=1.5cm,bottom=1.5cm}
\begin{titlepage}
	\subpdfbookmark{Titel}{pdf:title}
	\begin{center}
		\quad
		\vfill
		\Huge{
			\setstretch{2} \textbf{Fußballfangesänge – Pejorative Sprachhandlungen von Fußballfans am Beispiel "`Alemannia Aachen"'}
		}
		\vspace{5mm}
		\vfill
		\large{
			{\bfseries Masterarbeit\\
			\vspace{5mm}
			\normalfont \rmfamily zur Erlangung des akademischen Grades\\
			\bfseries Master of Education (M. Ed.)\\
			\vspace{5mm}
			\normalfont \rmfamily im Studiengang Grundschulbildung}
		}
		\\
		\vspace{1.5cm}
		\large{
			{vorgelegt von\\
			Martin Spoo\\
			Matrikelnr.\,212100872}
		}
		\vspace{1cm}
		\\
		\Large{
			{Universität Koblenz-Landau}\\
			{WS\,2014/2015}
		}
		\vspace{1cm}
		\begin{table}[b]
			\begin{center}
				\begin{tabular}{lr}
					Prüfer: & Apl. Prof. Dr. Hajo Diekmannshenke, \\
								&	Institut für Germanistik, Campus Koblenz \\
					Zweitprüfer: & Prof. Dr. Wolf-Andreas Liebert \\
					\vspace{0.25cm} \\
					Abgabetermin: & 04. Februar 2015 \\
					Datum: & \today
				\end{tabular}
			\end{center}
		\end{table}
	\end{center}
\end{titlepage}
\renewcommand{\baselinestretch}{1.1}
\restoregeometry



%% Eine leere Seite nach dem Titelblatt einfügen
\newpage

%% Seitenzähler nach dem Titelblatt auf 1 setzen
\setcounter{page}{1}

%% Inhaltsverzeichnis
\currentpdfbookmark{Inhaltsverzeichnis}{pdf:toc}
\tableofcontents
\newpage

%% Zeilenabstand auf 1,5fach
\onehalfspacing

%% Einleitung
\section{Einleitung} 
Die folgende Hausarbeit setzt sich mit dem Rassismus-Motiv der \glqq Harry Potter\grqq{}-Reihe von Joanne K. Rowling am Beispiel des Bandes \glqq Harry Potter und die Heiligtümer des Todes\grqq{} auseinander.
In der Betrachtung soll dabei zunächst auf wichtige Termini der Bücher in Bezug auf Rassismus eingegangen werden. 
Im weiteren Verlauf soll der Rassismus in der magischen Welt beleuchtet und an ausgewählten Textpassagen des siebten Bandes der Reihe mit dem Titel \glqq Harry Potter und die Heiligtümer des Todes\grqq{} belegt werden. 
Besonderes Augenmerk soll dabei auf der Figur des Tom Riddle alias Lord Voldemort und seiner Bedeutung als Führer der Todesser liegen.
Schlussendlich soll untersucht werden, ob und inwiefern sich Gemeinsamkeiten und Unterschiede zu historischen Entwicklungen in Bezug auf Rassismus ergeben. 
Dies soll am Beispiel des Nationalsozialismus deutlich gemacht werden. 
Dabei soll auch die Frage geklärt werden, ob die Figur des Lord Voldemort als  literarische Darstellung von Adolf Hitler dienen kann. 

Die Wahl des Themas ergab sich aus dem Interesse, das Thema \glqq Nationalsozialismus\grqq{} in den Kontext der Literatur zu bringen. 
Dabei bot sich die \glqq Harry Potter\grqq-Reihe aufgrund ihrer Themenvielfalt und Tiefe an. 
Des Weiteren übten die Werke auch auf mich persönlich eine große Faszination aus. 

Die \glqq Harry Potter\grqq-Reihe, bestehend aus 7 einzelnen Bänden, ist mit über 500 Millionen weltweit verkaufter Exemplare eine der erfolgreichsten literarischen Reihen. 
Literarisch lässt sie sich der Kinder- und Jugendliteratur, aber auch dem Genre der Fantasyliteratur zuordnen. 
Allerdings erfreuen sich die Bücher nicht nur bei Kindern und Jugendlichen großer Beliebtheit, sondern werden auch von Erwachsenen gelesen.
 Zudem bilden die 8 Verfilmungen der Bücher – ergänzend ist zu sagen, dass der siebte Band in zwei Filmen umgesetzt wurde – mit ca. 8 Mrd. US-\$ die finanziell erfolgreichste Literaturverfilmungsreihe aller Zeiten.

\glqq Harry Potter und die Heiligtümer des Todes\grqq{} schließt die Geschichte um den Zauberlehrling Harry Potter, der mit seinen Freunden Ron Weasley und Hermine Granger gegen den bösen Zauberer Tom Riddle, der sich selbst nur \glqq Lord Voldemort\grqq{} nennt, kämpft, ab.
Die Geschichte knüpft nahtlos an den vorherigen Band \glqq Harry Potter und der Halbblutprinz\grqq{} an, in dem der Schulleiter der Hogwarts-Schule für Hexerei und Zauberei, Albus Dumbledore, versucht, die magischen Gegenstände, in die Lord Voldemort Teile seiner Seele versiegelt hat, zu zerstören. 
Jedoch wird Dumbledore am Ende des 6.\,Bandes getötet, woraufhin sich Harry, Ron und Hermine allein auf die Suche nach den verbliebenen Gegenständen machen. Lord Voldemort und seine Anhänger, die Todesser, haben unterdessen das mächtigste politische Organ der Zaubererwelt, das Zaubereiministerium, unter ihre Kontrolle gebracht und begonnen, eine rassistisch motivierte \glqq Auslese\grqq{} denjenigen gegenüber vorzunehmen, die nach ihrer Auffassung nicht würdig sind, magische Fähigkeiten zu erlernen. 
Schlussendlich gelingt es Harry, Ron und Hermine, die verbliebenen, Horkruxe genannte, Teile von Voldemorts Seele zu zerstören und ihn und seine Helfer in der \glqq Schlacht um Hogwarts\grqq{} zu besiegen.

\newpage

%% Philosophieren in der Grundschule
\section{Philosophieren mit Kindern}

Das Philosophieren ist mittlerweile vor allem an den weiterführenden Schulen im deutschen Bildungssystem angekommen.
 In vielen Fällen wird es sogar als eigenständiges Unterrichtsfach angeboten. 
 Doch auch in der Grundschule gewinnt das Philosophieren mit Kindern stetig an Bedeutung, weshalb es von zentraler Bedeutung ist, zu klären, inwiefern es auch im Sachunterricht der 
 Grundschule Sinn macht, den Schülerinnen und Schülern ein solches Unterrichtsangebot zu machen. 
 
Zunächst soll der Fokus jedoch auf der Betrachtung des Philosophiebegriffs liegen und darauf, was man letztlich tut, wenn man philosophiert.



\subsection{Was versteht man unter Philosophie und Philosophieren?}

Der Begriff \glqq Philosophie\grqq{} setzt sich aus den beiden Ausdrücken \glqq philos\grqq{}, welcher Liebe bedeutet und \glqq sophia\grqq{}, was mit Weisheit übersetzt werden kann, zusammen\cite[S.\,8]{BB10}. 
An anderer Stelle, wie in  \cite{GT16}, wird der Begriff \glqq philos\grqq{} hingegen als \glqq Freund\grqq{} übersetzt. 
Von der Wortherkunft lässt sich daher sagen, dass sich die Philosophie als \glqq Liebe zur Weisheit\grqq{} oder \glqq Freundschaft zur Weisheit\grqq{} übersetzen lässt. 

Aufgabe der Philosophen als Philosophie betreibende Menschen und der Philosophie allgemein ist es, \glqq über wichtige Lebensfragen\grqq{} des Menschen und seiner Umwelt nachzudenken\cite[S.\,8]{BB10}. 

Brüning beschreibt, dass sich zwei zentrale Formen der Philosophie entwickelt haben. 
Die esoterische Philosophie beruht darauf, dass seit den Anfängen der griechischen Philosophietradition ein weitreichendes Wissen über die Welt und ihr Dasein gesammelt und zusammengetragen wurde. 
Dieses Wissen wurde dann an philosophischen Hochschulen und Universitäten gelehrt und in eine akademische Disziplin überführt, welche sich folglich Philosophie nennt. 
Darüber hinaus wird in der esoterischen Philosophie angenommen, dass lediglich diejenigen Menschen auch Philosophen sein können, die sich beruflich und mit der Absicht der Forschung mit den Fragen über die Welt beschäftigen.

Demgegenüber steht die exoterische Philosophie, welche ein entgegengesetztes Bild der Philosophie als Disziplin als gegeben ansieht. 
Demnach sind im Wesentlichen alle Menschen in der Lage, sich über die zentralen philosophischen Themen Gedanken zu machen. 
Dabei ist es unerheblich, ob dies aus persönlicher Neugierde oder aus dem Bedürfnis nach Forschungszuwachs geschieht. 
Ferner ist für die exoterische Philosophie auch nicht von Bedeutung, ob das Nachdenken über Philosophisches in einer regelmäßigen Auseinandersetzung stattfindet oder ob der Anspruch des Philosophen erhoben wird, haltbare Theorien zu entwickeln\cite[S.\,8]{BB10}.

Dr. Holm Bräuer, der an der TU Dresden im Institut für Philosophie tätig ist, bezeichnet die Philosophie als wissenschaftliche Disziplin, welche im Gegensatz zu anderen Einzelwissenschaften vor allem gemeingültige Fragen behandelt.
 Dabei sieht auch er vor allem die Frage nach dem Wesen der Welt, aber auch nach dem Sinn des Lebens oder den Ursachen für \glqq Vergangenes und Zukünftiges von religiösen oder mythischen Lehren und Legenden oder von Weltanschauungen\grqq{}\cite{PL16} als zentrale Fragen der Philosophie an.

 Ziel der Philosophie ist es dabei aus seiner Sicht, auf diese allgemein gefassten Fragen der Menschheit Erklärungen zu finden, welche auf der Basis von rationalen Zusammenhängen zustande kommen.
  Bräuer bezieht sich insbesondere auf die Grundfragen Immanuel Kants, der die philosophischen Probleme mit vier Fragen erstmals resümierte: \glqq Was ist der Mensch?\grqq{}, \glqq Was kann ich wissen?\grqq{}, \glqq Was soll ich tun?\grqq{}, \glqq Was darf ich hoffen?\grqq{}.  
  
Im Laufe der Geschichte hat sich die Philosophie durch die verschiedenartigen Fragen, mit denen sie sich beschäftigt, in Teildisziplinen untergliedert, die sich unter den Bereichen der Metaphysik, der Physik und der Ethik fassen lassen. 
Über diesen drei Teildisziplinen stehen einerseits die theoretische Philosophie, die sich vor allem auf den Erkenntnisgewinn und das Verstehen der Welt konzentriert, andererseits die praktische Philosophie, die die Aufmerksamkeit vor allem auf das Handeln des Menschen aus philosophischer Sichtweise richtet. 
Die Teilbereiche der Metaphysik und der Physik lassen sich der theoretischen Philosophie, die Ethik und dabei vor allem die Staatsphilosophie und die Rechtsphilosophie der praktischen Philosophie zuordnen. 

Die Tätigkeit des Philosophierens sieht Barbara Brüning vor allem durch einen Aufbau gekennzeichnet, der sich anhand von zentralen Merkmalen manifestiert. 
Am Beginn eines jeden Philosophierens sieht sie zunächst das Staunen, das dazu führt, das etwas zuvor als trivial Angesehenes von nun an kritisch hinterfragt wird und nach Begründungen für Existierendes gesucht wird. 
Aus diesem Staunen entspringt in einem weiteren Schritt das Fragen. 
Der Mensch möchte erfahren, warum sich Dinge verhalten, wie sie es tun und gehen damit \glqq erste Schritte auf dem Weg zu neuen Erkenntnissen\grqq{}\cite[S.\,10]{BB10}. 
Interessant ist an dieser Stelle auch die Erkenntnis John Lockes, auf den sich Brüning bezieht, der feststellte, dass bereits Kinder Fragen nach dem Wesen der Welt stellen würden. 
Daher empfiehlt Locke den Eltern grundsätzlich, den Drang der Kinder nach Wissen zu erhalten und durch Antworten, welche neue Fragen hervorrufen, weiterzuführen.

Die Fragen die sich der Mensch stellt, möchte er in der Folge selbstverständlich auch beantwortet wissen. 
Daher liegt der nachfolgende Schritt im Nachdenken, welches sowohl im Nachdenken des Einzelnen, aber auch im Diskurs mit Anderen vollzogen werden kann. 

Das Nachdenken vollzieht sich nach Brüning in einem Dreischritt. 
Zunächst müssen im philosophischen Diskurs Begriffe ge-- bzw. erklärt werden, um sich dem zentralen Gegenstand der jeweiligen Frage nähern zu können. 
Tauscht man sich beispielshalber über die Frage aus, ob es ein Leben nach dem Tod gibt, so muss zunächst geklärt werden, was das Leben eines Lebewesens ausmacht und was der Tod bedeutet. 
Somit wird zunächst eine Standortbestimmung der Begrifflichkeiten vorgenommen. 
Anschließend werden Gründe dafür gesucht, die eine Position bejahen oder verneinen können und darüber geurteilt, welche Argumente am stichhaltigsten sind, um eine Sicht zu untermauern. 
Im sokratischen Gespräch können solche Begründungen dann kontrovers diskutiert und dabei bestätigt oder widerlegt werden. 
Dabei muss das Gespräch keine feste Lösung ergeben, sondern es können auch sich ausschließende Antworten Seite an Seite stehen bleiben\cite[S.\,11]{BB10}.

Das Philosophieren kann und möchte trotz rational begründeter Annahmen nicht den Anspruch erheben, vollständig beweisbare Antworten zu liefern. 
Das hat zur Folge, dass jede philosophische Antwort Zweifler finden wird, die Gegenargumente heranziehen werden. 
Daher gehört zum Philosophieren auch die Erkenntnis, zu wissen, dass solche Antworten stets nur einen Teil der Realität abbilden und daher nur provisorischen Charakter haben können. 
Dazu gehört auch, dass bei jeder Diskussion auch die gänzlich gegensätzliche Ansicht als denkbar anerkannt werden muss. 
Die Erkenntnis darüber und die Zweifel an der Antwort auf eine Frage bringen es mit sich, dass die Gedanken des Zweifels weitergedacht werden müssen. 
Dadurch wird der Prozess des Überlegens am Leben erhalten und das Weiterdenken, wie Brüning es nennt, kommt zustande. 
Den Abschluss des Weiterdenkens bringt dann die Einsicht, \glqq dass der Prozess des Nachdenkens über ein philosophisches Problem unabgeschlossen bleibt\grqq{}\cite[S.\,13]{BB10} und es eine \glqq richtige\grqq{} Antwort nicht geben kann.

Am Schluss des Philosophierens wird das eigene Denken infrage gestellt, angezweifelt und unter Umständen auch korrigiert. 
Wird die gleiche Thematik erneut diskutiert, ist es demnach möglich, dass eine zuvor als korrekt angesehene Antwort inzwischen überarbeitet oder gänzlich neu gefasst wurde. 
Daher sieht es Barbara Brüning als besonders zentral an, im Philosophieren mit Kindern in der Grundschule die Antworten der Kinder stets dahingehend zu hinterfragen, ob auch die andere Sichtweise für möglich erachtet wird.


\newpage
\subsection{Eigenschaften philosophischer Unterrichtsgespräche}

Allgemein versteht man unter Unterrichtsgesprächen Gespräche, die im schulischen Kontext -- vor allem im Unterricht zwischen Lehrern und Schülern -- geführt werden. 
Im Gegensatz zu Alltagsgesprächen zwischen Fremden oder der Unterhaltung im familiären Rahmen sind diese Gespräche \glqq institutionell geprägt\grqq{}\cite[S.\,28]{HB15}. 
Daher folgen sie auch schulisch geprägten Normen und die verwendete Sprache ist eine auf diesen Kontext zugeschnittene. 
Feilke fand dazu heraus, dass \glqq die Schulsprache keine Bildungssprache sei, sondern ein Instrument der Erziehung zur Bildungssprache; denn ein kompetenter Sprachgebrauch werde orientiert an den Vorstellungen der Schule und Schulfächer aufgebaut.\grqq{}\cite[S.\,117]{HF13}

De Boer verweist auf Lüders, der davon ausgeht, dass sich das Unterrichtsgespräch in ein wiederkehrendes Muster einordnen lässt\cite[S.\,28]{HB15}:
 Zunächst geht von der Lehrperson eine \textit{initiation}, beispielsweise eine Frage, aus, auf die Schüler meistens mit einem \textit{reply} bzw. einer Antwort reagieren. 
 Schließlich wird die gegebene Antwort in der \textit{evaluation} hinterfragt und der Prozess beginnt erneut. 
 Es lässt sich daher sagen, dass in einem klassischen Unterrichtsgespräch immer eine Validierung vorgenommen wird, die sich bei Lüders in der \textit{evaluation} finden lässt. 
 Für das philosophische Unterrichtsgespräch ist jedoch zu sagen, dass es vor allem gilt, \glqq dem gemeinsamen Nachdenken Raum zu geben, unterschiedliche Denkwege auszutauschen, Fragen zu entwickeln und ein offenes Gespräch ohne Validierungszwänge zuzulassen\grqq{}\cite[S.\,159]{HD15}.
 
Das Unterrichtsgespräch befindet sich darüber hinaus in zwei zentralen Spannungsfeldern. 
Zum einen muss die Lehrkraft eine Organisation vornehmen, die für die Schüler einerseits durch den gewählten Zeitpunkt und andererseits durch Transparenz im Umgang Klarheit vermittelt, zum anderen führt eine detaillierte Planung zu repetierenden Abläufen\cite[S.\,31]{HB15}. 
Das zweite Spannungsfeld sieht die Lehrperson vor dem Problem, das Gespräch fachlich fundiert zu führen und darüber hinaus auch flexibel zu gestalten, um auf situationsbedingte Veränderungen reagieren zu können. 
Daher wird von den Lehrerinnen und Lehrern eine zunehmend selbstreflektierte Moderatorentätigkeit verlangt.

Ziel des philosophischen Gesprächs sollte es sein, einen Dialog zwischen den Schülerinnen und Schülern herzustellen. 
Martin Buber entwickelte dazu drei Merkmale, die für ihn ein dialogisches Gespräch ausmachen. 
Zum einen müssten sich die Gesprächspartner einander zuwenden, d.h. die Existenz des Anderen akzeptieren und nicht seine Gedanken und Handlungen widerspruchslos  übernehmen\cite[S.\,106]{MB15}.
Weiter erachtet es Buber für wichtig, dass die Gesprächspartner eigene Gedanken einbringen und diese unverkürzt wie uneingeschränkt zur Sprache bringen. 
Zuletzt sollte in einem Dialog auf die Selbstdarstellung verzichtet werden, da dann nicht die eigentlichen Positionen mitgeteilt werden, sondern die authentische Komponente des Gesprächs verloren geht.

Für philosophische Unterrichtsgespräche ergeben sich in Bezug auf die dialogische Ausgestaltung zusätzliche Merkmale, die nicht vernachlässigt werden dürfen. 
Grundlegend ist zu philosophischen Unterrichtsgesprächen zu sagen, dass es sich bei ihnen um sokratische Gespräche handelt. 
Diese Gesprächsform zeichnet sich vor allem dadurch aus, \glqq dass in ihm ein philosophischer Gegenstand diskutiert wird, bspw. Glück, Gerechtigkeit oder Freundschaft.\grqq{}\cite[S.\,27]{BB10}

Ein sokratisches Gespräch, wie es auch mit Schülern in der Grundschule geführt werden kann, gliedert sich in drei Phasen. 

In der ersten Phase, welche Brüning Vorbereitungsphase nennt, geht es vor allem darum, die zentrale philosophische Fragestellung zu klären. 
Dabei kann die Fragestellung sowohl die eines Kindes als auch die der Lehrperson sein. 
Daneben müssen Regeln für den Ablauf des Gespräches aufgestellt werden, welche im gesamten Gesprächsverlauf eingehalten werden müssen. 
Dazu können das aktive Zuhören, das Begründen der eigenen Meinung gehören aber auch, andere ausreden zu lassen\cite[S.\,31]{BB10}.
In der zweiten Phase kommt es zum eigentlichen philosophischen Gespräch. 
Das zuvor ausgewählte Thema wird mit den Schülerinnen und Schülern diskutiert und die Lehrerin bzw. der Lehrer übernehmen lediglich die Aufgabe des Moderators. 
Brüning weist ausdrücklich darauf hin, dass auch ein Schüler diese Tätigkeit wahrnehmen kann, wenn die Schülerinnen und Schüler bereits mit dem gemeinsamen Philosophieren vertraut sind. 
Während des Gespräches sollen auch unbekannte Begriffe geklärt und über die Thematik argumentativ gestritten werden. 
Gegen Ende des Gespräches soll sich dann ein gemeinsames Ergebnis herauskristallisieren. 
Dieses kann entweder in einer alleinigen Lösung oder aber in verschiedenen bestehen. 

An diese Phase schließt sich in einem dritten Schritt das Metagespräch an, in  dem das gemeinsame Nachdenken unterbrochen wird oder ein Anschluss an die Diskussion gefunden wird. 
Unter anderem beschreibt Brüning, dass in diesem Teil des sokratischen Gespräches die Einhaltung der Regeln erneut diskutiert und auf etwaige Verletzungen hingewiesen werden kann. 
Zudem können die Ergebnisse des Gespräches an dieser Stelle erneut zusammengefasst werden.

Während für ein Unterrichtsgespräch generell vor allem der gedankliche Lernprozess des Kollektivs im Vordergrund steht, ist dies allein für das philosophische Gespräch nicht mehr ausreichend. 
Vielmehr muss den Schülerinnen und Schülern näher gebracht werden, die Sichtweisen und Blickwinkel anderer zu verstehen und akzeptieren zu können\cite[S.\,234]{HDB15}. 
Auch muss für philosophische Unterrichtsgespräche das Rollenbild zwischen Lehrperson und den Schülerinnen und Schülern aufgebrochen werden. 
Bleibt es nämlich, wie bei dieser Gesprächsform üblich, dabei, dass ausschließlich die Lehrerin bzw. der Lehrer die Fragen stellt, werden die Schüler nicht in der Art und Weise für die Thematik zu begeistern sein, wie in dem Falle, in dem sie selbst ihre Fragen in den Raum stellen dürfen. 
Daher würde bei einer klassischen Anordnung im philosophischen Gespräch die Chance, Lernprozesse bei den Schülerinnen und Schülern zu initiieren, verpuffen.

Für de Boer wird das Philosophieren mit Kindern dann zum Unterrichtsprinzip, wenn Lehrerinnen bzw. Lehrer und Erwachsene im Allgemeinen dazu beitragen, dass das kategoriale Denken der Kinder in weitergehende und multidimensionale Denkweisen übertragen werden.


\newpage
\subsection{Konzeptionen des Philosophierens in der Grundschule: Warum sollte in der Grundschule philosophiert werden?}

In den letzten Jahren hat sich der pädagogische Blick auf das Philosophieren an Schulen kontinuierlich gewandelt. 
So wurden teilweise von Philosophen selbst Gesprächsanlässe und Konzeptionen geschaffen, die die Schülerinnen und Schüler als selbständige Menschen sehen und ihnen die Möglichkeit geben wollen, sich die Welt autonom zu erschließen\cite[S.\,617]{AN13}. 
In diesem Kontext wurde auch die Frage diskutiert, ob das Philosophieren mit Kindern als eigenes Unterrichtsfach angeboten oder in bereits vorhandenen Fachunterricht eingebettet werden soll. 

Professor Dr. Andreas Nießeler, der an der Universität Würzburg in den Bereichen Grundschulpädagogik und Theorie des Sachunterrichts, Philosophieren mit Kindern, Kulturanthropologische Theorie der Bildung und des Lernens und Bildungsphilosophie und pädagogische Anthropologie forscht und lehrt, stellt klar, dass das Ziel des Philosophierens mit Kindern aus pädagogischer Sicht vor allem darin besteht, \glqq junge Menschen mit ihren Fragen ernst zu nehmen\grqq{}\cite[S.\,617]{AN13}.
Hinzu kommt, dass die Schülerinnen und Schüler in die Lage versetzt werden sollen, ihre Deutung von der Welt ausdrücken zu können und sich eigenständig in ihrem Denken zu orientieren. 

Nießeler erwähnt dazu Herman Nohl, der in seiner 1922 veröffentlichten Schrift \glqq Philosophie in der Schule\grqq{} feststellt, dass der Philosophieunterricht \glqq anstatt als Einzelfach ein trauriges Dasein zu fristen die Gesamtheit der Fächer zu tragen habe und dessen Aufgabe eine Erziehung zur \glqq Vergeistigung unseres Seins, Denkens und Tuns durch Rückbeziehung alles Einzelnen auf Sinn und Grund, eine platonische Anleitung zur Erhebung in das höhere geistige Leben der Idee\grqq{} sei.\grqq{}\cite[S.\,618]{AN13}
Hier zeigt sich die Verortung der Philosophie in der Schule, die keinen eigenen Platz in der Fächerlandschaft der Grundschule zugewiesen bekommt und deren Ziel es ist, ein autonomes Denken unter den Schülerinnen und Schülern zu etablieren. 
Darüber hinaus spricht Andreas Nießeler die Ideen des Philosophen Leonard Nelson an, der als zentrale Unterrichtsmethode die sokratische Maieutik betrachtete. 

Diese Methode, auch Hebammenprinzip, sah nach Sokrates vor, durch gezielte Fragen das vorhandene Wissen des Befragten zutage zu fördern. 
Mithilfe dieser Maieutik wollte Nelson den Schülern die Unvollständigkeit des Wissens deutlich machen. 
Hauptsächlich ging es ihm darum, dass die Schülerinnen und Schüler nicht lernen sollten, was Philosophie ist, sondern sie sollten das Philosophieren als solches lernen und so selbst zu Philosophen werden\cite[S.\,618]{AN13}. 

Ferner bezieht sich Nießeler darauf, dass die ersten, wissenschaftlichen Untersuchungen vor allem in den USA stattfanden. 
Matthew Lipman, der 1974 das \glqq Institute for the Advancement of Philosophy for Children\grqq gründete, war zusammen mit Gareth B. Matthews einer der ersten, die ein Philosophiekonzept für die Durchführung mit Kindern entwickelten. 
Lipman war der überzeugung, dass Kinder bereits früh in der Lage sind, logisch zu denken und sah das logische Denken daher als eine Basisqualifikation an\cite[S.\,619]{AN13}. 
Matthews hingegen untersuchte vor allem philosophische Gespräche zu verschiedenen Themen wie \glqq Glück\grqq{}, \glqq Wünsche\grqq{} oder \glqq Gerechtigkeit\grqq{}. 
Er konnte durch Erinnerungsprotokolle ermitteln, dass in diesen Gesprächen bei den Kindern eine Entwicklung stattfindet und philosophische Gespräche es erlauben, die philosophischen Vorstellungen von Kindern in Relation zu bedeutenden Fragen untersuchen zu können.

Barbara Weber stellt in ihrem Beitrag \glqq Philosophieren mit Kindern: Wieso, Weshalb, Wozu?\grqq{} fest, dass das Philosophieren mit Kindern einerseits aus pädagogischer andererseits aus philosophischer Sicht in der Kritik stehe. 
Es werde argumentiert, man könne die Schülerinnen und Schüler überfordern, wenn man mit ihnen philosophische Themen erörtert, andererseits ist die Bedeutung der Philosophie in sich eine der größten Fragen, der sich die Philosophie entgegensieht. 
Weber hebt hervor, dass es Skeptiker gebe, die den Kindern nicht die Fähigkeiten zusprechen würden, sich ein umfassendes Wissen über philosophische Themen anzueignen, aber auch Befürworter, die in der Philosophie eine umfassende pädagogische Lösung sehen würden, mit deren Hilfe jegliche sozialen und inhaltlichen Kompetenzen erworben werden könnten\cite[S.\,623]{BW13}.
Jedoch sollten überhöhte Erwartungen an das Philosophieren mit Kindern vermieden werden. 

Entscheidend sei aus ihrer Sicht auch, welche Themen für das Philosophieren gewählt würden und ob diese philosophisch gesichert seien, was Ekkehard Martens 1999 bezweifelt habe\cite[S.\,624]{BW13}. 
Das Philosophieren in Unterrichtsgesprächen mit Kindern kann für Barbara Weber von großem Nutzen für die pädagogische Forschung sein. 
Nutzt man es als qualitative Methode, so könne man aus diesen Gesprächen Rückschlüsse daraus ziehen, wie Schülerinnen und Schüler über philosophische Themen denken und was sie dazu sagen.

Weber erstellt ein Schema, nach dem sich philosophische Gespräche grob gliedern lassen. 
So geht jedem philosophischen Gespräch ein Staunen und Fragen voraus, in dem die Schülerinnen und Schüler sich über eine Fragestellung klar werden und diese hinterfragen. 
Dem schließt sich der gemeinsame Dialog mit Anderen - das Denken-Sprechen - an, der die Positionen anderer in den Blick nimmt und gegebenenfalls auch die eigenen verändern kann. 
In einem weiteren Schritt kommt es zum Werten-Handeln, das zu einer Prüfung oder sogar Veränderung der eigenen Lebenswelt führen kann. 
Eigenes Handeln und individuelle Werte werden geprüft und dies führt zu weiterem Staunen und Fragen\cite[S.\,626]{BW13}.
Barbara Weber betrachtet das gesammelte Wissen der Philosophie nicht als abgeschlossen, sondern verweist auf Pierre Hadot, der in seinem Werk \glqq Wege zur Weisheit\grqq{} darauf hinwies, dass \glqq die Philosophie ein historisches Phänomen ist, das zu einer Zeit begonnen und sich bis heute weiterentwickelt hat.\grqq{}
Diese Sicht auf die Philosophie hat zur Folge, dass diese nicht mehr bloß Modelle behandelt, sondern sich auch mit philosophischen Verhaltens- und Lebensweisen beschäftigen müsse.

Kerstin Michalik verortet das Philosophieren in einem eigenen Unterrichtsprinzip, dass die Schülerinnen und Schüler in ihren persönlichen Lernprozessen unterstützen und ihnen die Möglichkeit bieten soll, ein differenziertes Weltbild zu entwickeln\cite[S.\,635]{KM13}. 
Weiter schließt Michalik in ihrem Beitrag an die Ausführungen von Andreas Nießeler an, der die Frage nach der Philosophie als eigenständigem Unterrichtsfach aufwarf. 
Sie sieht die Philosophie als Unterrichtsprinzip in den Fachunterricht eingebettet durch philosophische Gespräche mit den Kindern. 
Es geht dabei nicht um die Einführung neuer Inhalte, wie es in den anderen Fächern des Grundschulunterrichts der Fall ist, sondern um überfachliche Zugänge zu Fragen, mit denen sich die Schülerinnen und Schüler auseinander setzen. 
Dadurch eröffnet das Philosophieren mit Kindern vollkommen neue Möglichkeiten im interaktiven Lernen. 
Die Schüler werden durch die Gestaltung des Unterrichts und die Gelegenheit, eigene Fragen in den Unterricht einzubringen, an ihrem Lernstand abgeholt und es wird eine besondere Bedeutsamkeit für sie geschaffen.

Michalik stellt klar, dass Kinderfragen im alltäglichen Unterrichtsgeschehen keine besonders große Wichtigkeit zugeschrieben wird. 
Dem kann das Philosophieren als Unterrichtsprinzip entgegenwirken, indem es diese Fragen zum Ausgangspunkt des Unterrichts macht. 
Sie erörtert, dass der klassische Unterricht vor allem durch Lehrerfragen dominiert wird, die darauf ausgerichtet sind, vorhandenes oder zu erwerbenes Wissen abzufragen und zu festigen. 
Sie haben zudem einen \glqq fixierenden und einschränkenden Charakter\grqq{}\cite[S.\,635]{KM13}. 
Sie stellt jedoch den Stellenwert von Schülerfragen in Bezug auf den Lernprozess der Schülerinnen und Schüler heraus und knüpft an die Positionen des US-amerikanischen Medienwissenschaftlers Neil Postman an, der eine Perspektivenumkehr forderte, infolge derer das Stellen von Fragen als \glqq Kunst und Wissenschaft beizubringen\grqq{}\cite[S.\,636]{KM13} sei. 
Das Stellen von Fragen hängt zusätzlich eng mit dem Bildungsbegriff zusammen. 
So sagt der Philosoph Peter Bieri, dass man sich bilde, um etwas zu werden. 
Dabei gehe die Bildung immer von einer Neugierde und damit von eigenen Fragen aus.
Michalik weist darauf hin, dass es das Ziel des Philosophierens mit Kindern sein müsse, \glqq das Fragenstellen zu ermutigen und zu fördern und den Kindern zu zeigen, dass das gemeinsame Nachdenken über schwierige Fragen interessant und lohnenswert ist.\grqq{}\cite[S.\,637]{KM13}
Sie beschreibt, dass das klassische Unterrichtsgespräch vor allem von einer Lehrerrolle dominiert wird, die über umfassendes Wissen verfügt und Fragen stellt, die das Gespräch zu vorgefertigten Antworten lenken.

Daher müsse nach den Erkenntnissen Rainer Kokemohrs eine Modalisierung und Validierung stattfinden, um das Gespräch in Gang zu halten und offen zu gestalten.
Unter Modalisierung versteht Kokemohr \glqq die Öffnung des Gespräches hin zu konkurrierenden Lesarten.\grqq{}\cite[S.\,637]{KM13}
Das bedeutet, dass den Interpretationen und Denkweisen mehr Raum zugestanden wird.
Die Validierung hingegen bezeichnet die gegensätzliche Vorgehensweise, durch die das Ende eines Gespräches herbeigeführt wird, indem dieser Spielraum immer weiter eingeengt wird.
Die Schule stehe unter einem Validierungszwang, der dazu führe, dass einige mögliche Lern- und Verstehensprozesse verloren gehen würden.
 
Das Philosophieren mit Kindern kann nach Weber dazu beitragen, den Blick auf das Wissen, dass die Schule vermittelt, zu erneuern.
Werde durch die Schule bisher überwiegend Wissen vermittelt, dass sich durch begründbare Tatsachen charakterisiere, so könne das Philosophieren den Kindern zeigen, dass nicht alles in der Welt \glqq restlos vermessen, geordnet und erklärt ist, sondern auch Raum für Staunen, Nachdenklichkeit, Weiterfragen, Forschen bietet.\grqq{}\cite[S.\,639]{KM13}
Die Schülerinnen und Schüler sollten daher auch lernen, dass das Wissen des Menschen keineswegs allumfassend oder abgeschlossen ist, sondern, dass es immer Fragen gebe, die nicht abschließend beantwortet werden könnten.
Daher sei es wichtig, den Kinder diese Erfahrung der Begrenztheit von Wissenschaft zu ermöglichen und so \glqq Grundlagen für ein reflektiertes und differenziertes Welt- und Wissenschaftsbild zu legen.\grqq{}\cite[S.\,640]{KM13}
Für Michalik kann das Philosophieren dazu dienen, dem Lernen der Kinder einen Sinn zu geben und es vom Stigma des bloßen Bewältigens von Arbeitsaufträgen zu lösen.
Sie betont, dass Wissenschaft und Philosophie keine Gegensätze seien, sondern die Auseinandersetzung mit philosophischen Fragen sei ein zentraler Bestandteil für einen Unterricht, der sich an der Wissenschaft anlehnt.
  
Abseits der pädagogischen Begründungen, auf die vorhin eingegangen wurde, verweist Michalik in ihrem Artikel zudem auf verschiedene Studien, die im anglo-amerikanischen Raum entstanden sind und den positiven Einfluss auf die kognitive und sprachliche Entwicklung der Schülerinnen und Schüler belegen. 
Sie bezieht sich dabei auf die Studien, die, angelehnt an den kinderphilosophischen Ansatz von Lipman, \glqq kritisches, logisches, kreatives Denken (thinking skills) und Argumentations- und Gesprächsfähigkeiten\grqq{}\cite[S.\,643]{KM13} als relevante Aspekte ansehen. 
Demnach wurden Entwicklungen positiver Natur im Sozial- und Gruppenverhalten sowie hinsichtlich des Selbstwertgefühls, Selbstvertrauens und Selbstbewusstseins beobachtet.
Außerdem wurde in kanadischen Studien festgestellt, dass der gegenseitige Austausch unter Gleichaltrigen das Gesprächsbewusstsein und die Gesprächsfähigkeit fördern könne. 
Ergänzend werde das Denken der Schülerinnen und Schüler durch philosophische Gespräche \glqq zunehmend komplexer und multimodal im Sinne logischen, kreativen, verantwortlichen und meta-kognitiven Denkens.\grqq{}\cite[S.\,643]{KM13}

Mecklenburg-Vorpommern ist bisher das einzige Bundesland der Bundesrepublik Deutschland, in dem das Philosophieren mit Kindern ein eigenes Schulfach darstellt.
In diesem Bundesland haben die Fächer Philosophie und Philosophieren mit Kindern den Status des Ersatzfaches, das heißt, dass sie nur dann angeboten werden können, wenn auch gleichzeitig ein Angebot auf Religionsunterricht zur Verfügung steht.
Dieser Umstand kommt zustande, da der Religionsunterricht als einziges Fach im Fächerspektrum der Primarstufe eine besondere Stellung als ordentliches Unterrichtsfach einnimmt.
Das Ziel des Philosophierens ist es nach Silke Pfeiffer in Mecklenburg-Vorpommern, \glqq Kinder und Jugendliche bei ihrer Suche nach Orientierung und Sinn zu begleiten und ihnen Denk- und Verstehensangebote in existentiellen Fragen zu unterbreiten.\grqq{}\cite[S.\,652]{SP13}
Grundlage für das Philosophieverständnis, das hierfür zugrunde gelegt wird, ist es, die Interessen, intellektuellen Fähigkeiten etc. zu berücksichtigen.
Fachlich bezieht sich der Unterricht vor allem auf die Themenbereiche \glqq Familie\grqq{}, \glqq Natur\grqq{}, \glqq Konfliktbewältigung bzw. Gut und Böse\grqq{} sowie auf \glqq Fernseh- und Computerwelten\grqq{}.
Aus einem Interview mit einer Lehrenden und zwei Schülern heraus entwickelt Silke Pfeiffer, dass der Unterricht des Philosophierens mit Kindern vor allem schüler- und problemorientiert sei.
Darüber hinaus zeichne ihn ein besonderes Verhältnis zwischen Lehrperson und den Schülerinnen und Schülern aus.

Besonders positiv sei aus der Sicht der Teilnehmer, dass die inhaltliche Breite in einer neuen Dimension gegeben sei.
Jedoch gibt es auch Nachteile, die das Philosophieren als eigenes Unterrichtsfach mit sich bringen würde.
So sei es besonders schwierig, ein befriedigendes Zeitmanagement vorzunehmen, der Austausch unter den Fachkräften fände in einem unzureichenden Rahmen statt und die Benotung gestalte sich nicht zuletzt durch die Beschaffenheit des Faches komplex.

Es lässt sich zusammenfassend sagen, dass es verschiedene Faktoren gibt, die eine Etablierung des Philosophierens im Schulunterricht der Grundschule rechtfertigen. 
Das Philosophieren bietet den Lehrerinnen und Lehrern aber auch den Schülerinnen und Schülern abseits des gewöhnlichen Fachunterrichts die Möglichkeit, über Themen ins Gespräch zu kommen, die die Schüler interessiert und die ihnen persönlich wichtig sind. 
So kann das Philosophieren in der Grundschule das Selbstbewusstsein der Kinder, ihre Fragen einzubringen, aber auch das Selbstwertgefühl steigern. 
Dies kann jedoch nur gelingen, wenn die Lehrkraft bereit ist, die eigenen Gesprächsbeiträge zu minimieren und die Schülerinnen und Schüler zu Wort kommen zu lassen.
Das Philosophieren mit Kindern kann durch die Breite der Themen, mit denen es sich auseinander setzt, dazu beitragen, den Blick der Schüler auf das Wissen dahingehend zu verändern, dass Wissen nicht mehr bloß als rezipierbar wahrgenommen wird, sondern, dass es eine individuelle Bedeutung für jeden einzelnen Schüler entwickelt.
Das Philosophieren mit Kindern kann auch dazu beitragen, den Schülern den Sinn des Lernens zu erschließen, indem die Schülerfragen zum Ausgangspunkt des Unterrichts gemacht werden.
Gleichzeitig dürfen jedoch nicht zu hohe Erwartungen an dieses Unterrichtsprinzip gestellt werden, die dieses nicht in der Lage sein kann zu erfüllen. 
Es zeigt sich jedoch auch, dass das Philosophieren als eigenständiges Unterrichtsfach Vor- und Nachteile auf sich vereint.
So sei vor allem die Problematik der Notenfindung von Bedeutung, die bei einer Einbindung des Philosophierens in den Sachunterricht nicht in der Häufigkeit auftreten wird.
Schlussendlich können beim Philosophieren mit Kindern signifikante Verbesserungen in Sozial- und Fachkompetenzerwerb der Schülerinnen und Schüler nachgewiesen werden, wodurch es für den Lernprozess der Kinder einen elementaren Baustein darstellen kann.

\newpage
\subsection{Verortung des Philosophierens mit Kindern im Teilrahmenplan Sachunterricht Grundschule des Landes Rheinland-Pfalz}

Neben der fachlich-wissenschaftlichen Ebene ist in Bezug auf die Philosophie im Sachunterricht auch der pädagogische Blickwinkel von besonderer Bedeutung, um den Nutzen des Philosophierens mit Kindern in der Grundschule einordnen zu können. 
Dazu soll nun der Teilrahmenplan Grundschule für den Sachunterricht, der im Mai 2006 vom Ministerium für Bildung, Frauen und Jugend des Landes Rheinland-Pfalz herausgegeben wurde, in den Blick genommen werden. 

Ziel der Teilrahmenpläne, die für alle an Grundschulen in Rheinland-Pfalz angebotenen Fächer (Deutsch, Mathematik, Ethik, katholische/evangelische Religion, Sport, Kunst, Musik und Fremdsprachen) erarbeitet wurden, ist es, die Qualität und die Optimierung des Unterrichts voranzutreiben. 
Zudem bietet er Lehrerinnen und Lehrern die Möglichkeit, sich in Bezug auf die Kompetenzen, welche die Schülerinnen und Schüler am Ende der Grundschulzeit erworben haben sollten, in den jeweiligen Fächern zu informieren und zu orientieren. 
Gleichzeitig zeigt er den Lehrenden auch auf, welche fachlichen und didaktischen Voraussetzungen sie mitbringen müssen.

Grundsätzlich sieht der Teilrahmenplan die Entwicklung von Sprachfähigkeiten als untrennbar verknüpft mit dem Erwerb von Erfahrungswissen, da die Auseinandersetzung mit Umwelt Sprache benötige\cite[S.\,6]{MBFJ06}.
Im Sammeln von Erfahrungen wird dabei zwischen Primärerfahrungen, die das Kind in direkter Konfrontation macht und Sekundärerfahrungen, die durch Sprache zum Ausdruck gebracht werden, differenziert. 

Diese Primär- und Sekundärerfahrungen finden sich ebenfalls im Philosophieren, in dessen Rahmen die Schülerinnen und Schüler ihre Primärerfahrungen zu philosophischen Themen kundtun und die Mitschüler auf diese Weise Sekundärerfahrungen machen. 
Zusätzlich lernen die Kinder, ihre persönlichen Erfahrungen zu prüfen, mit denen der Mitschüler zu vergleichen, zu verifizieren und gegebenenfalls auch zu falsifizieren. 
Dieser Aspekt des kritischen Umgangs mit der eigenen Erfahrung wird auch vom Teilrahmenplan als zentraler Bestandteil des Spracherwerbs angesehen.

Der Teilrahmenplan weist ausdrücklich darauf hin, dass die Schülerinnen und Schüler Erfahrungen innerhalb und außerhalb der Sphären der Schule machen. 
Dies wird unter dem Schlagwort der \glqq Handlungskompetenz\grqq{} verortet. 
Das Philosophieren in der Klassengemeinschaft bietet in diesem Rahmen einen besonderen Raum für Fragen, die sich außerhalb der gewöhnlichen inhaltlichen Linien des Fachunterrichtes der Grundschule befinden und eröffnet den Kindern so völlig neue Zugänge. 
Der Kompetenzerwerb wird dabei nochmal in vier verschiedene Arten gegliedert: personal, sozial, methodisch und fachlich. 

Die Kinder können durch das Philosophieren auf allen Ebenen profitieren. 
Sie erwerben das Selbstbewusstsein, ihre Gedanken zu philosophischen Inhalten zu kommunizieren (personal), sie partizipieren an den Ansichten der Anderen (sozial), sie lernen, ihre Erfahrungen sprachlich auszudrücken (methodisch) und sie setzen sich mit für sie bedeutsamen philosophischen Fragestellungen auf ihrem kognitiven Niveau auseinander (fachlich).

Zusätzlich zu diesen allgemeinen Aspekten formuliert der Teilrahmenplan ein Leistungsprofil für den Sachunterricht, das angeben soll, welche Leistungen die Kinder im Rahmen ihrer Fähigkeiten am Ende der Grundschulzeit erbringen können. 
Für das Führen von philosophischen Gesprächen im Sachunterricht ist dabei vor allem von Bedeutung, dass die Schülerinnen und Schüler \glqq angemessene Darstellungsformen\grqq{}\cite[S.\,8]{MBFJ06} beherrschen.
Zu diesen zählen insbesondere die Gesprächsformen, oder die verschiedenen Darstellungen sprachlicher Beiträge wie z.B. Berichten, Darstellen und Referieren.

Darüber hinaus ist es wichtig, dass sich Kinder \glqq Fragen und Problemen aus ihren natürlichen, sozialen, kulturellen, technischen und wirtschaftlichen Erfahrungsbereichen mit Neugier und Selbstvertrauen\grqq{}\cite[S.\,8]{MBFJ06} zuwenden. 
Ebenfalls für philosophische Gespräche relevante Kompetenzen liegen im \glqq Auffinden, Erklären, Darstellen und Begründen von Strategien zur Lösung von Sachfragen (Bilden von Hypothesen, überprüfen von Annahmen, Experimentieren, Schlüsse ziehen, übertragen von Ergebnissen auf analoge Sachverhalte).\grqq{}\cite[S.\,8]{MBFJ06}
Denn in diesen Gesprächen steht der interaktive Austausch im Fokus des Sachunterrichts, so dass die Schülerinnen und Schüler diese Kompetenzen in besonderer Weise schulen.

Dem Sachunterricht kommt nach Ansicht des Teilrahmenplans eine ganz besondere Rolle im Fächerspektrum der Grundschule zu. 
Er diene \glqq nicht nur dem Erwerb von Kenntnissen, Fähigkeiten und Fertigkeiten im Umgang mit der sozialen und natürlichen Umwelt, sondern schließt die Förderung der sprachlichen, ästhetischen und interkulturellen Bildung, der wertbewussten Orientierung und des Verstehens ein.\grqq{}\cite[S.\,9]{MBFJ06} 
Diesem überfachlichen Anspruch kann das Philosophieren mit Kindern im Unterricht der Grundschule Rechnung tragen, indem, wie vorhin bereits angesprochen, auch außerschulische Dimensionen, die in den Fragen der Schülerinnen und Schüler zum Ausdruck kommen, Bestandteil der Auseinandersetzung werden.

Eine Besonderheit des Sachunterrichts stellt die Tatsache dar, dass dieser sich aus verschiedenen Themengebieten zusammensetzt, die vom Teilrahmenplan fünf Erfahrungsbereichen zugeordnet werden: Natur, Gesellschaft, Technik, Raum und Zeit. 
In diesem Zusammenhang liegt nahe, dass sich das Philosophieren mit Kindern in der Grundschule vor allem den Erfahrungsbereichen \glqq Gesellschaft\grqq{} und \glqq Zeit\grqq{} des Teilrahmenplans zuordnen lässt. 

Im Rahmen des Erfahrungsbereiches \glqq Gesellschaft\grqq{} lässt sich vor allem konstatieren, dass das Philosophieren dazu beitragen kann, dass sich die Kinder mit der Andersartigkeit der Mitmenschen auseinandersetzen in Bezug auf deren Bedürfnisse, Gefühle und Sehnsüchte. 
Diese Schlüsselkompetenz nennt der Teilrahmenplan unter Punkt 1, was ebenfalls dessen Bedeutung unterstreicht. 

Darüber hinaus ist die Antizipationsfähigkeit der Schüler, die sie im Rahmen des Sachunterrichts erwerben sollen, unabdingbar in ihrem späteren Leben, damit sie in die Lage versetzt werden, das Handeln, die Ansichten aber auch die Empfindungen Anderer besser verstehen zu können. 
Beim Philosophieren kann das beispielsweise bedeuten, dass die Kinder darüber nachdenken, was für sie Glück bedeutet, was für ihre Mitschüler Glück ist und warum sie Glück für sich so definieren, wie sie es tun. 

Auch die Perspektive der Zeit kann zur Einordnung des Philosophierens in den Kontext des Sachunterrichts herangezogen werden. 
Für die Schüler ist es wichtig, zu erfahren, dass Zeit etwas ist, was dem menschlichen Leben Struktur gibt und es möglich macht, in Epochen und Zeiträumen die Entwicklung menschlichen Denkens über Jahrtausende hinweg zu untersuchen und zu durchdringen. 
Dabei sieht der Teilrahmenplan unter anderem die \glqq Beurteilung von Entscheidungen und Handlungen\grqq{}\cite[S.\,15]{MBFJ06} unter Beachtung der vorherrschenden Lebensumstände als wesentlich an, um zum Beispiel die Ansichten von Philosophen im Vergleich zur heutigen Denkweise richtig einordnen zu können. 
Auf diese Weise bekommen die Schülerinnen und Schüler einen umfassenderen Blick auf die philosophischen Hintergründe.

Letztlich sieht der Teil"-rah"-men"-plan Grund"-schu"-le den Sach"-unter"-richt in der Pflicht, 
Anreize zu schaffen, die die von Natur aus vorhandene Neugier der Kinder wecken und sie in ihrem eigenständigen Sammeln von Erfahrungen begleiten sollen. 
Dazu sollten Lernumgebungen geschaffen werden, die diese Voraussetzungen ermöglichen. 

Als eine solche, gewinnbringende Lernumgebung kann das Philosophieren mit Kindern in der Grundschule verstanden werden, das sich neben den bereits behandelten, positiven Entwicklungen des Kompetenzerwerbs und der fachlichen Breite
auch dadurch auszeichnen kann, dass es die sonst üblichen Rollen und Rituale des Fachunterrichts der Primarstufe für die Kinder nutzbringend verändern kann.

\newpage
\subsection{Bezug zum Perspektivrahmen der Gesellschaft für die Didaktik des Sachunterrichts  e.V. (GDSU)}

Der Perspektivrahmen Sachunterricht, der 2013 in einer überarbeiteten Fassung erschienen ist, sieht seine Aufgabe darin, die Didaktik des Sachunterrichts zu fördern und Disziplin des wissenschaftlichen Diskurses im Hinblick auf Forschung und Lehre zu etablieren. 
Die Gesellschaft selbst sieht sich als Bund von Angehörigen von Hochschulen, Lehrerfortbildungen, Lehrerweiterbildungen und Schulen. 
Ziel des Perspektivrahmens soll es sein, Lehrerinnen und Lehrer darin zu unterstützen, kompetenzorientierten Sachunterricht planen, ausarbeiten, untersuchen und reflektieren zu können.

Dazu liefert er ein Kompetenzmodell, dass den Lehrenden einen Überblick über inhaltliche Perspektiven, perspektivenübergreifende sowie perspektivenbezogene Handlungsweisen geben soll. 
Diese Handlungsweisen werden im Folgenden weiter aufgeschlüsselt und die perspektivenbezogenen Handlungsweisen den fünf Inhaltsperspektiven, die den Sachunterricht in einen sozialwissenschaftlichen, naturwissenschaftlichen, geographischen, historischen und technischen Bereich gliedern, zugewiesen\cite[S.\,5f]{GDS13}.
 Zudem werden konkrete Unterrichtsbeispiele vorgestellt, die jedoch in der hier vorgenommenen Einordnung des Philosophierens mit Kindern in der Grundschule keine Rolle spielen.
 
Der Perspektivrahmen Sachunterricht sieht im Fach Sachunterricht einen \glqq zentralen Beitrag zu grundlegender Bildung\grqq{}\cite[S.\,9]{GDS13}.
Dabei wird der Begriff Bildung als charakteristisch für das Wesen des Menschen aufgefasst und daraus der Anspruch für den Sachunterricht abgeleitet, die Schülerinnen und Schüler zu einem verantwortungsbewussten Verhalten in ihrer Umgebung zu erziehen. 
Die vornehmliche Aufgabe sieht er darin, die Schülerinnen und Schüler zu einem Verständnis ihres natürlichen, kulturellen, sozialen und technischen Umfeldes zu bringen, in dessen Erwerbsprozess auch die Vorerfahrungen der Kinder aus Familie, Kindertagesstätten und anderen sozialen Einrichtungen einfließen sollen und diese ein darauf aufbauendes Lernen begünstigen sollen. 
In diesem Zusammenhang sieht der Perspektivrahmen Sachunterricht eine zweifache Herausforderung in der Anschlussfähigkeit des Sachunterrichts, der einerseits die Prämissen der Wissensstände der Kinder sowie deren Interessen und Fragen im Blick haben muss. 
Andererseits darf auch die Anschlussfähigkeit an das fachliche Wissen nicht ausgeklammert werden. 

Analog zum Teilrahmenplan des Landes Rheinland-Pfalz sieht auch der Perspektivrahmen Sachunterricht die Sprache eng verknüpft mit dem Sachunterricht selbst. 
So sei die Sprache im Kontext des Sachunterrichts einerseits ein Werkzeug des Austauschs und der Konstruktion von Inhalten, andererseits entwickle sich die Sprache aber auch im Laufe der Grundschulzeit mithilfe des Sachunterrichts, da es auch eine der Aufgaben der Lehrerinnen und Lehrer sei, die Alltagssprache der Kinder durch eigene Impulse und die Struktur des Sachunterrichts zu einer Fachsprache zu führen. 
Durch den Sachunterricht werde ein wesentlicher \glqq Beitrag zur sprachlichen Bildung von Schülerinnen und Schülern, wenn (häufig sinnlich wahrnehmbare) \glqq Sachen\grqq (wie Gegenstände oder auch Prozesse) zu benennen sind\grqq{}\cite[S.\,11]{GDS13} geleistet.

Wie bereits erwähnt legt der Perspektivrahmen Sachunterricht seinen Überlegungen ein Kompetenzmodell zugrunde, das zwischen perspektivenbezogenen und perspektivenübergreifenden Handlungsweisen in Bezug auf die inhaltlichen Schwerpunkte differenziert.
 Darüber hinaus wird zwischen den Dimensionen \glqq Konzepte/Themenbereiche\grqq{} und \glqq Denk-, Arbeits- und Handlungsweisen\grqq{} unterschieden, wobei sich erstere vor allem auf den deklarativen Wissenserwerb, die zweite auf das prozedurale Wissen fokussiert. 
 Es wird ferner darauf verwiesen, dass für die Planung von Unterricht beide Dimensionen gemeinsam zu denken seien.
 Nun soll näher auf die perspektivenübergreifenden Aspekte des Kompetenzmodells eingegangen werden und aufgezeigt werden, wie sich das Philosophieren mit Kindern in dieses Raster einordnen lässt. 
 Dabei werden die im Perspektivrahmen verwendeten Überschriften zum besseren Verständnis übernommen. 
Zudem wurde die Handlungsweise des Umsetzen/Handeln für den Kontext des Philosophierens ausgeklammert, da der Fokus des Philosophierens mehr auf dem Denken als dem Handeln liegt.

\newpage

\subsection{Perspektivenübergreifende Denk, Arbeits- und Handlungsweisen: Erkennen/Verstehen}


Das Verstehen bildet für den Lernprozess und den Kompetenzerwerb der Schülerinnen und Schüler in der Grundschule eine entscheidende Grundlage. 
Bereits mehrfach wurde auf den Einfluss der Vorerfahrungen der Kinder hingewiesen, denn für das Verstehen sind diese essenziell. 
Daher muss im Sachunterricht dafür Sorge getragen werden, dass Lernumgebungen konstruiert werden, in denen diese Verstehensprozesse zustande kommen können. 
Das Philosophieren bietet den Kindern dazu eine Möglichkeit, indem es diese Erfahrungen einbindet und Diskussionen anstößt, durch die die Kinder ihre Positionen argumentativ begründen, verteidigen und unter Umständen auch überdenken müssen. 

Dabei kommt auch der Lehrperson eine wichtige Rolle zu, die durch ihr Frageverhalten den Verstehensprozess einscheidend beeinflussen kann. 
Kobarg unterscheidet verschiedene Arten von Lehrerfragen: keine, offene und geschlossene Fragen. 
Offene Fragen zeichnen sich dadurch aus, dass sie keine bestimmte Antwort verlangen, während geschlossene Fragen genau darauf abzielen. 
Darüber hinaus gibt es Niveaus, denen diese Fragen zugeordnet werden können\cite[S.\,22]{HB15} und durch die der Lern- und Verstehensprozess mitbestimmt wird.
In diesem Zusammenhang ist es für die Lehrperson wichtig, besonders sogenannte \glqq Deep-Reasoning\grqq{}-Fragen zu stellen, die auf eine länger ausgestaltete Antwort abzielen und von den Kindern autonomes Denken einfordert. 

Solche Prozesse können auch kollektiv durch Partner- und Gruppenarbeiten geschehen, in denen die Kinder sich über ihre Ansichten und Argumente austauschen und so gemeinsam philosophische Ideen entwickeln. 
Ergänzend dazu lassen sich auch \glqq komplexe, problemhaltige Anforderungen, die eine Übertragung vorhandenen Wissens in neue Kontexte erfordern\grqq{}\cite[S.\,21]{GDS13}, wie es der Perspektivrahmen Sachunterricht beschreibt, in das Philosophieren einbinden, indem etwa bereits erworbenes Wissen in anderen Zusammenhängen wie einer Geschichte angewandt und übertragen wird. 

\newpage

\subsubsection{Perspektivenübergreifende Denk, Arbeits- und Handlungsweisen: Eigenständig erarbeiten}


Der Perspektivrahmen Sachunterricht sieht die Fähigkeit der Schülerinnen und Schüler, sich selbstständig neues Wissen anzueignen, als essenziell an, um sich in einer Welt, die von einer schnell wachsenden und sich verändernden Wissenslandschaft geprägt ist, zurechtzufinden. 
Um dies zu erreichen, müssen den Schülerinnen und Schülern im Sachunterricht auch Aufgaben gestellt werden, \glqq die Lernende aus eigenem Interesse entwickeln oder die sie sich zu eigen machen.\grqq{}\cite[S.\,22]{GDS13}
Das Entfalten von Wissen aus eigenem Antrieb kann im Sachunterricht durch das Philosophieren geleistet werden. 
Dadurch, dass die Kinder die Gelegenheit haben, in diesem Rahmen ihre persönlichen Fragen zu stellen und sich darüber mit Gleichaltrigen auszutauschen, erweitern sie ihr eigenes Weltwissen und reflektieren es selbstständig im Kontext der Klassengemeinschaft. 
Hinzu kommt, dass die Schüler beim Philosophieren auch die Möglichkeit erhalten, unterschiedliche Zugangsweisen zu Wissen kennenzulernen. 
So können im Kontext philosophischer Fragestellungen neben Printmedien oder Expertenbefragungen auch moderne Medien wie das Internet genutzt werden.
Zusätzlich lernen die Schüler durch das Philosophieren im Sachunterricht, eine eigene Reflexionskompetenz auszubilden. 
So sollen sie nach Meinung des Perspektivrahmens \glqq ihre selbst gewählten Lernwege erläutern, begründen und überprüfen.\grqq{}\cite[S.\,23]{GDS13}


\subsubsection{Perspektivenübergreifende Denk, Arbeits- und Handlungsweisen: Evaluieren/ Reflektieren}


Die Reflexion und die Evaluation der eigenen Positionen sind für Schülerinnen und Schüler wichtig, um das eigene Denken und Handeln nicht nur an den eigenen Interessen und Bedürfnissen auszurichten, sondern auch die der Mitmenschen im Blick zu haben. 
In diesem Kontext kann das Philosophieren einen entscheidenen Beitrag leisten, indem die Kinder Meinungen ausdrücken und diese anschließend durch den Erwerb von philosophischen Wissen bestätigt oder widerlegt werden können. 
Hinzu kommt, dass es für die Kinder wertvoll ist, ihre eigenen Vermutungen anderen gegenüberzustellen, um so etwaige Überschneidungen oder auch Differenzen ausfindig machen zu können. 

Ein konkretes Beispiel könnte das sogenannte \glqq Heinz-Dilemma\grqq{} von Lawrence Kohlberg sein, das an die \glqq Was soll ich tun?\grqq{}-Frage von Immanuel Kant anknüpft. 
In der Geschichte wird von einem Mann erzählt, dessen Frau schwer krank ist. 
Es gibt jedoch eine Medizin, die sie retten könnte.
 Allerdings verlangt der Apotheker den zehnfachen Preis des Herstellungspreises für eine Dosis, so dass sich der Mann das Medikament auch mithilfe seiner Freunde nicht leisten kann. 
 So stellt sich die Frage, ob er das Medikament stehlen soll oder nicht. 
 Die Schülerinnen und Schüler könnten einerseits vermuten, dass es besser wäre, es zu stehlen, weil der Mann damit seine Frau retten könne.
 Andererseits würde er damit gegen das Gesetz verstoßen, den Apotheker schädigen und unter Umständen eine Gefängnisstrafe riskieren. 
 Die Kinder könnten daraufhin reflektieren, was sie anstelle des Mannes tun oder aber auch nicht tun würden und sich darüber austauschen.

Die Möglichkeiten, die dieses Beispiels deutlich macht, werden auch vom Perspektivrahmen genauer benannt.
So verweist die Gesellschaft für die Didaktik des Sachunterrichts darauf, dass folgende, sogenannte \glqq Lernmöglichkeiten\grqq{} die Kinder unterstützen können:
Das Ansprechen von \glqq Vermutungen und Vorerfahrungen vor der Erarbeitung neuen Wissens\grqq{}\cite[S.\,23]{GDS13}, welche anschließend bejaht oder verneint werden, die Schaffung von Lernumgebungen, in denen die Kinder zu eigenen Annahmen andere Optionen entdecken, um den Inhalt gänzlich zu durchdringen und Phasen des Austauschs, der Reflexion und des Nachdenkens, in denen die Kinder für sie bedeutsame Fragen besprechen können.

\subsubsection{Perspektivenübergreifende Denk, Arbeits- und Handlungsweisen: Kommunizieren/Mit anderen zusammenarbeiten}


Die Konstruktion von Wissen in der Schule ist kein Prozess, den Schülerinnen und Schüler allein bewältigen können. 
Daher liegt es nahe, dass die Kommunikation und Interaktion eine weitere, elementare Kompetenz darstellt, die die Kinder im Laufe ihrer Grundschulzeit erwerben müssen.
 Die philosophischen Unterrichtsgespräche können die Kinder dabei in ihrer Kommunikationsfähigkeit unterstützen, da sie im Rahmen dieser Unterrichtsanordnung mehr Anreize bekommen, sich zu äußern, als in den üblichen Unterrichtsgesprächen, die den schulischen Alltag seit Jahren dominieren.

Hinzu kommt, dass das Philosophieren mit Kindern ein Baustein für die Kinder sein kann, um von der Alltagssprache langfristig zu einer detaillierteren Fachsprache zu gelangen, um in der Lage zu sein, Wissen und Sachverhalte adäquat versprachlichen zu können\cite[S.\,24]{GDS13}.
Das Kommunizieren in der Gruppe unterstützt die Kinder darin, gemeinsam mit ihren Mitschülern Ideen zu konzipieren und zu optimieren. Daher nennt der Perspektivrahmen Sachunterricht dies als eine weitere Lernmöglichkeit.

\subsubsection{Perspektivenübergreifende Denk, Arbeits- und Handlungsweisen: Den Sachen interessiert begegnen}


Letztlich wird der Lernerfolg der Schülerinnen und Schüler in der Grundschule auch dadurch mitbestimmt, inwiefern dieser ihre ganz persönliche Neugier und ihre Begeisterung wecken kann. 
Daher sollte auch dem Schülerinteresse besonders Rechnung getragen werden und dieses im Sachunterricht in die Unterrichtsplanung miteinbezogen werden.
 Dadurch, dass das Philosophieren mit Kindern nicht direkt an zentrale Vorgaben durch den Lehrplan gebunden ist, ergibt sich ein großes Potential für philosophische Gespräche, da diese genau auf eine bestimmte Lerngruppe abgestimmt werden können. 
Der Perspektivrahmen Sachunterricht hat die Bedeutung des Schülerinteresses als eine eigene, perspektivenübergreifende Denk-, Arbeits- und Handlungsweise erkannt und hebt so ihren Stellenwert für einen kompetenzorientierten Sachunterricht hervor. 
Hierbei tauchen auch die bereits angesprochenen Schülerfragen als ein Ausgangspunkt des Sachunterrichts auf, die das gesteigerte Interesse der Schülerinnen und Schüler wecken können und ihnen die Bedeutsamkeit eines bestimmten Unterrichtsinhaltes greifbar machen können. 

Grundlegend sei auch der bewusste Umgang der Lehrkraft mit \glqq Rückmeldungen, die wertschätzend die Anstrengung und die geleistete Arbeit beurteilen\grqq{}\cite[S.\,25]{GDS13}, da diese ebenfalls zu einer erhöhten Interessenshaltung der Schülerinnen und Schüler beitragen können. 
Letztendlich müsse auch die Gestaltung des Unterrichts an sich durch die Auswahl der Themen, die ästhetische Aufbereitung und die Einbindung der Schülerinnen und Schüler so ausgearbeitet werden, dass die behandelten Themen \glqq von den Schülerinnen und Schülern erlebt, nachvollzogen und bearbeitet werden\grqq{}\cite[S.\,25]{GDS13}.







\newpage


%% Was ist Glück? Definition und Entwicklung der Glücksphilosophie
\section{Was ist Glück?}
\subsection{Begriffsdefinition}

Um die Vorstellungen, die Grundschüler beim Philosophieren in der Klasse äußern, verstehen und analysieren zu können, empfiehlt es sich, der Analyse der Unterrichtssequenz eine Begriffsdefinition voranzustellen, die klären soll, was Glück ist und ob es einen Unterschied gibt, Glück zu \glqq haben\grqq{} oder \glqq glücklich\grqq{} zu sein.

Der Duden versteht Glück zunächst einmal als \glqq etwas, was Ergebnis des Zusammentreffens besonders günstiger Umstände ist\grqq{}\cite{D16} und nimmt damit eine erste Verortung des Begriffes vor. 
Demnach wird Glück nicht durch seine spezielle Beschaffenheit bestimmt, sondern ist unabhängig von der Artigkeit der Ereignisse die möglichst günstige Abfolge.
 Ein wichtiger Begriff für die Definition des Duden ist in diesem Zusammenhang auch der Zufall, da das Auftreten von Glück keiner statistisch messbaren Form von Regelmäßigkeit zu folgen scheint. 
 Zudem lässt die Definition der günstigen Abfolge von Ereignissen Spielraum für Interpretationen, welche Ereignisabfolgen von Menschen als günstig angesehen werden.  
 Der Duden nennt in einer zweiten Ebene die Sicht des personifizierten Glücks, das das Glück antropomorph betrachtet. 
 Dem Glück wird beispielsweise in der römischen Göttin Fortuna eine Figur zugeordnet, wie es im antiken römischen Reich für alle Lebensbereiche der Gesellschaft üblich war. 
 Abschließend unterscheidet der Duden genauer zwischen Glück haben und glücklich sein, indem er \glqq glücklich sein\grqq{} als positive Gemütsverfassung und \glqq Glück haben\grqq{} als vereinzelte glückliche Situationen beschreibt\cite{D16}.
  Demnach kann man in Situationen sein, in denen man zwar Glück hat, aber nicht zwangsläufig glücklich ist. 
 Andersherum gesagt kann man auch glücklich sein ohne Glück zu haben. 
 Es zeigt sich also, dass \glqq Glück haben\grqq{} und \glqq glücklich sein\grqq{} völlig verschiedene Zustände beschreiben können. 
 
Auch das philosophische Online-Wörterbuch Philosophie stellt eine solche Unterscheidbarkeit heraus. 
So sei es denkbar, dass manche Menschen durch den Erwerb oder den Besitz materieller Güter wie Macht oder Geld Glück empfinden werden. 
Andere wiederum werden vor allem durch mitmenschliche und innere Gefühle Glück erleben oder aber auch durch den Genuss\cite{GT16}. 
Auch hier zeigt sich erneut der individuelle Charakter der menschlichen Glückswahrnehmung. 

Dass jeder Mensch Glück als etwas anderes beschreiben kann, liegt für die Autorin Lic. phil. Gerhild Tesak vor allem  darin begründet, dass jeder Mensch durch seine individuelle Prägung, die er im Laufe des Lebens erfährt, und durch seine Interessen ein eigenständiges Glücksverständnis entwickele.
Demnach sei eine genaue Definition schlicht nicht möglich und die einzige, gültige Festlegung bestehe darin, das Glück als das höchste Ziel der menschlichen Existenz zu verstehen. 
 Sie stellt darüber hinaus fest, dass das Streben nach Glück nicht selbiges als Ziel erfolgreich anstreben kann. 
 Mit jener Glücksvorstellung verbundene Wünsche könnten zwar in Erfüllung gehen, jedoch ohne ein entsprechendes Gefühl von Glückseligkeit hervorzurufen\cite{GT16}.
 So gesehen sei Glück \glqq das begleitende Gefühl gelungenen Handelns\grqq{}.
 
Harald Schöndorf bezieht sich ebenfalls auf die Abgrenzung zwischen \glqq Glück haben\grqq{} und \glqq glücklich sein\grqq{}. 
Dabei versteht er das Glück im erstgenannten Kontext vorrangig als positives Ereignis, das sich \glqq zumindest bis zu einem gewissen Grad als Geschenk, Zufall oder göttliche Fügung erweist.\grqq{} 
Glücklich zu sein betrachtet Schöndorf jedoch in einem umfassenderen, semantischen Zusammenhang, der darin besteht, dass es sich dabei um Erfahrungen handelt, die die Zufriedenheit an sich übersteigen. 
Dabei nimmt er auch Bezug auf die \textit{Eudaimonia} des Aristoteles, welche das letztliche Ziel der menschlichen Existenz darstellt. 
So gesehen ist mit Glück im Kontext des \glqq glücklich seins\grqq{} auch eine gewisse Hoffnung auf dieses Ziel eng verknüpft.

Harald Schöndorf betont darüber hinaus auch die Gemeinsamkeiten der beiden Lesarten des Glücksbegriffes, die darin bestünden, dass das Glück keine Leistung oder etwas planbares sei, sondern lediglich als etwas Geschenktes zu erreichen sei. 
Es sei nichts Machbares, sondern stelle sich erst ein, wenn ein Mensch sein Leben kompromisslos akzeptieren könne\cite{WB13, S.175}.
Glück ist zudem nichts, was sich lediglich im irdischen Leben verortet. 
Dadurch, dass das irdische Leben des Menschen unvollkommen und defizitär ist, muss sich auch das alles erfassende Glück in einem transzendentalen Leben außerhalb der Sphäre der Erde befinden. 
Eine große Gefahr bestehe folglich darin, sich auf das Streben nach Glück zu fixieren, statt die erbrachten Leistungen zu wertschätzen und das Glück weiterhin als besonderes Geschenk wahrzunehmen\cite{WB13, S.175}.

Es lässt sich daher sagen, dass der Begriff \glqq Glück\grqq{} als solches nicht abschließend definiert, sondern nur angenähert werden kann. 
Viele Philosophen der Antike, des Mittelalters und auch der Neuzeit haben sich mit dem Glück beschäftigt und wie der Mensch sich ihm nähern kann. 
Daher sollen an dieser Stelle die zentralen Positionen der Philosophie angesprochen werden, um sie in einem weiteren Schritt mit den Aussagen von Grundschülern in Beziehung setzen zu können.


\subsection{Philosophische Positionen der Antike}

Bereits im antiken Griechenland befassten sich Philosophen mit den Fragen, was Glück sei und wie man es erlangen könne. 
Aristippos von Kyrene, ein Schüler Sokrates', der zwischen 435 und 335 v.Chr. in Kyrene lebte, unterscheidet in seiner Sicht auf das Glück vor allem zwei Gemütslagen der Seele, die entweder durch Lust oder durch Schmerz bestimmt waren\cite{DL67, S.116}.

Er differenziert dabei ausdrücklich nicht die Beschaffenheit der Lust, da es aus seiner Sicht keine höhere Ausprägung als die Lust als solche geben könne. 
Zusätzlich sei die Lust grundsätzlich aus Sicht aller Geschöpfe erstrebenswert, wobei der körperlichen Lust in dieser Beziehung eine besondere Bedeutung zukomme. 
Neben der Differenzierung zwischen Lust und Schmerz, denen sich die menschliche Seele hingeben könne, unterscheidet Aristippos zusätzlich zwischen dem Ziel und der Glückseligkeit. 
Demnach sei das Ziel die einfache Lust, während die Glückseligkeit in der Summe jeglicher Arten von Lust sei und damit allumfassender\cite{DL67, S.116}.
Daraus folgert er, dass die einzelne Lust durch ihre eigene Anziehung in sich lustvoll sei, die Glückseligkeit hingegen diese Anziehungskraft aus den verschiedensten Lüsten beziehe. 

Während Aristippos von Kyrene einerseits das Streben der menschlichen Seele nach Lust betont, so weist er andererseits auch energisch auf die Vermeidung von Schmerzen und Leid hin, welche nach seiner Ansicht dem Glück im Wege stünden. 
Letztlich sieht er die Glückseligkeit des Menschen als utopisches Ziel an, da dem Erreichen der Glückseligkeit körperliche Leiden, die auch die Seele beträfen, entgegenwirken würden\cite{DL67, S.119}.

Aristoteles stellt in seiner nikomachischen Ethik fest, dass jedes Handeln des Menschen zielorientiert ist. 
Der Mensch strebe durch sein unmittelbares Handeln ein Ziel an, das es zu erreichen gelte\cite{MF93, S.4}.
Darüber hinaus differenziert Aristoteles zwischen Dingen, die man ihrer selbst willen tue und solchen, die man tue, um ein höheres Ziel zu erreichen, aber auch Dinge, die sowohl zweckgebunden als auch als solches zielgerichtet sind wie z.B. eine sportliche Aktivität, die man vollziehen kann, weil sie Freude bereitet aber auch, weil sie für das eigene Wohlbefinden zuträglich sein kann. 
Seine Definition von Glück definiert Aristoteles aus der überzeugung heraus, dass der Mensch nach Vollkommenheit strebe. 
Daher sei es sinnvoll, sich mit dem höchsten menschlichen Ziel -- dem Glück -- näher zu beschäftigen und es zu ergründen. 

Zentraler Begriff seiner Ausführungen ist die \textit{Eudaimonia}, welcher sich mit \glqq gut leben\grqq{} oder \glqq gut handeln\grqq{} übersetzen lässt\cite{MF93, S.5}.
Mit ihr beschreibt er das zentrale Endziel, dass das menschliche Leben ausmache. 
Jedoch lässt sich dieser Begriff nicht eins zu eins mit der heutigen Vorstellungen eines glücklichen Lebens in Relation setzen.
Aristoteles betrachtet ein Leben erst dann als als eudaimon, welches eine \glqq objektive Gestalt und Qualität und nicht eine subjektive Stimmungslage des Lebens zum Inhalt hat.\grqq{}\cite{MF93, S.5}
 Folglich kann ein eudaimones Leben nur ein solch geartetes sein, dass durch einen guten Charakter und eine Lebensweise geprägt wird, die alles dafür tut, dass der eigene Lebensweg sich erfolgreich gestaltet. 
 Daran wird deutlich, dass der Glücksbegriff des Aristoteles sich grundlegend von dem neuzeitlichen des 21.Jahrhunderts unterscheidet.
 
Es lässt sich demnach sagen, dass die Sehnsucht des Menschen nach Geld, Gesundheit oder Vergnügen Teilziele sind im Gegensatz zur Eudaimonia. 
Demgegenüber ist die Eudaimonia das höchste Ziel menschlicher Existenz und so etwas, das sich nicht um seiner selbst willen erreichen lässt. 
Gleichzeitig versucht der Mensch jedoch auch, durch partielles Glück wie Wohlstand, Gesundheit oder Vergnügen sich der Eudamonia anzunähern und deshalb ist sie so gesehen auch etwas Zusammengesetztes. 

Aristoteles hält die Glückseligkeit des Menschen für ein Zusammenwirken dreier Güter, die er im folgenden ausführt: die seelischen, die körperlichen (Wohlbefinden, Stärke, Schönheit) und die äußeren wie z.B. Reichtum oder Ruhm\cite{DL67, S.257f}.
Eines allein sei nicht ausreichend, um Glückseligkeit zu erfahren, sondern alle drei müssten zusammenwirken. 
Das Gegenstück zur Glückseligkeit -- das unglückselige Leben -- sei jedoch bereits erreicht, wenn eines der genannten Güter nicht vorhanden wäre unabhängig davon wie gesund oder reich ein Mensch sei. 
In diesem Zusammenhang nennt er das Beispiel eines Weisen, der ein unglückseliges Leben führen würde, wenn er in Armut oder gesundheitlichem übel leben müsse.

Epikur sieht, ähnlich wie Aristippos, die Lust als ein zentrales Moment des Menschen an, um das persönliche Glück zu erreichen. 
Lust sieht er als erstrebenswert und Schmerz als zu vermeiden an. 
Allerdings ist er der Auffassung, dass der Mensch bescheiden sein müsse, um diese Lust zu erreichen.
Der Mensch müsse daher seine Bedürfnisse auf das Nötigste beschränken, um langfristig glücklich zu werden, da extreme Lust auch zu extremer Unlust führen könne.
Die größte Differenz zu Aristippos zeigt sich darin, dass Epikur neben den Zuständen der Lust und des Schmerzes auch einen dritten beschreibt, der sowohl lustvoll als auch schmerzhaft bezeichnet werden kann\cite{MF93, S.32}. 
Dabei erkennt Epikur durchaus an, dass sich der Mensch in einem Abhängigkeitsverhältnis zu einem natürlichen Trieb befindet. 
Jedoch sieht er diesen dadurch als unterbrochene Abhängigkeit an, da der Mensch jederzeit die Wahl habe, zu leben oder nicht zu leben\cite{MF93, S.36}.

Zu den Positionen Aristoteles' geht Epikur auf Distanz, da er das Glück nicht als zweckorientiertes Tun begreift, sondern ihm eher den Charakter eines \glqq zweckfreien Spiels\grqq{} zuschreibt, wie Maximilian Forschner dies nennt, \glqq das als solches, weil zwecklos, nicht ausgerichtet auf ein zu Erreichendes, stets vollendet ist.\grqq{}\cite{MF93, S.44}
In diesem \glqq Spiel\grqq{} könne letztlich nur mitspielen, wer sich von den \glqq Zwängen unbedingten Begehrens und Strebens befreit und in ästhetischer Distanz dem Spiel der Natur überlässt.\grqq{}\cite{MF93, S.44}
 
Auch die Stoa, eine philosophische Schule in Athen, beschäftigt sich mit dem Glück des Menschen. 
Die Lehre der antiken Stoa, die um 300 v. Chr. von Zenon von Kition begründet und von zahlreichen Vertretern bis hin zu Mark Aurel im 2.Jahrhundert n. Chr. fortgeführt wurde, grenzt sich klar ab von den Ansichten Epikurs und Aristoleles'. 
So schreibt Diogenes Laertius, dass die Vertreter der Stoa annahmen, dass der erste Trieb der menschlichen Natur die Selbsterhaltung sei und nicht etwa das Streben nach Lust\cite{DL67, Siebentes Buch S.48}.
Die Lust sahen sie daher nur als eine Art Begleiterscheinung, die \glqq den lebenden Wesen die heitere Stimmung und den Pflanzen das fröhliche Wachstum\grqq{} beschere. 

Zenon bekundet, dass das finale Ziel des Menschen ein Leben in Harmonie mit der Natur und demnach ein tugendhaftes Leben sei. 
Die Tugend wird von Stoikern wie Hekaton in wissenschaftliche und theoretische unterteilt. 
Als Beispiele nennt er an dieser Stelle die Einsicht und die Gerechtigkeit. 
So schreibt Laertius, untheoretische Tugenden seien so benannt, \glqq weil sie  nicht auf der Zustimmung des Verstandes beruhen, sondern eine Folgeerscheinung sind und auch bei Schwachköpfen sich finden, wie z.B. Gesundheit, Männlichkeit.\grqq{}\cite{DL67, Siebentes Buch S.50}
Demzufolge müsse es allgemeine Tugenden geben, die jedem Menschen zuteil sind und solche, die bei manchen vorhanden sind und bei anderen nicht. 
Es sei an dieser Stelle jedoch darauf hingewiesen, dass die Stoa im Tugendbegriff gespalten war, da die Stoiker untereinander andere Schwerpunkte setzten. 
So ging Panaitios davon aus, es gebe theoretische und praktische Tugenden, andere wiederum klassifizierten Tugend als logisch, physisch oder ethisch.

Die Stoa lehnt die Lust als notwendigen Parameter für das Glück ab. 
Zenon schreibt dazu folgendes: \glqq Lust ist das unvernünftige Frohgefühl über eine scheinbar begehrenswerte Sache.\grqq{}\cite{DL67, Siebentes Buch S.60}
Hier zeigt sich die Ablehnung der epikureischen Lehre, die die Lust zum Mittelpunkt ihrer Argumentation macht. 
Darüber hinaus widerspricht die Stoa Aristoteles, der der überzeugung war, dass auch der gesellschaftliche Stand zum vollkommenen Glück beitragen muss. 
Sie sieht das Glück in der Zuwendung zur Natur, da die Natur als Resultat der Weltvernunft, welche als wesensgleich mit dem griechischen Gott Zeus angesehen wird, die einzige legitime Quelle für ein glückliches Leben ist\cite{DL67, Siebentes Buch S.49}.
Daher ist das Glück aus Sicht der Stoiker eine gottgegebene Ordnung des natürlichen Lebens. 


\newpage

\subsection{Philosophische Positionen im Mittelalter}

Die Philosophie des Mittelalters wurde maßgeblich durch die Christianisierung Europas geprägt, weshalb es zustande kommt, dass sich vor allem Geistliche mit der Thematik des Glückes beschäftigt haben. 
Augustinus von Hippo, der im 5. Jahrhundert n. Chr. in Hippo im heutigen Algerien Bischof war und neben Hieronymus, Ambrosius von Mailand und Papst Gregor I. zu den vier spätantiken Kirchenvätern des Abendlandes zählt, widmete dem Glück in seinem Werk \glqq de beata vita -- über das Glück\grqq{} ein ganzes Buch.

Darin sagt Augustinus, dass grundsätzlich jeder Mensch nach Glück strebe: \glqq Wir alle wollen glücklich sein.\grqq{}\cite{A82, S.21}
Dabei sei das Glück nicht grundsätzlich an den Besitz des Begehrten gebunden, sondern an dessen Qualität. 
Wenn es sich dabei um Gutes handele, könne derjenige glücklich werden. 
Wenn es etwas Schlechtes sei, würde er unglücklich sein, obwohl er es besitzt. 
Er weißt zudem darauf hin, dass es dem Glück geradezu hinderlich sei, sich den Lüsten hinzugeben, da jegliches Unerlaubte größtes Unglück sei\cite{A82, S.21}.
Dagegen sei das Versäumen solch Unerlaubtem das geringere übel.

Im Folgenden macht Augustinus deutlich, dass niemand glücklich werden kann, der nicht das Begehrte besitzt, jedoch auch nicht jeder Mensch glücklich ist, der besitzt, was er begehrt. 
Damit leitet er über auf die Frage, was der Mensch sich verschaffen müsse, um glücklich sein zu können\cite{A82, S.23}.
Dazu bedürfe es etwas, was der Mensch besitzen könne, das unabhängig von Zufällen oder anderen Einflüssen sei und so von größerer Dauer. 
Denn Zufallsgüter seien vergänglich und könnten verloren gehen. 

So kommt er zu dem Schluss, dass der Mensch Gott haben muss, um glücklich sein zu können, da Gott das einzige sei, das unvergänglich und unabängig von Einflüssen sei. 
Diesen Gedanken führt er fort, indem er festhält, dass jeder glücklich sei, der Gott gefunden habe und dieser sei ein gnädiger Gott\cite{A82, S.41}. 
Wer jedoch noch nach Gott suchen müsse, könne nicht glücklich sein, da er noch nicht besitze, was er begehre. 
So konstruiert Augustinus das Unglück als Mangel an dem, was man sich wünscht und das Glück als den Zustand, den man erreicht, wenn man den gnädigen Gott findet. 
Daher ist für ihn Glück nichts, was durch materielle Güter erreicht werden kann, sondern nur durch die Gnade Gottes. 
Denn in diesem Zusammenhang nennt er das Beispiel von Orata, der durch die Erfindung der Fußbodenheizung in Bädern reich geworden sei. 
Dieser müsse jedoch stets gefürchtet haben, all sein Hab und Gut durch das Schicksal verlieren zu können und sei durch seine Unsicherheit demzufolge auch nicht fähig, glücklich zu sein\cite{A82, S.48}.

Augustinus definiert \glqq glücklich sein\grqq{} also als den Zustand, indem man keinen Mangel zu leiden hat, was er mit \glqq weise sein\grqq{} gleichsetzt. 
Ergänzend nennt er in seinen überlegungen, was für ihn Weisheit bedeutet: \glqq Ist sie doch nichts anderes als das Maß des Geistes, das heißt das, womit sich der Geist im Gleichgewicht hält, um weder ins übermaß auszuschweifen noch in die Unzulänglichkeit herabgedrückt zu werden.\grqq{}\cite{A82, S.59} 
An dieser Stelle beschreibt Augustinus die Weisheit als eine seelische Ausgeglichenheit, die nicht zu Verführungen wie Verschwendung, Machtgier, Hochmut, etc. neigt, wie er es beschreibt. 
Derjenige werde das Glück finden, der sich nicht \glqq dem Trug der Götzenbilder zuwendet, durch deren Gewicht er von Gott abfallen und versinken muß\grqq{}, da dieser dann auch kein Ungemach oder Mangel und damit auch kein Unglück fürchten müsse. 
Und so sieht er das Glück in einer tiefgehenden Verknüpfung zu Gott, der als das Vollkommene den Weg zum Glück darstellt.

Martin Luther, der die Reformation in Deutschland Ende des 15.Jahrhunderts mit anführte, kritisiert in seinen 95 Thesen unter anderem den Ablasshandel, der aus seiner Sicht das Seelenheil -- und damit auch das Glück -- der Menschen gefährde. 
Er schreibt in seiner 30.These: \glqq Niemand kann der Wahrhaftigkeit seiner Reue sicher sein; und noch viel weniger gilt das vom Resultat des vollkommenen Nachlasses.\grqq{}\cite{ML65, S.56}
Luther sieht den Menschen in einer tiefen Schuld gefangen, die ihn darin hindere, ins Himmelreich zu gelangen. 
Daher könne die Buße allein nicht zu einer Rettung führen, sondern lediglich die allumfassende Gnade Gottes. 
Er kritisiert zudem, dass sich gottlose Menschen oder auch Feinde Gottes, wie er sie nennt, das Heil im Reich Gottes erkaufen könnten. 
Der Kauf von Ablassbriefen rette aber keinen Gläubigen vor dem Fegefeuer, sondern wiege ihn in einer falschen Sicherheit\cite{ML65, S.58}.
Folglich lässt sich aus Martin Luthers 95 Thesen der Schluss ziehen, dass Glück für ihn nur durch ein Leben erreicht werden könne, dass sich an den Lehren und Geboten der Bibel orientiere und sich nicht durch Verführungen wie eben beispielsweise den Ablasshandel vom Weg abbringen lasse.


\newpage

\subsection{Philosophische Positionen der Moderne}

In der Moderne löste sich der Glücksbegriff wieder von theologischen Betrachtungsweise und es entwickelten sich im Zuge der Aufklärung Positionen, die wieder an die Ansichten der Antike anknüpften. 

Zu diesen Vertretern zählte unter anderem Immanuel Kant, der zu den bedeutendsten Philosophen der Aufklärung zählt. 
Kant lehnt den klassischen Glücksbegriff, wie er vor allem von Epikur vertreten wurde, ab, indem er schreibt: \glqq Das Wesentliche alles sittlichen Werths der Handlungen kommt darauf an, daß das moralische Gesetz unmittelbar den Willen bestimme. 
Geschieht die Willensbestimmung zwar gemäß dem moralischen Gesetze, aber nur vermittelst eines Gefühls, welcher Art es auch sei, das vorausgesetzt werden muß, damit jenes ein hinreichender Bestimmungsgrund des Willens werde, mithin nicht um des Gesetzes willen: so wird die Handlung zwar Legalität, aber nicht Moralität enthalten.\grqq{}\cite{IK74, S. 71}

Besonders deutlich wird an dieser Stelle die Ablehnung Kants gegenüber eines Glücksbegriffes, der durch ein Gefühl oder den Wunsch nach Glück bestimmt wird. 
Kant sieht das Glück vor allem in rational getroffenen Entscheidungen, die Gesetzen und Normen entsprechen. 
Zudem sieht er die Handlungen des Menschen als eine Kausalkette an, in der das Handeln des Menschen durch äußere Einflüsse und Neigungen bestimmt wird. 
Er schreibt: \glqq Von der anderen Seite ist es sich seiner doch auch als eines Stücks der Sinnenwelt bewußt, in welcher seine Handlungen als bloße Erscheinungen jener Kausalität angetroffen werden, deren Möglichkeit aber aus dieser, die wir nicht kennen, nicht eingesehen werden kann, sondern an deren Statt jene Handlungen als bestimmt durch andere Erscheinungen, nämlich Begierden und Neigungen, als zur Sinnenwelt gehörig eingesehen werden müssen.\grqq{}\cite{IK65, S.79f}

Unter Glückseligkeit versteht Kant \glqq die Befriedigung aller unserer Neigungen (sowohl extensive der Mannigfaltigkeit derselben, als intensive dem Grade und auch protensive der Dauer nach.)\grqq{}\cite{IK73, S.523} 
Kant ist der Auffassung, dass ein Mensch vor allem dann glücklich sein kann, wenn er würdig handelt und gibt damit eine Antwort auf die Frage, was der Mensch tun solle. 
Gleichzeitig verknüpft er diesen Standpunkt mit der Frage, was der Mensch hoffen dürfe und schreibt dazu: \glqq Ich sage demnach: daß eben sowohl, als die moralischen Principien nach der Vernunft in ihrem praktischen Gebrauche nothwendig sind, eben so nothwendig sei es auch nach der Vernunft, in ihrem theoretischen Gebrauch anzunehmen, daß jedermann die Glückseligkeit in demselben Maße zu hoffen Ursache habe, als er sich derselben in seinem Verhalten würdig gemacht hat, und daß also das System der Sittlichkeit mit dem der Glückseligkeit unzertrennlich, aber nur in der Idee der reinen Vernunft verbunden sei.\grqq{}\cite{IK73, S.525}

An dieser Stelle beschreibt Kant, dass das würdige Handeln in direkter Verbindung zur Hoffnung auf Glückseligkeit steht, so dass die beiden Fragen nach dem Tun und dem Hoffen in einem engen Zusammenhang betrachtet werden müssen. 
Gleichzeitig unterstreicht er mit dem Hinweis auf die Vernunft seine Ablehnung des Glücksbegriffs als einen von Gefühlen hervorgebrachten Zustand. 
Er betont ergänzend dazu, dass eine solch geartete Hoffnung auf die Glückseligkeit, verbunden mit der Würdigkeit zur Glückseligkeit, nur vorliegen könne, wenn nicht nur die Natur zugrunde gelegt würde, sondern vor allem, wenn es eine höchste Vernunft gebe, die moralischen Gesetzen unterworfen sei und ebenfalls hinzugezogen würde.
Für Kant ist eine Glückseligkeit ohne Sittlichkeit nicht denkbar, da er die Sittlichkeit als Voraussetzung für das Glück betrachtet. 
Erst wenn der Mensch im Sinne der Sittlichkeit handelt, kann er das Glück finden. 
Kant erarbeitet ein Bild von einer idealen Welt, die folglich auch einen idealen Ursprung, einen Schöpfer, haben muss\cite{IK73, S.526}.

John Stuart Mill, der im 19.Jahrhundert wirkte und als einer der bedeutsamsten Anhänger des Utilitarismus galt, lehnt seine Vorstellungen von Glück hingegen an die des antiken epikureischen Gedankenguts an. 
Der von ihm propagierte Utilitarismus geht davon aus, dass jede Handlung moralisch korrekt ist, die das Glück befördern kann und moralisch falsch ist, die das Glück gefährdet. 
Er schreibt: \glqq Die Auffassung, für die die Nützlichkeit oder das Prinzip des größten Glücks die Grundlage der Moral ist, besagt, daß Handlungen insoweit und in dem Maße moralisch richtig sind, als sie die Tendenz haben, Glück zu befördern, und insoweit moralisch falsch, als sie die Tendenz haben, das Gegenteil von Glück zu bewirken.\grqq{}\cite{JM94, S.13}
Er ergänzt diesen grundlegenden Gedanken noch, indem er das Glück (happiness) als Lust (pleasure) versteht sowie das Unglück (unhapiness) als Unlust oder das Fehlen von Lust definiert. 
Mill greift an dieser Stelle den epikureischen Begriff der \glqq Lust\grqq erneut auf. 
Mill sieht die Lust und die Vermeidung von Unlust als die zentralen Faktoren an, die wichtig sind, um dauerhaft glücklich zu sein. 
Dabei nimmt er an, dass \glqq alle anderen wünschenswerten Dinge (die nach utilitaristischer Auffassung ebenso vielfältig sind wie nach jeder anderen) entweder deshalb wünschenswert sind, weil sie selbst lustvoll sind oder weil sie Mittel sind zur Beförderung von Lust und zur Vermeidung von Unlust.\grqq{}\cite{JM94, S.13}

John Stuart Mill differenziert verschiedene Arten von Freuden aus, die nach seiner Auffassung in unterschiedlichem Maße zur Lust des Menschen beitragen. 
Er stellt daher fest, dass es unsinnig wäre, wenn die Wertigkeit einer Freude nur durch die Quantität und nicht auch durch deren Qualität mitbestimmt wird\cite{JM94, S.15}. 
Das heißt: Es gibt Dinge, die den Menschen häufiger Freude bereiten, als andere. 
Diese können jedoch zwangsläufig nicht von selbiger Qualität sein wie diese, die nicht in der gleichen Häufigkeit jedoch in einer größeren Intensität auftreten. 
Daraus folgt, dass der Mensch stets immer die Freuden vorzieht, die ihm den größtmöglichen Nutzen (lat. Utilitas = Nutzen) verschaffen. 
Aus seiner Sicht ist gerade die Form der Lebensführung willkommen, die ein größtmögliches Maß an Lust und ein möglichst geringes an Unlust beinhalte. 
Diese Art der Lebensführung bezeichnet Mill als den letzten Zweck des menschlichen Lebens\cite{JM94, S.21}.

Mills skizzierte Lebensführung, die zum Glück des Menschen führen soll, basiert auf einer Glücksvorstellung, die das Glück nicht als dauerhaften Zustand des Wohlbefindens begreift: 
\glqq Freilich: versteht man unter Glück das Fortdauern einer im höchsten Grade lustvollen Erregung, dann ist die Unerreichbarkeit von Glück nur zu offensichtlich. 
Der Zustand der überschwänglichkeit hält höchstens einige Augenblicke, in einigen Fällen -- mit Unterbrechungen -- auch Stunden und Tage an;\grqq{}\cite{JM94, S.23}
 Zudem vergleicht er das Glück mit einer auflodernden Flamme, die im Gegensatz zur Glut, nur gelegentlich ausbricht. 
 Damit meint Mill, dass das Leben nicht aus einem andauernden Zustand der Freude bestehe, sondern das ein glückliches Leben eines sei, dass durch Abschnitte der Lust und durch Abschnitte der Unlust bestimmt werde.
 
Das persönliche Glück des Einzelnen, das John Stuart Mill in seinem Werk \glqq Utilitarismus\grqq beschreibt, steht im engen Zusammenhang mit dem Glück der Gesellschaft. 
Nach Mill müssen \glqq Gesetze und gesellschaftliche Verhältnisse das Glück -- oder wie man es in der Praxis auch nennen kann -- die Interessen jedes einzelnen so weit wie möglich mit den Interessen des Ganzen in übereinstimmung bringen.\grqq{}\cite{JM94, S.30}
Damit sieht Mill sowohl den Einzelnen als auch die Gesellschaft in der Pflicht, alle Interessen zu vereinbaren und greift damit auch Aristoteles erneut auf, der bereits in der Antike die Verbindung mit den gesellschaftlichen Werten im Hinblick auf den Glücksbegriff verwies.

Friedrich Nietzsche (1844-1900) sieht das Glück im Gegensatz zu Kant oder Mill nicht als äußerliche Erscheinung an, sondern als etwas, dass dem Menschen in seiner Psyche innewohnt. 
Er schreibt dazu: \glqq Die Bestie in uns will belogen werden; Moral ist Notlüge, damit wir von ihr nicht zerrissen werden.\grqq{}\cite{FN06, S.57}
Nietzsche betrachtet die Moral als Irrtum, ohne den der Mensch ein Tier geblieben wäre. 
Jedoch hat er sich durch die Abgrenzung vom Tierischen auch strengere Gesetzmäßigkeiten auferlegt, denen er fortan unterliegt.

Für seine Ausführungen über das Glück des Menschen entfaltet Nietzsche drei Säulen, die zur Erlangung des Glücks notwendig sind: Die \textit{Gewohnheit}, \textit{die Schönheit} und \textit{den Unsinn}. 
Er schreibt über die Gewohnheit in seinem Werk \glqq Menschliches, Allzumenschliches\grqq{}: \glqq Die Lust in der Sitte. -- Eine wichtige Gattung der Lust und damit der Quelle der Moralität entsteht aus der Gewohnheit.\grqq{}\cite{FN06, S.84}
Dinge, die man aus einer Routine heraus tue, sieht Nietzsche als positiver konnotiert als andere Tätigkeiten, da sie auch durch die Erfahrung mitbeeinflusst werden. 
In diesem Zusammenhang sieht er die Sitte als die Verschmelzung des Angenehmen mit dem Nützlichen an und dass jede Sitte zur Gewohnheit, demnach also zur Lust werden kann.

Die zweite Säule des menschlichen Glücks beschreibt Nietzsche mit der Schönheit: \glqq Der langsame Pfeil der Schönheit. -- Die edelste Art der Schönheit ist die, welche nicht auf einmal hinreißt, welche nicht stürmische und berauschende Angriffe macht (eine solche erweckt leicht Ekel), sondern jene langsam einsickernde, welche man fast unbemerkt mit sich fortträgt und die einem im Traum einmal wiederbegegnet, endlich aber, nachdem sie lange mit Bescheidenheit an unserem Herzen gelegen, von uns ganz Besitz nimmt, unser Auge mit Tränen, unser Herz mit Sehnsucht füllt.\glqq{}\cite{FN06, S.131}
An diesem Punkt bekennt Nietzsche, dass für ihn die Schönheit des Glücks darin besteht, dass er das Glück als einen Zustand der langanhaltenden Zufriedenheit begreift und nicht etwa als wiederkehrende Phasen des Glücks, wie Mill dies in seiner utilitaristischen Betrachtung tut. 
Hier wird die Schönheit auch mit einer gewissen Unbeschwertheit assoziiert.

Nietzsches dritte Säule besteht in der menschlichen Freude am Unsinn, die er in jeder Situation vermutet, in der gelacht wird: \glqq Wie kann der Mensch Freude am Unsinn haben? Soweit nämlich auf der Welt gelacht wird, ist dies der Fall; ja man kann sagen, fast überall wo es Glück gibt, gibt es Freude am Unsinn.\grqq{}\cite{FN06, S.159}
Der Unsinn befreie den Menschen vom Zwang des \glqq Notwendigen, Zweckmäßigen und Erfahrungsgemäßen\grqq{}.\cite{FN06, S.159}


\subsection{Zusammenfassung}

In diesem Kapitel konnte festgestellt werden, dass sich der Glücksbegriff nicht in eine abschließende Definierbarkeit bringen lässt. 
Allerdings lassen sich allgemeingültige Parameter für das Glück festlegen, die zusammengefasst werden sollten.

Evident ist, dass Glück als Resultat optimaler Gegebenheiten für ein bestimmtes Ereignis angesehen werden kann, wobei die Art dieses Ereignisses für diese Sichtweise irrelevant ist. 
Von Wichtigkeit ist in dieser Betrachtung nur die Abfolge von Ereignissen. 
Darüber hinaus wurde -- vor allem in der Antike -- das Glück, wie beispielsweise auch Krieg oder Tod, mit Gottheiten assoziiert, die als Patrone dieser Zuständigkeitsfelder betrachtet wurden. 

Auch die Differenzierung, die die deutsche Sprache vornimmt, indem sie von \glqq Gl/"uck haben\grqq{} und \glqq glücklich sein\grqq{} spricht, ist von zentraler Bedeutung für den Glücksbegriff, da sie eine Fokussierung der Betrachtung vornimmt. 
Während \glqq Glück haben\grqq{} die vorhin beschriebene Abfolge von Ereignisabfolgen bezeichnet, lässt sich  \glqq glücklich sein\grqq{} als emotionale Regung des Menschen fassen. 
Die Unmöglichkeit, das Glück genauer zu definieren lässt sich auch damit erklären, dass jeder Mensch durch seine individuelle Prägung ein anderes Verständnis von Glück oder glücklich sein entwickelt und dadurch abweichende Einschätzungen entstehen können. 
Trotz dieser milieugeprägten Unterschiede lässt sich trotzdem konstatieren, dass das Glück als höchstes Ziel des Menschen gesehen werden kann.

Auch in der Antike wurden höchste Ziele f/"ur die menschliche Existenz formuliert, doch unterschieden sich diese deutlich voneinander. 
Während Aristippos die Glückseligkeit als den Zusammenschluss von Lust jeglicher Art verstand und diesen Zustand durch den Einschluss seelischer Leiden als utopisch betrachtete, sah Aristoteles den Menschen als zielstrebiges Wesen an, dass sich durch Teilziele wie Gesundheit oder Geld der Eudaimonia -- dem höchsten Ziel des Menschen -- annähern kann und will. 
Die Stoa wiederum widerspricht beiden Positionen, da sie die Lust ablehnt und das Glück in keiner Abhängigkeit des gesellschaftlichen Standes betrachtet. 
Für sie liegt das Glück in der Natur als gottgegebene Ordnung und als einzig legitime Quelle des Glücks.

Im Mittelalter wurde der Glücksbegriff stark in den Kontext des Gottesglaubens gestellt, so dass Gott als unabdingbare Instanz für das Glück des Menschen verstanden wurde. 
So lehnte Augustinus -- analog zur Stoa -- die Lust ab, da sie das Unglück eines Menschen sei. 
Er vertrat die Auffassung, dass der Mensch das Glück besitzen wolle, er jedoch sein eigenes Unglück verstärken würde, wenn es sich bei seinem Besitz um etwas schlechtes handele. 
Jeglichen Besitz des Menschen sah er als vergänglich an und daher nicht tauglich für dauerhaftes Glück. 
Daher sei der Mensch von Gott abhängig, da dieser das einzige Wesen sei, dass unabhängig von Einflüssen und unvergänglich sei und der Mensch brauche die Weisheit, um sich von Verführungen zu entfernen. 
Auch Luther sah den Menschen als von Gott abhängig an in Bezug auf das persönliche Glück, welches durch Versuchungen wie den Ablasshandel verstärkt würde. 
In der modernen Philosophie wurden die Positionen der Antike erneut aufgegriffen, was zur Folge hatte, dass sich Ansätze der antiken Glücksvorstellungen dort wieder finden lassen.
 
Immanuel Kant betrachtet das Glück als Folge rationaler Handlungen. 
Für ihn kann Glück nur dann entstehen, wenn sich das Handeln des Menschen an gesellschaftlichen Normen und Gesetzen orientiert und nicht an Wünschen und Gefühlen. 
Das Handeln sieht er als untrennbare Kausalität der Ereignisse bzw. als Ereigniskette an, die von außen durch seine Triebe und Neigungen bestimmt werden. 
Folglich sieht er die Glückseligkeit als eine Befriedigung dieser menschlichen Bedürfnisse an. 
Kant erachtet die Sittlichkeit menschlichen Handelns als Grundvoraussetzung dafür, glücklich zu sein und lehnt eine reine Zugrundelegung der Naturbetrachtung ab. 
Für ihn muss darüber hinaus eine schöpfende Instanz zur Sittlichkeit beitragen, indem er den sittlich handelnden Menschen den Weg zum Glück eröffnet.

John Stuart Mill hingegen sieht vor allem den Nutzen in seiner Glücksbetrachtung als zentral an. 
Er greift dabei die epikureischen Lehren auf, indem auch er feststellt, dass der Mensch zwischen Handlungen steht, die Lust erzeugen und solchen, die Lust behindern. 
Erstere sieht er als moralisch korrekt, letztere analog als moralisch schlecht an. 
Jedoch erweitert er das epikureische Weltbild, indem er lustvolle Handlungen ausdifferenziert nach solchen, die er quantitativ sind und jenen, die eher qualitativ sind. 
Dadurch gebe es Handlungen, die öfter vollzogen würden, um Lust zu generieren und solche die seltener stattfinden würden. 
Daraus schließt Mill, dass diese beiden in Relation nicht das selbe Maß an Lust enthalten können, so dass bestimmte Handlungen anderen vorgezogen werden können. 
Bei dieser Argumentation sieht Mill den Menschen stets im Kontext der Nützlichkeit, derer er unterworfen ist und die er als höchstes Ziel betrachtet. 
Das Glück ist laut Mill kein Zustand, der dauerhaft aufrecht erhalten werden kann, sondern das Leben werde bestimmt durch Zeitspannen der Lust und solche der Unlust. 
Dieser Zustand müsse von gesellschaftlichen Vorgaben und Gesetzen unterstützt werden, so dass auch der Gesellschaft eine Verantwortung für das personifizierte Glück zukommt.

Friedrich Nietzsche schlussendlich grenzt sich ab von Kant und Mill, da das Glück aus seiner Sichtweise etwas grundtypisch menschliches sei und ihm innewohne. 
Um das Glück erlangen zu können, gelte es für den Menschen drei Säulen in Einklang zu bringen: 
Die Gewohnheit solle dazu führen, durch routinierte Handlungen die persönliche Lust aufrecht zu erhalten, die Schönheit des Glücks, dass Nietzsche im Gegensatz zu Mill eben doch als etwas Dauerhaftes begreift und nicht als repetierende Momente und abschließend den Unsinn, der den Menschen durch Freude von seinen Zwängen befreien solle.

Es lässt sich also entdecken, dass sich die Glücksvorstellungen im Laufe der Jahrhunderte stark verändert haben und antike Positionen Eingang in die moderne Philosophie gefunden haben.
\newpage

%% Inhaltsanalytische Kriterien und zentrale Fragestellungen der Untersuchung
\section{Inhaltsanalytische Kriterien und zentrale Fragestellungen}

Nachdem die theoretische Grundlage zum Glücksbegriff im philosophischen Diskurs gelegt sowie eine Einordnung des Philosophierens mit Grundschülern in den didaktischen Zusammenhang des Rahmenplans Sachunterricht des Landes Rheinland-Pfalz und den Perspektivrahmen Sachunterricht des GDSU vorgenommen wurde, muss im Vorfeld der sich anschließenden Analyse geklärt werden, nach welchen Kriterien das vorliegende Transkript untersucht wurde und welche Fragestellungen besondere Beachtung finden sollen.

Im Fokus stehen dabei fünf zentrale Sichtweisen der Kinder auf das Glück: 
Es wird beleuchtet, inwiefern das Glück für die Schülerinnen und Schüler im Zufall besteht. 
Dabei wird zu klären sein, ob Glück nur in wünschenswerten Zufallssituationen auftreten kann oder aber auch in Zufällen, die weniger wünschenswert verlaufen. 
Darüber hinaus wird an dieser Stelle besonders auch auf das sogenannte \glqq Glück im Unglück\grqq{} verwiesen. 
Dem gegenübergestellt lassen sich im Gespräch auch Passagen finden, die dem Glück als Zufall widersprechen. 
Vielmehr kann das Glück aus dieser Betrachtung heraus auch als Resultat eigener Anstrengung verstanden werden.
Daher lässt sich sagen, dass die Frage, inwiefern solch konträre Glücksbegriffe den gleichen Gegenstand \glqq Glück\grqq{} darstellen sollen, von großer Bedeutung ist. 
Hinzu kommt die Frage, ob für ein Glück durch persönliche Leistungen eine Art Bestätigung durch die Mitmenschen notwendig ist oder nicht.

Eine weitere wichtige Kategorie umfasst die zwischenmenschlichen Beziehungen, zu denen sich die Schülerinnen und Schüler in Bezug auf das Glück äußern. 
Dabei werden einerseits unter dieser Kategorie sowohl Gedankengänge in Bezug auf Familie und Freunde in den Blick genommen, andererseits aber auch die Fragestellung zu klären sein, inwiefern solch geartete Beziehungen für das subjektive Glücksempfinden von Nöten sind oder nicht. 

Auch das Glück als emotionale Regung bzw. als Gefühl soll in der inhaltlichen Auseinandersetzung mit den Positionen der Kinder behandelt werden. 
Eine zentrale Frage dieser Kategorie wird sein, inwiefern arme Menschen fähig sind, Glück zu empfinden trotz ihrer misslichen Lage. 
Des Weiteren werden die Schüler in Textpassagen, die dieser Kategorie zugeordnet wurden, sich über die Art des Glücksgefühls austauschen und darüber, ob dieses Gefühl durch materielle Dinge ausgelöst werden muss oder von materiellen Abhängigkeiten als autonom angesehen werden kann. 
Die letzte Kategorie besteht im Glauben der Kinder an Glück: 
Helfen Glücksbringer, um Glück zu haben? Welche Arten von Gegenständen können Glücksbringer werden? 
Können Menschen Glücksbringer für andere Menschen sein?

Diese Formen des Glücksverständnisses lassen sich auch im Kontext philosophischer Vorstellungen, welche zuvor bereits erläutert wurden, betrachten. 
Dabei soll die Frage im Fokus stehen, inwiefern sich philosophische Richtungen wie z.B. der Epikureismus oder der Utilitarismus in Ansätzen bei den Aussagen der Schülerinnen und Schüler wiederfinden lassen. 
\newpage


%% Analyse des Gesprächs mit Grundschulkindern der Klasse 4 an der Grundschule Kesselheim
\section{Analyse des Gesprächs mit Grundschulkindern an der GS Kesselheim}

Für die Analyse des transkribierten philosophischen Unterrichtsgesprächs wurde die Methode der qualitativen Inhaltsanalyse gewählt. 
Der nachfolgende Absatz soll sich daher mit der Methodik der qualitativen Inhaltsanalyse auseinandersetzen. 
Zudem soll die Ausgangssituation der Tonaufnahme, die der Transkription zugrunde liegt, näher erläutert werden und die Transkriptionsregeln, nach denen dieses Transkript angefertigt wurde, herausgestellt werden. 
Anschließend werden die inhaltsanalytischen Erkenntnisse, die aus dem Transkript entnommen werden konnten, dargelegt und in einem weiteren Schritt mit den zu Beginn vorgestellten philosophischen Positionen in Beziehung gesetzt.

Schlussendlich sollen die gewonnen Ergebnisse dann final zusammengestellt werden.


\subsection{ Methodik}

Die qualitative Inhaltsanalyse stellt eine der wesentlichsten Methoden zur Analyse von Texten dar. 
Ihr Ziel ist es, \glqq Kommunikationsinhalte, die in Form von Texten vorliegen, wissenschaftlich zu analysieren.\grqq{}\cite[S.\,20]{WK07}
Jedoch lässt sich der Begriff \glqq Inhaltsanalyse\grqq{} nicht als bloße Auseinandersetzung mit den Inhalten von Kommunikation beschreiben. 
Mayring stellt fest, dass sich eine Begriffsklärung der Inhaltsanalyse bereits im wissenschaftlichen Diskurs als schwierig erwies. 

Während Jürgen Ritsert die Inhaltsanalyse als \glqq ein Untersuchungsinstrument zur Analyse des \glqq gesellschaftlichen\grqq{}, letztlich des \glqq ideologischen Gehalts\grqq{} von Texten\grqq{}\cite[S.\,11]{JR07} betrachtet und sie damit mehr in den gesellschaftlich--ideologischen Kontext als in den inhaltlichen rückt, beschränkt sich Alfred L. George in seiner Definition lediglich auf die Absichten, die der Sprechende mit seiner Aussage zum Ausdruck bringt: 
\glqq Kurz, Inhaltsanalyse wird verwendet als ein diagnostisches Instrument, um spezifische Schlußfolgerungen über bestimmte Aspekte des zielgerichteten Verhaltens (purpose behavior) des Sprechers zu ziehen.\grqq{}\cite[S.\,11]{AG07}
Mayring fasst die Beschaffenheit der Inhaltsanalyse in sechs Punkten zusammen: 
Er betont, dass die Inhaltsanalyse sich mit Kommunikation beschäftigt und sich diese meist in Form von Sprache niederschlägt. 
Er weist jedoch ausdrücklich darauf hin, dass auch Musik, Bilder, etc. zu einer analytischen Betrachtung herangezogen werden können. 
Daran angeschlossen liegt diese Kommunikation fixiert vor, das heißt, dass sich die Inhaltsanalyse mit Kommunikation auseinandersetzt, die in Texten, Bildern, Noten, etc. festhalten wurde, was er als symbolisches Material bezeichnet\cite[S.\,12]{PM07}.

Die Inhaltsanalyse hat ferner nicht zum Ziel, das Material zu interpretieren. 
Vielmehr wird darauf Wert gelegt, den Untersuchungsgegenstand systematisch zu betrachten. 
Daran schließt er auch an, dass die Inhaltsanalyse einer bestimmten Regelhaftigkeit folgt. 
Durch diese festen Regeln werde erreicht, dass jedermann eine solche Analyse nachvollziehen und hinterfragen könne. 
In die Analyse fließt auch der Bezug zur Theorie mit ein, der bei der Inhaltsanalyse für Mayring einen wichtigen Bezugspunkt darstellt. 
Daher ist es auch nicht Gegenstand der Analyse, einen Text zu referieren, sondern sie \glqq analysiert ihr Material unter einer theoretisch ausgewiesenen Fragestellung; die Ergebnisse werden vom jeweiligen Theoriehintergrund her interpretiert und auch die einzelnen Analyseschritte sind von theoretischen Überlegungen geleitet.\grqq{}\cite[S.\,12]{PM07}
 Der sechste Punkt besteht nach Mayring im Endeffekt darin, dass die Inhaltsanalyse das Material nicht seiner selbst willen untersucht, sondern, dass sie die Ausfindigmachung von Rückschlüssen auf Kommunikation erreichen will. 
 
Ihr Fokus liegt dabei nicht auf der Häufigkeit des Auftretens bestimmter sprachlicher Ausdrücke, wie es die quantitative Inhaltsanalyse zum Ziel hat. 
Die qualitative Inhaltsanalyse untersucht, welche inhaltlichen Aussagen in einer Textpassage geäußert werden und will sich diesen Inhalten wissenschaftlich nähern. 
Knapp sieht zwischen quantitativer und qualitativer Inhaltsanalyse ein Spannungsverhältnis, dass er zwischen dem Verstehen von Texten und der Objektivität von Analyseverfahren verortet. 
Er schreibt: \glqq Mit dem Spannungsverhältnis zwischen dem Verstehen einerseits und der Objektivität andererseits wird eine Grundproblematik der Inhaltsanalyse sichtbar. 
Sie spiegelt sich in der Kontroverse zwischen qualitativer und quantitativer Inhaltsanalyse, wobei erstere eher das Verstehen, letztere eher die Objektivität des Analyseverfahrens beansprucht.\grqq{}\cite[S.\,20f]{WK07}
Knapp stellt weiterführend fest, dass die Inhaltsanalyse zwei weitere wichtige Kriterien erfüllen muss, die er als Validität und Reliabilität bezeichnet. 
Die Reliabilität bezeichnet , dass unterschiedliche Forscher zu unterschiedlichen Zeitpunkten zum selben Ergebnis kommen müssten. 
Validität wiederum bezeichnet die Gültigkeit eines wissenschaftlichen Verfahrens. 
Es lässt sich demnach festhalten, dass die qualitative Inhaltsanalyse drei Kriterien erfüllen muss: Objektivität, Reliabilität und Validität.

Betrachtet man nun, was Texte -- also auch Transkripte von Unterrichtsgesprächen -- aus linguistischer Sicht letztlich sind, nämlich die Verschriftung von sprachlichem Handeln, liegt es nahe, die Methodik der qualitativen Inhaltsanalyse auch im Hinblick sprachtheoretischer Aspekte zu betrachten. 
Knapp nimmt in diesem Zusammenhang Bezug auf Ludwig Wittgenstein und dessen Zitat \glqq Die Bedeutung des Wortes ist sein Gebrauch in der Sprache\grqq{}\cite[S.\,21]{LW07}, woraus er den Schluss zieht, dass die Semantik von Wörter stets an soziale Handlungsweisen gebunden sind. 
Für die Inhaltsanalyse bedeutet das, dass kein Text eine allgemeingültige Bedeutung hat, sondern dass verschiedene Leser von Texten auch abweichende Inhalte aus diesen Texten ziehen werden.  
Zudem ist die Extrahierung von Textinhalten auch abhängig vom Vorwissen des Lesers sowie davon, ob ein Text zum ersten Mal gelesen wird oder ob er schon bekannt ist\cite[S.\,22]{WK07}.

Für eine inhaltsanalytische Betrachtung von Texten ist laut Knapp auch von Bedeutung, welche Aussagen in einem Text vorgenommen werden. 
So könne man aus der quantitativen Häufigkeit bestimmter sprachlicher Konstruktionen nicht direkt auf den Inhalt schließen, sondern man müsse den Inhalt kennen. 
Knapp illustriert diesen Zusammenhang sehr treffend anhand des Beispiels, dass ein Text, in dem die Worte \glqq Joschka Fischer\grqq{} und Berlin vorkommen, keinen hinreichenden Ansatzpunkt für die Bestimmung eines Themas des Textes geben. 
Analog dazu verhält es sich zum vorliegenden Transkript. 
Informationen wie das Umfeld der Erhebung, die Klassenstufe und die Gesprächssituation sind wichtige Informationen, die für das Verständnis und für eine fundierte, qualitative Inhaltsanalyse von Bedeutung sind. 
Es liegt daher auf der Hand, dass eine bloße Betrachtung der sprachlichen Ausdrucksweisen für ein substantiiertes Verständnis nicht ausreichend sind. 
Vielmehr muss der Sinn eines Satzes \glqq in einem aktiven und konstruktiven Prozess erschlossen werden, der mehr einbezieht als die vorliegende Äußerung.\grqq{}\cite[S.\,28]{WK07}

Werner Knapp weist neben dem Bezug auf die Beschaffenheit von Texten in Bezug auf ihren Inhalt sowie die Themen, die in ihnen entwickelt werden, in einem weiteren Schritt auf die Deutung sprachlicher Zeichen hin, die dazu beiträgt, eine wissenschaftliche Vergleichbarkeit von Analysen zu gewährleisten. 
Er nimmt dabei Bezug auf Hans-Jürgen Bucher  und Gerd Fritz, die in ihrem Beitrag \glqq Sprachtheorie, Kommunikationsanalyse, Inhaltsanalyse\grqq{} vier Aspekte nennen, die sie für wichtig erachten: 
Zunächst nennen sie das gemeinsame Wissen der Kommunikationsteilnehmer, welches u.a. das Wissen über eine Situation, in der gehandelt wird, umfasst. 
Ein weiterer liegt im Sequenzzusammenhang sprachlicher Handlungen, da die Bedeutungen von sprachlichen Handlungen auch von ihrer Positionierung abhängen können. 

Auch der thematische Zusammenhang ist von Signifikanz für das Verständnis, da Aussagen eines Textes immer im Kontext des Ganzen verstanden werden müssen. 
Wird dieser Zusammenhang nicht berücksichtigt, können Aussagen fehlgedeutet werden. 
Der letzte Aspekt liegt in der inneren Struktur von sprachlichen Handlungen begründet, die -- in Anlehnung an die Sprechakttheorie von John Searle -- den illokutionären Aspekt, der besagt, dass eine Äußerung stets eine Funktion hat, welche beispielsweise feststellend, bewertend, rechtfertigend etc. sein kann. 
Hinzu kommt der indem--Zusammenhang, welcher aussagt, dass eine Handlung durch Sprache vollzogen werden kann, indem eine andere vollzogen wird\cite[S.\, 137f]{HB80}.

Bucher und Fritz nennen zusätzlich Qualitätskriterien für Kommunikationsanalysen. 
So nennen sie zunächst das Prinzip der zusammenhängenden Betrachtung, das bedeutet, dass Handlungssequenzen betrachtet werden, statt isolierter Handlungen sowie Zusammenhänge statt einzelner Ausdrücke ausgewertet werden\cite[S.\,137f]{HB80}.
 Das zweite Prinzip ist das Prinzip der Explizitheit, das besagt, dass Regeln und Hintergrundannahmen formuliert werden sollen, um Interpretationen über die subjektive Meinung hinaus argumentativ stützen zu können. 
 Das dritte und letzte Prinzip beschreiben Bucher und Fritz als Prinzip der Reflexivität, nach dem der Forschende stets reflektieren müsse, welches Verständnis er von den Aussagen eines Textes hat und wie sich sein Sprachstil darstellt\cite[S.\,143]{HB80}.
 
 \newpage
 
 


\subsubsection{Gewinnung der Aufnahme}

Die Aufnahme, die zur inhaltlichen Analyse transkribiert wurde, entstand am 08. Dezember 2015 in der Grundschule Koblenz--Kesselheim. 
Sie wurde im Rahmen des Seminars \glqq Philosophieren mit Kindern im Sachunterricht\grqq{} im Wintersemester 2015/16 zwischen 8:10 und 8:40 in einem 4. Schuljahr der Grundschule Koblenz--Kesselheim von den Studierenden Martin Spoo und Pascal Gerigk erstellt. 

Das Seminar \glqq Philosophieren mit Kindern an der Grundschule\grqq{} fand im Kontext des Mastermoduls 10 im Studiengang Grundschulbildung für das Lehramt an Grundschulen an der Universität Koblenz-Landau, Campus Koblenz unter der Leitung von Prof. Dr. Heike de Boer statt. 
Es gliederte sich in einen theoretischen Teil, in dem die Studierenden mit den theoretischen Inhalten des Philosophierens mit Kindern vertraut gemacht wurden und in einen praktischen Teil, in dem die Studierenden eigene Erfahrungen mit philosophischen Gesprächen machen sollten.

An verschiedenen Grundschulen in Koblenz und zum Teil auch in anderen Grundschulen in Rheinland-Pfalz wurden dann von Gruppen zu je zwei Studierenden jeweils drei philosophische Gespräche geführt und diese mit Aufnahmegeräten aufgezeichnet. 
Anschließend wurden diese Aufnahmen nach den Transkriptionsregeln von GAT (Gesprächsanalytisches Transkriptionssystem), welche im Verlauf dieser Arbeit noch genauer geklärt werden, transkribiert, um den Studierenden einen Einblick in die Sprecherrolle als spätere Lehrerinnen und Lehrer zu geben. 

Schwerpunkt war dabei, dass die Studierenden das Gespräch -- im Gegensatz zum klassischen Unterrichtsgespräch -- nicht aktiv leiten sollten. 
Überdies sollten die Studierenden beachten, besonders offene Fragestellungen zu formulieren, um einen Lernprozess bei den Kindern in Gang zu bringen und um dualistische Antworten zu vermeiden. 
Außerdem ergab sich für die Studierenden die Gelegenheit, mit Schülern über philosophische Themen wie Freundschaft oder auch Glück ins Gespräch zu kommen und die Aussagen der Schülerinnen und Schüler auch im Hinblick auf die jeweiligen Klassenstufen hin zu untersuchen. 
So wurde den Studierenden nach den Unterrichtsbesuchen stets die Möglichkeit gegeben, sich über ihre gesammelten Erfahrungen miteinander auszutauschen.
Gegen Ende des Semesters wurde schließlich die Entwicklung der eigenen Sprecherrolle in den philosophischen Gesprächen reflektiert.

Das Transkript, das das Gespräch zwischen den Studierenden Martin Spoo und Pascal Gerigk und der Klasse beinhaltet, setzt in Minute 6:30 der Aufnahme ein, da die Studenten mit den Kindern zunächst über die Geschichte von Hans im Glück debattierten und unklare Begriffe mit den Schülern klärten. 
Diese Geschichte wurde als Einleitung für die Schülerinnen und Schüler in die Thematik genutzt.
Da diese Sequenz jedoch für die Untersuchung der Aussagen der Schülerinnen und Schüler irrelevant ist, wurde sie bei der Transkription des Gespräches ausgelassen.


\subsubsection{Transkriptionsregeln nach GAT}

Gegenstand der folgenden Analyse ist ein Basistranskript des vorhin erläuterten Gespräches, dass nach den Regeln des Gesprächsanalytischen Transkriptionssystems (GAT) für Basistranskripte erstellt wurde. 
Dieses System wurde im Jahre 1998 von einer Autorengruppe bestehend aus den Linguisten Margret Selting, Peter Auer, Birgit Barden, Jörg Bergmann Elizabeth Couper-Kuhlen, Susanne Günthner, Christoph Meier, Uta Quasthoff, Peter Schlobinski und Susanne Uhmann entwickelt, um ein einheitliches Untersuchungsinstrument für die forschende Erarbeitung von verschriftlichter gesprochener Sprache zu erhalten. 

Das GAT wurde nach bestimmten Kriterien entwickelt, die kurz erläutert werden sollen. 
Durch eine Gewährleistung der Ausbaubarkeit soll sicher gestellt werden, dass ein Transkript ohne vollständige Neubearbeitung ergänzt und verfeinert werden kann. 
Um auch der Lesbarkeit für Nicht--Linguisten gerecht zu werden, wurde auf die Verwendung von Darstellungen in phonetischer Schrift u. ä. bewusst verzichtet, welche jedoch je nach Anwendungsbereich des Transkriptes noch ergänzt werden können. 
Zudem wurde das System besonders ökonomisch geplant, indem jedem Transkriptionszeichen genau eine Bedeutung zugewiesen wurde und eine Robustheit über die verschiedenen Computersysteme hinweg erreicht, da in Transkripten nach den GAT-Regeln keine Sonderzeichen verwendet werden. 
Bei der Zuweisung der Transkriptionszeichen wurde darauf geachtet, diese nicht arbiträr -- also willkürlich -- zu setzen, sondern sie ikonischen Prinzipien folgen zu lassen. 
Schließlich wurden auch die Relevanz, die formbezogene Parametrisierung und die Kompatibilität mit anderen Systemen beachtet. 

Ziel des GAT ist es, sprachliche Handlungen zu veranschaulichen, die für die Forschung interessant sind oder deren Interesse für die Forschung nachgewiesen werden soll.
Dabei soll es verschiedene Merkmale von Gesprochenem miteinbeziehen, d.h. es sollen nicht allein Interpretationen im Text vorgenommen werden, sondern es sollen Interpretationen mit Berücksichtigung anderer Merkmale wie Tonhöhe, etc. mitberücksichtigt werden.
Daneben soll es zudem auch anschlussfähig sein an andere Transkriptionssysteme\cite[S.\,92f]{MS98}.

Allgemein besteht ein Transkript aus einem Transkriptkopf, der wichtige Informationen über die Herkunft des Transkriptes, den Aufnahmetag, den Ort, etc enthält und aus dem Gesprächstranskript selbst. 
Dieses Gesprächstranskript folgt grundsätzlich der zeitlichen Abfolge des Gespräches und es wurden Konventionen festgelegt, die ein solches Transkript erfüllen muss. 

Grundsätzlich ist darauf zu achten, einen systemunabhängigen Schriftyp zu verwenden. 
Daher wird in der Regel die Schriftart Courier in der Schriftgröße 10 verwendet. 
Außerdem wird das Transkript gänzlich in Minuskeln verfasst, da Majuskeln für die Akzentuierung bestimmter Passagen benötigt werden. 
Jede Zeile des Transkriptes wird zur besseren Lesbarkeit mit einer Nummer versehen, es werden drei Leerstellen gesetzt und es folgt die Bezeichnung des Sprechers. 
Nach erneuten drei Leerstellen folgt der eigentliche Transkriptionstext\cite[S.\,95]{MS98}.

Allgemein wird zwischen Basistranskripten und Feintranskripten unterschieden. 
Während Basistranskripte lediglich analytische Mindestanforderungen erfüllen müssen, werden in Feintranskripten unter anderem auch die Tonhöhen, Akzentuierungen oder Sprechgeschwindigkeiten betrachtet.
Jedoch lässt sich sagen, dass für eine qualitative Inhaltsanalyse ein Basistranskript für ausreichend erachtet werden kann, da letztlich nur die inhaltlichen Aussagen der Schülerinnen und Schüler von Bedeutung sind und nicht beispielsweise Tonhöhe oder Sprechgeschwindigkeiten.


\newpage


\subsection{Inhaltsanalyse}

Das Gespräch wird eröffnet durch die Frage der Studenten, was die Schülerinnen und Schüler unter dem Begriff Glück verstehen. 
Stefanie antwortet auf diese Frage nach einer Bedenkzeit von ca. 7 Sekunden für die Schüler darauf, dass Glück für sie in Situationen vorhanden sei, in denen sie getröstet würde. 
Damit richtet sie den ersten, definierenden Blick in Richtung eines Glücksbegriffes, der im Kontext zwischenmenschlicher Beziehungen steht. 
Da keine weiterführende Wortmeldung entsteht, entfaltet Stefanie eine weitere Dimension des Glücksbegriffs, den sie in zufälligen Ereignissen verortet, die für die Person, die in dieser Situation Glück empfindet, in einem wünschenswerten Abschluss einer Handlung besteht. 
Sie nennt in diesem Falle das Losen innerhalb der Klassengemeinschaft und verweist damit auf konkrete, in der Klasse vorherrschende Rituale. 

Mario knüpft an diese Perspektive der positiven Abfolge von Ereignissen an. 
Dabei nennt er das hypothetische Beispiel -- er verweist wortwörtlich darauf, dass ihm dies noch nicht selbst geschehen ist -- einer Person, die aus einer bestimmten Höhe herunterfalle und sich nicht festhalten könne. 
In diesem Kontext sei Glück, wenn sich diese Person nicht ernsthaft verletze oder andere Beeinträchtigungen infolge eines Sturzes zu erwarten habe. 
Darius weitet dieses Beispiel aus, da er konkret beschreibt, dass es Glück sei, sich in einer solchen Situation nicht den Arm gebrochen zu haben.
Er stimmt damit Marios Grundthese von Glück zu, die besagt, dass Glück in einer günstigen Abfolge von Ereignissen besteht, welche in diesem Fall die Vermeidung einer Verletzung zur Folge hat. 
Die Antwort des Studenten, dass es sich bei solchen Situationen um Glück im Unglück handele und die die Antworten von Mario und Darius nochmals resümierend auf den Punkt bringt, wird indes von den besagten Schülern aber auch von den anderen Mitschülerinnen und Mitschülern kommentarlos hingenommen.

Aysche hat im Gegensatz zu den Ansichten von Stefanie, Mario und Darius, die offensichtlich das Glück vor allem mit Zufall, aber auch -- im Falle von Stefanie -- mit der persönlichen Beziehung zwischen Menschen verbinden, eine vollkommen andere Ansicht, wie man sich dem Glücksbegriff nähern könne. 
Sie sieht das Glück als Ergebnis von eigenen erreichten Zielen und Leistungen und weist dabei auf den Sachverhalt hin, dass Glück für sie darin bestünde, dass sie gute Noten erreiche. 
Aysche argumentiert folglich aus einem starken Leistungsverständnis heraus und es lässt sich weiter feststellen, dass sie die Kernaussage des bekannten Sprichwortes \glqq Jeder ist seines Glückes Schmied\grqq{} für ihre Sicht auf das Glück heranzieht. 
Stefanie geht bei ihrer erneuten Wortmeldung nicht auf diese neugewonnene Perspektive von Aysche ein und erläutert erneut das Glück in Form des Kontaktes von Menschen. 
Wenn man in schwierigen Situationen für andere da sei und für sie sorgen würde, dann könnte folglich eine Person, die in einer solchen Situation ist, wieder Glück erfahren und empfinden. 

Während sich die Schüler zunächst also über die Beschaffenheit des Glücks austauschen und bereits erkennen, dass das Verständnis von Glück auch immer eine subjektiv dominierte Komponente in sich trägt, geht Stefanie nun darauf ein, dass es einen Dualismus von Glück und Pech geben müsse.
Sie zeichnet das Pech als klaren Gegenspieler zum Glück und untermauert diese These mit einem Beispiel. 
Demnach könne Pech für eine Person der Verlust eines wichtigen Gegenstandes oder etwas Vergleichbarem sein, den man für eine andere Person erarbeitet oder hergestellt hat. 
Ob es dabei unter Umständen um Geschenke geht, kann nicht abschließend aus dem Transkript heraus geklärt werden. 
Jedoch lässt sich die Vorstellung von Pech als Verlust von etwas klar herausstellen. 
Auf diesen Antagonismus zwischen Glück und Pech geht sie noch tiefergehend in Zeile 30-32 ein, da sie festhält, dass von ihrem Standpunkt aus alles das zum Glück gezählt werden könne, was \glqq richtig passiert\grqq{} und jegliches Geschehen, bei dem \glqq was schiefläuft\grqq{} zum Pech gezählt werden könne.
Mit ihren Formulierungen trifft Stefanie den Kern dieses Gegensatzes zwischen Ereignisketten, die erfolgreich oder wünschenswert verlaufen und solchen, die nicht das erwünschte Ergebnis zutage fördern. 

Stefanies Gedankengang zum Verhältnis von Glück und Pech ergänzt Mario mit einem Alltagsbeispiel aus dem Klassenkontext. 
Er beschreibt dazu eine Gesprächssituation im schulischen Unterricht, in der ein Schüler von der Lehrperson eine Frage gestellt bekommt und die zugehörige Antwort nicht nennen kann. 
Grundsätzlich würde man dieses Ereignis noch nicht per se als Pech bezeichnen. 
Mario legt jedoch Wert darauf, dass das Pech in diesem Falle darin bestünde, dass der Schüler grundsätzlich die Fragen des Lehrers bzw. der Lehrerin beantworten könne und er durch das fehlende Wissen um die passende Antwort zu ausgerechnet dieser Frage in einer vom Pech bestimmten Lage sei.  
Er spielt gewissermaßen wieder auf die Rolle des Zufalls beim Glücksbegriff an, der in diesem Falle jedoch eine andere Qualität hat als im Falle des Losens, dass Stefanie zu Beginn des Gespräches erläutert. 
Darius geht anschließend an Marios Ausführungen über das Pech im unterrichtlichen Kontext erneut auf das Pech als Verlust von etwas ein und nennt dazu mit dem Verlust eines teuren Gegenstandes ein weiteres Alltagsbeispiel.
Allerdings lässt sich sein Verlustverständnis in Bezug auf das Pech vor dem Kontext von materiellen Verlusten sehen während Stefanie im Folgenden auf den Verlust in Form eines verpassten Fluges hinweist. 
Konnten zu Beginn bereits Unterschiede in der Definition des Glücks festgestellt werden, so lassen sich auch Differenzen im Verständnis des Pechs ablesen. 
Während Stefanie das Pech zunächst allgemein fasst, sieht sie wenige Momente später das Pech vor allem im Verlust von wichtigen Gegenständen. 
Dieser Ansicht schließen sich Mario und Darius in der Folge an.

Auf die Frage, was man brauche, um glücklich zu sein, antwortet Stefanie erneut mit einem Verweis auf die menschliche Komponente des Glücksbegriffes, die sie bereits am Anfang des Gespräches schildert. 
Man brauche zum Glück vor allem die Familie -- sie nennt dazu noch den Hund -- und die Freunde. 
Auf das Verhältnis geht sie in Form eines konkreten Beispiels erneut ein, daher lässt sich sagen, dass ihr dieser Aspekt besonders wichtig zu sein scheint. 
Sie beschreibt, dass es für einen Menschen wichtig ist, ein gesundes Verhältnis zu seiner Familie zu haben, wobei die Mutter in diesem Falle als Repräsentantin der Familie dient. 
Des Weiteren verweist sie darauf, dass die eigenen menschlichen Bedürfnisse vernachlässigt werden könnten, wenn man sich um eine Erhaltung der familiären Bindungen nicht bemühen würde.
Mario stimmt ihr zu, da er Stefanies Meinung, ausreichend Essen, ein Dach über dem Kopf sowie eine Familie brauche man zum Glück, wiederholt. 

Darüber hinaus ergänzt er diese Bedürfnisse, die der Mensch aus seiner Sicht hat und nennt die Geborgenheit als zentralen Bestandteil der emotionalen Befindlichkeit, die der Mensch benötige, um letztendlich zum Glück zu gelangen. 
Damit geht er auf einen Aspekt ein, den Stefanie bereits nannte. 
Darius hingegen betrachtet die Bedürfnisse des Menschen aus der Sicht derer, die eine solche Familie nicht besitzen. 
Konkret bezieht er sich dabei auf Kinder im Kinderheim, die im Hinblick auf familiären Zuspruch und die Beziehungen zu Freunden isoliert seien.
 
Stefanie skizziert in Zeile 69-77 eine Situation, die zeigt, warum das Verhältnis zur Familie wichtig ist und schlägt damit inhaltlich die Brücke zurück zur Unterscheidung zwischen Glück und Pech. 
Darin beschreibt sie, dass es eine merkwürdige Situation sei, über andere Menschen wie die Mutter oder die Freundin zu lästern, aber gleichzeitig mit ihr zu spielen. 
Für die lästernde Person bestehe das Pech darin, dass die betroffene Freundin die Wahrheit erfahren könnte und infolge dessen den Kontakt abbricht und damit auch die zwischenmenschliche Beziehung zu dieser Freundin endet. 
Auch an dieser Stelle wird Stefanies Bild vom Glück im Verhältnis von Menschen zueinander deutlich. 

Auch Darius nennt eine solche Situation, in der Freunde einer Person sich als falsche Freunde herausstellen, da sie Geheimnisse verraten können und das Opfer belügen.
Mario führt den Gedanken über die Freunde weiter und geht dabei erneut auf das Pech als den Verlust von etwas Positivem ein, indem er darlegt, dass auch dann Pech vorliegen kann, wenn man nicht wisse, ob man bestimmte Menschen -- auch wenn man diese noch nicht lange kennt -- jemals wieder sieht. 
Und auch Stefanie kennt ein solches Beispiel und beschreibt die Begegnung mit einem Kind bei einer Überraschungsfeier, von dem sie nicht wissen konnte, ob sich beide jemals erneut treffen werden.

Nachdem die Kinder sich zu ihren Gedanken, was ein Mensch braucht, um glücklich zu sein, geäußert haben, lenken die Studierenden die Aufmerksamkeit der Schülerinnen und Schüler auf den Gedanken zurück, inwiefern Glück in zwischenmenschlichen Beziehungen bestehe und ob man grundsätzlich auch alleine glücklich werden könne. 
Nachdem Mario noch vorsichtig bekundet, dass man durchaus Freunde brauche, um glücklich zu werden, ist es erneut Stefanie, die die Diskussion voranbringt. 
Sie ist der Ansicht, dass ein Mensch auch alleine Glück empfinden könne und untermauert diese These mit einem Denkmuster aus ihrer Lebenswelt. 
Sie erzählt von Situationen, in denen ein Mensch ein Ziel vor Augen hat und dieses erreichen will. 
Für dieses angestrebte Ziel benötige diese Person jedoch nicht zwangsläufig Freunde, sondern sie werde ausschließlich durch das Erfolgserlebnis glücklich. 
Hinzu kommt, dass sich auch in diesem Beitrag von Stefanie der Aspekt des Glücks als eigene Leistung wiederfindet, den sie somit erneut entfaltet.
Mario hingegen sieht in der Fähigkeit des Menschen, auch alleine glücklich zu werden, gewisse Einschränkungen als gegeben an. 
Er stimmt Stefanie zwar zu, dass man generell in der Lage sei, auch ohne den Bezug zu anderen Menschen Glück empfinden zu können. 
Jedoch sei dieses Glück nicht von der gleichen Qualität wie jenes, das durch andere Menschen hervorgerufen werde. 
Daraus zieht er den Schluss, dass ein Mensch, der sich mit einem Freund austauschen könne, glücklicher werden könne als ein Mensch, der diese Möglichkeit nicht habe. 
Daniela bezieht klar Stellung und vertritt den Standpunkt, dass man ohne die Mitmenschen definitiv nicht glücklich werden könne. 
Ohne die Beziehung zu anderen Menschen würde man vereinsamen und nicht zum eigentlichen Zustand des persönlichen Glücks gelangen können.

Darius bringt in seiner Sichtweise zum Ausdruck, dass die zeitliche Komponente eine entscheidende Rolle bei der Frage spiele, ob man allein glücklich werden könne. 
Er bekennt zwar zunächst, dass man pauschal nicht sagen könne, dass man Freunde zum Glück brauche. 
Allerdings bringt er dann eine wesentliche Einschränkung zur Sprache. 
Der Zustand des Alleinseins beschränkt sich in seiner Darstellung auf einen Zeitraum von wenigen Stunden, in denen man sich auch allein beschäftigen und so auch eine gewisse Art von Glück empfinden könne. 
Dazu nennt er wiederholt die Ansicht, dass es bei Freundschaften zwischen Menschen nicht auf die quantitative Dimension ankäme, sondern auf die qualitative. 
Das bedeutet, dass es wichtiger sei, Freunde zu haben, die unterstützen und zuverlässig sind, wenn es besonders von Nöten ist, als eine große Gruppe von falschen Freunden um sich zu scharen, die sich jedoch nicht durch Zuverlässigkeit und Verlässlichkeit auszeichneten. 
Er ergänzt dieses Bild im Folgenden noch und legt besonderen Wert auf ein Bild von Freunden, die hinter den anderen stehen und diese nicht auslachen. 
An diesem Punkt zeigt sich folglich, in welcher Form Darius und seine Mitschüler erneut Bezüge zwischen der Frage, ob man allein glücklich sein könne und ihrer persönlichen Lebenswelt herzustellen bereit sind.

Einen weiteren Punkt in der Diskussion um das Glück ohne Beziehung zu anderen Menschen zieht Stefanie, die illustriert, dass es Menschen gäbe, die bewusst auf solche Kontakte verzichten würden. 
Aus diesem Verzicht folgert sie, dass sie anderen Mitmenschen mit Antipathie gegenüberträten. 
Gleichzeitig nimmt sie an, dass sich solche, eher introvertierte Menschen, jedoch menschliche Nähe wünschen würden und daher eine  Situation, in der sie ohne Freundschaften auskommen müssten, bei ihnen auf Unverständnis stoßen müssten.
In dem Beispiel, das sie daraufhin nennt wird jedoch noch eine andere Lesart dieser Introvertiertheit, die sie skizziert, deutlich: \glqq ähh es gibt zum beispiel jetz so ein mädchen das nur an sich denkt un dann immer petzt.\grqq{}

Hier beschreibt Stefanie eher ein Mädchen, dessen Charakter sich durch egoistisches Auftreten auszeichnet und sich in Form des \glqq Petzens\grqq{}, also des Meldens eines Mitschülers oder einer Mitschülerin, auf deren schlechtes Betragen oder andere Verfehlungen sie die Lehrerin oder den Lehrer hinweisen will, zeigt. 
Daher lassen sich aus ihrem Beitrag heraus einerseits Menschen ohne Bedürfnis nach Bindung, aber auch solche, die sich unkollegial andern gegenüber verhalten, beschreiben. 
Diesen Gedanken des unkollegialen Verhaltens spinnt sie weiter und legt dar, dass dieses Kind jedoch plötzlich ein anderes frage, ob sie gemeinsam etwas spielen wollen würden. 
Letztlich stellt Stefanie fest, dass die Absicht des Mädchens, mit dem anderen, dass sie gemeldet hat, auf Ablehnung stoßen müsse und somit eine klare Einordnung ihres Verhaltens vornehmen müsse. 
Allerdings spielen beide Kinder schlussendlich doch gemeinsam. 

In Stefanies Ausführungen findet sich zudem ein Bezug auf eine vergangene Sitzung zum Thema \glqq Freundschaft\grqq{}. 
Dabei bezieht sie sich auf die Auseinandersetzung mit dem Kinderbuch \glqq Irgendwie Anders\grqq{} von Kathryn Cave, dass die Studierenden mit der Klasse zuvor bearbeitet hatten. 
Die Geschichte handelt von der Figur \glqq Irgendwie Anders\grqq{}, die mit anderen Tieren in Kontakt treten und diese als Freunde gewinnen möchte. 
Dabei wird sie jedoch von diesen immer wieder zurückgewiesen mit der Begründung, dass Irgendwie Anders die gleichen Tätigkeiten nicht in der Form tätigen könne, wie die anderen Tiere. 
Gegen Ende der Geschichte trifft Irgendwie Anders auf ein anderes Wesen, dass sich als \glqq Etwas\grqq{} vorstellt und seine Freundschaft möchte. 
Zunächst schickt Irgendwie Anders das Etwas weg, ehe es erkennt, dass die beiden einander annehmen in ihrem Anderssein und schließlich Freunde werden.

Stefanie überträgt ihre Position auf die Geschichte von Irgendwie Anders und bekundet, dass ein solcher Umstand der Intoleranz nur schwer für die betroffene Person zu bewältigen sei. 
Letztendlich bringt Stefanie zum Ausdruck, dass Freunde auch für sie wichtig sind, um Glück zu erfahren. 
Dadurch macht sie die Beziehung von Freundschaft und Glück abermals klar.
Mario greift Stefanies Bezug zur Geschichte von Irgendwie Anders erneut auf und stellt klar, dass Irgendwie Anders in der Geschichte alleine nicht zum Glück finden konnte. 
Dieses Fazit der Geschichte überträgt er auf ihr Beispiel aus dem Sportunterricht und widerspricht ihrer These, dass man in einer solchen Lage ohne die Mitmenschen Glück empfinden könne. 
Dabei beruft er sich darauf, dass wahrscheinlich Stefanies Freunde oder auch die Lehrperson zu diesem Erfolg beigetragen hätten, indem sie ihr etwas beigebracht oder ihr Mut gemacht hätten. 
Für Mario besteht das eigentlich Glück in der Bestätigung der Anderen, die ihre Wertschätzung über eine Leistung wie das Bockspringen, das er konkret nennt, zum Ausdruck bringen. 

Ohne die Mitmenschen sei man nicht in der Lage, seine eigenen Erfolgserlebnisse zu teilen und so ein sich entwickelndes Glücksgefühl maximieren zu können. 
Den Aspekt des Teilens von positiven Erlebnissen ergänzt Mario schließlich erneut mit einem Beispiel aus seiner Perspektive, in dem er beschreibt, dass wenn man eine spannende Sendung im Fernsehen gesehen hätte, sich auch stets ein gewisses Mitteilungsbedürfnis ergeben würde, die Inhalte der Fernsehsendung mit jemanden teilen zu können. 
Damit lässt sich sagen, dass Mario offenbar der Auffassung ist, dass sich das Glück nicht bloß durch besonders erfolgreiche Phasen im Leben einstellen kann, sondern stets auch mit dem Zuspruch des Umfeldes in Zusammenhang steht. 
Daher ist aus seiner Sicht das Vorhandensein von Freunden und Bekanntschaften elementar für sein Verständnis von Faktoren, die für das Glücksempfinden wesentlich sind.

Nachdem sich die Schülerinnen und Schüler zunächst über eine Begriffsbestimmung, was Glück sein könne und über die Frage ausgetauscht haben, welche Dinge nötig seien, um Glück in seinem Leben empfinden zu können, greifen die Studierenden den emotionalen Aspekt des Glücks auf, indem sie die Frage in den Raum stellen, wie sich Glück anfühlen könnte und woher die Schülerinnen und Schüler wissen könnten, dass sie glücklich seien. 
Darius beschreibt das Gefühl des Glücks näher, indem er beschreibt, dass es für ihn aus dem Herzen komme. 
Man sei zufrieden, was sich durch herzhaftes Lachen ausdrücken könnte. 
Gleichzeitig unterscheidet er dieses Lachen von solchem, dass nicht aus einem tiefgehenden Glücksgefühl entspringe. 
Mario betrachtet das Glücksgefühl im Unterschied zu Darius eher aus einer materialistischen Ordnung heraus. 
Er sagt, dass man Glück fühlen könne, wenn sich persönliche Wünsche in einem bestimmten Rahmen erfüllen würden. 
In seiner Darstellung bezieht er sich darauf, dass das Glücksgefühl dadurch entstehen könne, dass man ein bestimmtes Kuscheltier bekommen würde, welches man sich vorher gewünscht habe. 
Darius knüpft an diese Vorstellung an und verweist darauf, dass sich ein solcher Wunsch unter Umständen auch nicht erfüllen kann. 
Da der darauffolgende Teil seines Beitrages auf der Aufnahme nicht klar verständlich war, lässt sich unglücklicherweise nicht feststellen, welche Schlussfolgerung Darius an dieser Stelle aus dieser Feststellung ziehen wollte. 

Stefanie stellt in Zeile 176-181 einen erneuten Rückbezug auf Marios Position her, indem sie erklärt, dass für sie Glück empfinden bedeute, die eigenen Grundbedürfnisse wie Nahrung und eine Heimat befriedigen zu können. 
Damit verbindet sie die menschlichen Bedürfnisse mit dem Glücksempfinden, woraus sich ableiten lässt, dass Stefanie offenbar ein Bild vom Glück als Emotion hat, dass geprägt ist von der Sorge um das eigene Wohlbefinden. 
Erst wenn der Mensch sein Auskommen gesichert habe, könne er wunschlos glücklich werden. 
Das heißt, dass für Stefanie das persönliche Glück an die Bedürfnisse des Körpers gebunden ist. 
Daraus entwickeln die Studierenden eine weiterführende Fragestellung, dass arme Menschen nach dieser Betrachtung nicht abschließend glücklich werden könnten, da sie ihre Bedürfnisse nur in beschränktem Maße erfüllen könnten durch den Umstand, dass sie durch den Mangel an finanziellen Ressourcen dazu nicht in der Lage seien. 

Darius widerspricht dieser Annahme zum Teil, da er betont, dass arme Menschen nicht grundsätzlich unglücklich seien. 
Diese Aussage präzisiert er mit einer Unterscheidung zwischen armen Menschen und Obdachlosen, indem er besonders auf jene Menschen verweist, die auf der Straße leben.
Es lässt sich annehmen, dass Darius davon ausgeht, dass Obdachlose demnach unglücklicher seien. 
Auch Stefanie schließt sich Darius' Auffassung an, dass arme Menschen nicht per se unglücklich sein müssten und es auf die Situation ankommen würde, in der sich diese Menschen befänden. 
So sei ein Obdachloser durchaus in der Lage, glücklich zu sein, wenn er die Nähe anderer Menschen, die ihn mit Almosen oder anderen Spenden unterstützen, spüren würde. 
Es zeigt sich wiederholt, dass das Glücksverständnis von Stefanie sehr eng mit der Wahrnehmung durch andere Menschen zusammenhängt. 
Dieser Eindruck wurde bereits durch Stefanies Aussagen, was der Mensch zum Glück benötige, bestätigt.

Darius geht davon aus, dass es auch Menschen gibt, die das Bedürfnis Anderer, armen Menschen zu helfen, für sich ausnutzen. 
Er beschreibt, dass sich diese bewusst als Hilfsbedürftige ausgeben würden, um die Hilfsbereitschaft vieler als Einnahmequelle auszunutzen und er verweist auch besonders auf Menschen, die anderen gewissermaßen helfen müssten, da sie sich in einer moralischen Verpflichtung sehen würden. 
Was die Fähigkeit armer Menschen, glücklich zu sein, angeht schließt er sich der Sichtweise Stefanies an, und wiederholt diese, dass es diesen Menschen ausreichen könne, von anderen einige Cent oder etwas zu Essen zu bekommen und nennt dabei auch einen konkreten Fall aus Koblenz-Kesselheim. 
Dem Aspekt des Spendens wendet sich auch Stefanie zu, die darauf verweist, dass man in Märkten auch mit wenig Geld Lebensmittel bekommen könne, die sich in einem bestimmten finanziellen Rahmen bewegen. 
Daraus lassen sich verschiedene Lesarten entwickeln. 
Einerseits kann sie damit ausdrücken, dass bereits wenig Geld ausreichen kann, damit sich Hilfsbedürftige selbst versorgen können, da diese von den Märkten aus Solidarität auch ohne den vollen Preis herausgegeben würden. 
Nach ihrem Bild von Glück könnten diese Menschen dann glücklich werden ungeachtet ihrer Armut. 
Sie könnte jedoch andererseits auch gemeint haben, dass sich Hilfsbedürftige von den Almosen dann etwas kaufen könnten, wenn andere im Markt bereit wären, weiteres Geld hinzuzugeben. 

Gegen Ende des Gespräches lenken die Studierenden die Aufmerksamkeit der Schülerinnen und Schüler auf Glücksbringer. 
Die Kinder sollen sich dazu äußern, ob sie Glücksbringer besitzen, ob sie Arten von Glücksbringern kennen und ob sie daran glauben, dass Glücksbringer das Glück fördern können. 
Darius antwortet darauf mit einer zunächst zurückhaltenden Zustimmung, was er mit dem Einschub \glqq eigentlich\grqq{} unterstreicht. 
Diese Annahme begründet er, indem er erzählt, mithilfe des Glaubens an seinen Glücksbringer dafür gesorgt zu haben, dass es am nächsten Tag nicht geregnet habe. 
Darius äußert damit offenkundig, dass er der Überzeugung ist, dass der Glaube an einen ausgewählten Gegenstand, der dann als Glücksbringer dient, Ereignisse so verändern kann, dass sie den eigenen Wünschen entsprechen. 
So gesehen ordnet er dem Menschen die Fähigkeit zu, Dinge durch seine Wünsche verändern zu können. 
Daran schließt er eine Situationsbeschreibung an, in der ein Mann bei \glqq Wer wird Millionär\grqq{} bei einer Frage über die Anzahl von Steinen, die ein Zauberwürfel besitze, mit Glück -- wie Darius es beschreibt -- die richtige Antwort gefunden habe. 
Hier stellt er einen Rückbezug zu Stefanies Ausführungen her, die feststellte, dass Glück dann vorliegen würde, wenn gewisse Ereignisketten zu Gunsten einer Person ausfallen. 
Er löst sich gleichzeitig aber inhaltlich vom Subthema der Glücksbringer.

Stefanie lenkt den Fokus wieder zurück zur Frage nach den Glücksbringern und bezieht sich ebenfalls auf den Anfang des Gespräches, als sie feststellte, dass sie Glück beim Losen hatte. 
Dieses erlebte Glück stellt sie in einen direkten Kausalzusammenhang zu ihrem Glücksbringer, den sie kurz zuvor laut eigener Aussage erhalten hatte. 
Gleichzeitig macht sie durch ihr Seufzen, dass sie während ihrer Ergänzung, zuvor keinen Glücksbringer besessen zu haben, deutlich, dass für sie Glücksbringer offenbar wichtig sind, um Glück herbeiführen zu können. 
Darius wirft daraufhin ein, selbst keinen Glücksbringer zu besitzen, obwohl er zuvor davon gesprochen hatte, dass ihm ein Glücksbringer Glück gebracht habe. 
Daran angeschlossen führt Stefanie ihren Gedanken fort und bezieht sich erneut auf ihren Beitrag.

Aysche erzählt von ihrem Glücksbringer, den eine Freundin für sie im Urlaub angefertigt hat. 
Diese Freundin habe zusätzlich noch einen Text zu diesem Glücksbringer geschrieben und ihn ihr dann überreicht. 
Folglich entfaltet Aysche ein Schema von Glücksbringern, die dann wirksam sind, wenn sie eine persönliche Bedeutung für einen Menschen besitzen. 
In diesem Fall besteht diese Bedeutung in der Beziehung zwischen Aysche und ihrer Freundin. 
Auch findet sich hier erneut die zwischenmenschliche Beziehung als Ausprägung des menschlichen Glücks wieder, das in Form des Glücksbringers geteilt werden kann. 
Mario knüpft an diese Sicht auf Glücksbringer an und sieht im Kontext eines Gegenstandes die Bedeutung als Glücksbringer.
Er erläutert, dass er für seine Eltern zum Nikolaustag Geschenke in Form von Schneemännern und Sternen gebastelt und diese bei einer Adventsfeier verschenkt hatte. 
Da er offensichtlich vorher gezählt hatte, wunderte er sich darüber, dass der Schneemann, den er eigentlich ausgemustert hatte, noch bei den Geschenken lag. 
Es stellte sich heraus, dass Marios Mutter diesen Schneemann mit einem aufgemalten Herz und einem schwarzen Hut ergänzt hatte und so für Mario daraus ein Glücksbringer wurde. 
Auch hier manifestiert sich die Vorstellung der Schülerinnen und Schüler, die Aysche zuvor bereits äußerte, dass ein Glücksbringer durch seine Bedeutung entsteht.
	
Auch wenn bisher nur Bezüge auf materielle Dinge als Glücksbringer seitens der Schülerinnen und Schüler vorgenommen wurde, hält Darius dagegen, dass auch Menschen als Glücksbringer füreinander dienen können. 
So sei sein Mitschüler Mario ein Glücksbringer bei der Fussball-Weltmeisterschaft gewesen, da erst dann ein Tor gefallen sei, als Mario sich am gemeinsamen Verfolgen des Spiels beteiligt habe. 
Daraus zieht Darius den Schluss, dass Mario ein Glücksbringer gewesen sein muss, der durch seine Anwesenheit als Glücksbringer für Darius in der Lage war, die Ereignisse positiv beeinflussen zu können. 
Neben diesem Glücksbringer -- seinem Freund Mario -- berichtet Darius auch von einer Münze aus Kroatien, die für ihn ein Glücksbringer sei, ohne näher auf die Geschichte des Glücksbringers einzugehen. 

Stefanie hingegen beschreibt, dass ein sogenannter Traumfänger ihrer Freundin Celine ein Glücksbringer sei, da dieser Alpträume von ihr fern halten könne und sie das Gefühl habe, besser schlafen zu können, wenn dieser Traumfänger in ihrer Nähe sei. 
Zudem stellt sie eine interessante These auf. 
Sie geht davon aus, dass Glücksbringer mit etwas versehen würden, was sie zu Glücksbringern machen würde.
Dieses Wissen über die Beschaffenheit drückt sich auch in ihrer Ausdrucksweise aus, da sie sagt: \glqq die machen irgend son zeug rein das weiß ich.\grqq{} 
Sie drückt damit ihre bedingungslose Überzeugung darüber aus, dass es ein Geheimnis geben müsse, dass Glücksbringer zu Glücksbringern mache. 
Da sie diesen Gedanken nicht weiter ausgestaltet, ist folglich anzunehmen, dass sie selbst keine genaue Vorstellung darüber hat, um was es sich dabei genau handeln könnte. 

Die Studierenden bringen die Aussagen der Schülerinnen und Schüler nochmals auf den Punkt. 
Glücksbringer könnten grundsätzlich alles sein, d.h. sie sind unabhängig von einem Trägermedium. 
Wichtig sei nur, dass es sich dabei um etwas handele, was für die betreffende Person bedeutsam sei und an dessen Glück bringende Kraft sie glauben könne. 
Daher kann man sagen, dass das, was Stephanie über das Besondere an Glücksbringern sagt, vor allem Glauben an diesen Gegenstand besteht.
Daran gebunden sei auch immer der Kontext, der einen Gegenstand zum Glücksbringer mache. 
So könne für den einen Menschen ein vierblättriges Kleeblatt ein Glücksbringer sein und für einen anderen wiederum etwas vollkommen Anderes.

Stefanie bringt einen weiteren neuen Aspekt ein in Bezug auf Menschen, die Glücksbringer sein können. 
Sie beschreibt eine Begebenheit, in der sie selbst für sich ein Glücksbringer war und erzählt von einem Fahrradausflug, den sie mit ihrem Vater unternehmen wollte.
Allerdings lehnte ihr Vater dies aufgrund des zu diesem Zeitpunkt zu schlechten Wetters ab. 
Da Mario und Darius sie darauf aufmerksam machen, dass sie den Bezug zum Thema \glqq Glücksbringer\grqq{} noch nicht verstanden haben, erklärt sie das Beispiel erneut. 
Sie erläutert, dass sie ihren Vater letztendlich dazu hätte überreden können, den Ausflug zu unternehmen, was für sie an dieser Stelle ein Glückserlebnis darstellte. 
Demzufolge deutet sie ihre Überzeugungskraft gegenüber dem Vater so, dass sie selbst in der Lage war, die Situation zu verändern. 
Das passt auch zu der Grundannahme, die Darius zuvor äußerte, dass Glücksbringer in der Lage sind, Einfluss zu nehmen. 
So sieht sich Stefanie als Glücksbringer für sich selbst. 
Diese Interpretation wird auch von den Studierenden vorgenommen und für die Mitschülerinnen und Mitschüler kundgetan.
Am Ende des Gespräches verknüpft Darius seinen Glücksbringer mit einer selbsterbrachten Leistung, die sich als gewonnenes Fußballspiel vollzieht. 
Er bezieht sich demnach wiederholt auf die Glücksvorstellung des Glücks als eigene Leistung und verbindet dieses Glück in einer kausalen Kohärenz mit dem Gedanken an seinen Glücksbringer.

Auch Stefanie geht erneut auf diese Kausalkette ein und belegt diese anhand dessen, dass auch ihr der Glücksbringer geholfen habe. 
Sie bezeichnet es als Glück, dass ihr Vater mit ihr trotz zeitlichen Engpasses einen Kinofilm ansah, den Stefanie gerne sehen wollte und obwohl sie, um der Schulpflicht in ausreichendem Maße nachkommen zu können, darauf angewiesen sei, genug zu schlafen. Mario schließt das Gespräch mit einer erneuten Repetition ab und verweist im Kontext seines Basketballtrainings auf die Dimension des Glücks als Resultat der eigenen Anstrengung.

Die Schülerinnen und Schüler haben mit ihren zahlreichen Alltagsbezügen, welche an dieser Stelle dargestellt wurden, deutlich gemacht, was ihnen beim Verständnis des Glücksbegriffs besonders wichtig ist. 
Glück ist demnach zusammengefasst eine günstige Abfolge von zusammenhängenden Ereignisketten, die, unterstützt durch Familie und Freunde und -- je nach dem, welchen Fokus der Schüler oder die Schülerin setzt -- durch Zufall oder eigene Leistung zustande kommt. 


\newpage



\subsection{Vergleich ausgewählter Schüleraussagen mit philosophischen Positionen}

Nachdem der Inhalt des philosophischen Gesprächs zwischen den Studierenden und den Schülerinnen und Schülern inhaltsanalytisch betrachtet wurde, sollen besonders aussagekräftige Textstellen des Transkriptes mit den philosophischen Positionen in Beziehung gesetzt werden, welche zu Beginn der Auseinandersetzung mit dem Glück erarbeitet wurden.

In den Zeilen 8 bis 10 und 12 bis 15 beschreiben Mario und Darius das Glück als eine Handlungskette mit vorteilhaftem Ausgang und verbinden diese Grundthese mit einem Beispiel aus ihrer Lebenswirklichkeit, indem sie erläutern, dass es Glück sein könne, wenn man von einem Baum nicht ganz nach unten stürzen bzw. sich nach einem Sturz nicht gefährlich verletzen würde. 
Damit schließen sie sich der allgemeinen Definition des Dudens an, der das Glück als etwas bezeichnet, \glqq was Ergebnis des Zusammentreffens besonders günstiger Umstände ist\grqq{}\cite{D16}.
Darüber hinaus zeigt sich in ihren Ausführungen allerdings auch eine Tendenz dahingehend, dass Glück für die beiden Schüler anscheinend auch mit der Vermeidung von Leid bzw. Schmerz zusammenhängt. 

Daher liegt es nahe, dass Mario und Darius ein Bild vom Glück an dieser Stelle des Gespräches äußern, dass gewissermaßen eine Einordnung des Glücks als Lust oder als Freude und des Pechs als Leid oder Schmerz voraussetzt. 
Da sie jedoch nicht auf eine weitere Differenzierung von Glück auslösenden Ereignissen eingehen, lässt sich keine feste Zuordnung zu einer philosophischen Position vornehmen. 
Vielmehr kann gezeigt werden, dass Mario und Darius die Schmerzvermeidung, wie sie sowohl Epikur und Aristippos als auch John Stuart Mill beschreiben, in ihr grundsätzliches Glücksverständnis übernommen zu haben scheinen. 
Ergänzend lässt sich auch ein Bezug zur antiken Stoa herleiten, da sich in der Abwendung von möglichem Unheil auch auf gewisse Weise ein Selbsterhaltungstrieb erkennen lässt, der von dieser philosophischen Schule angenommen wurde. 

Auch sehen die beiden Schüler das Glück im Kontext der Abwendung einer Verletzung als etwas immaterielles an, wodurch sich Parallelen zur mittelalterlichen Glücksphilosophie erkennen lassen. 
Auch Augustinus betrachtete das Glück als etwas vom Materiellen Unabhängiges. 
Allerdings folgerte er aus dieser Beschaffenheit des Glücks, dass Gott allein das Glück durch seine Unabhängigkeit von äußeren Einflüssen befördern könne. 
Bleibt man in diesem Bild, so wäre die Schlussfolgerung, dass Mario und Darius nach der Interpretation von Augustinus einen Glückszustand beschreiben, der als Resultat die Vermeidung der Verletzung herbeigeführt hat und dieses Ereignis durch Gott gesteuert wurde. 
Allerdings werden an dieser Stelle auch die Grenzen dieser Betrachtungsweise deutlich, da die Aussagen von Mario und Darius keinen Hinweis auf eine Verknüpfung mit Gott beinhalten.

Auch in Zeile 20 wird das Glück als etwas Nichtmaterielles begriffen und dabei aber auch gleichzeitig ein neuer Schwerpunkt gesetzt. 
Für Stefanie liegt das Glück in der Beziehung der Menschen begründet und kann sich dem Menschen als Helfen in der Not und Fürsprache in diffizilen Situationen offenbaren.
 Stefanie entfaltet auf der einen Seite ein Glücksverständnis, dass als durch christliche Nächstenliebe mitgeprägt bezeichnet werden kann. 
 Wenn der Mensch bereit ist, sich auch den Schwachen und Hilfsbedürftigen zu widmen, kann er in der Folge Glück erfahren. 
 
 Auf der anderen Seite lässt sich ihr Glücksverständnis im Kontext des Utilitarismus von John Stuart Mill betrachten. 
 So stellt das Helfen in der Not anderer Menschen eine Handlung dar, die als moralisch korrekt eingeordnet werden kann, da sie die Lust bzw. das Glück desjenigen, der Hilfe leistet, verstärken kann. 
 Gleichzeitig will der Mensch im utilitaristischen Sinne aus seinen Handlungen Nutzen ziehen. 
 Dieser lässt sich im Kontext dieses Beispiels in der Erlangung persönlichen Glücks durch Hilfestellung erkennen. 
 Die Abstufung der Freuden, die Mill vornimmt, lässt sich an der vorliegenden Textstelle ebenfalls illustrieren. 
 Je nach dem in welchem Umfang die Unterstützung anderer Menschen stattfindet, so lässt sich daraus auch ableiten, von welcher Qualität die daraus resultierende Freude ist. 
 Hilft man beispielsweise jemanden beim Umzug, so wird sich das Glücksgefühl voraussichtlich nur in einem äußerst begrenzten zeitlichen Rahmen zeigen. 
 Hilft man jedoch einem Menschen in einer schwierigen Situation, in der er wenig finanzielle Ressourcen hat, so wird sich dieses Glücksgefühl unter Umständen auch sehr viel später noch einstellen.
 
Innerhalb des Gespräches stellen die Schülerinnen und Schüler auch einen klaren Gegensatz zwischen dem Glück auf der einen Seite und dem Pech auf der anderen Seite her. 
Stefanie bringt diesen Gegensatz mit ihrer Aussage aus Zeile 29 bis 32 auf den Punkt:
\glqq weil ähm danach kann man auch also das is halt eben anderster als (unverständlich, 3 sek) weil is glück is was halt wenn was richtig passiert weißte? (unverständlich, 5 sek) pech is jetz was wenn was schiefläuft. (5.0)\grqq{}

Stefanie bezieht sich an dieser Stelle erneut auf die Grundthese des Glücks als wünschenswerter Abfolge von Ereignissen. 
Zugleich benennt sie diesen Gegensatz auch in Form von Lust und Schmerz oder in ihren Worten \glqq wenn was richtig passiert\grqq{} und \glqq wenn was schiefläuft\grqq{} und knüpft damit erneut an Aristippos, Epikur und Mill an. 
Die Verschiedenartigkeit von Pech wird von ihren Mitschülerin in der Folge durch ihre Beispiele manifestiert, die das Pech auch im schulischen Kontext des Unterrichtsgespräches, dem Verlust eines wertvollen Gegenstandes oder in dem unglücklichen Ereignis eines verpassten Flugzeugs verorten. 

Damit verweisen die Schülerinnen und Schüler auf eine subjektive Wahrnehmbarkeit des Pechs und die Abhängigkeit der Interpretation von Pech vom Kontext der Situation. 
Gleichzeitig wird deutlich, dass die Schülerinnen und Schüler sich darüber bewusst sind, dass es neben der Qualität von Freuden, die sich nach Mill voneinander unterscheiden lassen, auch das Pech ist, dass sich in seiner Ausprägung unterschiedlich beschreiben lässt.

Auf die Frage, was man zum Glück benötigt, antwortet Mario, dass man vor allem ein Zuhause, Essen und eine Familie zum Glück brauche, weil man sich ohne diese Grundversorgung nicht wohl fühlen könne.
 Mario verortet das Glück folglich auch in dieser Befriedigung von Grundbedürfnisse. 
 Auch Immanuel Kant dachte im Zeitalter der Aufklärung ähnlich. 
 Er sah das Glück vor allem als Summe rationaler Entscheidungen des Menschen und unabhängig von emotionalen Regungen. 
 
Diesem Aspekt Kants schließt sich Mario ausdrücklich nicht an, da er das Wohlbefinden des Menschen hervorhebend nennt. 
Der Mensch möchte in dieser Situation die Selbsterhaltung erreichen und strebt deshalb die Erfüllung seines Verlangens nach Heimat, Nahrung und menschlicher Bindung an. 
Insofern findet sich mit dem Drang nach Selbsterhaltung auch ein Aspekt stoischer Lehre in Marios Aussage. 
Mit seinem Hinweis, dass man sich nicht wohl fühlen könne, ohne das nötigste zu besitzen, nimmt Mario erneut Bezug auf Kant, der ein sittliches Leben als Grundvoraussetzung für das Glück des Menschen ansah. 
Demnach kann ein sittliches Leben also nicht stattfinden, wenn die Grundversorgung eines Menschen nicht stattfindet und folglich kann dieser Mensch nach kantischer Lesart das Glück nicht erreichen. 
Auch Darius beruft sich auf diese Bedürfnisse des Menschen und weist darauf hin, dass es auch Menschen gebe, die keine Familie und keine Freunde besitzen würden. 

In einem weiteren Schritt sieht Darius das Pech im Umfeld der Freundschaft zwischen Menschen verortet, was er in seinem Statement auch deutlich macht: 
\glqq wie jetzt/ zum thema freunde, wenn man zum beispiel was wichtiges hat und äh/ und das dann einem sagt. und der das dann sofort der ganzen schule verpetzt, ähm (.) dann (.) und bricht man ja auch den kontakt ab und dann/ jaja, ich mach das nie wieder und dann machen die das zehn minuten später wieder. ((mehrere SuS lachen)) kann doch auch sein, dann verrät man ein geheimnis und dann sagen die das wieder.\grqq{}

An dieser Textstelle wird deutlich, dass Darius die Beziehung zwischen Menschen, die Stefanie in Zeile 20 als Glück beschrieb, aus der gegensätzlichen Perspektive betrachtet. 
Aus seiner Sicht liegt für eine Person, die anderen Menschen etwas anvertraut und dann von diesen hintergangen oder enttäuscht wird, an dieser Stelle eine Form von Pech vor. 
Folglich kann die Beziehung von Menschen also sowohl vom Glück als auch vom Pech bestimmt werden. 

Anhand von Darius' Beispiel lässt sich auch erneut die utilitaristische Lehre von John Stuart Mill miteinbeziehen. 
Wenn man im Beispiel bleibend davon ausgeht, dass jede Handlung des Menschen einen Nutzen haben muss, so kann sich dieser lediglich in der Handlung derjenigen Personen vollziehen, die das Geheimnis des Freundes verraten haben. 
Daher stellt sich nun die Frage, welchen Nutzen sie daraus ziehen könnten. 
Darüber lässt sich freilich nur spekulieren. 
Es kann jedoch angenommen werden, dass die Freunde dem Opfer schaden wollen, um ein höheres, nicht weiter zu bestimmendes Ziel zu erreichen. 
Nach Mills Definition müssten sie demnach glücklich werden können, da in seinem Bild jede Handlung moralisch korrekt ist, die Glück hervorruft. 
Kommen die Freunde jedoch in einer späteren Situation zu der Einsicht, dass ihr Verhalten falsch war, so dreht sich auch die Moralität der Handlung ins Negative. 
Daraus folgt, dass ihr Tun den Verrätern als moralisch falsche Handlung ihnen auf dem Weg zum eigenen Glück im Weg stehen wird. 
Es zeigt sich an dieser Stelle einer der zentralen Kritikpunkte am Utilitarismus Mills, der zulässt, dass eine moralisch falsche Handlung durch den größeren, zugeschrieben Nutzen für die Gemeinschaft als legitim angesehen werden kann.

Eine weitere aussagekräftige Textstelle findet sich im Rahmen der Frage, ob man allein glücklich werden kann oder nicht.
 An die Frage schließt sich zunächst der Kurzbeitrag Marios an, der bekundet, dass der Mensch andere Menschen brauche, um Glück zu empfinden bzw. glücklich zu sein. 
 Stefanie entgegnet daraufhin, dass sie auch alleine durch ihr Handeln glücklich werden könne und dabei nicht von anderen Menschen abhängig sei.
 
Stefanie greift in ihren Ausführungen indirekt auf das Glücksverständnis Friedrich Nietzsches zurück, der das Glück nicht als etwas von außen Einwirkendes betrachtete, sondern annahm, dass das Glück etwas typisch Menschliches sei, dass seiner Psyche innewohne. 
Auch Stefanie stellt klar, dass sie durch ihre sportliche Leistung das Glück erreichen könne, ohne dabei von äußeren Einflüssen durch Freunde oder andere Mitmenschen abhängig zu sein. 
Ebenfalls findet sich in ihrer Äußerung eine Parallele zu Aristoteles, der formulierte, dass der Mensch auf dem Weg zur Eudaimonia Teilziele zu erreichen suche, welche in diesem Beispiel im Glücksgefühl durch Erfolg besteht. 
Spinnt man diesen Faden weiter, so ist Stefanie nach Aristoteles auf einem fortgeschrittenen Weg zur Eudaimonia, da sie in ihrem Leben viel für den eigenen Erfolg tun möchte. 
Sie widerspricht aber auch gleichzeitig der Annahme Aristoteles', dass der Mensch durch äußere Faktoren abhängig sei und betont dies ausdrücklich.

Mario stimmt Stefanie zwar in dem Punkt zu, dass der Mensch auch allein glücklich sein kann, deutet jedoch auch an, dass dieser Zustand mithilfe der Mitmenschen besser zu erreichen sei. 
Wenig später, in den Zeilen 147-161, widerlegt er gar Stefanies Beispiel aus dem Sportunterricht und erläutert, dass sie in dieser Situation doch von anderen Mitmenschen unterstützt wurde. 
Damit spricht Mario im weitesten Sinne aus der Tradition des Aristoteles, der das finale Ziel der Eudaimonia eben darin erkennt, sich auch gesellschaftlich einzubringen. 

Der Mensch müsse akzeptieren, dass er nicht allein für sich sein Glück finden könne, sondern auch abhängig sei von den anderen Gütern, welche sich in diesem Zusammenhang auch in der Zuwendung durch Freunde oder nahestehende Personen allgemein erfassen lässt. 
Zusammenfassend kann man also festhalten, dass man laut Stefanie allein seine erfolgreichen Taten benötigt, um das Glück zu erlangen. 
Mario hingegen sieht auch die Beziehung zu anderen Menschen und das Einbringen in die Gemeinschaft als zentral an, um glücklich zu werden.

Ein weiterer wichtiger Aspekt der Frage nach dem Glück, zu der sich die Schülerinnen und Schüler äußern, liegt in der Sicht von Glück als Gefühl begründet. 
Grundsätzlich lässt sich feststellen, dass die Schülerinnen und Schüler augenscheinlich annehmen, dass es so etwas wie ein determiniertes Glücksgefühl gebe muss.
Darius skizziert ein Bild des Glücksgefühls, dass sich vor allem durch eine allgemein positive Stimmungslage eines Menschen auszeichnet, die er darlegt.

Diese Auslegung widerspricht deutlich Immanuel Kants Glücksverständnis, der dieses vor allem aus dem rationalen Handeln des Menschen heraus begründet. 
Das Glück sei demnach eben keine Gefühlsregung, wie sie von Darius beschrieben wird, sondern eher das rationale Resultat der sich ergebenen Handlungsketten. 
Darius beschreibt die Glückseligkeit als ein Gefühl der inneren Zufriedenheit und der Freude, die sich auch in äußerlichen Regungen wie dem Lachen zeigen kann. 
Daraus resultiert ein deutlicher Widerspruch zu Kant, der jegliche Emotionalität bezüglich des Glücks kategorisch ablehnt und das Glück sogar lediglich als die Befriedigung der menschlichen Bedürfnisse bezeichnet, die eng an das Leben in Sittlichkeit gebunden ist. 

Dieser Zusammenhang von Glück und Emotion wird  in der Form weitergedacht, dass sich die Schülerinnen und Schüler auch damit befassen, inwiefern arme Menschen zum Glück fähig sind. 
Stefanie ist der Auffassung, dass Arme nur in beschränktem Maße glücklich werden können, da der Mangel an den nötigsten Dingen sie vom Glück entferne. 
Darin spiegelt sich ein Bezug zu Kant wieder, der -- wie bereits mehrfach erwähnt -- vor allem die Bedienung des Bedarfes des Menschen an bestimmten Gütern als Ziel des Menschen sah, um zum Glück zu gelangen. 
Folglich zeigt sich durch die Verbindung zu Kant, dessen Theorie sich nah an der der antiken Stoa bewegt, dass der Mensch einen Selbsterhaltungstrieb in sich habe, der bestimmt, ob er glücklich werden kann oder nicht. 
Stefanie nennt daran anschließend auch, dass es Menschen gebe, die die Armen in ihrer Not mit Spenden wie Essen oder Geld unterstützen. 

Hier lässt sich erneut John Stuart Mills Utilitarismus anwenden. 
Wenn ein Mensch einem anderen hilft, so handelt er für sich selbst moralisch korrekt, da er aus seinem helfenden Handeln einen direkten Nutzen für sich selbst beziehen kann, welcher darin bestehen wird, dass er sich durch sein Mitgefühl gegenüber dem armen Mitmenschen besser fühlen wird. 
Gleichzeitig vermeidet er damit den Schmerz, den es ausgelöst hätte, wenn er die Hilfe unterlassen und durch seine unmoralische Handlung Gewissensbisse bekommen hätte. 
Obendrein lässt sich das Handeln des guten Menschen, wie Stefanie es ausdrückt, auch dahin gehend charakterisieren, in welcher Qualität und Dauerhaftigkeit sich das neue Glück zeigt, dass dieser Mensch erworben hat. 
Denn durch sein Handeln hat er nicht nur die Lage eines ärmeren Menschen verbessert -- auch wenn nicht davon auszugehen ist, dass dies von Dauer sein wird -- , sondern er hat auch dazu beigetragen, sein eigenes Glück voranzutreiben. 
Daher ist davon auszugehen, dass eine solches Handeln seinen Mitmenschen gegenüber im Sinne Mills eine hohe Qualität für das Glück beinhalten wird.

Neben dem Utilitarismus kann auch die nikomachische Ethik des Aristoteles anhand dieses Beispiels Anwendung finden. 
So lässt sich sagen, dass ein Armer zwar durchaus in der Lage ist, durch sein charakterliches Wesen alles für das persönliche Glück zu tun. 
Jedoch ist er wie jeder andere Mensch abhängig vom Vorhandensein anderer Güter. 
Nach Aristoteles kann ein armer Mensch daher nicht zur Eudaimonia gelangen, da ihm äußerliche Güter augenscheinlich fehlen werden, auch wenn er vielleicht die seelischen und körperlichen besitzt. 
Dies wiederum hat zur Folge, dass ein Armer nach Aristoteles stets ein unglückseliges Leben führen werden muss, solange er nicht in der Lage ist, alle von ihm beschriebenen Güter auf sich zu vereinen. 

Der gute Mensch kann jedoch, wenn man davon ausgeht, dass er bereits die seelischen und körperlichen Güter besitzt, in der Lage sein, zur Eudaimonia zu gelangen, da er offenkundig über äußerliche Güter verfügt, die er mit dem armen Menschen zu teilen bereit ist. 
Doch auch der gute Mensch wird wie der arme Mensch ein unglückseliges Leben führen müssen, wenn er die aristotelischen Bedingungen der Eudaimonia nicht erfüllt. 
Demgegenüber werden in der epikureischen Lehre beide Figuren das Glück letztendlich dann erreichen können, wenn sie sich bescheiden auf ihre Grundbedürfnisse beschränken und jeglichen Schmerz zu vermeiden versuchen.
\newpage

%% Ergebnisse und Ausblick
\section{Ergebnisse und Ausblick}

Die inhaltliche Analyse und die Interpretation haben gezeigt, dass die Kinder des 4. Schuljahres der Grundschule Koblenz-Kesselheim über eine differenzierte Glücksvorstellung verfügen. 
Es konnte herausgearbeitet werden, dass die Schülerinnen und Schüler vor allem folgende Aspekte des Glücks als wichtig für ihr Verständnis ansehen:

Wiederholt findet sich im Gespräch mit den Kindern vor allem die Perspektive, in der das Glück auf der Beziehungsebene zwischen Menschen betrachtet wird. 
Die Kinder sehen das Glück in geschlossenen Freundschaften, in der Beziehung zur Familie oder aber auch in gemeinsamen Aktivitäten. 
Offensichtlich ist den Kindern besonders wichtig, zum Ausdruck zu bringen, dass sie ein besonderes Bedürfnis nach sozialer Sicherheit, Aufmerksamkeit und Wertschätzung haben. 
Dieses Merkmal von Glück, dass die Kinder aus ihrer Sicht entfalten, findet sich auch in der pädagogischen Literatur wieder. 
Münch und Wyrobnik beziehen sich auf eine ähnliche Erhebung, bei der eine Frankfurter Grundschulklasse mit 22 Kindern -- 11 Jungen und 11 Mädchen -- befragt wurde, was sie unter Glück verstehen. 
Auch sie beziehen sich auf den Beziehungsaspekt des gemeinsamen Spielens und Zusammenseins mit Freunden \cite[S.\,62]{JM11}.

An dieser Stelle zeigt sich auch, wie wichtig für Grundschulkinder der Kontakt zu Gleichaltrigen und der gegenseitige Austausch ist. 
Auch der gemeinsam verbrachten Zeit und den dabei gesammelten Erfahrungen kommt daher eine besondere Bedeutung zu. 
Dabei deuten die Schülerinnen und Schüler ebenfalls an, dass für sie Glück ohne das Zutun anderer Menschen nur schwer vorstellbar ist. 
Zwar sei der Mensch in begrenztem Maße und in einem gewissen zeitlichen Rahmen in der Lage, auch allein glücklich zu sein. 
Dieses Glück wird jedoch von den Kindern nicht in der selben Qualität begriffen, wie das durch Mitmenschen geschaffene Glück. 
Die individuellen Bedürfnisse des Menschen nach Heimat, Versorgung mit Lebensmitteln und mit der Geborgenheit der Familie wird von den Kindern als elementar angesehen. 
Daher ziehen sie daraus auch die Konsequenz, dass Glück für arme Menschen oder solche, die ohne familiäre Bindungen leben müssen, nur schwer zu denken ist. 
Auf diese Weise drücken die Kinder aus, wie wichtig ihnen vor allem ein sorgenfreies Leben, die Versorgung durch die Familie und Freundschaften sind. 

Zudem haben die Schülerinnen und Schüler vereinzelt auch das Bedürfnis nach Anerkennung ihrer Leistungen. 
Durch Aysches Bekenntnis am Anfang des Gespräches, dass sie Glück in guten Noten erkennt,  lässt sich der Schluss ziehen, dass die Kinder auch das Leistungsbewusstsein in ihr Glücksverständnis übernommen haben. 
Evident erscheint daher, dass die Kinder im Grundschulalter bereits unter einem signifikanten Leistungsdruck im Vorfeld des Übergangs zur weiterführenden Schule zu stehen scheinen. 

Ein weiterer wichtiger Punkt, den die Schülerinnen und Schüler erwähnen, ist das Motiv des Zufalls. 
Kinder verbinden mit dem Zufall nach Münch vor allem Glück, \glqq das ihnen ohne eigenes Zutun zufällig in den Schoß fällt.\grqq{} \cite[S.\,62]{JM11}
Im Gegensatz zu Münchs Ergebnissen nennen die Kinder in Kesselheim dabei jedoch nicht den Fund von Geld oder anderen Gegenständen, sondern beschreiben das Glück vor allem in Form der Ereigniskette, die dann zu einem -- für die Kinder -- positiven Abschluss kommt. 
Dadurch drücken sie Wünsche und Hoffnungen aus, die sie mit dem Glück verbinden, wenn diese in Erfüllung gehen. 

Den Zufall verorten die Kinder allerdings auch in vermiedenen, negativen Konsequenzen von Ereignisketten. 
Konkret benannten die Kinder eine mögliche Verletzung, die sich trotz eines Sturzes von einem Baum nicht einstellte. 
Daran lässt sich erkennen, dass auch Grundschulkinder bereits Erfahrungen mit dem Glück im Unglück gemacht haben und dieses ebenfalls Eingang in ihr Glücksverständnis gefunden hat.
Auch das Gefühl, dass die Schülerinnen und Schüler in Glückssituationen erleben, spielt eine wichtige Rolle. 
Die emotionale Komponente des Glücks wird von den Kindern vor allem mit positiver Mimik wie dem Lachen und einer Gefühlslage des Herzens verstanden. 
Dazu wird ausdrücklich das Lachen, welches durch Glück ausgelöst werde, von dem ohne solchen Kontext auftretenden differenziert. 
Damit zeigen die Schüler, dass sie in der Lage sind, emotionale Regungen im Kontext einer Handlung zu sehen und zu unterscheiden. 

Innerhalb des Gespräches ergaben sich durch die Analyse und Interpretationen auch Hinweise darauf, dass die Schülerinnen und Schüler in ihren Sichtweisen auf den Glücksbegriff antike und neuzeitliche Positionen aufgriffen. 
So konnte festgestellt werden, dass in den Glücksvorstellungen der Kinder bereits das Bedürfnis nach der Vermeidung von Leid vorhanden ist, dass in der Philosophie vor allem durch Epikur, Aristippos und John Stuart Mill vertreten wurde. 
Hinzu kommt, dass sich in den Beiträgen der Kinder nachweisen lassen konnte, dass es offensichtlich unterschiedliche Arten von Glück gibt und dass das Handeln der Schülerinnen und Schüler in unterschiedlichem Maße zu diesem Glück beitragen kann. 
Daher lässt sich festhalten, dass sich die philosophischen Vorstellungen der Kinder zum Teil den utilitaristischen Strömungen zuordnen lassen. 

Darüber hinaus lassen sich jedoch noch weitere philosophische Überschneidungen zwischen den Gedanken der Kinder und den philosophischen Positionen finden. 
So sind die Kinder zwar teilweise der Meinung, dass das Glück auch von den Mitmenschen mitbestimmt wird und nicht nur an die eigene Person gebunden ist. 
Jedoch kann man sagen, dass die Kinder nicht allgemein davon ausgehen, dass ein Mensch, wie Aristoteles es sagt, per se ein unglückliches Leben führen muss, weil ihm bestimmte Güter fehlen. 
Dies machen die Kinder daran deutlich, dass sie nicht kategorisch ausschließen, dass auch ärmere Mensch glücklich werden können.
Gleichwohl lässt sich festhalten, dass das Glück des Menschen als oberstes Lebensziel auch die Ansichten der Schülerinnen und Schüler mitbestimmt. 

Das philosophische Gespräch mit der Klasse hat deutlich gemacht, dass die Kinder eine Vorstellung davon haben, wie sich Glück anfühlen kann. 
Diese Verknüpfung von Emotion und Glück steht im eindeutigen Widerspruch zur Stoa und auch zu den Lehren Immanuel Kants, welche das Glück im Kontext einer gottgegebenen Ordnung betrachten.
Damit bringen die Schülerinnen und Schüler innerhalb des Gespräches  eine klare Gegenposition gegenüber einer philosophischen Lehrmeinung  zum Ausdruck. 
Vor allem die Position Kants, dass Glück lediglich durch ein rational erklärbares Handlungsmuster und nicht durch Gefühle gesteuert sei, ist eine Sicht der Dinge, die die Schülerinnen und Schüler nicht teilen. 
Dabei ließ sich aber auch ausmachen, dass die Kinder dieses Glücksgefühl augenscheinlich nicht genauer beschreiben konnten.
So gelang es lediglich, das Gefühl, das Glück auslöst, in einem gewissen Rahmen eingrenzen zu können.
Dies zeigt an dieser Stelle aber auch, dass sich die Kinder mit der Beschreibung der Gefühle, die sie mit Glück verbinden, schwer taten.

Weiterhin ist auch auffällig, dass sich die Positionen der Kirchenvertreter Augustinus und Luther sowie von Nietzsche in den Aussagen der Grundschüler nicht in der Häufigkeit erkennen ließen, wie etwa die Theorien der Antike oder der Aufklärung. 
Dies erscheint plausibel aufgrund der Tatsache, dass diese Vertreter ihr Glücksverständnis sehr spezifisch entwickelten und wenig Spielraum für Interpretation und Einordnung lassen. 

Denn sowohl Augustinus als auch Luther beziehen ihr Glücksverständnis auf die Rolle des allmächtigen Gottes, der durch seine Unabhängigkeit Ausgangspunkt für das Glück sein müsse. 
Darin spiegelt sich ein Weltbild wieder, dass sich so bei Kindern höchstwahrscheinlich nicht wiederfinden wird. 
Zudem fällt bezüglich Nietzsches Glücksbild auf, dass auch er, der sein Bild der drei Säulen vor allem aus psychologischer Perspektive  aufbaute, in der Betrachtung der Parallelen und Unterschiede zwischen den Positionen der Kinder und der Philosophie kaum eine Rolle spielte. 
Auch darin findet sich kein Denken wieder, das mit einer kindlichen Sicht auf das Glück zu vereinen zu sein scheint. 
Diesen drei Vertretern gemeinsam ist folglich auch, dass sie -- im Gegensatz zu den vorhin erläuterten Philosophen -- ein sehr eng gefasstes Glücksverständnis vertreten und daher eine Bezugnahme auf das Denken von Kindern kaum möglich ist. 

Es lässt sich resümieren, dass die zu Beginn gestellte Forschungsfrage, welche Vorstellungen Grundschulkinder von Glück haben, exemplarisch anhand der Untersuchung eines vierten Schuljahres beantwortet werden konnte. 
Darüber hinaus konnten auch die Vorstellungen dieser Schülerinnen und Schüler selbst herausgearbeitet werden. 

Neben den philosophischen Inhalten lassen sich im analysierten Gespräch die zentralen Gedanken des Rahmenplans Sachunterricht für die Grundschulen in Rheinland-Pfalz erkennen.
So bekamen die Schülerinnen und Schüler im Gespräch über das Glück die Möglichkeit, sich mit einem Thema zu befassen, von dem jeder Mensch ein Bild hat.
So waren die Kinder selbstbewusst, ihre Ideen und Gedankengänge offen zu diskutieren.
Dies lässt sich unter der personalen Kompetenz fassen.
Des Weiteren konnten sie an den Ansichten ihrer Mitschüler partizipieren (sozial), schulten die Kompetenz, eigene Erfahrungen kundzutun (methodisch) und setzten sich fachlich mit einem für sie greifbaren philosophischen Problem auseinander (fachlich).
Hinzu kommt, dass die gewählte Methodik des philosophischen Gespräches eindeutig in der Lage war, dem Forschungs- und Wissensdrang der Kinder in der Grundschule gerecht zu werden.
Andersherum ausgedrückt wäre kein solches Gespräch zustande gekommen, wenn die Schülerinnen und Schüler kein Interesse an einer Diskussion über Glück gehabt hätten.

Auch die perspektivenübergreifenden Denk-, Arbeits- und Handlungsweisen, wie sie der Perspektivrahmen Sachunterricht beschreibt, finden sich im untersuchten Gespräch wieder.
So lässt sich festhalten, dass die Schülerinnen und Schüler im Gespräch Eindrücke anderer Wertvorstellungen und Wissen der Mitschüler im Kontext des Erkennen/Verstehen sammeln konnten.
Zusätzlich mussten die Kinder ihre Positionen in der Diskussion argumentativ unterstützen und gegen Kritik verteidigen.
Eine weitere Denkweise des Perspektivrahmens, die durch das Gespräch aufgezeigt werden konnte, ist die des Evaluierens und Reflektierens. 
Die Kinder reflektierten ihre Erkenntnisse und bezogen zum Teil auch Stellung zu den Vermutungen anderer, denen sie Alternativen gegenüberstellten.
Zur Reflexionskompetenz passt an dieser Stelle auch die Kompetenz des Kommunizierens.
Die Schülerinnen und Schüler stellten ihre Meinung begründet dar und diskutieren miteinander über das Thema \glqq Glück\grqq{}.
Es zeigt sich, dass sowohl die philosophischen wie auch die pädagogischen Aspekte des Philosophierens anhand des Gesprächs mit den Kindern exemplarisch gezeigt werden konnten.

Die in dieser Masterarbeit behandelte Untersuchung stößt jedoch auch an Grenzen. 
So konnte im Rahmen der Erarbeitung lediglich eine Klasse untersucht werden. 
Um mögliche Unterschiede zwischen den Klassenstufen in den Blick zu nehmen, würde es sich daher auch anbieten, verschiedene Klassenstufen einer Schule zu einer Thematik zu befragen und diese Ergebnisse in Beziehung zueinander zu setzen.
Für den wissenschaftlichen Diskurs zu Glücksvorstellungen von Grundschulkindern können neben der an dieser Stelle gewählten Methodik auch andere Verfahren zum Einsatz kommen. 
So könnten philosophische Gespräche über Glück auch aus partizipatorischer Sicht betrachtet werden. 
Dabei würde  eine Partizipationsanalyse zum Einsatz kommen, die untersucht, inwiefern die Schülerinnen und Schüler sich aufeinander beziehen und wie sich ihr Glücksverständnis im Laufe des Gespräches weiterentwickelt. 
Diese Fragen konnten in dieser wissenschaftlichen Auseinandersetzung nicht behandelt werden, da der Fokus dieser Arbeit auf der inhaltlichen Komponente und nicht etwa auf der partizipatorischen des Gespräches lag. 

Zudem könnte auch interessant sein, welche Faktoren für die Entwicklung der Glücksvorstellungen der Schülerinnen und Schüler von Bedeutung sind. 
Entscheidende Fragen in diesem Kontext könnten sein, ob das soziale Umfeld Einfluss auf die Sichtweisen der Kinder nimmt. 
Dazu könnten Untersuchungen aus Schulen, deren Schüler eher aus sozial schwächerem Milieu kommen, mit denen von Schulen aus bürgerlichem und aus wohlhabenderen Milieus verglichen werden, um zu ergründen, ob und inwiefern der soziale Hintergrund der Kinder Einfluss auf ihr philosophisches Denken hat.
Zusätzlich dazu ergibt sich auch die Frage, ob man an Großstadtschulen andere Ergebnisse erzielen wird, als in ländlich geprägteren Gegenden.

Insgesamt kann man sagen, dass im Rahmen dieser Arbeit gezeigt werden konnte, dass nicht nur Erwachsene sondern auch Grundschüler bereits über ein differenziertes Glücksverständnis verfügen.
Die Untersuchung stützt auch die Definition des Begriffs \glqq Glück\grqq{}, die zu Beginn dargelegt wurde, dass es sich beim Glücksbegriff nicht um einen definierbaren und klar zu fassenden Untersuchungsgegestand handelt.
Vielmehr unterscheidet sich die Einordnung dessen, was ein Mensch als Glück sehen wird, von Mensch zu Mensch.
Und so man kann sagen, dass man nicht nur von zehn Erwachsenen zehn verschiedene Antworten auf die Frage, was Glück ist, bekommen wird, sondern auch von zehn Grundschulkindern.
\newpage


\appendix

%% Transkript
%\section{Transkripte}
\subsection{Spiel "`Alemannia Aachen - SG Wattenscheid 09"'}
Das folgende Transkript stammt aus einer Tonaufzeichnung der Begegnung "`Alemannia Aachen - SG Wattenscheid 09"', aufgezeichnet am 06.12.2014 im "`Tivoli"'-Stadion in Aachen. Es umfasst den Zeitraum 16:20 - 16:38, bezogen auf die Spielzeit des Fußballspiels.

\begin{dialogue}
	\speak{Stadionsprecher} vierundsechzigste spielminute; (.) \\
	tor (--) für die alemannia (--) \\
	torschütze, (---) wie in der ersten halbzeit; \\
	mit der nummer einundreißig- \\
	fabian
	\speak{Fans} GRAUDENZ
	\speak{S} damit der neue spielstand; (.) \\
	die alemannia, (.)
	\speak{F} ZWEI (.)
	\speak{S} wattenscheid, (.)
	\speak{F} NULL (.)
	\speak{S} danke danke? (.)
	\speak{F} BITTE BITTE;
\end{dialogue}

%\newpage


\clearpage
\bibliography{master_thesis}
\bibliographystyle{dinat}

\newpage
\newgeometry{left=1.5cm,right=1.5cm,head=1.5cm,bottom=1.5cm}
\currentpdfbookmark{Erklärung}{pdf:declaration}
\thispagestyle{empty}
\vspace*{3cm}
\begin{center}
	\Large{\textbf{Erklärung}}
\end{center}
\vspace*{1cm}
Hiermit bestätige ich, dass die vorliegende Arbeit von mir selbstständig verfasst wurde und ich keine anderen als die angegebenen Hilfsmittel - insbesondere keine im Quellenverzeichnis nicht benannten Internet-Quellen - benutzt habe und die Arbeit von mir vorher nicht in einem anderen Prüfungsverfahren eingereicht wurde.
Die eingereichte schriftliche Fassung entspricht der auf dem elektronischen Speichermedium. (CD-ROM)
\vspace*{1cm}\\
Koblenz, den \today
\vspace*{0.75cm}\\
Martin Spoo
\restoregeometry


\end{document}

