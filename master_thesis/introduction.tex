\section{Einleitung}
Wenn man zehn zufällig ausgesuchte Menschen danach fragen würde, was für sie Glück ist, so wird man voraussichtlich zehn völlig verschiedene Antworten erhalten.
 Für den einen mag es Glück sein, wenn er sich beim Fußballspielen nicht verletzt, für den anderen ist Glück ein positives Gefühl der Zufriedenheit und wieder ein anderer wird Glück als günstige Abfolge von Ereignissen betiteln. 
Schon im antiken Griechenland kamen führende Philosophen wie z.B. Platon oder Aristoteles nicht daran vorbei, sich diese Frage zu stellen, was Glück ist und wie man es erlangen kann.
 Bis heute hat die Frage des Glücks als eine der bedeutendsten der Philosophie insgesamt nichts an Aktualität eingebüßt. 
 Denn auch in der aktuellen Zeit, in der das Glück vieler Menschen u.a. durch die Flüchtlingskrise oder den internationalen Terrorismus gefährdet wird, stellt sich für jeden von uns die Frage, wie der Weg zum persönlichen Glück aussehen kann.
 Auch Kinder stellen Fragen zu den verschiedensten Teilbereichen der Philosophie und daher ist es von besonderer Bedeutung, die Vorstellungen von Kindern vom Lerngegenstand Glück in Bezug zu ihrer philosophischen Prägung hin zu untersuchen.

Der Frage, was Grundschulkinder über Glück denken und zu welchen Ergebnissen sie dabei kommen, will diese Masterarbeit auf den Grund gehen. 
Bevor diese Fragestellung behandelt werden kann, muss erst deutlich gemacht werden, worum es sich bei der Philosophie handelt und wodurch die Tätigkeit des Philosophierens gekennzeichnet ist.
Ergänzend dazu muss geklärt werden, warum das Philosophieren in der Grundschule wichtig ist und wie es in den schulischen Unterricht der Grundschule integriert werden kann. 
Darüber hinaus soll an dieser Stelle auch deutlich werden, inwiefern das Philosophieren in der Grundschule einen Beitrag zum Kompetenzerwerb nach den Vorgaben des Teilrahmenplans Grundschule des Landes Rheinland-Pfalz für den Sachunterricht leisten 
kann.
Auch der Perspektivrahmen Sachunterricht, der von der Gesellschaft für die Didaktik des Sachunterrichts (GDSU) veröffentlicht wird, soll zur Einordnung herangezogen werden.

Im weiteren Verlauf ist es wichtig, zu klären, worum es sich bei Glück im Allgemeinen handelt und und inwiefern sich  \glqq Glück haben\grqq{} und  \glqq glücklich sein\grqq{} 
in Eigenschaften voneinander unterscheiden lassen. 
Dazu sollen auch einige Positionen von Philosophen aus der Antike, des Mittelalters und der Neuzeit erläutert werden. 
Schlussendlich sollen anhand eines Transkriptes, das im Rahmen des universitären Seminars \glqq Philosophieren in der Grundschule\grqq{}  an der Universität Koblenz-Landau entstanden ist, einige philosophische Positionen der Kinder herausgearbeitet und mit 
den zuvor erarbeiteten Standpunkten bekannter Philosophen in Beziehung gesetzt und auf Gemeinsamkeiten und Unterschiede hin untersucht werden. 
Am Ende der Analyse wird dann ein Fazit gezogen, welche Schlüsse sich aus dem Unterrichtsgespräch mit den Kindern ergeben haben.