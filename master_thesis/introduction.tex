\section{Einleitung}
Spätestens seit der Fußballweltmeisterschaft im Jahr 2006 in Deutschland ist die Fankultur des Fußballs in unserem Land angekommen.
Die große Mehrheit der Deutschen fieberte mit der deutschen Nationalmannschaft und schuf sowohl in den Stadien als auch beim gemeinsamen Verfolgen der Spiele (Public Viewing u. Ä.) eine einzigartige Stimmung.
Und auch im heimischen Spielbetrieb der Bundesliga ist die Begeisterung der Menschen für den Fußball ungebrochen.
Wenn über 80000 Menschen regelmäßig zu den Heimspielen von Borussia Dortmund in den Signal-Iduna-Park kommen und ihre Mannschaft bis zum Abpfiff durch ihren Gesang unterstützen, so macht den Fußball offenbar mehr aus, als dass lediglich 22 Menschen einem Ball nachlaufen, wie es Skeptiker formulieren würden.

Auf inhaltlicher Ebene sind vor allem pejorative, also abwertende, Fangesänge bedeutsam, da sie im Gegensatz zu klassischen Gesängen, bei denen die Intention der Unterstützung für die favorisierte Mannschaft in der Regel klar ist, eine Vielzahl möglicher Funktionen eröffnen.
So können mit solchen Äußerungen nicht nur die eigene Mannschaft, sondern auch andere Personengruppen, die an einem Fußballspiel beteiligt sind, angesprochen werden.

Doch was ist das Besondere an den Gesängen, die die Fans in den Stadien anstimmen?
Handelt es sich dabei bloß um Unmutsbekundungen bzw. frenetischen Jubel, der manchmal in Grölen und Pfeifkonzerte übergeht?
Was tun Fans auf linguistischer Ebene, wenn sie am Wochenende im Stadion Schmährufe gegen die gegnerische Mannschaft und deren Anhänger oder gegebenenfalls auch gegen den Schiedsrichter skandieren?

Diesen Fragen und Denkanstößen soll sich die vorliegende Bachelorarbeit zum Thema "`Fußballfangesänge – Pejorative Sprachhandlungen von Fußballfans am Beispiel Alemannia Aachen"' widmen.
Dazu soll zunächst dargelegt werden, warum der Verein "`Alemannia Aachen"' als Untersuchungsgegenstand geeignet ist und als Anschauungsbeispiel für eine solche linguistische Untersuchung dienen kann.
Anschließend soll beleuchtet werden, was Zuschauer von Fans unterscheidet, um zu unterstreichen, welche Bedeutung Fans für den Sport haben.
Auch der Historie der Fankultur unter besonderer Berücksichtigung der Entstehung von Fangesängen sowie ihren Merkmalen soll in diesem Zusammenhang Rechnung getragen werden.
Darüber hinaus soll auf sprachliches Handeln, das von verschiedenen Vertretern der linguistischen Pragmatik betrachtet wurde, eingegangen werden und die Sprache unter dem Gesichtspunkt betrachtet werden, in welchem Verhältnis sie zur Gewalt steht.

Desweiteren wurde zur Analyse von Sprachhandlungen der Fans von Alemannia Aachen eine Tonaufnahme des Regionalligaspiels gegen den Verein "`SG Wattenscheid 09"' vom 06.12.2014 angefertigt, die in Auszügen transkribiert wurde.
Zudem wurden weitere Aufnahmen von Anhängern dieses Vereins, die aus einem Onlineforum für Fußballfans stammen, zur Analyse herangezogen.
Ziel soll es sein, festzustellen, ob sich auf Fangesänge Theorien der pragmatischen Linguistik anwenden lassen und welche Handlungen Fans mit diesen vollziehen.

Dies geschieht in drei Schritten:
Zunächst sollen die transkribierten Fangesänge auf inhaltlicher Ebene betrachtet werden.
Danach soll geklärt werden, welcher Kategorie von Fangesängen, die im Verlauf dieser Arbeit definiert werden sollen, diese Beispiele zuzuordnen sind.
Abschließend werden die gesammelten Erkenntnisse mit der Sprechakttheorie von John Searle verknüpft.