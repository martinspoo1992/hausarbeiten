\section{Inhaltsanalytische Kriterien und zentrale Fragestellungen}

Nachdem die theoretische Grundlage zum Glücksbegriff im philosophischen Diskurs gelegt sowie eine Einordnung des Philosophierens mit Grundschülern in den didaktischen Zusammenhang des Rahmenplans Sachunterricht des Landes Rheinland-Pfalz und den Perspektivrahmen Sachunterricht des GDSU vorgenommen wurde, muss im Vorfeld der sich anschließenden Analyse geklärt werden, nach welchen Kriterien das vorliegende Transkript untersucht wurde und welche Fragestellungen besondere Beachtung finden sollen.

Im Fokus stehen dabei fünf zentrale Sichtweisen der Kinder auf das Glück: 
Es wird beleuchtet, inwiefern das Glück für die Schülerinnen und Schüler im Zufall besteht. 
Dabei wird zu klären sein, ob Glück nur in wünschenswerten Zufallssituationen auftreten kann oder aber auch in Zufällen, die weniger wünschenswert verlaufen. 
Darüber hinaus wird an dieser Stelle besonders auch auf das sogenannte \glqq Glück im Unglück\grqq{} verwiesen. 
Dem gegenübergestellt lassen sich im Gespräch auch Passagen finden, die dem Glück als Zufall widersprechen. 
Vielmehr kann das Glück aus dieser Betrachtung heraus auch als Resultat eigener Anstrengung verstanden werden.
Daher lässt sich sagen, dass die Frage, inwiefern solch konträre Glücksbegriffe den gleichen Gegenstand \glqq Glück\grqq{} darstellen sollen, von großer Bedeutung ist. 
Hinzu kommt die Frage, ob für ein Glück durch persönliche Leistungen eine Art Bestätigung durch die Mitmenschen notwendig ist oder nicht.

Eine weitere wichtige Kategorie umfasst die zwischenmenschlichen Beziehungen, zu denen sich die Schülerinnen und Schüler in Bezug auf das Glück äußern. 
Dabei werden einerseits unter dieser Kategorie sowohl Gedankengänge in Bezug auf Familie und Freunde in den Blick genommen, andererseits aber auch die Fragestellung zu klären sein, inwiefern solch geartete Beziehungen für das subjektive Glücksempfinden von Nöten sind oder nicht. 

Auch das Glück als emotionale Regung bzw. als Gefühl soll in der inhaltlichen Auseinandersetzung mit den Positionen der Kinder behandelt werden. 
Eine zentrale Frage dieser Kategorie wird sein, inwiefern arme Menschen fähig sind, Glück zu empfinden trotz ihrer misslichen Lage. 
Des Weiteren werden die Schüler in Textpassagen, die dieser Kategorie zugeordnet wurden, sich über die Art des Glücksgefühls austauschen und darüber, ob dieses Gefühl durch materielle Dinge ausgelöst werden muss oder von materiellen Abhängigkeiten als autonom angesehen werden kann. 
Die letzte Kategorie besteht im Glauben der Kinder an Glück: 
Helfen Glücksbringer, um Glück zu haben? Welche Arten von Gegenständen können Glücksbringer werden? 
Können Menschen Glücksbringer für andere Menschen sein?

Diese Formen des Glücksverständnisses lassen sich auch im Kontext philosophischer Vorstellungen, welche zuvor bereits erläutert wurden, betrachten. 
Dabei soll die Frage im Fokus stehen, inwiefern sich philosophische Richtungen wie z.B. der Epikureismus oder der Utilitarismus in Ansätzen bei den Aussagen der Schülerinnen und Schüler wiederfinden lassen. 