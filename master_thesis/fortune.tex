\section{Was ist Glück?}
\subsection{Begriffsdefinition}

Um die Vorstellungen, die Grundschüler beim Philosophieren in der Klasse äußern, verstehen und analysieren zu können, empfiehlt es sich, der Analyse der Unterrichtssequenz eine Begriffsdefinition voranzustellen, die klären soll, was Glück ist und ob es einen Unterschied gibt, Glück zu \glqq haben\grqq{} oder \glqq glücklich\grqq{} zu sein.

Der Duden versteht Glück zunächst einmal als \glqq etwas, was Ergebnis des Zusammentreffens besonders günstiger Umstände ist\grqq{} und nimmt damit eine erste Verortung des Begriffes vor. 
Demnach wird Glück nicht durch seine spezielle Beschaffenheit bestimmt, sondern ist unabhängig von der Artigkeit der Ereignisse die möglichst günstige Abfolge.
 Ein wichtiger Begriff für die Definition des Duden ist in diesem Zusammenhang auch der Zufall, da das Auftreten von Glück keiner statistisch messbaren Form von Regelmäßigkeit zu folgen scheint. 
 Zudem lässt die Definition der günstigen Abfolge von Ereignissen Spielraum für Interpretationen, welche Ereignisabfolgen von Menschen als günstig angesehen werden.  
 Der Duden nennt in einer zweiten Ebene die Sicht des personifizierten Glücks, das das Glück antropomorph betrachtet. 
 Dem Glück wird beispielsweise in der römischen Göttin Fortuna eine Figur zugeordnet, wie es im antiken römischen Reich für alle Lebensbereiche der Gesellschaft üblich war. 
 Abschließend unterscheidet der Duden genauer zwischen Glück haben und glücklich sein, indem er \glqq glücklich sein\grqq{} als positive Gemütsverfassung und \glqq Glück haben\grqq{} als vereinzelte glückliche Situationen beschreibt.
  Demnach kann man in Situationen sein, in denen man zwar Glück hat, aber nicht zwangsläufig glücklich ist. 
 Andersherum gesagt kann man auch glücklich sein ohne Glück zu haben. 
 Es zeigt sich also, dass \glqq Glück haben\grqq{} und \glqq glücklich sein\grqq{} völlig verschiedene Zustände beschreiben können. 
 
Auch das philosophische Online-Wörterbuch Philosophie stellt eine solche Unterscheidbarkeit heraus. 
So sei es denkbar, dass manche Menschen durch den Erwerb oder den Besitz materieller Güter wie Macht oder Geld Glück empfinden werden. 
Andere wiederum werden vor allem durch mitmenschliche und innere Gefühle Glück erleben oder aber auch durch den Genuss. 
Auch hier zeigt sich erneut der individuelle Charakter der menschlichen Glückswahrnehmung. 

Dass jeder Mensch Glück als etwas anderes beschreiben kann, liegt für die Autorin Lic. phil. Gerhild Tesak vor allem  darin begründet, dass jeder Mensch durch seine individuelle Prägung, die er im Laufe des Lebens erfährt, und durch seine Interessen ein eigenständiges Glücksverständnis entwickele.
Demnach sei eine genaue Definition schlicht nicht möglich und die einzige, gültige Festlegung bestehe darin, das Glück als das höchste Ziel der menschlichen Existenz zu verstehen. 
 Sie stellt darüber hinaus fest, dass das Streben nach Glück nicht selbiges als Ziel erfolgreich anstreben kann. 
 Mit jener Glücksvorstellung verbundene Wünsche könnten zwar in Erfüllung gehen, jedoch ohne ein entsprechendes Gefühl von Glückseligkeit hervorzurufen.
 So gesehen sei Glück \glqq das begleitende Gefühl gelungenen Handelns\grqq{}.
 
Harald Schöndorf bezieht sich ebenfalls auf die Abgrenzung zwischen \glqq Glück haben\grqq{} und \glqq glücklich sein\grqq{}. 
Dabei versteht er das Glück im erstgenannten Kontext vorrangig als positives Ereignis, das sich \glqq zumindest bis zu einem gewissen Grad als Geschenk, Zufall oder göttliche Fügung erweist.\grqq{} 
Glücklich zu sein betrachtet Schöndorf jedoch in einem umfassenderen, semantischen Zusammenhang, der darin besteht, dass es sich dabei um Erfahrungen handelt, die die Zufriedenheit an sich übersteigen. 
Dabei nimmt er auch Bezug auf die \textit{Eudaimonia} des Aristoteles, welche das letztliche Ziel der menschlichen Existenz darstellt. 
So gesehen ist mit Glück im Kontext des \glqq glücklich seins\grqq{} auch eine gewisse Hoffnung auf dieses Ziel eng verknüpft.

Harald Schöndorf betont darüber hinaus auch die Gemeinsamkeiten der beiden Lesarten des Glücksbegriffes, die darin bestünden, dass das Glück keine Leistung oder etwas planbares sei, sondern lediglich als etwas Geschenktes zu erreichen sei. 
Es sei nichts Machbares, sondern stelle sich erst ein, wenn ein Mensch sein Leben kompromisslos akzeptieren könne.
Glück ist zudem nichts, was sich lediglich im irdischen Leben verortet. 
Dadurch, dass das irdische Leben des Menschen unvollkommen und defizitär ist, muss sich auch das alles erfassende Glück in einem transzendentalen Leben außerhalb der Sphäre der Erde befinden. 
Eine große Gefahr bestehe folglich darin, sich auf das Streben nach Glück zu fixieren, statt die erbrachten Leistungen zu wertschätzen und das Glück weiterhin als besonderes Geschenk wahrzunehmen.

Es lässt sich daher sagen, dass der Begriff \glqq Glück\grqq{} als solches nicht abschließend definiert, sondern nur angenähert werden kann. 
Viele Philosophen der Antike, des Mittelalters und auch der Neuzeit haben sich mit dem Glück beschäftigt und wie der Mensch sich ihm nähern kann. 
Daher sollen an dieser Stelle die zentralen Positionen der Philosophie angesprochen werden, um sie in einem weiteren Schritt mit den Aussagen von Grundschülern in Beziehung setzen zu können.


\subsection{Philosophische Positionen der Antike}

Bereits im antiken Griechenland befassten sich Philosophen mit den Fragen, was Glück sei und wie man es erlangen könne. 
Aristippos von Kyrene, ein Schüler Sokrates', der zwischen 435 und 335 v.Chr. in Kyrene lebte, unterscheidet in seiner Sicht auf das Glück vor allem zwei Gemütslagen der Seele, die entweder durch Lust oder durch Schmerz bestimmt waren.

Er differenziert dabei ausdrücklich nicht die Beschaffenheit der Lust, da es aus seiner Sicht keine höhere Ausprägung als die Lust als solche geben könne. 
Zusätzlich sei die Lust grundsätzlich aus Sicht aller Geschöpfe erstrebenswert, wobei der körperlichen Lust in dieser Beziehung eine besondere Bedeutung zukomme. 
Neben der Differenzierung zwischen Lust und Schmerz, denen sich die menschliche Seele hingeben könne, unterscheidet Aristippos zusätzlich zwischen dem Ziel und der Glückseligkeit. 
Demnach sei das Ziel die einfache Lust, während die Glückseligkeit in der Summe jeglicher Arten von Lust sei und damit allumfassender.
Daraus folgert er, dass die einzelne Lust durch ihre eigene Anziehung in sich lustvoll sei, die Glückseligkeit hingegen diese Anziehungskraft aus den verschiedensten Lüsten beziehe. 

Während Aristippos von Kyrene einerseits das Streben der menschlichen Seele nach Lust betont, so weist er andererseits auch energisch auf die Vermeidung von Schmerzen und Leid hin, welche nach seiner Ansicht dem Glück im Wege stünden. 
Letztlich sieht er die Glückseligkeit des Menschen als utopisches Ziel an, da dem Erreichen der Glückseligkeit körperliche Leiden, die auch die Seele beträfen, entgegenwirken würden.

Aristoteles stellt in seiner nikomachischen Ethik fest, dass jedes Handeln des Menschen zielorientiert ist. 
Der Mensch strebe durch sein unmittelbares Handeln ein Ziel an, das es zu erreichen gelte.
Darüber hinaus differenziert Aristoteles zwischen Dingen, die man ihrer selbst willen tue und solchen, die man tue, um ein höheres Ziel zu erreichen, aber auch Dinge, die sowohl zweckgebunden als auch als solches zielgerichtet sind wie z.B. eine sportliche Aktivität, die man vollziehen kann, weil sie Freude bereitet aber auch, weil sie für das eigene Wohlbefinden zuträglich sein kann. 
Seine Definition von Glück definiert Aristoteles aus der überzeugung heraus, dass der Mensch nach Vollkommenheit strebe. 
Daher sei es sinnvoll, sich mit dem höchsten menschlichen Ziel -- dem Glück -- näher zu beschäftigen und es zu ergründen. 

Zentraler Begriff seiner Ausführungen ist die \textit{Eudaimonia}, welcher sich mit \glqq gut leben\grqq{} oder \glqq gut handeln\grqq{} übersetzen lässt.
Mit ihr beschreibt er das zentrale Endziel, dass das menschliche Leben ausmache. 
Jedoch lässt sich dieser Begriff nicht eins zu eins mit der heutigen Vorstellungen eines glücklichen Lebens in Relation setzen.
Aristoteles betrachtet ein Leben erst dann als als eudaimon, welches eine \glqq objektive Gestalt und Qualität und nicht eine subjektive Stimmungslage des Lebens zum Inhalt hat.\grqq{} 
 Folglich kann ein eudaimones Leben nur ein solch geartetes sein, dass durch einen guten Charakter und eine Lebensweise geprägt wird, die alles dafür tut, dass der eigene Lebensweg sich erfolgreich gestaltet. 
 Daran wird deutlich, dass der Glücksbegriff des Aristoteles sich grundlegend von dem neuzeitlichen des 21.Jahrhunderts unterscheidet.
 
Es lässt sich demnach sagen, dass die Sehnsucht des Menschen nach Geld, Gesundheit oder Vergnügen Teilziele sind im Gegensatz zur Eudaimonia. 
Demgegenüber ist die Eudaimonia das höchste Ziel menschlicher Existenz und so etwas, das sich nicht um seiner selbst willen erreichen lässt. 
Gleichzeitig versucht der Mensch jedoch auch, durch partielles Glück wie Wohlstand, Gesundheit oder Vergnügen sich der Eudamonia anzunähern und deshalb ist sie so gesehen auch etwas Zusammengesetztes. 

Aristoteles hält die Glückseligkeit des Menschen für ein Zusammenwirken dreier Güter, die er im folgenden ausführt: die seelischen, die körperlichen (Wohlbefinden, Stärke, Schönheit) und die äußeren wie z.B. Reichtum oder Ruhm.
Eines allein sei nicht ausreichend, um Glückseligkeit zu erfahren, sondern alle drei müssten zusammenwirken. 
Das Gegenstück zur Glückseligkeit -- das unglückselige Leben -- sei jedoch bereits erreicht, wenn eines der genannten Güter nicht vorhanden wäre unabhängig davon wie gesund oder reich ein Mensch sei. 
In diesem Zusammenhang nennt er das Beispiel eines Weisen, der ein unglückseliges Leben führen würde, wenn er in Armut oder gesundheitlichem übel leben müsse.

Epikur sieht, ähnlich wie Aristippos, die Lust als ein zentrales Moment des Menschen an, um das persönliche Glück zu erreichen. 
Lust sieht er als erstrebenswert und Schmerz als zu vermeiden an. 
Allerdings ist er der Auffassung, dass der Mensch bescheiden sein müsse, um diese Lust zu erreichen.
Der Mensch müsse daher seine Bedürfnisse auf das Nötigste beschränken, um langfristig glücklich zu werden, da extreme Lust auch zu extremer Unlust führen könne.
Die größte Differenz zu Aristippos zeigt sich darin, dass Epikur neben den Zuständen der Lust und des Schmerzes auch einen dritten beschreibt, der sowohl lustvoll als auch schmerzhaft bezeichnet werden kann. 
Dabei erkennt Epikur durchaus an, dass sich der Mensch in einem Abhängigkeitsverhältnis zu einem natürlichen Trieb befindet. 
Jedoch sieht er diesen dadurch als unterbrochene Abhängigkeit an, da der Mensch jederzeit die Wahl habe, zu leben oder nicht zu leben.

Zu den Positionen Aristoteles' geht Epikur auf Distanz, da er das Glück nicht als zweckorientiertes Tun begreift, sondern ihm eher den Charakter eines \glqq zweckfreien Spiels\grqq{} zuschreibt, wie Maximilian Forschner dies nennt, \glqq das als solches, weil zwecklos, nicht ausgerichtet auf ein zu Erreichendes, stets vollendet ist.\grqq{}
In diesem \glqq Spiel\grqq{} könne letztlich nur mitspielen, wer sich von den \glqq Zwängen unbedingten Begehrens und Strebens befreit und in ästhetischer Distanz dem Spiel der Natur überlässt.\grqq{}
 
Auch die Stoa, eine philosophische Schule in Athen, beschäftigt sich mit dem Glück des Menschen. 
Die Lehre der antiken Stoa, die um 300 v. Chr. von Zenon von Kition begründet und von zahlreichen Vertretern bis hin zu Mark Aurel im 2.Jahrhundert n. Chr. fortgeführt wurde, grenzt sich klar ab von den Ansichten Epikurs und Aristoleles'. 
So schreibt Diogenes Laertius, dass die Vertreter der Stoa annahmen, dass der erste Trieb der menschlichen Natur die Selbsterhaltung sei und nicht etwa das Streben nach Lust.
Die Lust sahen sie daher nur als eine Art Begleiterscheinung, die \glqq den lebenden Wesen die heitere Stimmung und den Pflanzen das fröhliche Wachstum\grqq{} beschere. 

Zenon bekundet, dass das finale Ziel des Menschen ein Leben in Harmonie mit der Natur und demnach ein tugendhaftes Leben sei. 
Die Tugend wird von Stoikern wie Hekaton in wissenschaftliche und theoretische unterteilt. 
Als Beispiele nennt er an dieser Stelle die Einsicht und die Gerechtigkeit. 
So schreibt Laertius, untheoretische Tugenden seien so benannt, \glqq weil sie  nicht auf der Zustimmung des Verstandes beruhen, sondern eine Folgeerscheinung sind und auch bei Schwachköpfen sich finden, wie z.B. Gesundheit, Männlichkeit.\grqq{}
Demzufolge müsse es allgemeine Tugenden geben, die jedem Menschen zuteil sind und solche, die bei manchen vorhanden sind und bei anderen nicht. 
Es sei an dieser Stelle jedoch darauf hingewiesen, dass die Stoa im Tugendbegriff gespalten war, da die Stoiker untereinander andere Schwerpunkte setzten. 
So ging Panaitios davon aus, es gebe theoretische und praktische Tugenden, andere wiederum klassifizierten Tugend als logisch, physisch oder ethisch.

Die Stoa lehnt die Lust als notwendigen Parameter für das Glück ab. 
Zenon schreibt dazu folgendes: \glqq Lust ist das unvernünftige Frohgefühl über eine scheinbar begehrenswerte Sache.\grqq{}
Hier zeigt sich die Ablehnung der epikureischen Lehre, die die Lust zum Mittelpunkt ihrer Argumentation macht. 
Darüber hinaus widerspricht die Stoa Aristoteles, der der überzeugung war, dass auch der gesellschaftliche Stand zum vollkommenen Glück beitragen muss. 
Sie sieht das Glück in der Zuwendung zur Natur, da die Natur als Resultat der Weltvernunft, welche als wesensgleich mit dem griechischen Gott Zeus angesehen wird, die einzige legitime Quelle für ein glückliches Leben ist.
Daher ist das Glück aus Sicht der Stoiker eine gottgegebene Ordnung des natürlichen Lebens. 


\newpage

\subsection{Philosophische Positionen im Mittelalter}

Die Philosophie des Mittelalters wurde maßgeblich durch die Christianisierung Europas geprägt, weshalb es zustande kommt, dass sich vor allem Geistliche mit der Thematik des Glückes beschäftigt haben. 
Augustinus von Hippo, der im 5. Jahrhundert n. Chr. in Hippo im heutigen Algerien Bischof war und neben Hieronymus, Ambrosius von Mailand und Papst Gregor I. zu den vier spätantiken Kirchenvätern des Abendlandes zählt, widmete dem Glück in seinem Werk \glqq de beata vita -- über das Glück\grqq{} ein ganzes Buch.

Darin sagt Augustinus, dass grundsätzlich jeder Mensch nach Glück strebe: \glqq Wir alle wollen glücklich sein.\grqq{}
Dabei sei das Glück nicht grundsätzlich an den Besitz des Begehrten gebunden, sondern an dessen Qualität. 
Wenn es sich dabei um Gutes handele, könne derjenige glücklich werden. 
Wenn es etwas Schlechtes sei, würde er unglücklich sein, obwohl er es besitzt. 
Er weißt zudem darauf hin, dass es dem Glück geradezu hinderlich sei, sich den Lüsten hinzugeben, da jegliches Unerlaubte größtes Unglück sei.
Dagegen sei das Versäumen solch Unerlaubtem das geringere übel.

Im Folgenden macht Augustinus deutlich, dass niemand glücklich werden kann, der nicht das Begehrte besitzt, jedoch auch nicht jeder Mensch glücklich ist, der besitzt, was er begehrt. 
Damit leitet er über auf die Frage, was der Mensch sich verschaffen müsse, um glücklich sein zu können.
Dazu bedürfe es etwas, was der Mensch besitzen könne, das unabhängig von Zufällen oder anderen Einflüssen sei und so von größerer Dauer. 
Denn Zufallsgüter seien vergänglich und könnten verloren gehen. 

So kommt er zu dem Schluss, dass der Mensch Gott haben muss, um glücklich sein zu können, da Gott das einzige sei, das unvergänglich und unabängig von Einflüssen sei. 
Diesen Gedanken führt er fort, indem er festhält, dass jeder glücklich sei, der Gott gefunden habe und dieser sei ein gnädiger Gott. 
Wer jedoch noch nach Gott suchen müsse, könne nicht glücklich sein, da er noch nicht besitze, was er begehre. 
So konstruiert Augustinus das Unglück als Mangel an dem, was man sich wünscht und das Glück als den Zustand, den man erreicht, wenn man den gnädigen Gott findet. 
Daher ist für ihn Glück nichts, was durch materielle Güter erreicht werden kann, sondern nur durch die Gnade Gottes. 
Denn in diesem Zusammenhang nennt er das Beispiel von Orata, der durch die Erfindung der Fußbodenheizung in Bädern reich geworden sei. 
Dieser müsse jedoch stets gefürchtet haben, all sein Hab und Gut durch das Schicksal verlieren zu können und sei durch seine Unsicherheit demzufolge auch nicht fähig, glücklich zu sein.

Augustinus definiert \glqq glücklich sein\grqq{} also als den Zustand, indem man keinen Mangel zu leiden hat, was er mit \glqq weise sein\grqq{} gleichsetzt. 
Ergänzend nennt er in seinen überlegungen, was für ihn Weisheit bedeutet: \glqq Ist sie doch nichts anderes als das Maß des Geistes, das heißt das, womit sich der Geist im Gleichgewicht hält, um weder ins übermaß auszuschweifen noch in die Unzulänglichkeit herabgedrückt zu werden.\grqq{} 
An dieser Stelle beschreibt Augustinus die Weisheit als eine seelische Ausgeglichenheit, die nicht zu Verführungen wie Verschwendung, Machtgier, Hochmut, etc. neigt, wie er es beschreibt. 
Derjenige werde das Glück finden, der sich nicht \glqq dem Trug der Götzenbilder zuwendet, durch deren Gewicht er von Gott abfallen und versinken muß\grqq{}, da dieser dann auch kein Ungemach oder Mangel und damit auch kein Unglück fürchten müsse. 
Und so sieht er das Glück in einer tiefgehenden Verknüpfung zu Gott, der als das Vollkommene den Weg zum Glück darstellt.

Martin Luther, der die Reformation in Deutschland Ende des 15.Jahrhunderts mit anführte, kritisiert in seinen 95 Thesen unter anderem den Ablasshandel, der aus seiner Sicht das Seelenheil -- und damit auch das Glück -- der Menschen gefährde. 
Er schreibt in seiner 30.These: \glqq Niemand kann der Wahrhaftigkeit seiner Reue sicher sein; und noch viel weniger gilt das vom Resultat des vollkommenen Nachlasses.\grqq{}
Luther sieht den Menschen in einer tiefen Schuld gefangen, die ihn darin hindere, ins Himmelreich zu gelangen. 
Daher könne die Buße allein nicht zu einer Rettung führen, sondern lediglich die allumfassende Gnade Gottes. 
Er kritisiert zudem, dass sich gottlose Menschen oder auch Feinde Gottes, wie er sie nennt, das Heil im Reich Gottes erkaufen könnten. 
Der Kauf von Ablassbriefen rette aber keinen Gläubigen vor dem Fegefeuer, sondern wiege ihn in einer falschen Sicherheit.
Folglich lässt sich aus Martin Luthers 95 Thesen der Schluss ziehen, dass Glück für ihn nur durch ein Leben erreicht werden könne, dass sich an den Lehren und Geboten der Bibel orientiere und sich nicht durch Verführungen wie eben beispielsweise den Ablasshandel vom Weg abbringen lasse.


\newpage

\subsection{Philosophische Positionen der Moderne}

In der Moderne löste sich der Glücksbegriff wieder von theologischen Betrachtungsweise und es entwickelten sich im Zuge der Aufklärung Positionen, die wieder an die Ansichten der Antike anknüpften. 

Zu diesen Vertretern zählte unter anderem Immanuel Kant, der zu den bedeutendsten Philosophen der Aufklärung zählt. 
Kant lehnt den klassischen Glücksbegriff, wie er vor allem von Epikur vertreten wurde, ab, indem er schreibt: \glqq Das Wesentliche alles sittlichen Werths der Handlungen kommt darauf an, daß das moralische Gesetz unmittelbar den Willen bestimme. 
Geschieht die Willensbestimmung zwar gemäß dem moralischen Gesetze, aber nur vermittelst eines Gefühls, welcher Art es auch sei, das vorausgesetzt werden muß, damit jenes ein hinreichender Bestimmungsgrund des Willens werde, mithin nicht um des Gesetzes willen: so wird die Handlung zwar Legalität, aber nicht Moralität enthalten.\grqq{}

Besonders deutlich wird an dieser Stelle die Ablehnung Kants gegenüber eines Glücksbegriffes, der durch ein Gefühl oder den Wunsch nach Glück bestimmt wird. 
Kant sieht das Glück vor allem in rational getroffenen Entscheidungen, die Gesetzen und Normen entsprechen. 
Zudem sieht er die Handlungen des Menschen als eine Kausalkette an, in der das Handeln des Menschen durch äußere Einflüsse und Neigungen bestimmt wird. 
Er schreibt: \glqq Von der anderen Seite ist es sich seiner doch auch als eines Stücks der Sinnenwelt bewußt, in welcher seine Handlungen als bloße Erscheinungen jener Kausalität angetroffen werden, deren Möglichkeit aber aus dieser, die wir nicht kennen, nicht eingesehen werden kann, sondern an deren Statt jene Handlungen als bestimmt durch andere Erscheinungen, nämlich Begierden und Neigungen, als zur Sinnenwelt gehörig eingesehen werden müssen.\grqq{}

Unter Glückseligkeit versteht Kant \glqq die Befriedigung aller unserer Neigungen (sowohl extensive der Mannigfaltigkeit derselben, als intensive dem Grade und auch protensive der Dauer nach.)\grqq{} 
Kant ist der Auffassung, dass ein Mensch vor allem dann glücklich sein kann, wenn er würdig handelt und gibt damit eine Antwort auf die Frage, was der Mensch tun solle. 
Gleichzeitig verknüpft er diesen Standpunkt mit der Frage, was der Mensch hoffen dürfe und schreibt dazu: \glqq Ich sage demnach: daß eben sowohl, als die moralischen Principien nach der Vernunft in ihrem praktischen Gebrauche nothwendig sind, eben so nothwendig sei es auch nach der Vernunft, in ihrem theoretischen Gebrauch anzunehmen, daß jedermann die Glückseligkeit in demselben Maße zu hoffen Ursache habe, als er sich derselben in seinem Verhalten würdig gemacht hat, und daß also das System der Sittlichkeit mit dem der Glückseligkeit unzertrennlich, aber nur in der Idee der reinen Vernunft verbunden sei.\grqq{}

An dieser Stelle beschreibt Kant, dass das würdige Handeln in direkter Verbindung zur Hoffnung auf Glückseligkeit steht, so dass die beiden Fragen nach dem Tun und dem Hoffen in einem engen Zusammenhang betrachtet werden müssen. 
Gleichzeitig unterstreicht er mit dem Hinweis auf die Vernunft seine Ablehnung des Glücksbegriffs als einen von Gefühlen hervorgebrachten Zustand. 
Er betont ergänzend dazu, dass eine solch geartete Hoffnung auf die Glückseligkeit, verbunden mit der Würdigkeit zur Glückseligkeit, nur vorliegen könne, wenn nicht nur die Natur zugrunde gelegt würde, sondern vor allem, wenn es eine höchste Vernunft gebe, die moralischen Gesetzen unterworfen sei und ebenfalls hinzugezogen würde.
Für Kant ist eine Glückseligkeit ohne Sittlichkeit nicht denkbar, da er die Sittlichkeit als Voraussetzung für das Glück betrachtet. 
Erst wenn der Mensch im Sinne der Sittlichkeit handelt, kann er das Glück finden. 
Kant erarbeitet ein Bild von einer idealen Welt, die folglich auch einen idealen Ursprung, einen Schöpfer, haben muss.

John Stuart Mill, der im 19.Jahrhundert wirkte und als einer der bedeutsamsten Anhänger des Utilitarismus galt, lehnt seine Vorstellungen von Glück hingegen an die des antiken epikureischen Gedankenguts an. 
Der von ihm propagierte Utilitarismus geht davon aus, dass jede Handlung moralisch korrekt ist, die das Glück befördern kann und moralisch falsch ist, die das Glück gefährdet. 
Er schreibt: \glqq Die Auffassung, für die die Nützlichkeit oder das Prinzip des größten Glücks die Grundlage der Moral ist, besagt, daß Handlungen insoweit und in dem Maße moralisch richtig sind, als sie die Tendenz haben, Glück zu befördern, und insoweit moralisch falsch, als sie die Tendenz haben, das Gegenteil von Glück zu bewirken.\grqq{}
Er ergänzt diesen grundlegenden Gedanken noch, indem er das Glück (happiness) als Lust (pleasure) versteht sowie das Unglück (unhapiness) als Unlust oder das Fehlen von Lust definiert. 
Mill greift an dieser Stelle den epikureischen Begriff der \glqq Lust\grqq erneut auf. 
Mill sieht die Lust und die Vermeidung von Unlust als die zentralen Faktoren an, die wichtig sind, um dauerhaft glücklich zu sein. 
Dabei nimmt er an, dass \glqq alle anderen wünschenswerten Dinge (die nach utilitaristischer Auffassung ebenso vielfältig sind wie nach jeder anderen) entweder deshalb wünschenswert sind, weil sie selbst lustvoll sind oder weil sie Mittel sind zur Beförderung von Lust und zur Vermeidung von Unlust.\grqq{}

John Stuart Mill differenziert verschiedene Arten von Freuden aus, die nach seiner Auffassung in unterschiedlichem Maße zur Lust des Menschen beitragen. 
Er stellt daher fest, dass es unsinnig wäre, wenn die Wertigkeit einer Freude nur durch die Quantität und nicht auch durch deren Qualität mitbestimmt wird. 
Das heißt: Es gibt Dinge, die den Menschen häufiger Freude bereiten, als andere. 
Diese können jedoch zwangsläufig nicht von selbiger Qualität sein wie diese, die nicht in der gleichen Häufigkeit jedoch in einer größeren Intensität auftreten. 
Daraus folgt, dass der Mensch stets immer die Freuden vorzieht, die ihm den größtmöglichen Nutzen (lat. Utilitas = Nutzen) verschaffen. 
Aus seiner Sicht ist gerade die Form der Lebensführung willkommen, die ein größtmögliches Maß an Lust und ein möglichst geringes an Unlust beinhalte. 
Diese Art der Lebensführung bezeichnet Mill als den letzten Zweck des menschlichen Lebens.

Mills skizzierte Lebensführung, die zum Glück des Menschen führen soll, basiert auf einer Glücksvorstellung, die das Glück nicht als dauerhaften Zustand des Wohlbefindens begreift: 
\glqq Freilich: versteht man unter Glück das Fortdauern einer im höchsten Grade lustvollen Erregung, dann ist die Unerreichbarkeit von Glück nur zu offensichtlich. 
Der Zustand der überschwänglichkeit hält höchstens einige Augenblicke, in einigen Fällen -- mit Unterbrechungen -- auch Stunden und Tage an;\grqq{}
 Zudem vergleicht er das Glück mit einer auflodernden Flamme, die im Gegensatz zur Glut, nur gelegentlich ausbricht. 
 Damit meint Mill, dass das Leben nicht aus einem andauernden Zustand der Freude bestehe, sondern das ein glückliches Leben eines sei, dass durch Abschnitte der Lust und durch Abschnitte der Unlust bestimmt werde.
 
Das persönliche Glück des Einzelnen, das John Stuart Mill in seinem Werk \glqq Utilitarismus\grqq beschreibt, steht im engen Zusammenhang mit dem Glück der Gesellschaft. 
Nach Mill müssen \glqq Gesetze und gesellschaftliche Verhältnisse das Glück -- oder wie man es in der Praxis auch nennen kann -- die Interessen jedes einzelnen so weit wie möglich mit den Interessen des Ganzen in übereinstimmung bringen.\grqq{}
Damit sieht Mill sowohl den Einzelnen als auch die Gesellschaft in der Pflicht, alle Interessen zu vereinbaren und greift damit auch Aristoteles erneut auf, der bereits in der Antike die Verbindung mit den gesellschaftlichen Werten im Hinblick auf den Glücksbegriff verwies.

Friedrich Nietzsche (1844-1900) sieht das Glück im Gegensatz zu Kant oder Mill nicht als äußerliche Erscheinung an, sondern als etwas, dass dem Menschen in seiner Psyche innewohnt. 
Er schreibt dazu: \glqq Die Bestie in uns will belogen werden; Moral ist Notlüge, damit wir von ihr nicht zerrissen werden.\grqq{}
Nietzsche betrachtet die Moral als Irrtum, ohne den der Mensch ein Tier geblieben wäre. 
Jedoch hat er sich durch die Abgrenzung vom Tierischen auch strengere Gesetzmäßigkeiten auferlegt, denen er fortan unterliegt.

Für seine Ausführungen über das Glück des Menschen entfaltet Nietzsche drei Säulen, die zur Erlangung des Glücks notwendig sind: Die \textit{Gewohnheit}, \textit{die Schönheit} und \textit{den Unsinn}. 
Er schreibt über die Gewohnheit in seinem Werk \glqq Menschliches, Allzumenschliches\grqq{}: \glqq Die Lust in der Sitte. -- Eine wichtige Gattung der Lust und damit der Quelle der Moralität entsteht aus der Gewohnheit.\grqq{}
Dinge, die man aus einer Routine heraus tue, sieht Nietzsche als positiver konnotiert als andere Tätigkeiten, da sie auch durch die Erfahrung mitbeeinflusst werden. 
In diesem Zusammenhang sieht er die Sitte als die Verschmelzung des Angenehmen mit dem Nützlichen an und dass jede Sitte zur Gewohnheit, demnach also zur Lust werden kann.

Die zweite Säule des menschlichen Glücks beschreibt Nietzsche mit der Schönheit: \glqq Der langsame Pfeil der Schönheit. -- Die edelste Art der Schönheit ist die, welche nicht auf einmal hinreißt, welche nicht stürmische und berauschende Angriffe macht (eine solche erweckt leicht Ekel), sondern jene langsam einsickernde, welche man fast unbemerkt mit sich fortträgt und die einem im Traum einmal wiederbegegnet, endlich aber, nachdem sie lange mit Bescheidenheit an unserem Herzen gelegen, von uns ganz Besitz nimmt, unser Auge mit Tränen, unser Herz mit Sehnsucht füllt.\glqq{}
An diesem Punkt bekennt Nietzsche, dass für ihn die Schönheit des Glücks darin besteht, dass er das Glück als einen Zustand der langanhaltenden Zufriedenheit begreift und nicht etwa als wiederkehrende Phasen des Glücks, wie Mill dies in seiner utilitaristischen Betrachtung tut. 
Hier wird die Schönheit auch mit einer gewissen Unbeschwertheit assoziiert.

Nietzsches dritte Säule besteht in der menschlichen Freude am Unsinn, die er in jeder Situation vermutet, in der gelacht wird: \glqq Wie kann der Mensch Freude am Unsinn haben? Soweit nämlich auf der Welt gelacht wird, ist dies der Fall; ja man kann sagen, fast überall wo es Glück gibt, gibt es Freude am Unsinn.\grqq{}
Der Unsinn befreie den Menschen vom Zwang des \glqq Notwendigen, Zweckmäßigen und Erfahrungsgemäßen\grqq{}.


\subsection{Zusammenfassung}

In diesem Kapitel konnte festgestellt werden, dass sich der Glücksbegriff nicht in eine abschließende Definierbarkeit bringen lässt. 
Allerdings lassen sich allgemeingültige Parameter für das Glück festlegen, die zusammengefasst werden sollten.

Evident ist, dass Glück als Resultat optimaler Gegebenheiten für ein bestimmtes Ereignis angesehen werden kann, wobei die Art dieses Ereignisses für diese Sichtweise irrelevant ist. 
Von Wichtigkeit ist in dieser Betrachtung nur die Abfolge von Ereignissen. 
Darüber hinaus wurde -- vor allem in der Antike -- das Glück, wie beispielsweise auch Krieg oder Tod, mit Gottheiten assoziiert, die als Patrone dieser Zuständigkeitsfelder betrachtet wurden. 

Auch die Differenzierung, die die deutsche Sprache vornimmt, indem sie von \glqq Gl/"uck haben\grqq{} und \glqq glücklich sein\grqq{} spricht, ist von zentraler Bedeutung für den Glücksbegriff, da sie eine Fokussierung der Betrachtung vornimmt. 
Während \glqq Glück haben\grqq{} die vorhin beschriebene Abfolge von Ereignisabfolgen bezeichnet, lässt sich  \glqq glücklich sein\grqq{} als emotionale Regung des Menschen fassen. 
Die Unmöglichkeit, das Glück genauer zu definieren lässt sich auch damit erklären, dass jeder Mensch durch seine individuelle Prägung ein anderes Verständnis von Glück oder glücklich sein entwickelt und dadurch abweichende Einschätzungen entstehen können. 
Trotz dieser milieugeprägten Unterschiede lässt sich trotzdem konstatieren, dass das Glück als höchstes Ziel des Menschen gesehen werden kann.

Auch in der Antike wurden höchste Ziele f/"ur die menschliche Existenz formuliert, doch unterschieden sich diese deutlich voneinander. 
Während Aristippos die Glückseligkeit als den Zusammenschluss von Lust jeglicher Art verstand und diesen Zustand durch den Einschluss seelischer Leiden als utopisch betrachtete, sah Aristoteles den Menschen als zielstrebiges Wesen an, dass sich durch Teilziele wie Gesundheit oder Geld der Eudaimonia -- dem höchsten Ziel des Menschen -- annähern kann und will. 
Die Stoa wiederum widerspricht beiden Positionen, da sie die Lust ablehnt und das Glück in keiner Abhängigkeit des gesellschaftlichen Standes betrachtet. 
Für sie liegt das Glück in der Natur als gottgegebene Ordnung und als einzig legitime Quelle des Glücks.

Im Mittelalter wurde der Glücksbegriff stark in den Kontext des Gottesglaubens gestellt, so dass Gott als unabdingbare Instanz für das Glück des Menschen verstanden wurde. 
So lehnte Augustinus -- analog zur Stoa -- die Lust ab, da sie das Unglück eines Menschen sei. 
Er vertrat die Auffassung, dass der Mensch das Glück besitzen wolle, er jedoch sein eigenes Unglück verstärken würde, wenn es sich bei seinem Besitz um etwas schlechtes handele. 
Jeglichen Besitz des Menschen sah er als vergänglich an und daher nicht tauglich für dauerhaftes Glück. 
Daher sei der Mensch von Gott abhängig, da dieser das einzige Wesen sei, dass unabhängig von Einflüssen und unvergänglich sei und der Mensch brauche die Weisheit, um sich von Verführungen zu entfernen. 
Auch Luther sah den Menschen als von Gott abhängig an in Bezug auf das persönliche Glück, welches durch Versuchungen wie den Ablasshandel verstärkt würde. 
In der modernen Philosophie wurden die Positionen der Antike erneut aufgegriffen, was zur Folge hatte, dass sich Ansätze der antiken Glücksvorstellungen dort wieder finden lassen.
 
Immanuel Kant betrachtet das Glück als Folge rationaler Handlungen. 
Für ihn kann Glück nur dann entstehen, wenn sich das Handeln des Menschen an gesellschaftlichen Normen und Gesetzen orientiert und nicht an Wünschen und Gefühlen. 
Das Handeln sieht er als untrennbare Kausalität der Ereignisse bzw. als Ereigniskette an, die von außen durch seine Triebe und Neigungen bestimmt werden. 
Folglich sieht er die Glückseligkeit als eine Befriedigung dieser menschlichen Bedürfnisse an. 
Kant erachtet die Sittlichkeit menschlichen Handelns als Grundvoraussetzung dafür, glücklich zu sein und lehnt eine reine Zugrundelegung der Naturbetrachtung ab. 
Für ihn muss darüber hinaus eine schöpfende Instanz zur Sittlichkeit beitragen, indem er den sittlich handelnden Menschen den Weg zum Glück eröffnet.

John Stuart Mill hingegen sieht vor allem den Nutzen in seiner Glücksbetrachtung als zentral an. 
Er greift dabei die epikureischen Lehren auf, indem auch er feststellt, dass der Mensch zwischen Handlungen steht, die Lust erzeugen und solchen, die Lust behindern. 
Erstere sieht er als moralisch korrekt, letztere analog als moralisch schlecht an. 
Jedoch erweitert er das epikureische Weltbild, indem er lustvolle Handlungen ausdifferenziert nach solchen, die er quantitativ sind und jenen, die eher qualitativ sind. 
Dadurch gebe es Handlungen, die öfter vollzogen würden, um Lust zu generieren und solche die seltener stattfinden würden. 
Daraus schließt Mill, dass diese beiden in Relation nicht das selbe Maß an Lust enthalten können, so dass bestimmte Handlungen anderen vorgezogen werden können. 
Bei dieser Argumentation sieht Mill den Menschen stets im Kontext der Nützlichkeit, derer er unterworfen ist und die er als höchstes Ziel betrachtet. 
Das Glück ist laut Mill kein Zustand, der dauerhaft aufrecht erhalten werden kann, sondern das Leben werde bestimmt durch Zeitspannen der Lust und solche der Unlust. 
Dieser Zustand müsse von gesellschaftlichen Vorgaben und Gesetzen unterstützt werden, so dass auch der Gesellschaft eine Verantwortung für das personifizierte Glück zukommt.

Friedrich Nietzsche schlussendlich grenzt sich ab von Kant und Mill, da das Glück aus seiner Sichtweise etwas grundtypisch menschliches sei und ihm innewohne. 
Um das Glück erlangen zu können, gelte es für den Menschen drei Säulen in Einklang zu bringen: 
Die Gewohnheit solle dazu führen, durch routinierte Handlungen die persönliche Lust aufrecht zu erhalten, die Schönheit des Glücks, dass Nietzsche im Gegensatz zu Mill eben doch als etwas Dauerhaftes begreift und nicht als repetierende Momente und abschließend den Unsinn, der den Menschen durch Freude von seinen Zwängen befreien solle.

Es lässt sich also entdecken, dass sich die Glücksvorstellungen im Laufe der Jahrhunderte stark verändert haben und antike Positionen Eingang in die moderne Philosophie gefunden haben.