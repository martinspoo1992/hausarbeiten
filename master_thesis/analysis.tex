\section{Gesprächsanalyse mit Grundschulkindern an der GS Kesselheim}

Für die Analyse des transkribierten philosophischen Unterrichtsgesprächs wurde die Methode der qualitativen Inhaltsanalyse gewählt. 
Der nachfolgende Absatz soll sich daher mit der Methodik der qualitativen Inhaltsanalyse auseinandersetzen. 
Zudem soll die Ausgangssituation der Tonaufnahme, die der Transkription zugrunde liegt, näher erläutert werden und die Transkriptionsregeln, nach denen dieses Transkript angefertigt wurde, herausgestellt werden. 
Anschließend werden die inhaltsanalytischen Erkenntnisse, die aus dem Transkript entnommen werden konnten, dargelegt und in einem weiteren Schritt mit den zu Beginn vorgestellten philosophischen Positionen in Beziehung gesetzt.

Schlussendlich sollen die gewonnen Ergebnisse dann final zusammengestellt werden.


\subsection{ Methodik}

Die qualitative Inhaltsanalyse stellt eine der wesentlichsten Methoden zur Analyse von Texten dar. 
Ihr Ziel ist es, \glqq Kommunikationsinhalte, die in Form von Texten vorliegen, wissenschaftlich zu analysieren.\grqq{} \cite[S.\,20]{WK07}
Jedoch lässt sich der Begriff \glqq Inhaltsanalyse\grqq{} nicht als bloße Auseinandersetzung mit den Inhalten von Kommunikation beschreiben. 
Mayring stellt fest, dass sich eine Begriffsklärung der Inhaltsanalyse bereits im wissenschaftlichen Diskurs als schwierig erwies. 

Während Jürgen Ritsert die Inhaltsanalyse als \glqq ein Untersuchungsinstrument zur Analyse des \glqq gesellschaftlichen\grqq{}, letztlich des \glqq ideologischen Gehalts\grqq{} von Texten\grqq{} \cite[S.\,11]{JR07} betrachtet und sie damit mehr in den ge"-sell"-schaft"-lich--i"-de"-o"-lo"-gisch"-en Kontext als in den inhaltlichen rückt, beschränkt sich Alfred L. George in seiner Definition lediglich auf die Absichten, die der Sprechende mit seiner Aussage zum Ausdruck bringt: 
\glqq Kurz, Inhaltsanalyse wird verwendet als ein diagnostisches Instrument, um spezifische Schlußfolgerungen über bestimmte Aspekte des zielgerichteten Verhaltens (purpose behavior) des Sprechers zu ziehen.\grqq{} \cite[S.\,11]{AG07}
Mayring fasst die Beschaffenheit der Inhaltsanalyse in sechs Punkten zusammen: 
Er betont, dass die Inhaltsanalyse sich mit Kommunikation beschäftigt und sich diese meist in Form von Sprache niederschlägt. 
Er weist jedoch ausdrücklich darauf hin, dass auch Musik, Bilder, etc. zu einer analytischen Betrachtung herangezogen werden können. 
Daran angeschlossen liegt diese Kommunikation fixiert vor, das heißt, dass sich die Inhaltsanalyse mit Kommunikation auseinandersetzt, die in Texten, Bildern, Noten, etc. festhalten wurde, was er als symbolisches Material bezeichnet\cite[S.\,12]{PM07}.

Die Inhaltsanalyse hat ferner nicht zum Ziel, das Material zu interpretieren. 
Vielmehr wird darauf Wert gelegt, den Untersuchungsgegenstand systematisch zu betrachten. 
Daran schließt er auch an, dass die Inhaltsanalyse einer bestimmten Regelhaftigkeit folgt. 
Durch diese festen Regeln werde erreicht, dass jedermann eine solche Analyse nachvollziehen und hinterfragen könne. 
In die Analyse fließt auch der Bezug zur Theorie mit ein, der bei der Inhaltsanalyse für Mayring einen wichtigen Bezugspunkt darstellt. 
Daher ist es auch nicht Gegenstand der Analyse, einen Text zu referieren, sondern sie \glqq analysiert ihr Material unter einer theoretisch ausgewiesenen Fragestellung; die Ergebnisse werden vom jeweiligen Theoriehintergrund her interpretiert und auch die einzelnen Analyseschritte sind von theoretischen Überlegungen geleitet.\grqq{} \cite[S.\,12]{PM07}
 Der sechste Punkt besteht nach Mayring im Endeffekt darin, dass die Inhaltsanalyse das Material nicht seiner selbst willen untersucht, sondern, dass sie die Ausfindigmachung von Rückschlüssen auf Kommunikation erreichen will. 
 
Ihr Fokus liegt dabei nicht auf der Häufigkeit des Auftretens bestimmter sprachlicher Ausdrücke, wie es die quantitative Inhaltsanalyse zum Ziel hat. 
Die qualitative Inhaltsanalyse untersucht, welche inhaltlichen Aussagen in einer Textpassage geäußert werden und will sich diesen Inhalten wissenschaftlich nähern. 
Knapp sieht zwischen quantitativer und qualitativer Inhaltsanalyse ein Spannungsverhältnis, dass er zwischen dem Verstehen von Texten und der Objektivität von Analyseverfahren verortet. 
Er schreibt: \glqq Mit dem Spannungsverhältnis zwischen dem Verstehen einerseits und der Objektivität andererseits wird eine Grundproblematik der Inhaltsanalyse sichtbar. 
Sie spiegelt sich in der Kontroverse zwischen qualitativer und quantitativer Inhaltsanalyse, wobei erstere eher das Verstehen, letztere eher die Objektivität des Analyseverfahrens beansprucht.\grqq{} \cite[S.\,20f]{WK07}
Knapp stellt weiterführend fest, dass die Inhaltsanalyse zwei weitere wichtige Kriterien erfüllen muss, die er als Validität und Reliabilität bezeichnet. 
Die Reliabilität bezeichnet , dass unterschiedliche Forscher zu unterschiedlichen Zeitpunkten zum selben Ergebnis kommen müssten. 
Validität wiederum bezeichnet die Gültigkeit eines wissenschaftlichen Verfahrens. 
Es lässt sich demnach festhalten, dass die qualitative Inhaltsanalyse drei Kriterien erfüllen muss: Objektivität, Reliabilität und Validität.

Betrachtet man nun, was Texte -- also auch Transkripte von Unterrichtsgesprächen -- aus linguistischer Sicht letztlich sind, nämlich die Verschriftung von sprachlichem Handeln, liegt es nahe, die Methodik der qualitativen Inhaltsanalyse auch im Hinblick sprachtheoretischer Aspekte zu betrachten. 
Knapp nimmt in diesem Zusammenhang Bezug auf Ludwig Wittgenstein und dessen Zitat \glqq Die Bedeutung des Wortes ist sein Gebrauch in der Sprache\grqq{} \cite[S.\,21]{LW07}, woraus er den Schluss zieht, dass die Semantik von Wörter stets an soziale Handlungsweisen gebunden sind. 
Für die Inhaltsanalyse bedeutet das, dass kein Text eine allgemeingültige Bedeutung hat, sondern dass verschiedene Leser von Texten auch abweichende Inhalte aus diesen Texten ziehen werden.  
Zudem ist die Extrahierung von Textinhalten auch abhängig vom Vorwissen des Lesers sowie davon, ob ein Text zum ersten Mal gelesen wird oder ob er schon bekannt ist\cite[S.\,22]{WK07}.

Für eine inhaltsanalytische Betrachtung von Texten ist laut Knapp auch von Bedeutung, welche Aussagen in einem Text vorgenommen werden. 
So könne man aus der quantitativen Häufigkeit bestimmter sprachlicher Konstruktionen nicht direkt auf den Inhalt schließen, sondern man müsse den Inhalt kennen. 
Knapp illustriert diesen Zusammenhang sehr treffend anhand des Beispiels, dass ein Text, in dem die Worte \glqq Joschka Fischer\grqq{} und Berlin vorkommen, keinen hinreichenden Ansatzpunkt für die Bestimmung eines Themas des Textes geben. 
Analog dazu verhält es sich zum vorliegenden Transkript. 
Informationen wie das Umfeld der Erhebung, die Klassenstufe und die Gesprächssituation sind wichtige Informationen, die für das Verständnis und für eine fundierte, qualitative Inhaltsanalyse von Bedeutung sind. 
Es liegt daher auf der Hand, dass eine bloße Betrachtung der sprachlichen Ausdrucksweisen für ein substantiiertes Verständnis nicht ausreichend sind. 
Vielmehr muss der Sinn eines Satzes \glqq in einem aktiven und konstruktiven Prozess erschlossen werden, der mehr einbezieht als die vorliegende Äußerung.\grqq{} \cite[S.\,28]{WK07}

Werner Knapp weist neben dem Bezug auf die Beschaffenheit von Texten in Bezug auf ihren Inhalt sowie die Themen, die in ihnen entwickelt werden, in einem weiteren Schritt auf die Deutung sprachlicher Zeichen hin, die dazu beiträgt, eine wissenschaftliche Vergleichbarkeit von Analysen zu gewährleisten. 
Er nimmt dabei Bezug auf Hans-Jürgen Bucher  und Gerd Fritz, die in ihrem Beitrag \glqq Sprachtheorie, Kommunikationsanalyse, Inhaltsanalyse\grqq{} vier Aspekte nennen, die sie für wichtig erachten: 
Zunächst nennen sie das gemeinsame Wissen der Kommunikationsteilnehmer, welches u.a. das Wissen über eine Situation, in der gehandelt wird, umfasst. 
Ein weiterer liegt im Sequenzzusammenhang sprachlicher Handlungen, da die Bedeutungen von sprachlichen Handlungen auch von ihrer Positionierung abhängen können. 

Auch der thematische Zusammenhang ist von Signifikanz für das Verständnis, da Aussagen eines Textes immer im Kontext des Ganzen verstanden werden müssen. 
Wird dieser Zusammenhang nicht berücksichtigt, können Aussagen fehlgedeutet werden. 
Der letzte Aspekt liegt in der inneren Struktur von sprachlichen Handlungen begründet, die -- in Anlehnung an die Sprechakttheorie von John Searle -- den illokutionären Aspekt, der besagt, dass eine Äußerung stets eine Funktion hat, welche beispielsweise feststellend, bewertend, rechtfertigend etc. sein kann. 
Hinzu kommt der indem--Zusammenhang, welcher aussagt, dass eine Handlung durch Sprache vollzogen werden kann, indem eine andere vollzogen wird\cite[S.\, 137f]{HB80}.

Bucher und Fritz nennen zusätzlich Qualitätskriterien für Kommunikationsanalysen. 
So nennen sie zunächst das Prinzip der zusammenhängenden Betrachtung, das bedeutet, dass Handlungssequenzen betrachtet werden, statt isolierter Handlungen sowie Zusammenhänge statt einzelner Ausdrücke ausgewertet werden\cite[S.\,137f]{HB80}.
 Das zweite Prinzip ist das Prinzip der Explizitheit, das besagt, dass Regeln und Hintergrundannahmen formuliert werden sollen, um Interpretationen über die subjektive Meinung hinaus argumentativ stützen zu können. 
 Das dritte und letzte Prinzip beschreiben Bucher und Fritz als Prinzip der Reflexivität, nach dem der Forschende stets reflektieren müsse, welches Verständnis er von den Aussagen eines Textes hat und wie sich sein Sprachstil darstellt\cite[S.\,143]{HB80}.
 
 \newpage
 
 


\subsubsection{Gewinnung der Aufnahme}

Die Aufnahme, die zur inhaltlichen Analyse transkribiert wurde, entstand am 08. Dezember 2015 in der Grundschule Koblenz--Kesselheim. 
Sie wurde im Rahmen des Seminars \glqq Philosophieren mit Kindern im Sachunterricht\grqq{} im Wintersemester 2015/16 zwischen 8:10 und 8:40 in einem 4. Schuljahr der Grundschule Koblenz--Kesselheim von den Studierenden Martin Spoo und Pascal Gerigk erstellt. 

Das Seminar \glqq Philosophieren mit Kindern an der Grundschule\grqq{} fand im Kontext des Mastermoduls 10 im Studiengang Grundschulbildung für das Lehramt an Grundschulen an der Universität Koblenz-Landau, Campus Koblenz unter der Leitung von Prof. Dr. Heike de Boer statt. 
Es gliederte sich in einen theoretischen Teil, in dem die Studierenden mit den theoretischen Inhalten des Philosophierens mit Kindern vertraut gemacht wurden und in einen praktischen Teil, in dem die Studierenden eigene Erfahrungen mit philosophischen Gesprächen machen sollten.

An verschiedenen Grundschulen in Koblenz und zum Teil auch in anderen Grundschulen in Rheinland-Pfalz wurden dann von Gruppen zu je zwei Studierenden jeweils drei philosophische Gespräche geführt und diese mit Aufnahmegeräten aufgezeichnet. 
Anschließend wurden diese Aufnahmen nach den Transkriptionsregeln von GAT (Gesprächsanalytisches Transkriptionssystem), welche im Verlauf dieser Arbeit noch genauer geklärt werden, transkribiert, um den Studierenden einen Einblick in die Sprecherrolle als spätere Lehrerinnen und Lehrer zu geben. 

Schwerpunkt war dabei, dass die Studierenden das Gespräch -- im Gegensatz zum klassischen Unterrichtsgespräch -- nicht aktiv leiten sollten. 
Überdies sollten die Studierenden beachten, besonders offene Fragestellungen zu formulieren, um einen Lernprozess bei den Kindern in Gang zu bringen und um dualistische Antworten zu vermeiden. 
Außerdem ergab sich für die Studierenden die Gelegenheit, mit Schülern über philosophische Themen wie Freundschaft oder auch Glück ins Gespräch zu kommen und die Aussagen der Schülerinnen und Schüler auch im Hinblick auf die jeweiligen Klassenstufen hin zu untersuchen. 
So wurde den Studierenden nach den Unterrichtsbesuchen stets die Möglichkeit gegeben, sich über ihre gesammelten Erfahrungen miteinander auszutauschen.
Gegen Ende des Semesters wurde schließlich die Entwicklung der eigenen Sprecherrolle in den philosophischen Gesprächen reflektiert.

Das Transkript, das das Gespräch zwischen den Studierenden Martin Spoo und Pascal Gerigk und der Klasse beinhaltet, setzt in Minute 6:30 der Aufnahme ein, da die Studenten mit den Kindern zunächst über die Geschichte von Hans im Glück debattierten und unklare Begriffe mit den Schülern klärten. 
Diese Geschichte wurde als Einleitung für die Schülerinnen und Schüler in die Thematik genutzt.
Da diese Sequenz jedoch für die Untersuchung der Aussagen der Schülerinnen und Schüler irrelevant ist, wurde sie bei der Transkription des Gespräches ausgelassen.


\subsubsection{Transkriptionsregeln nach GAT}

Gegenstand der folgenden Analyse ist ein Basistranskript des vorhin erläuterten Gespräches, dass nach den Regeln des Gesprächsanalytischen Transkriptionssystems (GAT) für Basistranskripte erstellt wurde. 
Dieses System wurde im Jahre 1998 von einer Autorengruppe bestehend aus den Linguisten Margret Selting, Peter Auer, Birgit Barden, Jörg Bergmann Elizabeth Couper-Kuhlen, Susanne Günthner, Christoph Meier, Uta Quasthoff, Peter Schlobinski und Susanne Uhmann entwickelt, um ein einheitliches Untersuchungsinstrument für die forschende Erarbeitung von verschriftlichter gesprochener Sprache zu erhalten. 

Das GAT wurde nach bestimmten Kriterien entwickelt, die kurz erläutert werden sollen. 
Durch eine Gewährleistung der Ausbaubarkeit soll sicher gestellt werden, dass ein Transkript ohne vollständige Neubearbeitung ergänzt und verfeinert werden kann. 
Um auch der Lesbarkeit für Nicht--Linguisten gerecht zu werden, wurde auf die Verwendung von Darstellungen in phonetischer Schrift u. ä. bewusst verzichtet, welche jedoch je nach Anwendungsbereich des Transkriptes noch ergänzt werden können. 
Zudem wurde das System besonders ökonomisch geplant, indem jedem Transkriptionszeichen genau eine Bedeutung zugewiesen wurde und eine Robustheit über die verschiedenen Computersysteme hinweg erreicht, da in Transkripten nach den GAT-Regeln keine Sonderzeichen verwendet werden. 
Bei der Zuweisung der Transkriptionszeichen wurde darauf geachtet, diese nicht arbiträr -- also willkürlich -- zu setzen, sondern sie ikonischen Prinzipien folgen zu lassen. 
Schließlich wurden auch die Relevanz, die formbezogene Parametrisierung und die Kompatibilität mit anderen Systemen beachtet. 

Ziel des GAT ist es, sprachliche Handlungen zu veranschaulichen, die für die Forschung interessant sind oder deren Interesse für die Forschung nachgewiesen werden soll.
Dabei soll es verschiedene Merkmale von Gesprochenem miteinbeziehen, d.h. es sollen nicht allein Interpretationen im Text vorgenommen werden, sondern es sollen Interpretationen mit Berücksichtigung anderer Merkmale wie Tonhöhe, etc. mitberücksichtigt werden.
Daneben soll es zudem auch anschlussfähig sein an andere Transkriptionssysteme\cite[S.\,92f]{MS98}.

Allgemein besteht ein Transkript aus einem Transkriptkopf, der wichtige Informationen über die Herkunft des Transkriptes, den Aufnahmetag, den Ort, etc enthält und aus dem Gesprächstranskript selbst. 
Dieses Gesprächstranskript folgt grundsätzlich der zeitlichen Abfolge des Gespräches und es wurden Konventionen festgelegt, die ein solches Transkript erfüllen muss. 

Grundsätzlich ist darauf zu achten, einen systemunabhängigen Schriftyp zu verwenden. 
Daher wird in der Regel die Schriftart Courier in der Schriftgröße 10 verwendet. 
Außerdem wird das Transkript gänzlich in Minuskeln verfasst, da Majuskeln für die Akzentuierung bestimmter Passagen benötigt werden. 
Jede Zeile des Transkriptes wird zur besseren Lesbarkeit mit einer Nummer versehen, es werden drei Leerstellen gesetzt und es folgt die Bezeichnung des Sprechers. 
Nach erneuten drei Leerstellen folgt der eigentliche Transkriptionstext\cite[S.\,95]{MS98}.

Allgemein wird zwischen Basistranskripten und Feintranskripten unterschieden. 
Während Basistranskripte lediglich analytische Mindestanforderungen erfüllen müssen, werden in Feintranskripten unter anderem auch die Tonhöhen, Akzentuierungen oder Sprechgeschwindigkeiten betrachtet.
Jedoch lässt sich sagen, dass für eine qualitative Inhaltsanalyse ein Basistranskript für ausreichend erachtet werden kann, da letztlich nur die inhaltlichen Aussagen der Schülerinnen und Schüler von Bedeutung sind und nicht beispielsweise Tonhöhe oder Sprechgeschwindigkeiten.


\newpage


\subsection{Inhaltsanalyse}

Das Gespräch wird eröffnet durch die Frage der Studenten, was die Schülerinnen und Schüler unter dem Begriff Glück verstehen. 
Stefanie antwortet auf diese Frage nach einer Bedenkzeit von ca. 7 Sekunden für die Schüler darauf, dass Glück für sie in Situationen vorhanden sei, in denen sie getröstet würde. 
Damit richtet sie den ersten, definierenden Blick in Richtung eines Glücksbegriffes, der im Kontext zwischenmenschlicher Beziehungen steht. 
Da keine weiterführende Wortmeldung entsteht, entfaltet Stefanie eine weitere Dimension des Glücksbegriffs, den sie in zufälligen Ereignissen verortet, die für die Person, die in dieser Situation Glück empfindet, in einem wünschenswerten Abschluss einer Handlung besteht. 
Sie nennt in diesem Falle das Losen innerhalb der Klassengemeinschaft und verweist damit auf konkrete, in der Klasse vorherrschende Rituale. 

Mario knüpft in Zeile 8 bis 10 an diese Perspektive der positiven Abfolge von Ereignissen an. 
Dabei nennt er das hypothetische Beispiel -- er verweist wortwörtlich darauf, dass ihm dies noch nicht selbst geschehen ist -- einer Person, die aus einer bestimmten Höhe herunterfalle und sich nicht festhalten könne. 
In diesem Kontext sei Glück, wenn sich diese Person nicht ernsthaft verletze oder andere Beeinträchtigungen infolge eines Sturzes zu erwarten habe. 
Darius weitet dieses Beispiel aus, da er in den Zeilen 12 bis 15 konkret beschreibt, dass es Glück sei, sich in einer solchen Situation nicht den Arm gebrochen zu haben.
Er stimmt damit Marios Grundthese von Glück zu, die besagt, dass Glück in einer günstigen Abfolge von Ereignissen besteht, welche in diesem Fall die Vermeidung einer Verletzung zur Folge hat. 
Die Antwort des Studenten, dass es sich bei solchen Situationen um Glück im Unglück handele und die die Antworten von Mario und Darius nochmals resümierend auf den Punkt bringt, wird indes von den besagten Schülern aber auch von den anderen Mitschülerinnen und Mitschülern kommentarlos hingenommen.

Aysche hat im Gegensatz zu den Ansichten von Stefanie, Mario und Darius, die offensichtlich das Glück vor allem mit Zufall, aber auch -- im Falle von Stefanie -- mit der persönlichen Beziehung zwischen Menschen verbinden, eine vollkommen andere Ansicht, wie man sich dem Glücksbegriff nähern könne. 
Sie sieht das Glück als Ergebnis von eigenen erreichten Zielen und Leistungen und weist dabei auf den Sachverhalt hin, dass Glück für sie darin bestünde, dass sie gute Noten erreiche.
Dies bringt sie in Zeile 17 zum Ausdruck. 
Aysche argumentiert folglich aus einem starken Leistungsverständnis heraus und es lässt sich weiter feststellen, dass sie die Kernaussage des bekannten Sprichwortes \glqq Jeder ist seines Glückes Schmied\grqq{} für ihre Sicht auf das Glück heranzieht. 
Stefanie geht bei ihrer erneuten Wortmeldung nicht auf diese neugewonnene Perspektive von Aysche ein und erläutert erneut das Glück in Form des Kontaktes von Menschen. 
Wenn man in schwierigen Situationen für andere da sei und für sie sorgen würde, dann könnte folglich eine Person, die in einer solchen Situation ist, wieder Glück erfahren und empfinden. 

Während sich die Schüler zunächst also über die Beschaffenheit des Glücks austauschen und bereits erkennen, dass das Verständnis von Glück auch immer eine subjektiv dominierte Komponente in sich trägt, geht Stefanie nun darauf ein, dass es einen Dualismus von Glück und Pech geben müsse (Zeile 23-27).
Sie zeichnet das Pech als klaren Gegenspieler zum Glück und untermauert diese These mit einem Beispiel. 
Demnach könne Pech für eine Person der Verlust eines wichtigen Gegenstandes oder etwas Vergleichbarem sein, den man für eine andere Person erarbeitet oder hergestellt hat. 
Ob es dabei unter Umständen um Geschenke geht, kann nicht abschließend aus dem Transkript heraus geklärt werden. 
Jedoch lässt sich die Vorstellung von Pech als Verlust von etwas klar herausstellen. 
Auf diesen Antagonismus zwischen Glück und Pech geht sie noch tiefergehend in Zeile 30-32 ein, da sie festhält, dass von ihrem Standpunkt aus alles das zum Glück gezählt werden könne, was \glqq richtig passiert\grqq{} und jegliches Geschehen, bei dem \glqq was schiefläuft\grqq{} zum Pech gezählt werden könne.
Mit ihren Formulierungen trifft Stefanie den Kern dieses Gegensatzes zwischen Ereignisketten, die erfolgreich oder wünschenswert verlaufen und solchen, die nicht das erwünschte Ergebnis zutage fördern. 

Stefanies Gedankengang zum Verhältnis von Glück und Pech ergänzt Mario mit einem Alltagsbeispiel aus dem Klassenkontext (Zeile 33-37). 
Er beschreibt dazu eine Gesprächssituation im schulischen Unterricht, in der ein Schüler von der Lehrperson eine Frage gestellt bekommt und die zugehörige Antwort nicht nennen kann. 
Grundsätzlich würde man dieses Ereignis noch nicht per se als Pech bezeichnen. 
Mario legt jedoch Wert darauf, dass das Pech in diesem Falle darin bestünde, dass der Schüler grundsätzlich die Fragen des Lehrers bzw. der Lehrerin beantworten könne und er durch das fehlende Wissen um die passende Antwort zu ausgerechnet dieser Frage in einer vom Pech bestimmten Lage sei.  
Er spielt gewissermaßen wieder auf die Rolle des Zufalls beim Glücksbegriff an, der in diesem Falle jedoch eine andere Qualität hat als im Falle des Losens, dass Stefanie zu Beginn des Gespräches erläutert. 
Darius geht anschließend an Marios Ausführungen über das Pech im unterrichtlichen Kontext erneut auf das Pech als Verlust von etwas ein und nennt dazu mit dem Verlust eines teuren Gegenstandes ein weiteres Alltagsbeispiel (Zeile 39-41).
Allerdings lässt sich sein Verlustverständnis in Bezug auf das Pech vor dem Kontext von materiellen Verlusten sehen während Stefanie im Folgenden auf den Verlust in Form eines verpassten Fluges hinweist. 
Konnten zu Beginn bereits Unterschiede in der Definition des Glücks festgestellt werden, so lassen sich auch Differenzen im Verständnis des Pechs ablesen. 
Während Stefanie das Pech zunächst allgemein fasst, sieht sie wenige Momente später das Pech vor allem im Verlust von wichtigen Gegenständen. 
Dieser Ansicht schließen sich Mario und Darius in der Folge an.

Auf die Frage, was man brauche, um glücklich zu sein, antwortet Stefanie erneut mit einem Verweis auf die menschliche Komponente des Glücksbegriffes, die sie bereits am Anfang des Gespräches schildert. 
Man brauche zum Glück vor allem die Familie -- sie nennt dazu noch den Hund -- und die Freunde. 
Auf das Verhältnis geht sie in Zeile 48-56 in Form eines konkreten Beispiels erneut ein, daher lässt sich sagen, dass ihr dieser Aspekt besonders wichtig zu sein scheint. 
Sie beschreibt, dass es für einen Menschen wichtig ist, ein gesundes Verhältnis zu seiner Familie zu haben, wobei die Mutter in diesem Falle als Repräsentantin der Familie dient. 
Des Weiteren verweist sie darauf, dass die eigenen menschlichen Bedürfnisse vernachlässigt werden könnten, wenn man sich um eine Erhaltung der familiären Bindungen nicht bemühen würde.
Mario stimmt ihr zu, da er Stefanies Meinung, ausreichend Essen, ein Dach über dem Kopf sowie eine Familie brauche man zum Glück, wiederholt. 

Darüber hinaus ergänzt er diese Bedürfnisse, die der Mensch aus seiner Sicht hat und nennt die Geborgenheit als zentralen Bestandteil der emotionalen Befindlichkeit, die der Mensch benötige, um letztendlich zum Glück zu gelangen (Zeile 57-62). 
Damit geht er auf einen Aspekt ein, den Stefanie bereits nannte. 
Darius hingegen betrachtet die Bedürfnisse des Menschen aus der Sicht derer, die eine solche Familie nicht besitzen. 
Konkret bezieht er sich dabei auf Kinder im Kinderheim, die im Hinblick auf familiären Zuspruch und die Beziehungen zu Freunden isoliert seien.
 
Stefanie skizziert in den Zeilen 69 bis 77 eine Situation, die zeigt, warum das Verhältnis zur Familie wichtig ist und schlägt damit inhaltlich die Brücke zurück zur Unterscheidung zwischen Glück und Pech. 
Darin beschreibt sie, dass es eine merkwürdige Situation sei, über andere Menschen wie die Mutter oder die Freundin zu lästern, aber gleichzeitig mit ihr zu spielen. 
Für die lästernde Person bestehe das Pech darin, dass die betroffene Freundin die Wahrheit erfahren könnte und infolge dessen den Kontakt abbricht und damit auch die zwischenmenschliche Beziehung zu dieser Freundin endet. 
Auch an dieser Stelle wird Stefanies Bild vom Glück im Verhältnis von Menschen zueinander deutlich. 

Auch Darius nennt eine solche Situation, in der Freunde einer Person sich als falsche Freunde herausstellen, da sie Geheimnisse verraten können und das Opfer belügen (Zeile 81-89).
Mario führt den Gedanken über die Freunde weiter und geht dabei erneut auf das Pech als den Verlust von etwas Positivem ein, indem er darlegt, dass auch dann Pech vorliegen kann, wenn man nicht wisse, ob man bestimmte Menschen -- auch wenn man diese noch nicht lange kennt -- jemals wieder sieht. 
Und auch Stefanie kennt ein solches Beispiel und beschreibt die Begegnung mit einem Kind bei einer Überraschungsfeier, von dem sie nicht wissen konnte, ob sich beide jemals erneut treffen werden.

Nachdem die Kinder sich zu ihren Gedanken, was ein Mensch braucht, um glücklich zu sein, geäußert haben, lenken die Studierenden die Aufmerksamkeit der Schülerinnen und Schüler auf den Gedanken zurück, inwiefern Glück in zwischenmenschlichen Beziehungen bestehe und ob man grundsätzlich auch alleine glücklich werden könne. 
Nachdem Mario noch vorsichtig bekundet, dass man durchaus Freunde brauche, um glücklich zu werden, ist es erneut Stefanie, die die Diskussion voranbringt (Zeile 110-115). 
Sie ist der Ansicht, dass ein Mensch auch alleine Glück empfinden könne und untermauert diese These mit einem Denkmuster aus ihrer Lebenswelt. 
Sie erzählt von Situationen, in denen ein Mensch ein Ziel vor Augen hat und dieses erreichen will. 
Für dieses angestrebte Ziel benötige diese Person jedoch nicht zwangsläufig Freunde, sondern sie werde ausschließlich durch das Erfolgserlebnis glücklich. 
Hinzu kommt, dass sich auch in diesem Beitrag von Stefanie der Aspekt des Glücks als eigene Leistung wiederfindet, den sie somit erneut entfaltet.
Mario hingegen sieht in der Fähigkeit des Menschen, auch alleine glücklich zu werden, gewisse Einschränkungen als gegeben an (Zeile 116-118). 
Er stimmt Stefanie zwar zu, dass man generell in der Lage sei, auch ohne den Bezug zu anderen Menschen Glück empfinden zu können. 
Jedoch sei dieses Glück nicht von der gleichen Qualität wie jenes, das durch andere Menschen hervorgerufen werde. 
Daraus zieht er den Schluss, dass ein Mensch, der sich mit einem Freund austauschen könne, glücklicher werden könne als ein Mensch, der diese Möglichkeit nicht habe. 
Daniela bezieht klar Stellung und vertritt den Standpunkt, dass man ohne die Mitmenschen definitiv nicht glücklich werden könne. 
Ohne die Beziehung zu anderen Menschen würde man vereinsamen und nicht zum eigentlichen Zustand des persönlichen Glücks gelangen können.

Darius bringt in seiner Sichtweise zum Ausdruck, dass die zeitliche Komponente eine entscheidende Rolle bei der Frage spiele, ob man allein glücklich werden könne (Zeile 122-129). 
Er bekennt zwar zunächst, dass man pauschal nicht sagen könne, dass man Freunde zum Glück brauche. 
Allerdings bringt er dann eine wesentliche Einschränkung zur Sprache. 
Der Zustand des Alleinseins beschränkt sich in seiner Darstellung auf einen Zeitraum von wenigen Stunden, in denen man sich auch allein beschäftigen und so auch eine gewisse Art von Glück empfinden könne. 
Dazu nennt er wiederholt die Ansicht, dass es bei Freundschaften zwischen Menschen nicht auf die quantitative Dimension ankäme, sondern auf die qualitative. 
Das bedeutet, dass es wichtiger sei, Freunde zu haben, die unterstützen und zuverlässig sind, wenn es besonders von Nöten ist, als eine große Gruppe von falschen Freunden um sich zu scharen, die sich jedoch nicht durch Zuverlässigkeit und Verlässlichkeit auszeichneten. 
Er ergänzt dieses Bild im Folgenden noch und legt besonderen Wert auf ein Bild von Freunden, die hinter den anderen stehen und diese nicht auslachen. 
An diesem Punkt zeigt sich folglich, in welcher Form Darius und seine Mitschüler erneut Bezüge zwischen der Frage, ob man allein glücklich sein könne und ihrer persönlichen Lebenswelt herzustellen bereit sind.

Einen weiteren Punkt in der Diskussion um das Glück ohne Beziehung zu anderen Menschen zieht Stefanie, die illustriert, dass es Menschen gäbe, die bewusst auf solche Kontakte verzichten würden (Zeile 134-146) . 
Aus diesem Verzicht folgert sie, dass sie anderen Mitmenschen mit Antipathie gegenüberträten. 
Gleichzeitig nimmt sie an, dass sich solche, eher introvertierte Menschen, jedoch menschliche Nähe wünschen würden und daher eine  Situation, in der sie ohne Freundschaften auskommen müssten, bei ihnen auf Unverständnis stoßen müssten.
In dem Beispiel, das sie daraufhin nennt wird jedoch noch eine andere Lesart dieser Introvertiertheit, die sie skizziert, deutlich: \glqq ähh es gibt zum beispiel jetz so ein mädchen das nur an sich denkt un dann immer petzt.\grqq{}

Hier beschreibt Stefanie eher ein Mädchen, dessen Charakter sich durch egoistisches Auftreten auszeichnet und sich in Form des \glqq Petzens\grqq{}, also des Meldens eines Mitschülers oder einer Mitschülerin, auf deren schlechtes Betragen oder andere Verfehlungen sie die Lehrerin oder den Lehrer hinweisen will, zeigt. 
Daher lassen sich aus ihrem Beitrag heraus einerseits Menschen ohne Bedürfnis nach Bindung, aber auch solche, die sich unkollegial andern gegenüber verhalten, beschreiben. 
Diesen Gedanken des unkollegialen Verhaltens spinnt sie weiter und legt dar, dass dieses Kind jedoch plötzlich ein anderes frage, ob sie gemeinsam etwas spielen wollen würden. 
Letztlich stellt Stefanie fest, dass die Absicht des Mädchens, mit dem anderen, dass sie gemeldet hat, auf Ablehnung stoßen müsse und somit eine klare Einordnung ihres Verhaltens vornehmen müsse. 
Allerdings spielen beide Kinder schlussendlich doch gemeinsam. 

In Stefanies Ausführungen findet sich zudem ein Bezug auf eine vergangene Sitzung zum Thema \glqq Freundschaft\grqq{}. 
Dabei bezieht sie sich auf die Auseinandersetzung mit dem Kinderbuch \glqq Irgendwie Anders\grqq{} von Kathryn Cave, dass die Studierenden mit der Klasse zuvor bearbeitet hatten. 
Die Geschichte handelt von der Figur \glqq Irgendwie Anders\grqq{}, die mit anderen Tieren in Kontakt treten und diese als Freunde gewinnen möchte. 
Dabei wird sie jedoch von diesen immer wieder zurückgewiesen mit der Begründung, dass Irgendwie Anders die gleichen Tätigkeiten nicht in der Form tätigen könne, wie die anderen Tiere. 
Gegen Ende der Geschichte trifft Irgendwie Anders auf ein anderes Wesen, dass sich als \glqq Etwas\grqq{} vorstellt und seine Freundschaft möchte. 
Zunächst schickt Irgendwie Anders das Etwas weg, ehe es erkennt, dass die beiden einander annehmen in ihrem Anderssein und schließlich Freunde werden.

Stefanie überträgt ihre Position auf die Geschichte von Irgendwie Anders und bekundet, dass ein solcher Umstand der Intoleranz nur schwer für die betroffene Person zu bewältigen sei (Zeile 142-146). 
Letztendlich bringt Stefanie zum Ausdruck, dass Freunde auch für sie wichtig sind, um Glück zu erfahren. 
Dadurch macht sie die Beziehung von Freundschaft und Glück abermals klar.
Mario greift Stefanies Bezug zur Geschichte von Irgendwie Anders erneut auf und stellt klar, dass Irgendwie Anders in der Geschichte alleine nicht zum Glück finden konnte. 
Dieses Fazit der Geschichte überträgt er auf ihr Beispiel aus dem Sportunterricht und widerspricht ihrer These, dass man in einer solchen Lage ohne die Mitmenschen Glück empfinden könne. 
Dabei beruft er sich darauf, dass wahrscheinlich Stefanies Freunde oder auch die Lehrperson zu diesem Erfolg beigetragen hätten, indem sie ihr etwas beigebracht oder ihr Mut gemacht hätten. 

Für Mario besteht das eigentlich Glück in der Bestätigung der Anderen, die ihre Wertschätzung über eine Leistung wie das Bockspringen, das er konkret nennt, zum Ausdruck bringen (Zeile 147-161). 
Ohne die Mitmenschen sei man nicht in der Lage, seine eigenen Erfolgserlebnisse zu teilen und so ein sich entwickelndes Glücksgefühl maximieren zu können. 
Den Aspekt des Teilens von positiven Erlebnissen ergänzt Mario schließlich erneut mit einem Beispiel aus seiner Perspektive, in dem er beschreibt, dass wenn man eine spannende Sendung im Fernsehen gesehen hätte, sich auch stets ein gewisses Mitteilungsbedürfnis ergeben würde, die Inhalte der Fernsehsendung mit jemanden teilen zu können. 
Damit lässt sich sagen, dass Mario offenbar der Auffassung ist, dass sich das Glück nicht bloß durch besonders erfolgreiche Phasen im Leben einstellen kann, sondern stets auch mit dem Zuspruch des Umfeldes in Zusammenhang steht. 
Daher ist aus seiner Sicht das Vorhandensein von Freunden und Bekanntschaften elementar für sein Verständnis von Faktoren, die für das Glücksempfinden wesentlich sind.

Nachdem sich die Schülerinnen und Schüler zunächst über eine Begriffsbestimmung, was Glück sein könne und über die Frage ausgetauscht haben, welche Dinge nötig seien, um Glück in seinem Leben empfinden zu können, greifen die Studierenden den emotionalen Aspekt des Glücks auf, indem sie die Frage in den Raum stellen, wie sich Glück anfühlen könnte und woher die Schülerinnen und Schüler wissen könnten, dass sie glücklich seien. 
Darius beschreibt das Gefühl des Glücks näher, indem er beschreibt, dass es für ihn aus dem Herzen komme (Zeile 165-169). 
Man sei zufrieden, was sich durch herzhaftes Lachen ausdrücken könnte. 
Gleichzeitig unterscheidet er dieses Lachen von solchem, dass nicht aus einem tiefgehenden Glücksgefühl entspringe. 
Mario betrachtet das Glücksgefühl im Unterschied zu Darius eher aus einer materialistischen Ordnung heraus. 
Er sagt, dass man Glück fühlen könne, wenn sich persönliche Wünsche in einem bestimmten Rahmen erfüllen würden. 
In seiner Darstellung bezieht er sich darauf, dass das Glücksgefühl dadurch entstehen könne, dass man ein bestimmtes Kuscheltier bekommen würde, welches man sich vorher gewünscht habe. 
Darius knüpft an diese Vorstellung an und verweist darauf, dass sich ein solcher Wunsch unter Umständen auch nicht erfüllen kann. 
Da der darauffolgende Teil seines Beitrages auf der Aufnahme nicht klar verständlich war, lässt sich unglücklicherweise nicht feststellen, welche Schlussfolgerung Darius an dieser Stelle aus dieser Feststellung ziehen wollte. 

Stefanie stellt in Zeile 176-181 einen erneuten Rückbezug auf Marios Position her, indem sie erklärt, dass für sie Glück empfinden bedeute, die eigenen Grundbedürfnisse wie Nahrung und eine Heimat befriedigen zu können. 
Damit verbindet sie die menschlichen Bedürfnisse mit dem Glücksempfinden, woraus sich ableiten lässt, dass Stefanie offenbar ein Bild vom Glück als Emotion hat, dass geprägt ist von der Sorge um das eigene Wohlbefinden. 
Erst wenn der Mensch sein Auskommen gesichert habe, könne er wunschlos glücklich werden. 
Das heißt, dass für Stefanie das persönliche Glück an die Bedürfnisse des Körpers gebunden ist. 
Daraus entwickeln die Studierenden eine weiterführende Fragestellung, dass arme Menschen nach dieser Betrachtung nicht abschließend glücklich werden könnten, da sie ihre Bedürfnisse nur in beschränktem Maße erfüllen könnten durch den Umstand, dass sie durch den Mangel an finanziellen Ressourcen dazu nicht in der Lage seien. 

Darius widerspricht dieser Annahme in den Zeilen 184 bis 186 zum Teil, da er betont, dass arme Menschen nicht grundsätzlich unglücklich seien. 
Diese Aussage präzisiert er mit einer Unterscheidung zwischen armen Menschen und Obdachlosen, indem er besonders auf jene Menschen verweist, die auf der Straße leben.
Es lässt sich annehmen, dass Darius davon ausgeht, dass Obdachlose demnach unglücklicher seien. 
Auch Stefanie schließt sich Darius' Auffassung an, dass arme Menschen nicht per se unglücklich sein müssten und es auf die Situation ankommen würde, in der sich diese Menschen befänden. 
So sei ein Obdachloser durchaus in der Lage, glücklich zu sein, wenn er die Nähe anderer Menschen, die ihn mit Almosen oder anderen Spenden unterstützen, spüren würde. 
Es zeigt sich wiederholt, dass das Glücksverständnis von Stefanie sehr eng mit der Wahrnehmung durch andere Menschen zusammenhängt. 
Dieser Eindruck wurde bereits durch Stefanies Aussagen, was der Mensch zum Glück benötige, bestätigt.

Darius geht davon aus, dass es auch Menschen gibt, die das Bedürfnis Anderer, armen Menschen zu helfen, für sich ausnutzen. 
Er beschreibt, dass sich diese bewusst als Hilfsbedürftige ausgeben würden, um die Hilfsbereitschaft vieler als Einnahmequelle auszunutzen und er verweist auch besonders auf Menschen, die anderen gewissermaßen helfen müssten, da sie sich in einer moralischen Verpflichtung sehen würden (Zeile 195-202). 
Was die Fähigkeit armer Menschen, glücklich zu sein, angeht schließt er sich der Sichtweise Stefanies an, und wiederholt diese, dass es diesen Menschen ausreichen könne, von anderen einige Cent oder etwas zu Essen zu bekommen und nennt dabei auch einen konkreten Fall aus Koblenz-Kesselheim. 
Dem Aspekt des Spendens wendet sich auch Stefanie zu, die darauf verweist, dass man in Märkten auch mit wenig Geld Lebensmittel bekommen könne, die sich in einem bestimmten finanziellen Rahmen bewegen. 
Daraus lassen sich verschiedene Lesarten entwickeln. 
Einerseits kann sie damit ausdrücken, dass bereits wenig Geld ausreichen kann, damit sich Hilfsbedürftige selbst versorgen können, da diese von den Märkten aus Solidarität auch ohne den vollen Preis herausgegeben würden. 
Nach ihrem Bild von Glück könnten diese Menschen dann glücklich werden ungeachtet ihrer Armut. 
Möglich ist auch, dass Stefanie mit dieser Aussage zum Ausdruck bringen möchte, dass sich Hilfsbedürftige von den Almosen dann etwas kaufen können, wenn Mitmenschen bereit wären, weiteres Geld hinzuzugeben. 

Gegen Ende des Gespräches lenken die Studierenden die Aufmerksamkeit der Schülerinnen und Schüler auf Glücksbringer. 
Die Kinder sollen sich dazu äußern, ob sie Glücksbringer besitzen, ob sie Arten von Glücksbringern kennen und ob sie daran glauben, dass Glücksbringer das Glück fördern können. 
Darius antwortet darauf mit einer zunächst zurückhaltenden Zustimmung, was er mit dem Einschub \glqq eigentlich\grqq{} unterstreicht. 
Diese Annahme begründet er, indem er erzählt, mithilfe des Glaubens an seinen Glücksbringer dafür gesorgt zu haben, dass es am nächsten Tag nicht geregnet habe. 
Darius äußert damit offenkundig, dass er der Überzeugung ist, dass der Glaube an einen ausgewählten Gegenstand, der dann als Glücksbringer dient, Ereignisse so verändern kann, dass sie den eigenen Wünschen entsprechen. 
So gesehen ordnet er dem Menschen die Fähigkeit zu, Dinge durch seine Wünsche verändern zu können. 
Daran schließt er eine Situationsbeschreibung an, in der ein Mann bei \glqq Wer wird Millionär\grqq{} bei einer Frage über die Anzahl von Steinen, die ein Zauberwürfel besitze, mit Glück -- wie Darius es beschreibt -- die richtige Antwort gefunden habe (Zeile 224-232). 
Hier stellt er einen Rückbezug zu Stefanies Ausführungen her, die feststellte, dass Glück dann vorliegen würde, wenn gewisse Ereignisketten zu Gunsten einer Person ausfallen. 
Er löst sich gleichzeitig aber inhaltlich vom Subthema der Glücksbringer.

Stefanie lenkt den Fokus wieder zurück zur Frage nach den Glücksbringern und bezieht sich ebenfalls auf den Anfang des Gespräches, als sie feststellte, dass sie Glück beim Losen hatte. 
Dieses erlebte Glück stellt sie in einen direkten Kausalzusammenhang zu ihrem Glücksbringer, den sie kurz zuvor laut eigener Aussage erhalten hatte. 
Gleichzeitig macht sie durch ihr Seufzen, dass sie während ihrer Ergänzung, zuvor keinen Glücksbringer besessen zu haben, deutlich, dass für sie Glücksbringer offenbar wichtig sind, um Glück herbeiführen zu können. 
Darius wirft daraufhin ein, selbst keinen Glücksbringer zu besitzen, obwohl er zuvor davon gesprochen hatte, dass ihm ein Glücksbringer Glück gebracht habe. 
Daran angeschlossen führt Stefanie ihren Gedanken fort und bezieht sich erneut auf ihren Beitrag.

Aysche erzählt von ihrem Glücksbringer, den eine Freundin für sie im Urlaub angefertigt hat. 
Diese Freundin habe zusätzlich noch einen Text zu diesem Glücksbringer geschrieben und ihn ihr dann überreicht (Zeile 242-245). 
Folglich entfaltet Aysche ein Schema von Glücksbringern, die dann wirksam sind, wenn sie eine persönliche Bedeutung für einen Menschen besitzen. 
In diesem Fall besteht diese Bedeutung in der Beziehung zwischen Aysche und ihrer Freundin. 
Auch findet sich hier erneut die zwischenmenschliche Beziehung als Ausprägung des menschlichen Glücks wieder, das in Form des Glücksbringers geteilt werden kann. 
Mario knüpft an diese Sicht auf Glücksbringer an und sieht im Kontext eines Gegenstandes die Bedeutung als Glücksbringer.
Er erläutert, dass er für seine Eltern zum Nikolaustag Geschenke in Form von Schneemännern und Sternen gebastelt und diese bei einer Adventsfeier verschenkt hatte. 
Da er offensichtlich vorher gezählt hatte, wunderte er sich darüber, dass der Schneemann, den er eigentlich ausgemustert hatte, noch bei den Geschenken lag. 
Es stellte sich heraus, dass Marios Mutter diesen Schneemann mit einem aufgemalten Herz und einem schwarzen Hut ergänzt hatte und so für Mario daraus ein Glücksbringer wurde (Zeile 249-261). 
Auch hier manifestiert sich die Vorstellung der Schülerinnen und Schüler, die Aysche zuvor bereits äußerte, dass ein Glücksbringer durch seine Bedeutung entsteht.
	
Auch wenn bisher nur Bezüge auf materielle Dinge als Glücksbringer seitens der Schülerinnen und Schüler vorgenommen wurden, hält Darius dagegen, dass auch Menschen als Glücksbringer füreinander dienen können. 
So sei sein Mitschüler Mario ein Glücksbringer bei der Fussball-Weltmeisterschaft gewesen, da erst dann ein Tor gefallen sei, als Mario sich am gemeinsamen Verfolgen des Spiels beteiligt habe. 
Daraus zieht Darius den Schluss, dass Mario ein Glücksbringer gewesen sein muss, der durch seine Anwesenheit als Glücksbringer für Darius in der Lage war, die Ereignisse positiv beeinflussen zu können. 
Neben diesem Glücksbringer -- seinem Freund Mario -- berichtet Darius auch von einer Münze aus Kroatien, die für ihn ein Glücksbringer sei, ohne näher auf die Geschichte des Glücksbringers einzugehen. 

Stefanie hingegen beschreibt in den Zeilen 270 bis 273, dass ein sogenannter Traumfänger ihrer Freundin ein Glücksbringer sei, da dieser Alpträume von ihr fern halten könne und sie das Gefühl habe, besser schlafen zu können, wenn dieser Traumfänger in ihrer Nähe sei. 
Zudem stellt sie eine interessante These auf. 
Sie geht davon aus, dass Glücksbringer mit etwas versehen würden, was sie zu Glücksbringern machen würde.
Dieses Wissen über die Beschaffenheit drückt sich auch in ihrer Ausdrucksweise aus, da sie sagt: \glqq die machen irgend son zeug rein das weiß ich.\grqq{}(Zeile 273f) 
Sie drückt damit ihre bedingungslose Überzeugung darüber aus, dass es ein Geheimnis geben müsse, dass Glücksbringer zu Glücksbringern mache. 
Da sie diesen Gedanken nicht weiter ausgestaltet, ist folglich anzunehmen, dass sie selbst keine genaue Vorstellung darüber hat, um was es sich dabei genau handeln könnte. 

Die Studierenden bringen die Aussagen der Schülerinnen und Schüler in der Folge nochmals auf den Punkt. 
Glücksbringer könnten grundsätzlich alles sein, d.h. sie sind unabhängig von einem Trägermedium. 
Wichtig sei nur, dass es sich dabei um etwas handele, was für die betreffende Person bedeutsam sei und an dessen Glück bringende Kraft sie glauben könne. 
Daher kann man sagen, dass das, was Stephanie über das Besondere an Glücksbringern sagt, vor allem Glauben an diesen Gegenstand besteht.
Daran gebunden sei auch immer der Kontext, der einen Gegenstand zum Glücksbringer mache. 
So könne für den einen Menschen ein vierblättriges Kleeblatt ein Glücksbringer sein und für einen anderen wiederum etwas vollkommen Anderes.

Stefanie bringt einen weiteren neuen Aspekt ein in Bezug auf Menschen, die Glücksbringer sein können. 
Sie beschreibt eine Begebenheit, in der sie selbst für sich ein Glücksbringer war und erzählt von einem Fahrradausflug, den sie mit ihrem Vater unternehmen wollte.
Allerdings lehnte ihr Vater dies aufgrund des zu diesem Zeitpunkt zu schlechten Wetters ab. 
Da Mario und Darius sie darauf aufmerksam machen, dass sie den Bezug zum Thema \glqq Glücksbringer\grqq{} noch nicht verstanden haben, erklärt sie das Beispiel erneut. 
Sie erläutert, dass sie ihren Vater letztendlich dazu hätte überreden können, den Ausflug zu unternehmen, was für sie an dieser Stelle ein Glückserlebnis darstellte. 
Demzufolge deutet sie ihre Überzeugungskraft gegenüber dem Vater so, dass sie selbst in der Lage war, die Situation zu verändern. 

Das passt auch zu der Grundannahme, die Darius zuvor äußerte, dass Glücksbringer in der Lage sind, Einfluss zu nehmen. 
So sieht sich Stefanie als Glücksbringer für sich selbst. 
Diese Auslegung wird auch von den Studierenden vorgenommen und für die Mitschülerinnen und Mitschüler kenntlich gemacht.
Am Ende des Gespräches verknüpft Darius seinen Glücksbringer mit einer persönlichen Leistung, die sich in Form eines gewonnenes Fußballspiel vollzieht. 
Er bezieht sich demnach wiederholt auf die Glücksvorstellung des Glücks als eigene Leistung und verbindet dieses Glück in einer kausalen Kohärenz mit dem Gedanken an seinen Glücksbringer.

Auch Stefanie geht erneut auf diese Kausalkette ein und belegt diese anhand dessen, dass auch ihr der Glücksbringer geholfen habe. 
Sie bezeichnet es als Glück, dass ihr Vater mit ihr trotz zeitlichen Engpasses einen Kinofilm ansah, den Stefanie gerne sehen wollte und obwohl sie, um der Schulpflicht in ausreichendem Maße nachkommen zu können, darauf angewiesen sei, genug zu schlafen. Mario schließt das Gespräch mit einer erneuten Repetition ab und verweist im Kontext seines Basketballtrainings auf die Dimension des Glücks als Resultat der eigenen Anstrengung.

Die Schülerinnen und Schüler haben mit ihren zahlreichen Alltagsbezügen, welche an dieser Stelle dargestellt wurden, deutlich gemacht, was ihnen beim Verständnis des Glücksbegriffs besonders wichtig ist. 
Glück ist demnach zusammengefasst eine günstige Abfolge von zusammenhängenden Ereignisketten, die, unterstützt durch Familie und Freunde und -- je nach dem, welchen Fokus der Schüler oder die Schülerin setzt -- durch Zufall oder eigene Leistung zustande kommt. 


\newpage



\subsection{Vergleich ausgewählter Schüleraussagen mit philosophischen Positionen}

Nachdem der Inhalt des philosophischen Gesprächs zwischen den Studierenden und den Schülerinnen und Schülern inhaltsanalytisch betrachtet wurde, sollen besonders aussagekräftige Textstellen des Transkriptes mit den philosophischen Positionen in Beziehung gesetzt werden, welche zu Beginn der Auseinandersetzung mit dem Glück erarbeitet wurden.

In den Zeilen 8 bis 10 und 12 bis 15 beschreiben Mario und Darius das Glück als eine Handlungskette mit vorteilhaftem Ausgang und verbinden diese Grundthese mit einem Beispiel aus ihrer Lebenswirklichkeit, indem sie erläutern, dass es Glück sein könne, wenn man von einem Baum nicht ganz nach unten stürzen bzw. sich nach einem Sturz nicht gefährlich verletzen würde. 
Damit schließen sie sich der allgemeinen Definition des Dudens an, der das Glück als etwas bezeichnet, \glqq was Ergebnis des Zusammentreffens besonders günstiger Umstände ist\grqq{} \cite{D16}.
Darüber hinaus zeigt sich in ihren Ausführungen allerdings auch eine Tendenz dahingehend, dass Glück für die beiden Schüler anscheinend auch mit der Vermeidung von Leid bzw. Schmerz zusammenhängt. 

Daher liegt es nahe, dass Mario und Darius ein Bild vom Glück an dieser Stelle des Gespräches äußern, dass gewissermaßen eine Einordnung des Glücks als Lust oder als Freude und des Pechs als Leid oder Schmerz voraussetzt. 
Da sie jedoch nicht auf eine weitere Differenzierung von Glück auslösenden Ereignissen eingehen, lässt sich keine feste Zuordnung zu einer philosophischen Position vornehmen. 
Vielmehr kann gezeigt werden, dass Mario und Darius die Schmerzvermeidung, wie sie sowohl Epikur und Aristippos als auch John Stuart Mill beschreiben, in ihr grundsätzliches Glücksverständnis übernommen zu haben scheinen. 
Ergänzend lässt sich auch ein Bezug zur antiken Stoa herleiten, da sich in der Abwendung von möglichem Unheil auch auf gewisse Weise ein Selbsterhaltungstrieb erkennen lässt, der von dieser philosophischen Schule angenommen wurde. 

Auch sehen die beiden Schüler das Glück im Kontext der Abwendung einer Verletzung als etwas immaterielles an, wodurch sich Parallelen zur mittelalterlichen Glücksphilosophie erkennen lassen. 
Auch Augustinus betrachtete das Glück als etwas vom Materiellen Unabhängiges. 
Allerdings folgerte er aus dieser Beschaffenheit des Glücks, dass Gott allein das Glück durch seine Unabhängigkeit von äußeren Einflüssen befördern könne. 
Bleibt man in diesem Bild, so wäre die Schlussfolgerung, dass Mario und Darius nach der Interpretation von Augustinus einen Glückszustand beschreiben, der als Resultat die Vermeidung der Verletzung herbeigeführt hat und dieses Ereignis durch Gott gesteuert wurde. 
Allerdings werden an dieser Stelle auch die Grenzen dieser Betrachtungsweise deutlich, da die Aussagen von Mario und Darius keinen Hinweis auf eine Verknüpfung mit Gott beinhalten.

Auch in Zeile 20 wird das Glück als etwas Nichtmaterielles begriffen und dabei aber auch gleichzeitig ein neuer Schwerpunkt gesetzt. 
Für Stefanie liegt das Glück in der Beziehung der Menschen begründet und kann sich dem Menschen als Helfen in der Not und Fürsprache in diffizilen Situationen offenbaren.
 Stefanie entfaltet auf der einen Seite ein Glücksverständnis, dass als durch christliche Nächstenliebe mitgeprägt bezeichnet werden kann. 
 Wenn der Mensch bereit ist, sich auch den Schwachen und Hilfsbedürftigen zu widmen, kann er in der Folge Glück erfahren. 
 
 Auf der anderen Seite lässt sich ihr Glücksverständnis im Kontext des Utilitarismus von John Stuart Mill betrachten. 
 So stellt das Helfen in der Not anderer Menschen eine Handlung dar, die als moralisch korrekt eingeordnet werden kann, da sie die Lust bzw. das Glück desjenigen, der Hilfe leistet, verstärken kann. 
 Gleichzeitig will der Mensch im utilitaristischen Sinne aus seinen Handlungen Nutzen ziehen. 
 Dieser lässt sich im Kontext dieses Beispiels in der Erlangung persönlichen Glücks durch Hilfestellung erkennen. 
 Die Abstufung der Freuden, die Mill vornimmt, lässt sich an der vorliegenden Textstelle ebenfalls illustrieren. 
 Je nach dem in welchem Umfang die Unterstützung anderer Menschen stattfindet, so lässt sich daraus auch ableiten, von welcher Qualität die daraus resultierende Freude ist. 
 Hilft man beispielsweise jemanden beim Umzug, so wird sich das Glücksgefühl voraussichtlich nur in einem äußerst begrenzten zeitlichen Rahmen zeigen. 
 Hilft man jedoch einem Menschen in einer schwierigen Situation, in der er wenig finanzielle Ressourcen hat, so wird sich dieses Glücksgefühl unter Umständen auch sehr viel später noch einstellen.
 
Innerhalb des Gespräches stellen die Schülerinnen und Schüler auch einen klaren Gegensatz zwischen dem Glück auf der einen Seite und dem Pech auf der anderen Seite her. 
Stefanie bringt diesen Gegensatz mit ihrer Aussage aus Zeile 29 bis 32 auf den Punkt:
\glqq weil ähm danach kann man auch also das is halt eben anderster als (unverständlich, 3 sek) weil is glück is was halt wenn was richtig passiert weißte? (unverständlich, 5 sek) pech is jetz was wenn was schiefläuft. (5.0)\grqq{}

Stefanie bezieht sich an dieser Stelle erneut auf die Grundthese des Glücks als wünschenswerter Abfolge von Ereignissen. 
Zugleich benennt sie diesen Gegensatz auch in Form von Lust und Schmerz oder in ihren Worten \glqq wenn was richtig passiert\grqq{} und \glqq wenn was schiefläuft\grqq{} und knüpft damit erneut an Aristippos, Epikur und Mill an. 
Die Verschiedenartigkeit von Pech wird von ihren Mitschülerin in der Folge durch ihre Beispiele manifestiert, die das Pech auch im schulischen Kontext des Unterrichtsgespräches, dem Verlust eines wertvollen Gegenstandes oder in dem unglücklichen Ereignis eines verpassten Flugzeugs verorten. 

Damit verweisen die Schülerinnen und Schüler auf eine subjektive Wahrnehmbarkeit des Pechs und die Abhängigkeit der Interpretation von Pech vom Kontext der Situation. 
Gleichzeitig wird deutlich, dass die Schülerinnen und Schüler sich darüber bewusst sind, dass es neben der Qualität von Freuden, die sich nach Mill voneinander unterscheiden lassen, auch das Pech ist, dass sich in seiner Ausprägung unterschiedlich beschreiben lässt.

Auf die Frage, was man zum Glück benötigt, antwortet Mario, dass man vor allem ein Zuhause, Essen und eine Familie zum Glück brauche, weil man sich ohne diese Grundversorgung nicht wohl fühlen könne.
 Mario verortet das Glück folglich auch in dieser Befriedigung von Grundbedürfnisse. 
 Auch Immanuel Kant dachte im Zeitalter der Aufklärung ähnlich. 
 Er sah das Glück vor allem als Summe rationaler Entscheidungen des Menschen und unabhängig von emotionalen Regungen. 
 
Diesem Aspekt Kants schließt sich Mario ausdrücklich nicht an, da er das Wohlbefinden des Menschen hervorhebend nennt. 
Der Mensch möchte in dieser Situation die Selbsterhaltung erreichen und strebt deshalb die Erfüllung seines Verlangens nach Heimat, Nahrung und menschlicher Bindung an. 
Insofern findet sich mit dem Drang nach Selbsterhaltung auch ein Aspekt stoischer Lehre in Marios Aussage. 
Mit seinem Hinweis, dass man sich nicht wohl fühlen könne, ohne das nötigste zu besitzen, nimmt Mario erneut Bezug auf Kant, der ein sittliches Leben als Grundvoraussetzung für das Glück des Menschen ansah. 
Demnach kann ein sittliches Leben also nicht stattfinden, wenn die Grundversorgung eines Menschen nicht stattfindet und folglich kann dieser Mensch nach kantischer Lesart das Glück nicht erreichen. 
Auch Darius beruft sich auf diese Bedürfnisse des Menschen und weist darauf hin, dass es auch Menschen gebe, die keine Familie und keine Freunde besitzen würden. 

In einem weiteren Schritt sieht Darius das Pech im Umfeld der Freundschaft zwischen Menschen verortet, was er in seinem Statement auch deutlich macht: 
\glqq wie jetzt/ zum thema freunde, wenn man zum beispiel was wichtiges hat und äh/ und das dann einem sagt. und der das dann sofort der ganzen schule verpetzt, ähm (.) dann (.) und bricht man ja auch den kontakt ab und dann/ jaja, ich mach das nie wieder und dann machen die das zehn minuten später wieder. ((mehrere SuS lachen)) kann doch auch sein, dann verrät man ein geheimnis und dann sagen die das wieder.\grqq{}

An dieser Textstelle wird deutlich, dass Darius die Beziehung zwischen Menschen, die Stefanie in Zeile 20 als Glück beschrieb, aus der gegensätzlichen Perspektive betrachtet. 
Aus seiner Sicht liegt für eine Person, die anderen Menschen etwas anvertraut und dann von diesen hintergangen oder enttäuscht wird, an dieser Stelle eine Form von Pech vor. 
Folglich kann die Beziehung von Menschen also sowohl vom Glück als auch vom Pech bestimmt werden. 

Anhand von Darius' Beispiel lässt sich auch erneut die utilitaristische Lehre von John Stuart Mill miteinbeziehen. 
Wenn man im Beispiel bleibend davon ausgeht, dass jede Handlung des Menschen einen Nutzen haben muss, so kann sich dieser lediglich in der Handlung derjenigen Personen vollziehen, die das Geheimnis des Freundes verraten haben. 
Daher stellt sich nun die Frage, welchen Nutzen sie daraus ziehen könnten. 
Darüber lässt sich freilich nur spekulieren. 
Es kann jedoch angenommen werden, dass die Freunde dem Opfer schaden wollen, um ein höheres, nicht weiter zu bestimmendes Ziel zu erreichen. 
Nach Mills Definition müssten sie demnach glücklich werden können, da in seinem Bild jede Handlung moralisch korrekt ist, die Glück hervorruft. 
Kommen die Freunde jedoch in einer späteren Situation zu der Einsicht, dass ihr Verhalten falsch war, so dreht sich auch die Moralität der Handlung ins Negative. 
Daraus folgt, dass ihr Tun den Verrätern als moralisch falsche Handlung ihnen auf dem Weg zum eigenen Glück im Weg stehen wird. 
Es zeigt sich an dieser Stelle einer der zentralen Kritikpunkte am Utilitarismus Mills, der zulässt, dass eine moralisch falsche Handlung durch den größeren, zugeschrieben Nutzen für die Gemeinschaft als legitim angesehen werden kann.

Eine weitere aussagekräftige Textstelle findet sich im Rahmen der Frage, ob man allein glücklich werden kann oder nicht.
 An die Frage schließt sich zunächst der Kurzbeitrag Marios an, der bekundet, dass der Mensch andere Menschen brauche, um Glück zu empfinden bzw. glücklich zu sein. 
 Stefanie entgegnet daraufhin, dass sie auch alleine durch ihr Handeln glücklich werden könne und dabei nicht von anderen Menschen abhängig sei.
 
Stefanie greift in ihren Ausführungen indirekt auf das Glücksverständnis Friedrich Nietzsches zurück, der das Glück nicht als etwas von außen Einwirkendes betrachtete, sondern annahm, dass das Glück etwas typisch Menschliches sei, dass seiner Psyche innewohne. 
Auch Stefanie stellt klar, dass sie durch ihre sportliche Leistung das Glück erreichen könne, ohne dabei von äußeren Einflüssen durch Freunde oder andere Mitmenschen abhängig zu sein. 
Ebenfalls findet sich in ihrer Äußerung eine Parallele zu Aristoteles, der formulierte, dass der Mensch auf dem Weg zur Eudaimonia Teilziele zu erreichen suche, welche in diesem Beispiel im Glücksgefühl durch Erfolg besteht. 
Spinnt man diesen Faden weiter, so ist Stefanie nach Aristoteles auf einem fortgeschrittenen Weg zur Eudaimonia, da sie in ihrem Leben viel für den eigenen Erfolg tun möchte. 
Sie widerspricht aber auch gleichzeitig der Annahme Aristoteles', dass der Mensch durch äußere Faktoren abhängig sei und betont dies ausdrücklich.

Mario stimmt Stefanie zwar in dem Punkt zu, dass der Mensch auch allein glücklich sein kann, deutet jedoch auch an, dass dieser Zustand mithilfe der Mitmenschen besser zu erreichen sei. 
Wenig später, in den Zeilen 147-161, widerlegt er gar Stefanies Beispiel aus dem Sportunterricht und erläutert, dass sie in dieser Situation doch von anderen Mitmenschen unterstützt wurde. 
Damit spricht Mario im weitesten Sinne aus der Tradition des Aristoteles, der das finale Ziel der Eudaimonia eben darin erkennt, sich auch gesellschaftlich einzubringen. 

Der Mensch müsse akzeptieren, dass er nicht allein für sich sein Glück finden könne, sondern auch abhängig sei von den anderen Gütern, welche sich in diesem Zusammenhang auch in der Zuwendung durch Freunde oder nahestehende Personen allgemein erfassen lässt. 
Zusammenfassend kann man also festhalten, dass man laut Stefanie allein seine erfolgreichen Taten benötigt, um das Glück zu erlangen. 
Mario hingegen sieht auch die Beziehung zu anderen Menschen und das Einbringen in die Gemeinschaft als zentral an, um glücklich zu werden.

Ein weiterer wichtiger Aspekt der Frage nach dem Glück, zu der sich die Schülerinnen und Schüler äußern, liegt in der Sicht von Glück als Gefühl begründet. 
Grundsätzlich lässt sich feststellen, dass die Schülerinnen und Schüler augenscheinlich annehmen, dass es so etwas wie ein determiniertes Glücksgefühl gebe muss.
Darius skizziert ein Bild des Glücksgefühls, dass sich vor allem durch eine allgemein positive Stimmungslage eines Menschen auszeichnet, die er darlegt.

Diese Auslegung widerspricht deutlich Immanuel Kants Glücksverständnis, der dieses vor allem aus dem rationalen Handeln des Menschen heraus begründet. 
Das Glück sei demnach eben keine Gefühlsregung, wie sie von Darius beschrieben wird, sondern eher das rationale Resultat der sich ergebenen Handlungsketten. 
Darius beschreibt die Glückseligkeit als ein Gefühl der inneren Zufriedenheit und der Freude, die sich auch in äußerlichen Regungen wie dem Lachen zeigen kann. 
Daraus resultiert ein deutlicher Widerspruch zu Kant, der jegliche Emotionalität bezüglich des Glücks kategorisch ablehnt und das Glück sogar lediglich als die Befriedigung der menschlichen Bedürfnisse bezeichnet, die eng an das Leben in Sittlichkeit gebunden ist. 

Dieser Zusammenhang von Glück und Emotion wird  in der Form weitergedacht, dass sich die Schülerinnen und Schüler auch damit befassen, inwiefern arme Menschen zum Glück fähig sind. 
Stefanie ist der Auffassung, dass Arme nur in beschränktem Maße glücklich werden können, da der Mangel an den nötigsten Dingen sie vom Glück entferne. 
Darin spiegelt sich ein Bezug zu Kant wieder, der -- wie bereits mehrfach erwähnt -- vor allem die Bedienung des Bedarfes des Menschen an bestimmten Gütern als Ziel des Menschen sah, um zum Glück zu gelangen. 
Folglich zeigt sich durch die Verbindung zu Kant, dessen Theorie sich nah an der der antiken Stoa bewegt, dass der Mensch einen Selbsterhaltungstrieb in sich habe, der bestimmt, ob er glücklich werden kann oder nicht. 
Stefanie nennt daran anschließend auch, dass es Menschen gebe, die die Armen in ihrer Not mit Spenden wie Essen oder Geld unterstützen. 

Hier lässt sich erneut John Stuart Mills Utilitarismus anwenden. 
Wenn ein Mensch einem anderen hilft, so handelt er für sich selbst moralisch korrekt, da er aus seinem helfenden Handeln einen direkten Nutzen für sich selbst beziehen kann, welcher darin bestehen wird, dass er sich durch sein Mitgefühl gegenüber dem armen Mitmenschen besser fühlen wird. 
Gleichzeitig vermeidet er damit den Schmerz, den es ausgelöst hätte, wenn er die Hilfe unterlassen und durch seine unmoralische Handlung Gewissensbisse bekommen hätte. 
Obendrein lässt sich das Handeln des guten Menschen, wie Stefanie es ausdrückt, auch dahin gehend charakterisieren, in welcher Qualität und Dauerhaftigkeit sich das neue Glück zeigt, dass dieser Mensch erworben hat. 
Denn durch sein Handeln hat er nicht nur die Lage eines ärmeren Menschen verbessert -- auch wenn nicht davon auszugehen ist, dass dies von Dauer sein wird -- , sondern er hat auch dazu beigetragen, sein eigenes Glück voranzutreiben. 
Daher ist davon auszugehen, dass eine solches Handeln seinen Mitmenschen gegenüber im Sinne Mills eine hohe Qualität für das Glück beinhalten wird.

Neben dem Utilitarismus kann auch die nikomachische Ethik des Aristoteles anhand dieses Beispiels Anwendung finden. 
So lässt sich sagen, dass ein Armer zwar durchaus in der Lage ist, durch sein charakterliches Wesen alles für das persönliche Glück zu tun. 
Jedoch ist er wie jeder andere Mensch abhängig vom Vorhandensein anderer Güter. 
Nach Aristoteles kann ein armer Mensch daher nicht zur Eudaimonia gelangen, da ihm äußerliche Güter augenscheinlich fehlen werden, auch wenn er vielleicht die seelischen und körperlichen besitzt. 
Dies wiederum hat zur Folge, dass ein Armer nach Aristoteles stets ein unglückseliges Leben führen werden muss, solange er nicht in der Lage ist, alle von ihm beschriebenen Güter auf sich zu vereinen. 

Der gute Mensch kann jedoch, wenn man davon ausgeht, dass er bereits die seelischen und körperlichen Güter besitzt, in der Lage sein, zur Eudaimonia zu gelangen, da er offenkundig über äußerliche Güter verfügt, die er mit dem armen Menschen zu teilen bereit ist. 
Doch auch der gute Mensch wird wie der arme Mensch ein unglückseliges Leben führen müssen, wenn er die aristotelischen Bedingungen der Eudaimonia nicht erfüllt. 
Demgegenüber werden in der epikureischen Lehre beide Figuren das Glück letztendlich dann erreichen können, wenn sie sich bescheiden auf ihre Grundbedürfnisse beschränken und jeglichen Schmerz zu vermeiden versuchen.