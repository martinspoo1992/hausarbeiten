\section{Sprache und Gewalt}
Um die Verknüpfung zwischen der Sprache und Cybermobbing nachzuvollziehen, ist es zunächst sinnvoll, den Bezug zwischen Sprache und Gewalt zu erörtern und zu beleuchten, worum es sich bei Gewalt handelt.
Allgemein bezeichnet Gewalt die „Macht, Befugnis, das Recht und die Mittel, über jemanden, etwas zu bestimmen, zu herrschen.“ \cite{DG13}

Sybille Krämer sieht den Begriff der Gewalt in einem „ambivalenten Bedeutungsfeld“, „changierend zwischen konstruktiven und negativen Konnotationen.“ \cite[S.\,35]{SKR07}
Demnach kann unter „Gewalt“ sowohl ausgeübte Amts- beziehungsweise Staatsgewalt, aber auch verübte Gewalt in Bezug auf Gewalttaten verstanden werden.
Krämer nennt dazu die lateinischen Ausdrücke „Potestas“ für Staatsgewalt und „Violentia“ für Gewalttaten.
Anhand der lateinischen Bezeichnungen wird die Ambiguität des Gewaltbegriffs greifbar.
Hauptunterschied zwischen den genannten Gewaltbegriffen ist dabei, dass Gewalt als Potestas eine Form der Machtausübung darstellt, die sich vorherrschenden Gesetzen unterordnet, während Violentia bewusst mit bestehenden Regeln bricht.
Im Rahmen dieser Gewaltdefinitionen steht die sprachliche Gewalt in Zusammenhang zur violenten Form der Gewalt.

Betrachtet man die Verben der deutschen Sprache, die semantisch im Zusammenhang mit Aggression stehen, so fällt auf, dass sie alle einen tatsächlichen Vorgang des Handelns beschreiben.
Daraus folgt für Krämer, dass aus einer verbalen Äußerung nicht ohne weiteres geschlossen werden kann, dass sie die Funktion der Verletzung oder Beleidigung erfüllen soll.
Daher muss bei sprachlicher Gewalt immer miteinbezogen werden, in welchem Kontext eine Äußerung getätigt wird, wer an der Konversation beteiligt ist und wie die Äußerung formuliert wurde.
Dementsprechend sind verletzende Worte „nicht einfach Bestandteil der Sprache als System, sondern ein Phänomen des kulturell eingebetteten Sprachgebrauches“. \cite[S.\,35]{SKR07}

Auch König und Stathi sehen nicht die Sprache als solche als gewalttätig an, sondern die Äußerungen und kommunikative Akte, die in bestimmten Situationen vollzogen werden. \cite[S.\,47]{EK07}
Weiter führen sie an, dass jegliche nicht-verbalen Akte sowie deren Komponenten ebenfalls gewalttätig sein können.
Darunter fassen sie Intonation, Lautstärke, Körpersprache, Mimik, Gestik, etc. Aber auch Handlungen, die unterlassen werden, können daher in bestimmten, kommunikativen Kontexten als aggressiv und verletzend eingestuft werden.
Ekkehard König und Katerina Stathi gehen folglich in ihrer Definition, worum es sich bei sprachlicher Gewalt handelt, einen Schritt weiter und ergänzen ihre Definition der sprachlichen Gewalt neben der artikulatorischen Ebene noch um die Ebene der Körpersprache.

Daher lässt sich über sprachliche Gewalt sagen, dass damit nicht die reine Ebene der Wörter gemeint ist, sondern Sprache immer in einem kontextuellen Umfeld verletzend wird und ohne jenen Kontext in der Regel keinen, violenten Hintergrund hat.
Zudem lässt sich sprachliche Gewalt nicht nur auf das System der menschlichen Sprache, sondern auch auf dessen Verhalten beziehen.
