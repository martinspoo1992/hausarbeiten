\section{Gemeinsamkeiten und Unterschiede zwischen Cybermobbing und Mobbing}
Insgesamt ist der Übergang vom Mobbing in der Schule oder am Arbeitsplatz hin zum Cybermobbing fließend.
Jedoch weist Cybermobbing einige Merkmale des Mobbings, wie es schon seit längerer Zeit bekannt ist, auf und stellt daher eher eine andere Ausprägungsform von Mobbing dar.

Grundsätzlich gemein ist beiden Formen, dass sie das Ziel haben, das Opfer zu erniedrigen und herabzusetzen.
Hinzu kommt, dass sich die soziale Struktur eines solchen Konfliktes beim Cybermobbing ähnlich verhält, da es sowohl beim Cybermobbing als auch beim Mobbing verschiedene Rollen der beteiligten Personen des sozialen Umfeldes geben kann. \cite[S.\,453-459]{CS13}
Nach Salmivalli handelt es sich dabei um die Rollen der Täter, Helfer, Verstärker, die durch ihr Auftreten den oder die Täter in ihrem Handeln bestärken, aber auch Verteidiger und Außenstehende, welche etwaige Handlungen ignorieren oder auch dulden.

Demgegenüber lassen sich allerdings auch Unterschiede zwischen Mobbing und Cybermobbing feststellen.
Während der Täter beim Mobbing in der Regel klar als solcher identifizierbar ist, so ist es ihm beim Cybermobbing besser möglich, anonym zu bleiben. \cite{TK14}
Dies erschwert ein Vorgehen gegen das Cybermobbing und verstärkt das Gefühl der Hilflosigkeit des Opfers.
Ein weiterer Unterschied besteht darin, dass das Opfer und der Täter in keinem direkten Kommunikationsverhältnis stehen, d. h. dass der Täter keine unmittelbare Reaktion des Opfers zu erwarten hat.
Durch die Anonymität des Internets fühlen sich gegebenenfalls auch mehrere Täter darin bestärkt, ein Opfer zu schikanieren, woraus resultieren kann, dass auch Kinder und Jugendliche zu Tätern werden können, die dies ohne die Plattform Internet vermutlich nicht tun würden.

Die genannten Punkte führen letztlich zu einer herabgesetzten Hemmschwelle für die Täter, da sich diese sicher fühlen und daher besonders brutal vorgehen können.

Aus der fehlenden, direkten Kommunikationssituation zwischen Opfer und Täter führt auch, dass das Opfer noch weniger Anhaltspunkte hat, die Gründe des Mobbings verstehen zu können.
Aus der mittlerweile großen Verbreitung des Internets auf verschiedenen Endgeräten wie Laptops und vor allem Smartphones, resultiert für den Täter die Möglichkeit, das Opfer ohne zeitliche Begrenzung wie beispielsweise den Unterrichtsbeginn oder dessen Ende, zu erreichen.
Dies hat zur Folge, dass ein Opfer von Cybermobbing im Gegensatz zum klassischen Mobbing nie Gelegenheit findet, sich der Mobbingsituation entziehen zu können und dieser Umstand zu den erwähnten, psychischen Effekten auf das Opfer beitragen kann.

Abschließend lässt sich feststellen, dass Cybermobbing eine gefährliche Ausprägung des Mobbings darstellt.
Sowohl Täter als auch Opfer sind sich oft nicht bewusst, dass Inhalte, die im Internet veröffentlicht wurden, gar nicht oder nur sehr schwer entfernt werden können.
Für die Opfer ist Cybermobbing in seinen Folgen besonders belastend, da sie rund um die Uhr zur Zielscheibe der Täter werden können.
Zudem ist die Zahl der Beteiligten im Gegensatz zum klassischen Mobbing nicht auf eine bestimmte Gruppe begrenzt, sondern für jeden im Internet zugänglich.
Auf diese Weise ist der Mobbingprozess für das Opfer dauerhaft präsent.

Der Fall von Amanda Todd, der als stellvertretendes Beispiel für unzählige andere Fälle steht, macht deutlich, wie groß das Spektrum der möglichen Langzeitfolgen für die Opfer sein kann.
Daraus erwächst für die Gesellschaft und - auf die Schule bezogen - für Lehrer, Eltern und Schüler die Verantwortung, frühzeitig Mobbing und seine Ausprägungen zu erkennen und einzudämmen.
