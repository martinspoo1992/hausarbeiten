\section{Einleitung}
Die vorliegende Hausarbeit setzt sich mit den sprachlichen Besonderheiten des Cybermobbings auseinander.
Hierfür stellt sich die Frage, inwiefern sich Cybermobbing von Mobbing in seinen allgemeinen Merkmalen wie auch auf sprachlicher Ebene unterscheidet.

Um den Bezug zwischen Sprache und Cybermobbing herzustellen, soll sich der erste Abschnitt mit der Verbindung sprachlicher Äußerungen und Gewalt auseinandersetzen.
Anschließend soll beschrieben werden, worum es sich bei Cybermobbing im Allgemeinen handelt und worin seine Eigenschaften liegen.
Dabei sollen auch einige Modelle, die das Phänomen Cybermobbing beschreiben, betrachtet werden.
In einem weiteren Schritt sollen sprachliche Äußerungen von Cybermobbing an einem  Beispiel analysiert werden und auf bekannte, linguistische Modelle übertragen werden.
Hierbei soll gezeigt werden, dass sich beispielsweise die Sprechakttheorie von John Searle auch auf sprachliche Handlungen des Cybermobbings anwenden lässt.
Schlussendlich sollen Gemeinsamkeiten und Unterschiede zwischen Cybermobbing und Mobbing dargestellt werden.

Cybermobbing stellt ein weitreichendes, gesellschaftliches Problem dar.
Welche Folgen diese spezielle Form von Mobbing haben kann, wurde besonders deutlich durch den Selbstmord der 15-jährigen kanadischen Schülerin Amanda Todd, der zu einer größeren, medialen Präsenz von Cybermobbing führte.
Nachdem sie sich in einem Chatroom vor einem Fremden entblößte, wurde sie von diesem mit Bildern erpresst, die sie nackt zeigen.
Als sie sich jedoch weigert, veröffentlicht der Täter die Bilder von ihr, was dazu führte, dass auch ihre Mitschüler begannen, sie zu schikanieren.
Trotz mehrerer Umzüge und Schulwechsel verbesserte sich Todds Situation nicht und sie geriet in eine Phase, in der sie sich selbst regelmäßig verletzte.
Im Oktober 2012 nahm sie sich letztlich das Leben.
Im April 2014 wurde der mutmaßliche Täter in den Niederlanden gefasst und von kanadischer Seite seine Auslieferung beantragt.

Doch auch in Deutschland wird Cybermobbing zu einem wachsenden Problem.
Laut der JIM-Studie des Medienpädagogischen Forschungsverbundes Südwest des Jahres 2013, bei der 1170 Jugendliche zwischen 12 und 19 Jahren befragt wurden, kennen 32\,\% der Befragten eine Person, die bereits einmal über das Internet oder das Handy gemobbt wurde.
7\,\% gaben an, bereits selbst Opfer von Cybermobbing geworden zu sein.
Dabei sind mehr Mädchen als Jungen betroffen.
Auch von der Schulform scheint das Ausmaß des Mobbings abzuhängen.
Während von den Gymnasiasten nur 4\,\% angaben, schon einmal im Internet gemobbt worden zu sein, waren es an den Realschulen 10\,\% und an Hauptschulen 11\,\%. \cite[S.\,44]{SF13}
 
Wie das Beispiel von Amanda Todd zeigt, ist Cybermobbing ein Problem, bei dem Schüler, Eltern und Lehrer eng zusammenarbeiten müssen, um Folgen wie Suizid oder Selbstverletzungen der Opfer zu verhindern.
