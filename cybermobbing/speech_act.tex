\section{Sprechakte beim Cybermobbing}
Im Folgenden soll Cybermobbing im Hinblick auf die Anwendbarkeit des Sprechaktmodells von John Searle untersucht werden.
Hierzu dient ein Artikel der deutschen Wochenzeitung „Die Zeit“ aus der Ausgabe 47 des Jahres 2013 als Beispiel.
Darin beschreibt das Opfer – ein 15-jähriges Mädchen, dessen Name in Lea geändert wurde, - wie sie innerhalb weniger Wochen zum Opfer wurde, weil sie mit einem Jungen zusammen war, den ihre Mitschülerinnen als „Macho“ bezeichneten und auch die Mitschüler nicht mochten. \cite{LT13}
Die Täter – einige ihrer Mitschüler – veröffentlichten in einem Onlineforum den neuen Spitznamen „Analia“.
Bereits in diesem Anfangsstadium des Cybermobbings zeigt sich eine vulgärsprachliche Tendenz der Täter.

Im weiteren Verlauf setzt sich das Mobbing in einer Whatsapp-Chatgruppe fort, in der die Täter weitere Diffamierungen vornehmen: „Rosen sind rot. Ich hab nen Schal. Sie wurde gefickt, heftig anal.“ \cite[S.\,2]{LT13}
Hier zeigt sich eine deutlich aggressive und sexistische Ausdrucksweise, mit der dem Opfer Lea zugeschrieben wird, analen Geschlechtsverkehr ausgeübt zu haben.
Hinzu kommt, dass die Täter mit dieser Formulierung ausdrücken, dass sie davon ausgehen, dass Lea mit ihrem damaligen Freund Geschlechtsverkehr hatte.
Diese inhaltliche Aussage wird in Reimform in Anspielung an das Gedicht „Rosen sind rot, Veilchen sind blau, ich hab dich gern, das weiß ich genau.“ aus dem Jahre 1784, dessen Autor nicht bekannt ist, vorgenommen.

Hier zeigt sich vorrangig die Ausprägungsform des Flamings, wie es Nancy Willard in ihrem Werk „Cyberbullying und Cyberthreats“ beschreibt.
An diesem Beispiel wird zudem deutlich, wie die sogenannte „Netikette“, also das angemessene sprachliche Verhalten in elektronischen Kommunikationsmedien, verletzt wird.
Hinzu kommt, dass nach Kleinke sprachliche Gewalt anders als physische Gewalt häufig weniger offensichtliche, subtile Züge trägt. \cite[S.\,314]{SK07}

Auf die vorhin zitierte Äußerung lässt sich ergänzend die Sprechakttheorie von John Searle anwenden, der in seiner Theorie von vier verschiedenen Akten ausgeht, die er als Äußerungsakt, propositionaler Akt, illokutionärer Akt und perluktionärer Akt bezeichnet. \cite[S.\,210]{AL04}
Der Äußerungsakt umfasst dabei sowohl die Verwendung von Stimm- oder Schreibwerkzeugen, als auch die Realisierung der Äußerung durch „abstrakte Muster eines Sprachsystems“ wie z.B. Phoneme, Morpheme, etc. Anders als John Austin, der in seiner Theorie die vorhin genannten Eigenschaften zusammen mit der Verknüpfung zwischen der Sprache und der Lebenswirklichkeit als Lokution zusammenfasst, bezeichnet Searle sie als eigenständigen Sprechakt, der als propositionaler Akt bezeichnet wird.

Ferner umfasst der illokutionäre Akt die Funktion, die mit der getätigten Sprachhandlung erfüllt werden soll.
Dies können beispielsweise Drohungen, Feststellungen oder Befehle sein.
Der perlokutionäre Akt schließt die Sprachhandlung ab, indem er von der angesprochenen Person eine Reaktion wie z.B. eine Entschuldigung oder eine Rechtfertigung verlangt.
Neben der dargestellten Einteilungen des Sprechaktes in verschiedene Teilakte, hat Searl fünf Sprechaktklassifikationen definiert.
Er unterscheidet dabei zwischen repräsentativen Sprechakten, mit denen Aussagen über Weltdarstellungen getroffen werden, direktiven Sprechakten, die Forderungen an den Rezipienten stellen, kommissiven Sprechakten, die dem Sprecher Verpflichtungen abverlangen, expressiven Sprechakten, welche soziale Kontakte bedienen und deklarativen Sprechakten, die zum Ausdruck festgelegter Abläufe dienen. \cite[S.\,218]{AL04}

Im vorliegenden Beispiel besteht der Äußerungsakt darin, dass der Verfasser der Nachricht seine sprachliche Handlung in Form eines Gedichtes vollzieht, welche er mithilfe des Mobiltelefons, das als Trägermedium verwendet wird, übermittelt.
Dabei formuliert er seine Aussagen in Lautketten, die mit Worten der deutschen Sprache verschriftlicht werden können.
Er befolgt dabei mit Ausnahme des letzten „Verses“ die Wortfolge der deutschen Standardsprache für Verbzweitsätze (\cite[S.\,398]{PE06}), die auf der Abfolge von Subjekt, Prädikat und Objekt fußt bzw. bei der das Verb stets an zweiter Stelle steht, da er „Sie wurde gefickt, heftig anal.“ formuliert.
Hier ist die geänderte Satzstellung offensichtlich der Einhaltung des Reimschemas geschuldet.
Für den illokutionären Akt der getätigten Äußerung lässt sich festhalten, dass der Verfasser der Sprachnachricht mit dieser die Sprechhandlung der Beleidigung und Bloßstellung vollzieht bzw. vollziehen möchte.

Bezogen auf den perlokutionären Sprechakt lässt sich anhand des Beispiels keine konkrete Reaktion herleiten, die der Sprecher ausdrücken möchte.
Da es sich jedoch um eine eindeutig pejorative Aussage handelt, kann man davon ausgehen, dass der Sprecher erwartet, dass sich das Opfer durch die Verse des Täters herabgewürdigt fühlt.

Entsprechend der Sprechaktklassifikation nach Searle lassen sich die beschriebenen Sprachhandlungen als Repräsentativa bezeichnen, bei denen Behauptungen über eine Person aufgestellt werden, was sich in der Formulierung der Äußerung zeigt.
