\section{Definition}
Unter dem Begriff „Cybermobbing“ finden sich verschiedene Definitionen.
Der Duden beispielsweise bezeichnet Cybermobbing als das „Schikanieren von Personen über das Internet“. \cite{DC13}
Das Bundesministerium für Familie, Senioren, Frauen und Jugend versteht unter Cybermobbing „die Beleidigung, Bedrohung, Bloßstellung oder Belästigung von Personen mithilfe neuer Kommunikationsmedien – z.B. über Handy, E-Mails, Websites, Foren, Chats und Communities“ und präzisiert diese Definition noch. \cite{BFS11}
Der Begriff leitet sich ab vom englischen Verb „to mob“, welches umgangssprachlich „anpöbeln“ bedeutet.
Oftmals findet sich auch die Bezeichnung „Cyberbullying“, die sich vom englischen Wort „to bully“ ableitet, was übersetzt „jemanden tyrannisieren bedeutet.
Beide Begriffe werden in diesem Zusammenhang synonym verwendet.

Die Besonderheit dieser Form des Mobbings liegt darin, dass die Opfer von den Tätern nicht persönlich angegriffen werden, sondern mithilfe moderner Kommunikationsmedien diffamiert werden.
Im Gegensatz zum „klassischen“ Mobbing ist durch die fehlende Face-to-face-Kommunikation die Hemmschwelle der Täter deutlich herabgesetzt.
Hinzu kommt, dass die Äußerungen über die Opfer einer breiteren Öffentlichkeit zugänglich sind, wenn Menschen in sozialen Netzwerken oder Foren gemobbt werden.
Zudem lassen sich Bilder, Videos oder andere Datenträger, die das Opfer herabsetzen, gar nicht oder nur schwer entfernen, was zu einer Langzeitwirkung für das Opfer führt.

Nancy Willard unterscheidet acht Verschiedene Ausprägungen von Cyberbullying. \cite{NW08}
Das „Flaming“ bezeichnet dabei das Beleidigen und Beschimpfen im öffentlichen Bereich.
Unter „Harassment“ versteht sie die Belästigung des Opfers durch wiederholte und gezielte Attacken.
Das Verbreiten von Gerüchten über das Opfer wird als „Denigration“ bezeichnet.
Dabei werden beispielsweise Fotos oder Filmaufnahmen des Opfers im Internet veröffentlicht, um die sozialen Kontakte des Opfers zu kappen.
Weitere Formen sind „Impersonation“ (Auftreten unter fremder Indentität), „Outing and Trickery“ (Bloßstellung und Betrügerei), „Exclusion“ (Ausschließen), „Cyberstalking“ (fortwährende Belästigung und Verfolgung) sowie „Cyberthreats“ (offene Androhung von Gewalt).

Darüber hinaus lässt sich nach Heinz Leymann Mobbing am Arbeitsplatz in verschiedene Phasen unterteilen. \cite[S.\,59]{HL93}
Leymann hat festgestellt, dass Mobbing entsteht, wenn ein Konflikt nicht aufgelöst werden kann und sich niemand dessen annimmt, so dass der Konflikt im Raume stehen bleibt.
Im Laufe der Zeit tritt der ursprünglich sachliche Disput in den Hintergrund und eine persönliche Auseinandersetzung entsteht.
Die zweite Phase zeichnet sich dadurch aus, dass eine beteiligte Person des Konflikts in die Rolle des Opfers gedrängt wird.
Die Täter beginnen das Opfer zu beleidigen und ihn zu schaden.
Laut Leymann können die Reaktionen der Opfer sowohl aggressiv als auch introvertiert sein.

In der nächsten Phase wirkt sich das Mobbing auch auf das Verhalten des Opfers am Arbeitsplatz aus.
Das Opfer kann sich nicht mehr auf seine Arbeit fokussieren und ihm unterlaufen Fehler oder er fehlt aufgrund der psychischen Belastung.
Für den Arbeitgeber wird der betroffene Mitarbeiter zum Problem, was zur Folge hat, dass das Opfer nun nicht nur gegen die Täter, sondern auch gegen den Vorgesetzten ankämpfen muss und so eine weitere psychische Belastung entsteht.
Die letzte Phase nach Leymann führt letztendlich zu einem Bruch des Arbeitsverhältnisses.
Dies kann sowohl vom Mitarbeiter selbst, als auch vom Vorgesetzten ausgehen.
Ein Teil der Betroffenen haben lebenslang psychische Probleme, was zur Arbeitsunfähigkeit führen kann.
Die Gründe, warum bestimmte Menschen zu Opfer von Cybermobbing werden, liegen nach Petra Grimm vor allem im Anders-Sein, der sozialen Isolation und der geringen Beliebtheit in der Schule. \cite[S.\,230]{PG08}
