\RequirePackage{ifvtex}
\documentclass[12pt,2headlines,a4paper,oneside,bibtotoc]{scrartcl}

\usepackage{ngerman}
\usepackage[utf8]{inputenc}
\usepackage[T1]{fontenc}
\usepackage{setspace}
\usepackage[left=40mm,top=25mm,right=30mm,bottom=35mm,nohead,nofoot]{geometry}
\usepackage[font=small,format=plain,labelfont=bf,up,textfont=it,up]{caption}
\usepackage{scrpage2}
\usepackage{graphicx}
\usepackage{tikz}
\usepackage{hyperref}
\usepackage{float}
\usepackage{longtable}
\usepackage[numbers]{natbib}
\usepackage{soul}

\setcounter{LTchunksize}{100}

\usetikzlibrary{shapes,arrows}

\pagestyle{scrheadings}
\ihead{\headmark}
\ohead{Martin Spoo}
\ifoot{Sprachliches Handeln bei Cybermobbing}
\cfoot{}
\ofoot{\pagemark}
\setheadsepline{0.4pt}
\setfootsepline{0.4pt}
\addtolength{\headsep}{10mm}
\addtolength{\footskip}{10mm}
\setlength{\headheight}{1.1\baselineskip}

%\bibliographystyle{plainde}

\renewcommand{\figurename}{Abb.}

\begin{document}

%% Titelseite
\thispagestyle{empty}
\newgeometry{inner=1.5cm,outer=1.5cm,head=1.5cm,bottom=1.5cm}
\begin{titlepage}
	\subpdfbookmark{Titel}{pdf:title}
	\begin{center}
		\quad
		\vfill
		\Huge{
			\setstretch{2} \textbf{Philosophische Vorstellungen von Kindern zum Thema \glqq Glück\grqq}
		}
		\vspace{5mm}
		\vfill
		\large{
			{\bfseries Masterarbeit\\
			\vspace{5mm}
			\normalfont \rmfamily zur Erlangung des akademischen Grades\\
			\bfseries Master of Education (M. Ed.)\\
			\vspace{5mm}
			\normalfont \rmfamily im Studiengang Grundschulbildung}
		}
		\\
		\vspace{1.5cm}
		\large{
			{vorgelegt von\\
			Martin Spoo\\
			Matrikelnr.\,212100872}
		}
		\vspace{1cm}
		\\
		\Large{
			{Universität Koblenz-Landau}\\
			{SS\, 2016}
		}
		\vspace{1cm}
		\begin{table}[b]
			\begin{center}
				\begin{tabular}{lr}
					Prüfer: &  Prof. Dr. Heike de Boer , \\
								&	Institut für Grundschulbildung, Campus Koblenz \\
					Zweitprüfer: & Dr. Nicole Henrich \\
					\vspace{0.25cm} \\
					Abgabetermin: & 24. August 2016 \\
					Datum: & \today
				\end{tabular}
			\end{center}
		\end{table}
	\end{center}
\end{titlepage}
\renewcommand{\baselinestretch}{1.1}
\restoregeometry



\newpage

%% Seitenzähler nach dem Titelblatt auf 1 setzen
\setcounter{page}{1}

%% Inhaltsverzeichnis
\tableofcontents
\newpage

%% Zeilenabstand auf 1,5fach
\onehalfspacing

%% Einleitung
\section{Einleitung}
In der folgenden Hausarbeit soll ein synoptischer Vergleich der biblischen Textstelle Mk 15,38-41 durchgeführt werden. Dabei liegt das besonderere Augenmerk auf der Untersuchung der Evangelien auf inhaltliche und sprachliche Differenzen. Um einen Einstieg zu ermöglichen, will ich die Methode des synoptischen Vergleiches kurz erläutern.
Beim synoptischen Vergleich handelt es sich um eine Untersuchungsmethode, bei der vergleichbare Textpassagen der synoptischen Evangelien tabellarisch nebeneinander gestellt werden. In dieser Hausarbeit wird es sich dabei um die zu Beginn genannte Passage des Markusevangeliums sowie Lk 23,45-49 und Mt 27,51-56 handeln.
Der synoptische Vergleich bietet uns die Möglichkeit, eine Textstelle aus den synoptischen Evangelien nach Markus, Lukas und Matthäus näher zu untersuchen. Nach der Zweiquellentheorie geht man davon aus, dass das Markusevangelium als erstes entstand und als Vorlage für Matthäus und Lukas diente. Diese haben sich wiederum zusätzlich noch einer Logienquelle Q und ihres jeweiligen Sondergutes bedient.
Als Forschungsfrage leitet sich daher ab: „Inwiefern kann man die Zweiquellentheorie anhand der Textpassage Mk 15,38-41 bestätigen?“
Die Wahl des Themas der Passion für einen synoptischen Vergleich ergab sich daraus, dass mir die Passion bereits aus dem Religionsunterricht bekannt ist  und sie zu einem der Themen der Bibel gehört, die mich am meisten interessieren. Da die Größe der Textstelle auf vier Verse vorgegeben war, musste eine Stelle gewählt werden, die inhaltlich und sprachlich genug Untersuchungspunkte bietet. Des Weiteren lässt sich anhand der Leidensgeschichte Jesu gut darstellen, worauf die Evangelisten den inhaltlichen Schwerpunkt in ihren Evangelien legten, um ihre jeweiligen Zielgruppen anzusprechen. 
\newpage

%% Sprache und Gewalt
\section{Sprache und Gewalt}
Um die Verknüpfung zwischen der Sprache und Cybermobbing nachzuvollziehen, ist es zunächst sinnvoll, den Bezug zwischen Sprache und Gewalt zu erörtern und zu beleuchten, worum es sich bei Gewalt handelt.
Allgemein bezeichnet Gewalt die „Macht, Befugnis, das Recht und die Mittel, über jemanden, etwas zu bestimmen, zu herrschen.“ \cite{DG13}

Sybille Krämer sieht den Begriff der Gewalt in einem „ambivalenten Bedeutungsfeld“, „changierend zwischen konstruktiven und negativen Konnotationen.“ \cite[S.\,35]{SKR07}
Demnach kann unter „Gewalt“ sowohl ausgeübte Amts- beziehungsweise Staatsgewalt, aber auch verübte Gewalt in Bezug auf Gewalttaten verstanden werden.
Krämer nennt dazu die lateinischen Ausdrücke „Potestas“ für Staatsgewalt und „Violentia“ für Gewalttaten.
Anhand der lateinischen Bezeichnungen wird die Ambiguität des Gewaltbegriffs greifbar.
Hauptunterschied zwischen den genannten Gewaltbegriffen ist dabei, dass Gewalt als Potestas eine Form der Machtausübung darstellt, die sich vorherrschenden Gesetzen unterordnet, während Violentia bewusst mit bestehenden Regeln bricht.
Im Rahmen dieser Gewaltdefinitionen steht die sprachliche Gewalt in Zusammenhang zur violenten Form der Gewalt.

Betrachtet man die Verben der deutschen Sprache, die semantisch im Zusammenhang mit Aggression stehen, so fällt auf, dass sie alle einen tatsächlichen Vorgang des Handelns beschreiben.
Daraus folgt für Krämer, dass aus einer verbalen Äußerung nicht ohne weiteres geschlossen werden kann, dass sie die Funktion der Verletzung oder Beleidigung erfüllen soll.
Daher muss bei sprachlicher Gewalt immer miteinbezogen werden, in welchem Kontext eine Äußerung getätigt wird, wer an der Konversation beteiligt ist und wie die Äußerung formuliert wurde.
Dementsprechend sind verletzende Worte „nicht einfach Bestandteil der Sprache als System, sondern ein Phänomen des kulturell eingebetteten Sprachgebrauches“. \cite[S.\,35]{SKR07}

Auch König und Stathi sehen nicht die Sprache als solche als gewalttätig an, sondern die Äußerungen und kommunikative Akte, die in bestimmten Situationen vollzogen werden. \cite[S.\,47]{EK07}
Weiter führen sie an, dass jegliche nicht-verbalen Akte sowie deren Komponenten ebenfalls gewalttätig sein können.
Darunter fassen sie Intonation, Lautstärke, Körpersprache, Mimik, Gestik, etc. Aber auch Handlungen, die unterlassen werden, können daher in bestimmten, kommunikativen Kontexten als aggressiv und verletzend eingestuft werden.
Ekkehard König und Katerina Stathi gehen folglich in ihrer Definition, worum es sich bei sprachlicher Gewalt handelt, einen Schritt weiter und ergänzen ihre Definition der sprachlichen Gewalt neben der artikulatorischen Ebene noch um die Ebene der Körpersprache.

Daher lässt sich über sprachliche Gewalt sagen, dass damit nicht die reine Ebene der Wörter gemeint ist, sondern Sprache immer in einem kontextuellen Umfeld verletzend wird und ohne jenen Kontext in der Regel keinen, violenten Hintergrund hat.
Zudem lässt sich sprachliche Gewalt nicht nur auf das System der menschlichen Sprache, sondern auch auf dessen Verhalten beziehen.


%% Definition 'Cybermobbing'
\section{Definition}
Unter dem Begriff „Cybermobbing“ finden sich verschiedene Definitionen.
Der Duden beispielsweise bezeichnet Cybermobbing als das „Schikanieren von Personen über das Internet“. \cite{DC13}
Das Bundesministerium für Familie, Senioren, Frauen und Jugend versteht unter Cybermobbing „die Beleidigung, Bedrohung, Bloßstellung oder Belästigung von Personen mithilfe neuer Kommunikationsmedien – z.B. über Handy, E-Mails, Websites, Foren, Chats und Communities“ und präzisiert diese Definition noch. \cite{BFS11}
Der Begriff leitet sich ab vom englischen Verb „to mob“, welches umgangssprachlich „anpöbeln“ bedeutet.
Oftmals findet sich auch die Bezeichnung „Cyberbullying“, die sich vom englischen Wort „to bully“ ableitet, was übersetzt „jemanden tyrannisieren bedeutet.
Beide Begriffe werden in diesem Zusammenhang synonym verwendet.

Die Besonderheit dieser Form des Mobbings liegt darin, dass die Opfer von den Tätern nicht persönlich angegriffen werden, sondern mithilfe moderner Kommunikationsmedien diffamiert werden.
Im Gegensatz zum „klassischen“ Mobbing ist durch die fehlende Face-to-face-Kommunikation die Hemmschwelle der Täter deutlich herabgesetzt.
Hinzu kommt, dass die Äußerungen über die Opfer einer breiteren Öffentlichkeit zugänglich sind, wenn Menschen in sozialen Netzwerken oder Foren gemobbt werden.
Zudem lassen sich Bilder, Videos oder andere Datenträger, die das Opfer herabsetzen, gar nicht oder nur schwer entfernen, was zu einer Langzeitwirkung für das Opfer führt.

Nancy Willard unterscheidet acht Verschiedene Ausprägungen von Cyberbullying. \cite{NW08}
Das „Flaming“ bezeichnet dabei das Beleidigen und Beschimpfen im öffentlichen Bereich.
Unter „Harassment“ versteht sie die Belästigung des Opfers durch wiederholte und gezielte Attacken.
Das Verbreiten von Gerüchten über das Opfer wird als „Denigration“ bezeichnet.
Dabei werden beispielsweise Fotos oder Filmaufnahmen des Opfers im Internet veröffentlicht, um die sozialen Kontakte des Opfers zu kappen.
Weitere Formen sind „Impersonation“ (Auftreten unter fremder Indentität), „Outing and Trickery“ (Bloßstellung und Betrügerei), „Exclusion“ (Ausschließen), „Cyberstalking“ (fortwährende Belästigung und Verfolgung) sowie „Cyberthreats“ (offene Androhung von Gewalt).

Darüber hinaus lässt sich nach Heinz Leymann Mobbing am Arbeitsplatz in verschiedene Phasen unterteilen. \cite[S.\,59]{HL93}
Leymann hat festgestellt, dass Mobbing entsteht, wenn ein Konflikt nicht aufgelöst werden kann und sich niemand dessen annimmt, so dass der Konflikt im Raume stehen bleibt.
Im Laufe der Zeit tritt der ursprünglich sachliche Disput in den Hintergrund und eine persönliche Auseinandersetzung entsteht.
Die zweite Phase zeichnet sich dadurch aus, dass eine beteiligte Person des Konflikts in die Rolle des Opfers gedrängt wird.
Die Täter beginnen das Opfer zu beleidigen und ihn zu schaden.
Laut Leymann können die Reaktionen der Opfer sowohl aggressiv als auch introvertiert sein.

In der nächsten Phase wirkt sich das Mobbing auch auf das Verhalten des Opfers am Arbeitsplatz aus.
Das Opfer kann sich nicht mehr auf seine Arbeit fokussieren und ihm unterlaufen Fehler oder er fehlt aufgrund der psychischen Belastung.
Für den Arbeitgeber wird der betroffene Mitarbeiter zum Problem, was zur Folge hat, dass das Opfer nun nicht nur gegen die Täter, sondern auch gegen den Vorgesetzten ankämpfen muss und so eine weitere psychische Belastung entsteht.
Die letzte Phase nach Leymann führt letztendlich zu einem Bruch des Arbeitsverhältnisses.
Dies kann sowohl vom Mitarbeiter selbst, als auch vom Vorgesetzten ausgehen.
Ein Teil der Betroffenen haben lebenslang psychische Probleme, was zur Arbeitsunfähigkeit führen kann.
Die Gründe, warum bestimmte Menschen zu Opfer von Cybermobbing werden, liegen nach Petra Grimm vor allem im Anders-Sein, der sozialen Isolation und der geringen Beliebtheit in der Schule. \cite[S.\,230]{PG08}


%% Sprechakte
\section{Sprechakte beim Cybermobbing}
Im Folgenden soll Cybermobbing im Hinblick auf die Anwendbarkeit des Sprechaktmodells von John Searle untersucht werden.
Hierzu dient ein Artikel der deutschen Wochenzeitung „Die Zeit“ aus der Ausgabe 47 des Jahres 2013 als Beispiel.
Darin beschreibt das Opfer – ein 15-jähriges Mädchen, dessen Name in Lea geändert wurde, - wie sie innerhalb weniger Wochen zum Opfer wurde, weil sie mit einem Jungen zusammen war, den ihre Mitschülerinnen als „Macho“ bezeichneten und auch die Mitschüler nicht mochten. \cite{LT13}
Die Täter – einige ihrer Mitschüler – veröffentlichten in einem Onlineforum den neuen Spitznamen „Analia“.
Bereits in diesem Anfangsstadium des Cybermobbings zeigt sich eine vulgärsprachliche Tendenz der Täter.

Im weiteren Verlauf setzt sich das Mobbing in einer Whatsapp-Chatgruppe fort, in der die Täter weitere Diffamierungen vornehmen: „Rosen sind rot. Ich hab nen Schal. Sie wurde gefickt, heftig anal.“ \cite[S.\,2]{LT13}
Hier zeigt sich eine deutlich aggressive und sexistische Ausdrucksweise, mit der dem Opfer Lea zugeschrieben wird, analen Geschlechtsverkehr ausgeübt zu haben.
Hinzu kommt, dass die Täter mit dieser Formulierung ausdrücken, dass sie davon ausgehen, dass Lea mit ihrem damaligen Freund Geschlechtsverkehr hatte.
Diese inhaltliche Aussage wird in Reimform in Anspielung an das Gedicht „Rosen sind rot, Veilchen sind blau, ich hab dich gern, das weiß ich genau.“ aus dem Jahre 1784, dessen Autor nicht bekannt ist, vorgenommen.

Hier zeigt sich vorrangig die Ausprägungsform des Flamings, wie es Nancy Willard in ihrem Werk „Cyberbullying und Cyberthreats“ beschreibt.
An diesem Beispiel wird zudem deutlich, wie die sogenannte „Netikette“, also das angemessene sprachliche Verhalten in elektronischen Kommunikationsmedien, verletzt wird.
Hinzu kommt, dass nach Kleinke sprachliche Gewalt anders als physische Gewalt häufig weniger offensichtliche, subtile Züge trägt. \cite[S.\,314]{SK07}

Auf die vorhin zitierte Äußerung lässt sich ergänzend die Sprechakttheorie von John Searle anwenden, der in seiner Theorie von vier verschiedenen Akten ausgeht, die er als Äußerungsakt, propositionaler Akt, illokutionärer Akt und perluktionärer Akt bezeichnet. \cite[S.\,210]{AL04}
Der Äußerungsakt umfasst dabei sowohl die Verwendung von Stimm- oder Schreibwerkzeugen, als auch die Realisierung der Äußerung durch „abstrakte Muster eines Sprachsystems“ wie z.B. Phoneme, Morpheme, etc. Anders als John Austin, der in seiner Theorie die vorhin genannten Eigenschaften zusammen mit der Verknüpfung zwischen der Sprache und der Lebenswirklichkeit als Lokution zusammenfasst, bezeichnet Searle sie als eigenständigen Sprechakt, der als propositionaler Akt bezeichnet wird.

Ferner umfasst der illokutionäre Akt die Funktion, die mit der getätigten Sprachhandlung erfüllt werden soll.
Dies können beispielsweise Drohungen, Feststellungen oder Befehle sein.
Der perlokutionäre Akt schließt die Sprachhandlung ab, indem er von der angesprochenen Person eine Reaktion wie z.B. eine Entschuldigung oder eine Rechtfertigung verlangt.
Neben der dargestellten Einteilungen des Sprechaktes in verschiedene Teilakte, hat Searl fünf Sprechaktklassifikationen definiert.
Er unterscheidet dabei zwischen repräsentativen Sprechakten, mit denen Aussagen über Weltdarstellungen getroffen werden, direktiven Sprechakten, die Forderungen an den Rezipienten stellen, kommissiven Sprechakten, die dem Sprecher Verpflichtungen abverlangen, expressiven Sprechakten, welche soziale Kontakte bedienen und deklarativen Sprechakten, die zum Ausdruck festgelegter Abläufe dienen. \cite[S.\,218]{AL04}

Im vorliegenden Beispiel besteht der Äußerungsakt darin, dass der Verfasser der Nachricht seine sprachliche Handlung in Form eines Gedichtes vollzieht, welche er mithilfe des Mobiltelefons, das als Trägermedium verwendet wird, übermittelt.
Dabei formuliert er seine Aussagen in Lautketten, die mit Worten der deutschen Sprache verschriftlicht werden können.
Er befolgt dabei mit Ausnahme des letzten „Verses“ die Wortfolge der deutschen Standardsprache für Verbzweitsätze (\cite[S.\,398]{PE06}), die auf der Abfolge von Subjekt, Prädikat und Objekt fußt bzw. bei der das Verb stets an zweiter Stelle steht, da er „Sie wurde gefickt, heftig anal.“ formuliert.
Hier ist die geänderte Satzstellung offensichtlich der Einhaltung des Reimschemas geschuldet.
Für den illokutionären Akt der getätigten Äußerung lässt sich festhalten, dass der Verfasser der Sprachnachricht mit dieser die Sprechhandlung der Beleidigung und Bloßstellung vollzieht bzw. vollziehen möchte.

Bezogen auf den perlokutionären Sprechakt lässt sich anhand des Beispiels keine konkrete Reaktion herleiten, die der Sprecher ausdrücken möchte.
Da es sich jedoch um eine eindeutig pejorative Aussage handelt, kann man davon ausgehen, dass der Sprecher erwartet, dass sich das Opfer durch die Verse des Täters herabgewürdigt fühlt.

Entsprechend der Sprechaktklassifikation nach Searle lassen sich die beschriebenen Sprachhandlungen als Repräsentativa bezeichnen, bei denen Behauptungen über eine Person aufgestellt werden, was sich in der Formulierung der Äußerung zeigt.


%% Gemeinsamkeiten und Unterschiede
\section{Gemeinsamkeiten und Unterschiede zwischen Cybermobbing und Mobbing}
Insgesamt ist der Übergang vom Mobbing in der Schule oder am Arbeitsplatz hin zum Cybermobbing fließend.
Jedoch weist Cybermobbing einige Merkmale des Mobbings, wie es schon seit längerer Zeit bekannt ist, auf und stellt daher eher eine andere Ausprägungsform von Mobbing dar.

Grundsätzlich gemein ist beiden Formen, dass sie das Ziel haben, das Opfer zu erniedrigen und herabzusetzen.
Hinzu kommt, dass sich die soziale Struktur eines solchen Konfliktes beim Cybermobbing ähnlich verhält, da es sowohl beim Cybermobbing als auch beim Mobbing verschiedene Rollen der beteiligten Personen des sozialen Umfeldes geben kann. \cite[S.\,453-459]{CS13}
Nach Salmivalli handelt es sich dabei um die Rollen der Täter, Helfer, Verstärker, die durch ihr Auftreten den oder die Täter in ihrem Handeln bestärken, aber auch Verteidiger und Außenstehende, welche etwaige Handlungen ignorieren oder auch dulden.

Demgegenüber lassen sich allerdings auch Unterschiede zwischen Mobbing und Cybermobbing feststellen.
Während der Täter beim Mobbing in der Regel klar als solcher identifizierbar ist, so ist es ihm beim Cybermobbing besser möglich, anonym zu bleiben. \cite{TK14}
Dies erschwert ein Vorgehen gegen das Cybermobbing und verstärkt das Gefühl der Hilflosigkeit des Opfers.
Ein weiterer Unterschied besteht darin, dass das Opfer und der Täter in keinem direkten Kommunikationsverhältnis stehen, d. h. dass der Täter keine unmittelbare Reaktion des Opfers zu erwarten hat.
Durch die Anonymität des Internets fühlen sich gegebenenfalls auch mehrere Täter darin bestärkt, ein Opfer zu schikanieren, woraus resultieren kann, dass auch Kinder und Jugendliche zu Tätern werden können, die dies ohne die Plattform Internet vermutlich nicht tun würden.

Die genannten Punkte führen letztlich zu einer herabgesetzten Hemmschwelle für die Täter, da sich diese sicher fühlen und daher besonders brutal vorgehen können.

Aus der fehlenden, direkten Kommunikationssituation zwischen Opfer und Täter führt auch, dass das Opfer noch weniger Anhaltspunkte hat, die Gründe des Mobbings verstehen zu können.
Aus der mittlerweile großen Verbreitung des Internets auf verschiedenen Endgeräten wie Laptops und vor allem Smartphones, resultiert für den Täter die Möglichkeit, das Opfer ohne zeitliche Begrenzung wie beispielsweise den Unterrichtsbeginn oder dessen Ende, zu erreichen.
Dies hat zur Folge, dass ein Opfer von Cybermobbing im Gegensatz zum klassischen Mobbing nie Gelegenheit findet, sich der Mobbingsituation entziehen zu können und dieser Umstand zu den erwähnten, psychischen Effekten auf das Opfer beitragen kann.

Abschließend lässt sich feststellen, dass Cybermobbing eine gefährliche Ausprägung des Mobbings darstellt.
Sowohl Täter als auch Opfer sind sich oft nicht bewusst, dass Inhalte, die im Internet veröffentlicht wurden, gar nicht oder nur sehr schwer entfernt werden können.
Für die Opfer ist Cybermobbing in seinen Folgen besonders belastend, da sie rund um die Uhr zur Zielscheibe der Täter werden können.
Zudem ist die Zahl der Beteiligten im Gegensatz zum klassischen Mobbing nicht auf eine bestimmte Gruppe begrenzt, sondern für jeden im Internet zugänglich.
Auf diese Weise ist der Mobbingprozess für das Opfer dauerhaft präsent.

Der Fall von Amanda Todd, der als stellvertretendes Beispiel für unzählige andere Fälle steht, macht deutlich, wie groß das Spektrum der möglichen Langzeitfolgen für die Opfer sein kann.
Daraus erwächst für die Gesellschaft und - auf die Schule bezogen - für Lehrer, Eltern und Schüler die Verantwortung, frühzeitig Mobbing und seine Ausprägungen zu erkennen und einzudämmen.


\newpage
\bibliography{cybermobbing}
\bibliographystyle{dinat}

\newpage
\newgeometry{left=1.5cm,right=1.5cm,head=1.5cm,bottom=1.5cm}
\currentpdfbookmark{Erklärung}{pdf:declaration}
\thispagestyle{empty}
\vspace*{3cm}
\begin{center}
	\Large{\textbf{Erklärung}}
\end{center}
\vspace*{1cm}
Hiermit bestätige ich, dass die vorliegende Arbeit von mir selbstständig verfasst wurde und ich keine anderen als die angegebenen Hilfsmittel - insbesondere keine im Quellenverzeichnis nicht benannten Internet-Quellen - benutzt habe und die Arbeit von mir vorher nicht in einem anderen Prüfungsverfahren eingereicht wurde.
Die eingereichte schriftliche Fassung entspricht der auf dem elektronischen Speichermedium. (CD-ROM)
\vspace*{1cm}\\
Koblenz, den \today
\vspace*{0.75cm}\\
Martin Spoo
\restoregeometry


\end{document}

