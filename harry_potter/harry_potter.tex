\RequirePackage{ifvtex}
\documentclass[12pt,2headlines,a4paper,oneside,bibtotoc]{scrartcl}

\usepackage{ngerman}
\usepackage[utf8]{inputenc}
\usepackage[T1]{fontenc}
\usepackage{setspace}
\usepackage[left=40mm,top=25mm,right=30mm,bottom=35mm,nohead,nofoot]{geometry}
\usepackage[font=small,format=plain,labelfont=bf,up,textfont=it,up]{caption}
\usepackage{scrpage2}
\usepackage{graphicx}
\usepackage{tikz}
\usepackage{hyperref}
\usepackage{float}
\usepackage{longtable}
\usepackage[numbers]{natbib}
%\usepackage{BeamerColor}
%\usepackage{soul}

\setcounter{LTchunksize}{100}

\usetikzlibrary{shapes,arrows}

\pagestyle{scrheadings}
\ihead{\headmark}
\ohead{Martin Spoo}
\ifoot{Das Rassismus-Motiv der \glqq{}Harry Potter\grqq{}-Reihe}
\cfoot{}
\ofoot{\pagemark}
\setheadsepline{0.4pt}
\setfootsepline{0.4pt}
\addtolength{\headsep}{10mm}
\addtolength{\footskip}{10mm}
\setlength{\headheight}{1.1\baselineskip}

%\bibliographystyle{plainde}

\renewcommand{\figurename}{Abb.}

\begin{document}

%% Titelseite
\thispagestyle{empty}
\newgeometry{inner=1.5cm,outer=1.5cm,head=1.5cm,bottom=1.5cm}
\begin{titlepage}
	\subpdfbookmark{Titel}{pdf:title}
	\begin{center}
		\quad
		\vfill
		\Huge{
			\setstretch{2} \textbf{Philosophische Vorstellungen von Kindern zum Thema \glqq Glück\grqq}
		}
		\vspace{5mm}
		\vfill
		\large{
			{\bfseries Masterarbeit\\
			\vspace{5mm}
			\normalfont \rmfamily zur Erlangung des akademischen Grades\\
			\bfseries Master of Education (M. Ed.)\\
			\vspace{5mm}
			\normalfont \rmfamily im Studiengang Grundschulbildung}
		}
		\\
		\vspace{1.5cm}
		\large{
			{vorgelegt von\\
			Martin Spoo\\
			Matrikelnr.\,212100872}
		}
		\vspace{1cm}
		\\
		\Large{
			{Universität Koblenz-Landau}\\
			{SS\, 2016}
		}
		\vspace{1cm}
		\begin{table}[b]
			\begin{center}
				\begin{tabular}{lr}
					Prüfer: &  Prof. Dr. Heike de Boer , \\
								&	Institut für Grundschulbildung, Campus Koblenz \\
					Zweitprüfer: & Dr. Nicole Henrich \\
					\vspace{0.25cm} \\
					Abgabetermin: & 24. August 2016 \\
					Datum: & \today
				\end{tabular}
			\end{center}
		\end{table}
	\end{center}
\end{titlepage}
\renewcommand{\baselinestretch}{1.1}
\restoregeometry



\newpage

%% Seitenzähler nach dem Titelblatt auf 1 setzen
\setcounter{page}{1}

%% Inhaltsverzeichnis
\tableofcontents
\newpage

%% Zeilenabstand auf 1,5fach
\onehalfspacing

%% Einleitung
\section{Einleitung}
In der folgenden Hausarbeit soll ein synoptischer Vergleich der biblischen Textstelle Mk 15,38-41 durchgeführt werden. Dabei liegt das besonderere Augenmerk auf der Untersuchung der Evangelien auf inhaltliche und sprachliche Differenzen. Um einen Einstieg zu ermöglichen, will ich die Methode des synoptischen Vergleiches kurz erläutern.
Beim synoptischen Vergleich handelt es sich um eine Untersuchungsmethode, bei der vergleichbare Textpassagen der synoptischen Evangelien tabellarisch nebeneinander gestellt werden. In dieser Hausarbeit wird es sich dabei um die zu Beginn genannte Passage des Markusevangeliums sowie Lk 23,45-49 und Mt 27,51-56 handeln.
Der synoptische Vergleich bietet uns die Möglichkeit, eine Textstelle aus den synoptischen Evangelien nach Markus, Lukas und Matthäus näher zu untersuchen. Nach der Zweiquellentheorie geht man davon aus, dass das Markusevangelium als erstes entstand und als Vorlage für Matthäus und Lukas diente. Diese haben sich wiederum zusätzlich noch einer Logienquelle Q und ihres jeweiligen Sondergutes bedient.
Als Forschungsfrage leitet sich daher ab: „Inwiefern kann man die Zweiquellentheorie anhand der Textpassage Mk 15,38-41 bestätigen?“
Die Wahl des Themas der Passion für einen synoptischen Vergleich ergab sich daraus, dass mir die Passion bereits aus dem Religionsunterricht bekannt ist  und sie zu einem der Themen der Bibel gehört, die mich am meisten interessieren. Da die Größe der Textstelle auf vier Verse vorgegeben war, musste eine Stelle gewählt werden, die inhaltlich und sprachlich genug Untersuchungspunkte bietet. Des Weiteren lässt sich anhand der Leidensgeschichte Jesu gut darstellen, worauf die Evangelisten den inhaltlichen Schwerpunkt in ihren Evangelien legten, um ihre jeweiligen Zielgruppen anzusprechen. 
\newpage

%% Begriffsdefinitionen
\section{Definitionen}
Um den Einstieg in das Thema \glqq Rassismus-Motiv der  \glq Harry Potter\grq-Reihe\grqq{} zu erleichtern, soll im folgenden Abschnitt exemplarisch auf wichtige Begriffe wie \glqq Muggel\grqq, \glqq Halbblüter\grqq{} und das rassistische Pejorativum \glqq Schlammblut\grqq{} eingegangen werden.
Unter \glqq Muggeln\grqq{} (engl. \emph{muggles}) werden in den Romanen von Joanne K. Rowling Menschen ohne magische Begabung verstanden, wie in \cite[S. \,61]{JKR97} zu lesen ist. 
Abgeleitet wird der Begriff vom englischen Wort \emph{mug}, der sich umgangssprachlich mit \glqq jmd. abziehen\grqq{} übersetzen lässt. 
Mit diesem Ausdruck wird deutlich gemacht, dass \glqq normale\grqq{} Menschen von magischen Wesen als hilflos oder auch dümmlich angesehen werden und mit diesem herabsetzenden Ausdruck von magischen Wesen abgegrenzt werden sollen.
Muggel sind im Gegensatz zu magischen Wesen nicht in der Lage, bestimmte Kreaturen der magischen Welt zu sehen.

Von Muggeln zu unterscheiden sind sogenannte \glqq Halbblüter\grqq. 
Hat ein Zauberer sowohl ein Elternteil ohne Zauberkräfte, als auch eines mit Zauberkräften, ist das Kind folglich halb magischen \glqq Blutes\grqq, sobald sich die magische Begabung entwickelt hat. 
Zu ihnen zählt auch der Protagonist Harry Potter, da seine Eltern der Zauberer James Potter und die Muggelstämmige Hexe Lily Potter sind.
Halbblüter werden häufig von Zauberern, die rein magische Eltern haben und daher \glqq Reinblüter\grqq{} genannt werden, als minderwertiger angesehen \cite[S.\,220]{JKR03}. 
Hat ein Zauberer nur Muggel als Eltern, wie es beispielsweise bei Harry Potters bester Freundin Hermine Granger der Fall ist, verwenden besonders rassistisch eingestellte reinblütige Zaubererfamilien das Schimpfwort \glqq Schlammblut\grqq, das auf die - aus ihrer Sicht -  \glqq unreine\grqq{} Verwandtschaft mit Muggeln verweist.

Jedoch zeigt sich anhand des Beispiels von Hermine Granger, dass die Verwandtschaft mit Muggeln keineswegs Einfluss auf die magischen Fähigkeiten hat, da sie als eine der begabtesten Schülerinnen von Hogwarts gilt und ihre Eltern Muggel sind.
Ergänzend kann man sagen, dass in den Büchern deutlich wird, dass die meisten Zauberer der Auffassung sind, dass die Fähigkeit, Magie einzusetzen, ein Talent ist, dass in keinem kausalen Zusammenhang zur familiären Herkunft steht und daher auch durch Menschen erlernt werden kann, die ausschließlich nichtmagische Vorfahren haben. 




\newpage

%% Rassismus in der magischen Welt
\section{Rassismus in der magischen Welt}
Die rassistische Ideologie, die in der \glqq Harry Potter\grqq-Reihe beschrieben und von dunklen Magiern in der magischen Welt auch nach außen propagiert wird, fußt auf der Annahme, dass die Fähigkeit, Magie anwenden zu können, auf der familiären Abstammung beruht. Daher werden auch nur Magier, die ausschließlich magische Menschen als Vorfahren haben, als würdig angesehen, ihre Fähigkeiten nutzen zu lernen. Unter den Nachfolgern Lord Voldemorts, des bösesten Zauberers, geht die Ablehnung von Muggeln sogar so weit, dass sie die Verwandtschaft und den Kontakt zu Familienmitgliedern, die Kontakt mit \glqq unerwünschten\grqq Kreaturen haben, leugnen. Dies wird deutlich bei einer Konferenz der Todesser, bei der sie sich über die Gefangennahme von Harry Potter beraten, deutlich: \glqq Wir – Narzissa und ich (Bellatrix Lestrange d. Red.) – haben unsere Schwester nicht mehr zu Gesicht bekommen, seit sie den Schlammblüter geheiratet hat. Diese Göre hat mit keiner von uns etwas zu tun, ebenso wenig wie irgendein Biest, das sie heiratet.\cite[S.18]{JKR10}\grqq
Die Todesser lehnen jeglichen Kontakt zu Muggeln und Muggelstämmigen ab, da sie diese als minderwertig erachten. Voldemort selbst spricht davon, man müsse \glqq das Krebsgeschwür wegschneiden, das uns verseucht, bis nur noch die von wahrem Blut zurückbleiben.\cite[S.19]{JKR10}\grqq Gegen Ende des ersten Kapitels zeigt sich die besondere Grausamkeit Voldemorts und seiner Anhänger. Der gefangen genommene Professorin für \glqq Muggelkunde\grqq Charity Burbage, die über der Gruppe von Todessern an der Decke zur Schau gestellt wurde, werden \glqq Verbrechen\grqq wie Aufklärung über Muggel vorgeworfen. Sie habe nicht nur \glqq den Verstand von Zaubererkindern\grqq verdorben und besudelt, sondern auch behauptet, die schwindende Zahl von Reinblütern sei ein \glqq wünschenswertes Phänomen\grqq (\cite[S.19]{JKR10}). Daraufhin wird sie mit einem Todesfluch getötet.
Als die Todesser im weiteren Verlauf der Geschichte auch das zentrale Organ für magische Angelegenheite, das sogenannte Zaubereiministerium, unter ihre Kontrolle bringen, wird die rassistische Ideologie bereits in der Eingangshalle klar gemacht. Auf Seite 249 beschreibt Joanne K. Rowling ein schwarzes steinernes Denkmal, dass einen Zauberer und eine Hexe zeigt, die auf Thronen sitzen.\cite[S.249]{JKR10} Auf dem Sockel, der von stilisierten Muggeln getragen wird, sind die Worte \glqq Magie ist Macht\grqq eingraviert. In Kapitel 13 beschreibt Rowling die Einrichtung einer \glqq Registrierungskommission für Muggelstämmige\grqq , deren Aufgabe es ist, alle Muggelstämmigen zu listen und ihre Blutsherkunft zu überprüfen. Zudem müssen sie sich einem Verhör unterziehen und nachweisen, welchem Zauberer sie ihren Zauberstab entwendet haben\cite [S.268]{JKR10}. Daran lassen sich die fragwürdigen Methoden der Todesser erkennen, mit denen sie Muggel und Muggelgeborene entrechten und bloßstellen. 
Die Formen des Rassismus, wie er von Rowling in ihrem Buch in die Handlung miteinbezogen wird, werfen die Frage auf, woher die Inspiration kommen könnte und ob und in welcher Form es solche Vorgänge bereits in der Vergangenheit gab. Dieser Frage und derjenigen, ob sich Parallelen erkennen lassen, widmet sich der folgende Absatz.







\newpage

%% Parallelen zu historischen Formen von Rassismus
\section {Parallelen}
Anhand der bereits behandelten Textauszüge aus \glqq Harry Potter und die Heiligtümer des Todes\grqq{} lassen sich Parallelen zu historischen Formen des Rassismus im Dritten Reich herstellen.
Wie im Dritten Reich, gab es auch während der Gewaltherrschaft Voldemorts und der Todesser eine hierarchische Ordnung menschlicher \glqq Rassen\grqq. 
Die Nationalsozialisten erachteten die \glqq arische\grqq{} Rasse der Deutschen, die als besonders rein galt, gegenüber der \glqq jüdischen\grqq{} als überlegen. 
Ebenso lehnten sie Beziehung zwischen Deutschen und Juden kategorisch ab, da dies zu einer Verwässerung des \glqq deutschen\grqq{} Bluts führen würde.
Jene strikte Trennung zwischen Deutschen und Juden wurde in den \glqq Nürnberger Gesetzen\grqq{} aus dem Jahre 1935 von den Nationalsozialisten gesetzlich verankert\cite[S.\,72]{MW121}.
Eine ähnliche Haltung vertreten die dunklen Zauberer in Bezug auf Muggel und Muggelstämmige, woraus sich ein Bezugspunkt zur Eugenik der NS-Zeit, deren Ziel die \glqq Rassenhygiene\grqq{} der \glqq arischen\grqq{} Rasse war, ergibt.
Des Weiteren lässt sich feststellen, dass die Beschreibung der Übernahme des Zaubereiministeriums durch die Todesser und die folgenden \glqq Schauprozesse\grqq{} sich in ähnlicher Form während der nationalsozialistischen Diktatur wiederfinden lassen.
Diese Prozesse wurden vorrangig gegen politische Gegner, wie z.B. der Prozess gegen Claus Schenk Graf von Stauffenberg, geführt \footnote{http://www.bpb.de/izpb/10390/der-militaerische-widerstand?p=1}. 
In den Büchern lässt sich ein Bezug zwischen diesen Schauprozessen und ähnlichen Vorgängen der \glqq Registrierungskommission für Muggelstämmige\grqq{} ziehen. 
Auch hier zeigt sich in Kapitel 13, dass die Todesser Muggelstämmige in eben diesen Prozessen demütigen und vorführen.
Auch die Übernahme des Ministeriums selbst und der Versuch einer  vollständigen Umgestaltung der Gesellschaft hin zu einer rassistischen Diktatur, stellt eine literarische Beeinflussung durch historische Begebenheiten dar.
Denn nach der \glqq Machtergreifung\grqq{} der Nationalsozisalisten wurde die gesamte Gesellschaft im Sinne der NS-Ideologie zu einer \glqq Volksgemeinschaft\grqq{} umgestaltet. 
Bemerkenswert ist auch die Beschreibung, wie akribisch Voldemorts Gefolgsleute den \glqq Erhalt\grqq{} der reinblütigen Zauberer vorrantreiben.
Der Besuch der Hogwarts-Schule für Hexerei und Zauberei ist fortan nur Reinblütigen gestattet und auch vorherige Fächer, die sich mit Kenntnissen über Muggel beschäftigten, werden abgeschafft. 
Ähnlich wie die Nationalsozialisten in den Jahren 1933 bis 1945 versuchen auch Voldemort und die Todesser ihre rassistischen Ideale durch eine \glqq Gleichschaltung\grqq{} der Medien zu forcieren.
Dies zeigt sich ebenfalls in \cite[S.\,256]{JKR10}: \glqq Harry schlich näher heran, obwohl die Angestellten hier so konzentriert ihrer Arbeit nachgingen, dass sie seine vom Teppich gedämpften Schritte wohl kaum bemerken würden, und ließ ein fertige Broschüre von dem Stapel neben einer jungen Hexe gleiten. Unter dem Tarnumhang betrachtete er sie näher. Auf ihrem rosa Deckblatt prangte ein goldener Titel: SCHLAMMBLÜTER und die Gefahren, die sie für eine friedliche reinblütige Gesellschaft darstellen\grqq.
Eine weitere Parallele lässt sich auch in der im vorherigen Abschnitt beschriebenen Formulierung \glqq Magie ist Macht\grqq{} auf dem Marmorsockel im Zaubereiministerium ziehen, welche als Anspielung auf die Parolen  \glqq Arbeit macht frei\grqq{} oder \glqq Jedem das Seine\grqq{} interpretiert werden kann, welche die Nationalsozialisten über den Eingängen einiger Konzentrationslager anbringen ließen.
Sie dienten als zynische Beschreibung der \glqq Erziehung\grqq, zu der die Lager beitragen sollten.
Sowohl der Rassismus der Nationalsozialisten, als auch der in der \glqq Harry Potter\grqq{}-Reihe beschriebene, besaßen mit Adolf Hitler auf der einen Seite und Tom Riddle alias Lord Voldemort auf der anderen Seite wichtige Führungspersönlichkeiten.
Jedoch muss festgehalten werden, dass dem Personenkult um Adolf Hitler eine weit größere Bedeutung zuzuschreiben ist und dieser in Bezug auf Voldemort nicht von Rowling beschrieben wird.
Trotzdem verbindet beide Figuren die Tatsache, dass sie nicht nur von ihren Gegnern, sondern auch von ihren Unterstützern gefürchet wurden\cite[S.\,15]{JKR10}. 
Neben den untersuchten Gemeinsamkeiten lassen sich aber auch  Unterschiede zwischen der rassistischen Ideologie der NS-Zeit und dem Rassismus-Motiv von \glqq Harry Potter und die Heiligtümer des Todes\grqq{} erkennen.
Zum einen lässt sich sagen, dass die Wege zur Macht zwischen Voldemort und Hitler unterschiedlich aussehen.
Während Hitler zunächst versuchte, durch einen Putsch die Macht in Deutschland zu übernehmen, war er nach einem Gefängnisaufenthalt jedoch entschlossen, die Macht durch die Wahl zum Reichskanzler zu übernehmen\cite[S.\,1-99]{MW121}. 
Demgegenüber übernahm Voldemort und die Todesser gewaltsam die Kontrolle über die magische Welt\cite[S.\,13]{JKR10}. 
Hinzu kommt, dass Voldemort und seine Anhänger zwar Muggel diskriminieren und für minderwertig halten, es ihnen jedoch nur um die Erhaltung der Reinblüter und nicht wie den Nationalsozialisten um eine völlige Vernichtung der Juden geht\cite[S.43]{MW123}.
Daraus ergibt sich eine unterschiedliche Zielsetzung.
Ein weiterer Unterschied lässt sich darin finden, dass der Widerstand gegen die Machtübernahme Voldemorts als eine gemeinsame Gegenbewegung verstanden wird, während sich im Dritten Reich aus den verschiedensten Interessengruppen (Kirche, Politik, Kultur) Widerstand gegen die nationalsozialistische Diktatur formierte.
Alles in allem lässt sich feststellen, dass der Rassismus in \glqq Harry Potter\grqq{} starke Bezüge zum Dritten Reich besitzt, die sich in Parallelen zur Ideologie, Gleichschaltung und Entrechtung missliebiger Personengruppen zeigen.
Auch die Figur des Tom Riddle, der als Lord Voldemort die magische Welt in Angst versetzt, wird von Rowling als ein ähnlicher \glqq Führer\grqq{} wie Adolf Hitler gezeichnet, der schon ganz zu Beginn seine Grausamkeit gegen Muggel und ihre Helfer unterstreicht.
Daher lässt sich die Figur des Lord Voldemort durchaus als Hommage an den deutschen Diktator verstehen.
Rowling verstand es, die Ideologie der NS-Zeit aus britischer Sicht als Motiv in das Werk miteinfließen zu lassen, ohne dabei die Haupthandlung von Harry und seinen Freunden aus dem Blickfeld des Lesers verschwinden zu lassen. 
Die erste Hälfte des Buches zeigt sich klar durchzogen von der dunklen Seite der Magie, die jedoch in der zweiten Hälfte schlussendlich mit der Niederlage Voldemorts und der Todesser besiegt wird.

\newpage

%%
%% Literaturverzeichnis
%%
\newpage
\bibliography{harry_potter}
\bibliographystyle{dinat}

\newpage
\newgeometry{left=1.5cm,right=1.5cm,head=1.5cm,bottom=1.5cm}
\currentpdfbookmark{Erklärung}{pdf:declaration}
\thispagestyle{empty}
\vspace*{3cm}
\begin{center}
	\Large{\textbf{Erklärung}}
\end{center}
\vspace*{1cm}
Hiermit bestätige ich, dass die vorliegende Arbeit von mir selbstständig verfasst wurde und ich keine anderen als die angegebenen Hilfsmittel - insbesondere keine im Quellenverzeichnis nicht benannten Internet-Quellen - benutzt habe und die Arbeit von mir vorher nicht in einem anderen Prüfungsverfahren eingereicht wurde.
Die eingereichte schriftliche Fassung entspricht der auf dem elektronischen Speichermedium. (CD-ROM)
\vspace*{1cm}\\
Koblenz, den \today
\vspace*{0.75cm}\\
Martin Spoo
\restoregeometry


\end{document}
