\section {Parallelen}
Anhand der bereits behandelten Textauszüge aus \glqq Harry Potter und die Heiligtümer des Todes\grqq{} lassen sich Parallelen zu historischen Formen des Rassismus im Dritten Reich herstellen.
Wie im Dritten Reich, gab es auch während der Gewaltherrschaft Voldemorts und der Todesser eine hierarchische Ordnung menschlicher \glqq Rassen\grqq. 
Die Nationalsozialisten erachteten die \glqq arische\grqq{} Rasse der Deutschen, die als besonders rein galt, gegenüber der \glqq jüdischen\grqq{} als überlegen. 
Ebenso lehnten sie Beziehung zwischen Deutschen und Juden kategorisch ab, da dies zu einer Verwässerung des \glqq deutschen\grqq{} Bluts führen würde.
Jene strikte Trennung zwischen Deutschen und Juden wurde in den \glqq Nürnberger Gesetzen\grqq{} aus dem Jahre 1935 von den Nationalsozialisten gesetzlich verankert\cite[S.72]{MW121}.
Eine ähnliche Haltung vertreten die dunklen Zauberer in Bezug auf Muggel und Muggelstämmige, woraus sich ein Bezugspunkt zur Eugenik der NS-Zeit, deren Ziel die \glqq Rassenhygiene\grqq{} der \glqq arischen\grqq{} Rasse war, ergibt.
Des Weiteren lässt sich feststellen, dass die Beschreibung der Übernahme des Zaubereiministeriums durch die Todesser und die folgenden \glqq Schauprozesse\grqq{} sich in ähnlicher Form während der nationalsozialistischen Diktatur wiederfinden lassen.
Diese Prozesse wurden vorrangig gegen politische Gegner, wie z.B. der Prozess gegen Claus Schenk Graf von Stauffenberg, geführt \footnote{http://www.bpb.de/izpb/10390/der-militaerische-widerstand?p=1}. 
In den Büchern lässt sich ein Bezug zwischen diesen Schauprozessen und ähnlichen Vorgängen der \glqq Registrierungskommission für Muggelstämmige\grqq{} ziehen. 
Auch hier zeigt sich in Kapitel 13, dass die Todesser Muggelstämmige in eben diesen Prozessen demütigen und vorführen.
Auch die Übernahme des Ministeriums selbst und der Versuch einer  vollständigen Umgestaltung der Gesellschaft hin zu einer rassistischen Diktatur, stellt eine literarische Beeinflussung durch historische Begebenheiten dar.
Denn nach der \glqq Machtergreifung\grqq{} der Nationalsozisalisten wurde die gesamte Gesellschaft im Sinne der NS-Ideologie zu einer \glqq Volksgemeinschaft\grqq{} umgestaltet. 
Bemerkenswert ist auch die Beschreibung, wie akribisch Voldemorts Gefolgsleute den \glqq Erhalt\grqq{} der reinblütigen Zauberer vorrantreiben.
Der Besuch der Hogwarts-Schule für Hexerei und Zauberei ist fortan nur Reinblütigen gestattet und auch vorherige Fächer, die sich mit Kenntnissen über Muggel beschäftigten, werden abgeschafft. 
Ähnlich wie die Nationalsozialisten in den Jahren 1933 bis 1945 versuchen auch Voldemort und die Todesser ihre rassistischen Ideale durch eine \glqq Gleichschaltung\grqq{} der Medien zu forcieren.
Dies zeigt sich ebenfalls in \cite[S.256]{JKR10}: \glqq Harry schlich näher heran, obwohl die Angestellten hier so konzentriert ihrer Arbeit nachgingen, dass sie seine vom Teppich gedämpften Schritte wohl kaum bemerken würden, und ließ ein fertige Broschüre von dem Stapel neben einer jungen Hexe gleiten. Unter dem Tarnumhang betrachtete er sie näher. Auf ihrem rosa Deckblatt prangte ein goldener Titel: SCHLAMMBLÜTER und die Gefahren, die sie für eine friedliche reinblütige Gesellschaft darstellen\grqq.
Eine weitere Parallele lässt sich auch in der im vorherigen Abschnitt beschriebenen Formulierung \glqq Magie ist Macht\grqq{} auf dem Marmorsockel im Zaubereiministerium ziehen, welche als Anspielung auf die Parolen  \glqq Arbeit macht frei\grqq{} oder \glqq Jedem das Seine\grqq{} interpretiert werden kann, welche die Nationalsozialisten über den Eingängen einiger Konzentrationslager anbringen ließen.
Sie dienten als zynische Beschreibung der \glqq Erziehung\grqq, zu der die Lager beitragen sollten.
Sowohl der Rassismus der Nationalsozialisten, als auch der in der \glqq Harry Potter\grqq{}-Reihe beschriebene, besaßen mit Adolf Hitler auf der einen Seite und Tom Riddle alias Lord Voldemort auf der anderen Seite wichtige Führungspersönlichkeiten.
Jedoch muss festgehalten werden, dass dem Personenkult um Adolf Hitler eine weit größere Bedeutung zuzuschreiben ist und dieser in Bezug auf Voldemort nicht von Rowling beschrieben wird.
Trotzdem verbindet beide Figuren die Tatsache, dass sie nicht nur von ihren Gegnern, sondern auch von ihren Unterstützern gefürchet wurden\cite[S.15]{JKR10}. 
Neben den untersuchten Gemeinsamkeiten lassen sich aber auch  Unterschiede zwischen der rassistischen Ideologie der NS-Zeit und dem Rassismus-Motiv von \glqq Harry Potter und die Heiligtümer des Todes\grqq{} erkennen.
Zum einen lässt sich sagen, dass die Wege zur Macht zwischen Voldemort und Hitler unterschiedlich aussehen.
Während Hitler zunächst versuchte, durch einen Putsch die Macht in Deutschland zu übernehmen, war er nach einem Gefängnisaufenthalt jedoch entschlossen, die Macht durch die Wahl zum Reichskanzler zu übernehmen\cite[S.1-99]{MW121}. 
Demgegenüber übernahm Voldemort und die Todesser gewaltsam die Kontrolle über die magische Welt\cite[S.13]{JKR10}. 
Hinzu kommt, dass Voldemort und seine Anhänger zwar Muggel diskriminieren und für minderwertig halten, es ihnen jedoch nur um die Erhaltung der Reinblüter und nicht wie den Nationalsozialisten um eine völlige Vernichtung der Juden geht\cite[S.43]{MW123}.
Daraus ergibt sich eine unterschiedliche Zielsetzung.
Ein weiterer Unterschied lässt sich darin finden, dass der Widerstand gegen die Machtübernahme Voldemorts als eine gemeinsame Gegenbewegung verstanden wird, während sich im Dritten Reich aus den verschiedensten Interessengruppen (Kirche, Politik, Kultur) Widerstand gegen die nationalsozialistische Diktatur formierte.
Alles in allem lässt sich feststellen, dass der Rassismus in \glqq Harry Potter\grqq{} starke Bezüge zum Dritten Reich besitzt, die sich in Parallelen zur Ideologie, Gleichschaltung und Entrechtung missliebiger Personengruppen zeigen.
Auch die Figur des Tom Riddle, der als Lord Voldemort die magische Welt in Angst versetzt, wird von Rowling als ein ähnlicher \glqq Führer\grqq{} wie Adolf Hitler gezeichnet, der schon ganz zu Beginn seine Grausamkeit gegen Muggel und ihre Helfer unterstreicht.
Daher lässt sich die Figur des Lord Voldemort durchaus als Hommage an den deutschen Diktator verstehen.
Rowling verstand es, die Ideologie der NS-Zeit aus britischer Sicht als Motiv in das Werk miteinfließen zu lassen, ohne dabei die Haupthandlung von Harry und seinen Freunden aus dem Blickfeld des Lesers verschwinden zu lassen. 
Die erste Hälfte des Buches zeigt sich klar durchzogen von der dunklen Seite der Magie, die jedoch in der zweiten Hälfte schlussendlich mit der Niederlage Voldemorts und der Todesser besiegt wird.
