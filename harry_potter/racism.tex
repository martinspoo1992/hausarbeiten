\section{Rassismus in der magischen Welt}
Die rassistische Ideologie, die in der \glqq Harry Potter\grqq-Reihe beschrieben und von dunklen Magiern in der magischen Welt auch nach außen propagiert wird, fußt auf der Annahme, dass die Fähigkeit, Magie anwenden zu können, auf der familiären Abstammung beruht. Daher werden auch nur Magier, die ausschließlich magische Menschen als Vorfahren haben, als würdig angesehen, ihre Fähigkeiten nutzen zu lernen. Unter den Nachfolgern Lord Voldemorts, des bösesten Zauberers, geht die Ablehnung von Muggeln sogar so weit, dass sie die Verwandtschaft und den Kontakt zu Familienmitgliedern, die Kontakt mit \glqq unerwünschten\grqq Kreaturen haben, leugnen. Dies wird deutlich bei einer Konferenz der Todesser, bei der sie sich über die Gefangennahme von Harry Potter beraten, deutlich: \glqq Wir – Narzissa und ich (Bellatrix Lestrange d. Red.) – haben unsere Schwester nicht mehr zu Gesicht bekommen, seit sie den Schlammblüter geheiratet hat. Diese Göre hat mit keiner von uns etwas zu tun, ebenso wenig wie irgendein Biest, das sie heiratet.\cite[S.18]{JKR10}\grqq
Die Todesser lehnen jeglichen Kontakt zu Muggeln und Muggelstämmigen ab, da sie diese als minderwertig erachten. Voldemort selbst spricht davon, man müsse \glqq das Krebsgeschwür wegschneiden, das uns verseucht, bis nur noch die von wahrem Blut zurückbleiben.\cite[S.19]{JKR10}\grqq Gegen Ende des ersten Kapitels zeigt sich die besondere Grausamkeit Voldemorts und seiner Anhänger. Der gefangen genommene Professorin für \glqq Muggelkunde\grqq Charity Burbage, die über der Gruppe von Todessern an der Decke zur Schau gestellt wurde, werden \glqq Verbrechen\grqq wie Aufklärung über Muggel vorgeworfen. Sie habe nicht nur \glqq den Verstand von Zaubererkindern\grqq verdorben und besudelt, sondern auch behauptet, die schwindende Zahl von Reinblütern sei ein \glqq wünschenswertes Phänomen\grqq (\cite[S.19]{JKR10}). Daraufhin wird sie mit einem Todesfluch getötet.
Als die Todesser im weiteren Verlauf der Geschichte auch das zentrale Organ für magische Angelegenheite, das sogenannte Zaubereiministerium, unter ihre Kontrolle bringen, wird die rassistische Ideologie bereits in der Eingangshalle klar gemacht. Auf Seite 249 beschreibt Joanne K. Rowling ein schwarzes steinernes Denkmal, dass einen Zauberer und eine Hexe zeigt, die auf Thronen sitzen.\cite[S.249]{JKR10} Auf dem Sockel, der von stilisierten Muggeln getragen wird, sind die Worte \glqq Magie ist Macht\grqq eingraviert. In Kapitel 13 beschreibt Rowling die Einrichtung einer \glqq Registrierungskommission für Muggelstämmige\grqq , deren Aufgabe es ist, alle Muggelstämmigen zu listen und ihre Blutsherkunft zu überprüfen. Zudem müssen sie sich einem Verhör unterziehen und nachweisen, welchem Zauberer sie ihren Zauberstab entwendet haben\cite [S.268]{JKR10}. Daran lassen sich die fragwürdigen Methoden der Todesser erkennen, mit denen sie Muggel und Muggelgeborene entrechten und bloßstellen. 
Die Formen des Rassismus, wie er von Rowling in ihrem Buch in die Handlung miteinbezogen wird, werfen die Frage auf, woher die Inspiration kommen könnte und ob und in welcher Form es solche Vorgänge bereits in der Vergangenheit gab. Dieser Frage und derjenigen, ob sich Parallelen erkennen lassen, widmet sich der folgende Absatz.






