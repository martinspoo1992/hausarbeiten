\section{Definitionen}
Um den Einstieg in das Thema \glqq Rassismus-Motiv der  \glq Harry Potter\grq-Reihe\grqq{} zu erleichtern, soll im folgenden Abschnitt exemplarisch auf wichtige Begriffe wie \glqq Muggel\grqq, \glqq Halbblüter\grqq{} und das rassistische Pejorativum \glqq Schlammblut\grqq{} eingegangen werden.
Unter \glqq Muggeln\grqq{} (engl. muggles) werden in den Romanen von Joanne K. Rowling Menschen ohne magische Begabung verstanden, wie in \cite[S. 61]{JKR97} zu lesen ist. Abgeleitet wird der Begriff vom englischen Wort mug, der sich umgangssprachlich mit \glqq jmd. abziehen\grqq{} übersetzen lässt. Mit diesem Ausdruck wird deutlich gemacht, dass \glqq normale\grqq{} Menschen von magischen Wesen als hilflos oder auch dümmlich angesehen werden und mit diesem herabsetzenden Ausdruck von magischen Wesen abgegrenzt werden sollen. Muggel sind im Gegensatz zu magischen Wesen nicht in der Lage, bestimmte Kreaturen der magischen Welt zu sehen.
Von Muggeln zu unterscheiden sind sogenannte \glqq Halbblüter\grqq. Hat ein Zauberer sowohl ein Elternteil ohne Zauberkräfte, als auch eines mit Zauberkräften, ist das Kind folglich halb magischen \glqq Blutes\grqq, wenn sich die Zauberkräfte entwickeln. Zu ihnen zählt auch der Protagonist Harry Potter, da seine Eltern der Zauberer James Potter und die Muggelstämmige Hexe Lily Potter sind.
Halbblüter werden häufig von Zauberern, die rein magische Eltern haben, daher \glqq Reinblüter\grqq{} genannt, als minderwertiger angesehen (\cite[S.220]{JKR03}). Hat ein Zauberer nur Muggel als Eltern, wie es beispielsweise bei Harry Potters bester Freundin Hermine Granger der Fall ist, verwenden besonders rassistisch eingestellte reinblütige Zaubererfamilien das Schimpfwort \glqq Schlammblut\grqq, das auf die - aus ihrer Sicht -  \glqq unreine\grqq{} Verwandtschaft mit Muggeln verweist.
Jedoch zeigt sich anhand des Beispiels von Hermine Granger, dass die Verwandtschaft mit Muggeln keineswegs Einfluss auf die magischen Fähigkeiten hat, da sie als eine der begabtesten Schülerinnen von Hogwarts gilt und eine rassisch bedingte Begabung ausgeschlossen ist.
Tatsächlich wird in den Büchern jedoch deutlich, dass die meisten Zauberer der Auffassung sind, dass die Fahigkeit, Magie einzusetzen, ein Talent ist, dass nur begrenzt erblich ist und auch durch Menschen erlernt werden kann, die ausschließlich nichtmagische Vorfahren haben. 

