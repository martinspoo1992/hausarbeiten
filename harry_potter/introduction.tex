\section{Einleitung} 
Die folgende Hausarbeit setzt sich mit dem Rassismus-Motiv der \glqq Harry Potter\grqq{}-Reihe von Joanne K. Rowling am Beispiel des Bandes \glqq Harry Potter und die Heiligtümer des Todes\grqq{} auseinander.
In der Betrachtung soll dabei zunächst auf wichtige Termini der Bücher in Bezug auf Rassismus eingegangen werden. 
Im weiteren Verlauf soll der Rassismus in der magischen Welt beleuchtet und an ausgewählten Textpassagen des siebten Bandes der Reihe mit dem Titel \glqq Harry Potter und die Heiligtümer des Todes\grqq{} belegt werden. 
Besonderes Augenmerk soll dabei auf der Figur des Tom Riddle alias Lord Voldemort und seiner Bedeutung als Führer der Todesser liegen.
Schlussendlich soll untersucht werden, ob und inwiefern sich Gemeinsamkeiten und Unterschiede zu historischen Entwicklungen in Bezug auf Rassismus ergeben. 
Dies soll am Beispiel des Nationalsozialismus deutlich gemacht werden. 
Dabei soll auch die Frage geklärt werden, ob die Figur des Lord Voldemort als  literarische Darstellung von Adolf Hitler dienen kann. 

Die Wahl des Themas ergab sich aus dem Interesse, das Thema \glqq Nationalsozialismus\grqq{} in den Kontext der Literatur zu bringen. 
Dabei bot sich die \glqq Harry Potter\grqq-Reihe aufgrund ihrer Themenvielfalt und Tiefe an. 
Des Weiteren übten die Werke auch auf mich persönlich eine große Faszination aus. 

Die \glqq Harry Potter\grqq-Reihe, bestehend aus 7 einzelnen Bänden, ist mit über 500 Millionen weltweit verkaufter Exemplare eine der erfolgreichsten literarischen Reihen. 
Literarisch lässt sie sich der Kinder- und Jugendliteratur, aber auch dem Genre der Fantasyliteratur zuordnen. 
Allerdings erfreuen sich die Bücher nicht nur bei Kindern und Jugendlichen großer Beliebtheit, sondern werden auch von Erwachsenen gelesen.
 Zudem bilden die 8 Verfilmungen der Bücher – ergänzend ist zu sagen, dass der siebte Band in zwei Filmen umgesetzt wurde – mit ca. 8 Mrd. US-\$ die finanziell erfolgreichste Literaturverfilmungsreihe aller Zeiten.

\glqq Harry Potter und die Heiligtümer des Todes\grqq{} schließt die Geschichte um den Zauberlehrling Harry Potter, der mit seinen Freunden Ron Weasley und Hermine Granger gegen den bösen Zauberer Tom Riddle, der sich selbst nur \glqq Lord Voldemort\grqq{} nennt, kämpft, ab.
Die Geschichte knüpft nahtlos an den vorherigen Band \glqq Harry Potter und der Halbblutprinz\grqq{} an, in dem der Schulleiter der Hogwarts-Schule für Hexerei und Zauberei, Albus Dumbledore, versucht, die magischen Gegenstände, in die Lord Voldemort Teile seiner Seele versiegelt hat, zu zerstören. 
Jedoch wird Dumbledore am Ende des 6.\,Bandes getötet, woraufhin sich Harry, Ron und Hermine allein auf die Suche nach den verbliebenen Gegenständen machen. Lord Voldemort und seine Anhänger, die Todesser, haben unterdessen das mächtigste politische Organ der Zaubererwelt, das Zaubereiministerium, unter ihre Kontrolle gebracht und begonnen, eine rassistisch motivierte \glqq Auslese\grqq{} denjenigen gegenüber vorzunehmen, die nach ihrer Auffassung nicht würdig sind, magische Fähigkeiten zu erlernen. 
Schlussendlich gelingt es Harry, Ron und Hermine, die verbliebenen, Horkruxe genannte, Teile von Voldemorts Seele zu zerstören und ihn und seine Helfer in der \glqq Schlacht um Hogwarts\grqq{} zu besiegen.
