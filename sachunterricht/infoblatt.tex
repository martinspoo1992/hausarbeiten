\section{Infoblatt zum Experiment}

\begin{itemize}
	\item{Themenbereich: Wasser}
	\item{Klassenstufe: 4a, Grundschule Koblenz-Moselweiß}
	\item{Arbeitstitel: \glqq Das Schmelzen von Eis \grqq{}}
	\item{Material: Glasschalen, Wasser, Eiswürfel}
	\item{Kurzbeschreibung des Versuchs: In eine Glasschale wird kaltes Wasser und ein Eiswürfel gegeben. In eine zweite Glasschale wird ein Glas mit der Öffnung nach unten platziert und die Glasschale mit Wasser gefüllt. Anschließend werden auf dem Glas Eiswürfel platziert. Die Schüler markieren den Wasserstand mit einem Filzstift und beobachten die Veränderung des Wasserstandes.}
	\item{Mögliche Sicherheitshinweise: Vorsichtig sein mit den Eiswürfeln. Diese sollten nicht zu lange in der Hand gehalten werden.}
	\item{Ablauf: Sitzkreis, Fragerunde zum Thema Wasser und Eis als Impuls}
	\item{in der Mitte Materialien, die verwendet werden; Befragung der Schüler, ob sie Vorstellungen haben, was bei dem Experiment passieren könnte}
	\item{danach finden sich die Schüler in Gruppen zusammen, der Versuchsablauf wird erklärt und die Schüler experimentieren selbstständig; Ergebnisse werden in das Forscherheft eingetragen}
	\item{Abschluss des Experimentes erneut im Sitzkreis, wo sich die Schüler über ihre Ergebnisse austauschen sollen}
	\item{Erklärung der Beobachtung an der Tafel}
	\item{mögliche Fragen für die Fragerunde:
		\begin{itemize}
		\item{Wo gibt es Eis auf der Erde?}
		\item{Warum gibt es Eisgletscher auf der Erde?}
		\item{Wie ändert sich der Wasserstand, wenn Eis schmilzt?}
		\item{Was passiert mit Wasser, wenn es gefriert?}
		\item{Welche Folgen könnte es haben, wenn das Eis der Gletscher schmilzt?}
		
		\end{itemize}
		}
\end{itemize}
