\section{Einleitung} 
Die folgende Hausarbeit widmet sich dem Thema \glqq Das Schmelzen von Eis\grqq{} und soll einen Einblick in den Sachunterricht geben.
Als Grundlage dient dabei ein Experiment, das am 25.06.2014 mit einer Gruppe von Viertklässlern an der Grundschule Koblenz-Moselweiß durchgeführt wurde.
Zunächst sollen das Experiment selbst und seine Durchführung vorgestellt werden.
Einen wichtigen Teil soll dabei auch die Darstellung der physikalischen Abläufe einnehmen.
Anschließend sollen die Erfahrungen, die bei der Durchführung seitens des Studierenden gemacht wurden, behandelt werden.
Außerdem soll das Experiment exemplarisch in zwei Perspektiven des Perspektivrahmens Sachunterricht eingeordnet werden.

Der Sachunterricht nimmt im Unterrichtsspektrum der Grundschule eine besondere Rolle ein, da er fünf Perspektiven in sich vereint: die sozial- und kulturwissenschaftliche, die raumbezogene, die naturbezogene, die historische und die technische Perspektive.
Diese werden im Perspektivrahmen Sachunterricht der Gesellschaft für Didaktik des Sachunterrichts (GDSU) benannt.
Laut diesem Perspektivrahmen hat der Sachunterricht die Aufgabe, Schülerinnen und Schüler dabei zu unterstützen, \glqq  sich in ihrer Umwelt zurechtzufinden, diese angemessen zu verstehen und mitzugestalten, systematisch und reflektiert zu lernen und Voraussetzungen für späteres Lernen zu erwerben\grqq{}{\cite[S.\,2]{GPS02}}.
Daraus resultiert für die Lehrperson, dass sie ein fundiertes Hintergrundwissen benötigt, um auf die Fragen der Schülerinnen und Schüler fachlich genau eingehen zu können.
Hinzu kommt, dass der Sachunterricht durch seine breit gefächerten Perspektiven eine Vielzahl von Themen abdeckt und so die Grundlagen für den naturwissenschaftlichen, kulturwissenschaftlichen, technischen oder auch geschichtlichen Unterricht an weiterführenden Schulen schafft.
Im folgenden Abschnitt soll das durchgeführte Experiment vorgestellt und die wissenschaftlichen Grundlagen der Beobachtung erläutert werden.
