\section{Reflexion des Experimentes}
In der Nachbereitung des Experimentes zeigte sich, dass die Experimente vom Aufbau her zwar recht einfach, vom physikalischen Inhalt her aber durchaus anspruchsvoll waren.
So mussten zunächst Fachbegriffe wie \emph{Volumen} und \emph{Dichte}, aber auch die molekulare Struktur des Wassers verstanden sein, um sie den Schülern didaktisch vermitteln zu können.
Folglich erforderte die Unterrichtseinheit physikalischen und chemischen Sachverstand des Studenten.
Hinzu kamen die Abläufe der Eisschmelze bei Gletschern und die Unterscheidung zwischen schwimmendem Eis und vereisten Landflächen.

Eine weitere Herausforderung bestand in der Einbettung des Experimentes in eine Form der Unterrichtsreihe, welche sich schlussendlich mit der Eisschmelze, die durch den Klimawandel mitverursacht wird, beschäftigte.
Dazu wurden im Vorfeld der Planung Fragen erarbeitet, die als Einleitung eines Unterrichtsgespräches dienten.
Mit dem Unterrichtsgespräch sollte gleichzeitig abgesteckt werden, welchen Wissenstand die Schülerinnen und Schüler bereits haben und welche Lücken vorhanden sind.
Bei der Erarbeitung dieser Fragen wurde darauf geachtet, dass sie zu Beginn sammelnden Charakter hatten und mit dem Verlauf der Unterrichtseinheit immer präziser in Richtung des Zielthemas führen sollten.
An dieser Stelle zeigte sich während der Unterrichtseinheit, dass die Schülerinnen und Schüler über ein fundiertes Basiswissen verfügten, was den Umgang mit dem Thema vereinfachte.

Während der Durchführung des Experimentes zeigten sich alltägliche Schwierigkeiten, die auch in späteren Unterrichtseinheiten des Sachunterrichtes vorkommen können und deshalb eine genaue Planung des Unterrichts notwendig machen.
Es zeigte sich, dass sich der Schmelzvorgang der Eiswürfel ohne Zuhilfenahme technischer Hilfsmittel aufgrund der an diesem Tag niedrigen Lufttemperatur als zeitlich ausgedehnt darstellte.
Deshalb ging man im Laufe der Unterrichtsstunde dazu über, diesen Prozess mithilfe von Heizstrahlern zu beschleunigen.
Zudem war es hilfreich, wie sich an diesem Experiment beispielhaft zeigte, ausreichend Material wie z.B. Eiswürfel bereitstellen zu können, um allen Schülerinnen und Schülern die aktive Teilnahme am Experiment zu ermöglichen.
Ergänzend ließ sich im Nachhinein attestieren, dass eine gute zeitliche Planung für eine Doppelstunde vorgenommen werden sollte, um zu vermeiden, Zeit überbrücken zu müssen.

Alles in allem lässt sich als Ergebnis festhalten, dass die Durchführung und Einbettung des Experimentes in eine Unterrichtsreihe erfolgreich war.
Das nötige Hintergrundwissen, um die vorliegenden wissenschaftlichen Sachverhalte darzustellen, war vorhanden.
Auch der Umgang mit den Schülern im Unterrichtsgespräch und während der Experimentierphase verlief reibungslos.
Allerdings stieß man aufgrund der fehlenden Erfahrung mit der didaktischen Umsetzung von Inhalten im Sachunterricht an individuelle Grenzen, die das erstmalige Handeln als verantwortliche Lehrperson erschwerten.

Im folgenden Kapitel soll das Experiment in einem weiteren Schritt in den Perspektivrahmen des Sachunterrichts eingeordnet werden.
