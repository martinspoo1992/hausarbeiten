\section{Versuchsbeschreibung \glqq Das Schmelzen von Eis\grqq{}}
Im Rahmen des Seminars \glqq Beobachten im Sachunterricht am Beispiel der Lernwerkstatt\grqq{} wurden zu Beginn des Semesters Gruppen zu je zwei Studierenden, sogenannte Tandems, gebildet, die dann eigenständig Experimente zum Thema \glqq Wasser\grqq{} oder \glqq Luft\grqq{} ausarbeiten sollten, welche dann in einer 3. Klasse und einer 4. Klasse der Grundschule Koblenz-Moselweiß durchgeführt wurden.
Im Rahmen des Themas \glqq Wasser\grqq{} wurde schließlich das Thema \glqq Eisschmelze und ihr Einfluss auf den Wasserspiegel\grqq{} ausgesucht.
Ziel war es, den Schülerinnen und Schülern die Folgen des Klimawandels anhand eines einfachen Modells der Gletscherschmelze näherzubringen.
Als Material benötigte man Glasschalen, Gläser, Wasser und Eiswürfel.

Das Experiment bestand darin, dass Glasschalen mit Wasser gefüllt wurden und der Wasserstand mit einem Filzstift markiert wurde.
Daraufhin wurden in die Schalen einige Eiswürfel gegeben.
Anschließend beobachtete man das Experiment, bis die Eiswürfel geschmolzen waren und markierte den neuen Wasserspiegel mit einem Filzstift.
In einem weiteren Schritt wurde ein Glas mit dem Boden nach oben in eine Schale gestellt.
Auf das Glas wurden Eiswürfel gelegt und in die Schale Wasser gefüllt.
Zum Schluss wurden die gesammelten Ergebnisse in ein sogennantes \glqq Forscherheft\grqq{} eingetragen und die vor dem Experiment geäußerten Vermutungen mit den Beobachtungen des Experimentes verglichen.

Es ließ sich beobachten, dass der Wasserspiegel gestiegen war, nachdem die Eiswürfel geschmolzen waren.
Hier zeigt sich die Dichteanomalie des Wassers.
Anders als bei anderen Stoffen erhöht sich bei Wasser die Dichte zwischen 0 und ca. 4 Grad Celsius.
Das hat zur Folge, dass der Wasserstand zunächst fallen müsste und wieder steigt, sobald die Wassertemperatur 4 Grad Celsius überschritten hat.
Dadurch, dass das Wasser nicht gleichmäßig umgerührt wurde, ist der Effekt nicht ausgeprägt genug, um beobachtet zu werden.
Im zweiten Teil des Experimentes zeigte sich der gleiche Effekt.
Die Eiswürfel, die sich auf dem Glas befanden, wechselten im Laufe der Zeit vom festen in den flüssigen Aggregatszustand.
Dadurch füllte sich die Vertiefung des Glases, in dem die Eiswürfel lagen, mit Schmelzwasser, das nach einer gewissen Zeit über den Rand des Glases in die Glasschale ablief.
Dieses Phänomen begründet sich dadurch, dass eine Menge Wasser mehr Volumen einnimmt, als die gleiche Menge an gefrorenem Wasser, wenn die Temperatur nicht größer als 4 Grad Celsius ist.
Schmilzt das Eis, so wechselt es vom festen in den flüssigen Aggregatszustand und das Volumen des Wassers im Glas wird größer.
Dadurch steigt der Wasserspiegel an.

Das Experiment liefert eine einfache Darstellung der Vorgänge der Gletscherschmelze, wobei sich der erste Teil auf schwimmende Eisflächen und sich der zweite auf gefrorene Landmassen bezieht.
Ergänzend lässt sich dazu sagen, dass der Anstieg des Meerespiegels bei schmelzenden schwimmenden Eismassen, deren größter Anteil aufgrund der Auftriebskräfte unterhalb der Wasseroberfläche liegt, deutlich geringer ausfällt als bei Eismassen, die an Land schmelzen.
Das liegt daran, dass schwimmende Körper soviel Wasser verdrängen, wie ihrer Gewichtskraft entspricht.
Daraus resultiert auch, dass ein Glas Wasser, in dem sich Eiswürfel befinden, zunächst nicht überläuft, da die Eiswürfel nur so viel Schmelzwasser erzeugen, wie sie zuvor verdrängt haben.
Dieses Phänomen geht auf den griechischen Gelehrten Archimedes zurück und wird daher \emph{archimedisches Prinzip} genannt.

