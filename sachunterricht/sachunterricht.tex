\RequirePackage{ifvtex}
\documentclass[12pt,2headlines,a4paper,oneside,bibtotoc]{scrartcl}

\usepackage{ngerman}
\usepackage[utf8]{inputenc}
\usepackage[T1]{fontenc}
\usepackage{setspace}
\usepackage[left=40mm,top=25mm,right=30mm,bottom=35mm,nohead,nofoot]{geometry}
\usepackage[font=small,format=plain,labelfont=bf,up,textfont=it,up]{caption}
\usepackage{scrpage2}
\usepackage{graphicx}
\usepackage{tikz}
\usepackage[breaklinks=true]{hyperref}
\usepackage{float}
\usepackage{longtable}
\usepackage[numbers]{natbib}
%\usepackage{BeamerColor}
%\usepackage{soul}

\setcounter{LTchunksize}{100}

\usetikzlibrary{shapes,arrows}

\pagestyle{scrheadings}
\ihead{\headmark}
\ohead{Martin Spoo}
\ifoot{Beobachten im Sachunterricht am Beispiel der Lernwerkstatt}
\cfoot{}
\ofoot{\pagemark}
\setheadsepline{0.4pt}
\setfootsepline{0.4pt}
\addtolength{\headsep}{10mm}
\addtolength{\footskip}{10mm}
\setlength{\headheight}{1.1\baselineskip}

%\bibliographystyle{plainde}

\renewcommand{\figurename}{Abb.}

\begin{document}

%% Titelseite
\thispagestyle{empty}
\newgeometry{inner=1.5cm,outer=1.5cm,head=1.5cm,bottom=1.5cm}
\begin{titlepage}
	\subpdfbookmark{Titel}{pdf:title}
	\begin{center}
		\quad
		\vfill
		\Huge{
			\setstretch{2} \textbf{Philosophische Vorstellungen von Kindern zum Thema \glqq Glück\grqq}
		}
		\vspace{5mm}
		\vfill
		\large{
			{\bfseries Masterarbeit\\
			\vspace{5mm}
			\normalfont \rmfamily zur Erlangung des akademischen Grades\\
			\bfseries Master of Education (M. Ed.)\\
			\vspace{5mm}
			\normalfont \rmfamily im Studiengang Grundschulbildung}
		}
		\\
		\vspace{1.5cm}
		\large{
			{vorgelegt von\\
			Martin Spoo\\
			Matrikelnr.\,212100872}
		}
		\vspace{1cm}
		\\
		\Large{
			{Universität Koblenz-Landau}\\
			{SS\, 2016}
		}
		\vspace{1cm}
		\begin{table}[b]
			\begin{center}
				\begin{tabular}{lr}
					Prüfer: &  Prof. Dr. Heike de Boer , \\
								&	Institut für Grundschulbildung, Campus Koblenz \\
					Zweitprüfer: & Dr. Nicole Henrich \\
					\vspace{0.25cm} \\
					Abgabetermin: & 24. August 2016 \\
					Datum: & \today
				\end{tabular}
			\end{center}
		\end{table}
	\end{center}
\end{titlepage}
\renewcommand{\baselinestretch}{1.1}
\restoregeometry



\newpage

%% Seitenzähler nach dem Titelblatt auf 1 setzen
\setcounter{page}{1}

%% Inhaltsverzeichnis
\tableofcontents
\newpage

%% Zeilenabstand auf 1,5fach
\onehalfspacing

%% Einleitung
\section{Einleitung}
In der folgenden Hausarbeit soll ein synoptischer Vergleich der biblischen Textstelle Mk 15,38-41 durchgeführt werden. Dabei liegt das besonderere Augenmerk auf der Untersuchung der Evangelien auf inhaltliche und sprachliche Differenzen. Um einen Einstieg zu ermöglichen, will ich die Methode des synoptischen Vergleiches kurz erläutern.
Beim synoptischen Vergleich handelt es sich um eine Untersuchungsmethode, bei der vergleichbare Textpassagen der synoptischen Evangelien tabellarisch nebeneinander gestellt werden. In dieser Hausarbeit wird es sich dabei um die zu Beginn genannte Passage des Markusevangeliums sowie Lk 23,45-49 und Mt 27,51-56 handeln.
Der synoptische Vergleich bietet uns die Möglichkeit, eine Textstelle aus den synoptischen Evangelien nach Markus, Lukas und Matthäus näher zu untersuchen. Nach der Zweiquellentheorie geht man davon aus, dass das Markusevangelium als erstes entstand und als Vorlage für Matthäus und Lukas diente. Diese haben sich wiederum zusätzlich noch einer Logienquelle Q und ihres jeweiligen Sondergutes bedient.
Als Forschungsfrage leitet sich daher ab: „Inwiefern kann man die Zweiquellentheorie anhand der Textpassage Mk 15,38-41 bestätigen?“
Die Wahl des Themas der Passion für einen synoptischen Vergleich ergab sich daraus, dass mir die Passion bereits aus dem Religionsunterricht bekannt ist  und sie zu einem der Themen der Bibel gehört, die mich am meisten interessieren. Da die Größe der Textstelle auf vier Verse vorgegeben war, musste eine Stelle gewählt werden, die inhaltlich und sprachlich genug Untersuchungspunkte bietet. Des Weiteren lässt sich anhand der Leidensgeschichte Jesu gut darstellen, worauf die Evangelisten den inhaltlichen Schwerpunkt in ihren Evangelien legten, um ihre jeweiligen Zielgruppen anzusprechen. 
\newpage

%% Holocaust als Unterrichtsthema 
\section{Versuchsbeschreibung \glqq Das Schmelzen von Eis\grqq{}}
Im Rahmen des Seminars \glqq Beobachten im Sachunterricht am Beispiel der Lernwerkstatt\grqq{} wurden zu Beginn des Sommersemesters 2014 Gruppen zu je zwei Studierenden, sogenannte Tandems, gebildet, die dann eigenständig Experimente zum Thema \glqq Wasser\grqq{} oder \glqq Luft\grqq{} ausarbeiten sollten, welche dann in einer 3. Klasse und einer 4. Klasse der Grundschule Koblenz-Moselweiß durchgeführt wurden.
Im Rahmen des Themas \glqq Wasser\grqq{} wurde schließlich das Thema \glqq Eisschmelze und ihr Einfluss auf den Wasserspiegel\grqq{} ausgesucht.
Ziel war es, den Schülerinnen und Schülern die Folgen des Klimawandels anhand eines einfachen Modells der Gletscherschmelze näherzubringen.
Als Material benötigte man Glasschalen, Gläser, Wasser und Eiswürfel.

Das Experiment bestand darin, dass Glasschalen mit Wasser gefüllt wurden und der Wasserstand mit einem Filzstift markiert wurde.
Daraufhin wurden in die Schalen einige Eiswürfel gegeben.
Anschließend beobachtete man das Experiment, bis die Eiswürfel geschmolzen waren und markierte den neuen Wasserspiegel mit einem Filzstift.
In einem weiteren Schritt wurde ein Glas mit dem Boden nach oben in eine Schale gestellt.
Auf das Glas wurden Eiswürfel gelegt und in die Schale Wasser gefüllt.
Zum Schluss wurden die gesammelten Ergebnisse in ein sogennantes \glqq Forscherheft\grqq{} eingetragen und die vor dem Experiment geäußerten Vermutungen mit den Beobachtungen des Experimentes verglichen.

Es ließ sich beobachten, dass der Wasserspiegel gestiegen war, nachdem die Eiswürfel geschmolzen waren.
Hier zeigt sich die Dichteanomalie des Wassers.
Anders als bei anderen Stoffen erhöht sich bei Wasser die Dichte zwischen 0 und ca. 4 Grad Celsius.
Das hat zur Folge, dass der Wasserstand zunächst fallen müsste und wieder steigt, sobald die Wassertemperatur 4 Grad Celsius überschritten hat.
Dadurch, dass das Wasser nicht gleichmäßig umgerührt wurde, ist der Effekt nicht ausgeprägt genug, um beobachtet zu werden.
Im zweiten Teil des Experimentes zeigte sich der gleiche Effekt.
Die Eiswürfel, die sich auf dem Glas befanden, wechselten im Laufe der Zeit vom festen in den flüssigen Aggregatszustand.
Dadurch füllte sich die Vertiefung des Glases, in dem die Eiswürfel lagen, mit Schmelzwasser, das nach einer gewissen Zeit über den Rand des Glases in die Glasschale ablief.
Dieses Phänomen begründet sich dadurch, dass eine Menge Wasser mehr Volumen einnimmt, als die gleiche Menge an gefrorenem Wasser, wenn die Temperatur nicht größer als 4 Grad Celsius ist.
Schmilzt das Eis, so wechselt es vom festen in den flüssigen Aggregatszustand und das Volumen des Wassers im Glas wird größer.
Dadurch steigt der Wasserspiegel an.

Das Experiment liefert eine einfache Darstellung der Vorgänge der Gletscherschmelze, wobei sich der erste Teil auf schwimmende Eisflächen und sich der zweite auf gefrorene Landmassen bezieht.
Ergänzend lässt sich dazu sagen, dass der Anstieg des Meerespiegels bei schmelzenden schwimmenden Eismassen, deren größter Anteil aufgrund der Auftriebskräfte unterhalb der Wasseroberfläche liegt, deutlich geringer ausfällt als bei Eismassen, die an Land schmelzen.
Das liegt daran, dass schwimmende Körper soviel Wasser verdrängen, wie ihrer Gewichtskraft entspricht.
Daraus resultiert auch, dass ein Glas Wasser, in dem sich Eiswürfel befinden, zunächst nicht überläuft, da die Eiswürfel nur so viel Schmelzwasser erzeugen, wie sie zuvor verdrängt haben.
Dieses Phänomen geht auf den griechischen Gelehrten Archimedes zurück und wird daher \emph{archimedisches Prinzip} genannt.



%% Reflexion 
\newpage
\section{Reflexion des Experimentes}
In der Nachbereitung des Experimentes zeigte sich, dass die Experimente vom Aufbau her zwar recht einfach, vom physikalischen Inhalt her aber durchaus anspruchsvoll waren.
So mussten zunächst Fachbegriffe wie \emph{Volumen} und \emph{Dichte}, aber auch die molekulare Struktur des Wassers verstanden sein, um sie den Schülern didaktisch vermitteln zu können.
Folglich erforderte die Unterrichtseinheit physikalischen und chemischen Sachverstand des Studenten.
Hinzu kamen die Abläufe der Eisschmelze bei Gletschern und die Unterscheidung zwischen schwimmendem Eis und vereisten Landflächen.

Eine weitere Herausforderung bestand in der Einbettung des Experimentes in eine Form der Unterrichtsreihe, welche sich schlussendlich mit der Eisschmelze, die durch den Klimawandel mitverursacht wird, beschäftigte.
Dazu wurden im Vorfeld der Planung Fragen erarbeitet, die als Einleitung eines Unterrichtsgespräches dienten.
Mit dem Unterrichtsgespräch sollte gleichzeitig abgesteckt werden, welchen Wissenstand die Schülerinnen und Schüler bereits haben und welche Lücken vorhanden sind.
Bei der Erarbeitung dieser Fragen wurde darauf geachtet, dass sie zu Beginn sammelnden Charakter hatten und mit dem Verlauf der Unterrichtseinheit immer präziser in Richtung des Zielthemas führen sollten.
An dieser Stelle zeigte sich während der Unterrichtseinheit, dass die Schülerinnen und Schüler über ein fundiertes Basiswissen verfügten, was den Umgang mit dem Thema vereinfachte.

Während der Durchführung des Experimentes zeigten sich alltägliche Schwierigkeiten, die auch in späteren Unterrichtseinheiten des Sachunterrichtes vorkommen können und deshalb eine genaue Planung des Unterrichts notwendig machen.
Es zeigte sich, dass sich der Schmelzvorgang der Eiswürfel ohne Zuhilfenahme technischer Hilfsmittel aufgrund der an diesem Tag niedrigen Lufttemperatur als zeitlich ausgedehnt darstellte.
Deshalb ging man im Laufe der Unterrichtsstunde dazu über, diesen Prozess mithilfe von Heizstrahlern zu beschleunigen.
Zudem war es hilfreich, wie sich an diesem Experiment beispielhaft zeigte, ausreichend Material wie z.B. Eiswürfel bereitstellen zu können, um allen Schülerinnen und Schülern die aktive Teilnahme am Experiment zu ermöglichen.
Ergänzend ließ sich im Nachhinein attestieren, dass eine gute zeitliche Planung für eine Doppelstunde vorgenommen werden sollte, um zu vermeiden, Zeit überbrücken zu müssen.

Alles in allem lässt sich als Ergebnis festhalten, dass die Durchführung und Einbettung des Experimentes in eine Unterrichtsreihe erfolgreich war.
Das nötige Hintergrundwissen, um die vorliegenden wissenschaftlichen Sachverhalte darzustellen, war vorhanden.
Auch der Umgang mit den Schülern im Unterrichtsgespräch und während der Experimentierphase verlief reibungslos.
Allerdings stieß man aufgrund der fehlenden Erfahrung mit der didaktischen Umsetzung von Inhalten im Sachunterricht an individuelle Grenzen, die das erstmalige Handeln als verantwortliche Lehrperson erschwerten.

Im folgenden Kapitel soll das Experiment in einem weiteren Schritt in den Perspektivrahmen des Sachunterrichts eingeordnet werden.


%% Bezug des Experimentes zum Perspektivrahmen Sachunterricht
\section{Bezug des Experimentes zum Perspektivrahmen Sachunterricht}
Das beschriebene Experiment lässt sich modellhaft in die Interdisziplinarität des Perspektivrahmens einordnen.
Das Experiment an sich wird der naturwissenschaftlichen Perspektive zugeordnet, die den physikalischen Zusammenhang von Eisschmelze und Wasserspiegel darstellt.
Jedoch können wir diesen Vorgang auch mit Fokus auf die anderen Perspektiven betrachten.
Beispielsweise kann der Sachverhalt des Klimawandels, der durch das Experiment erklärt werden soll, auch aus raumbezogener Perspektive heraus beschrieben werden, indem man sich die Frage stellt, welche Gebiete der Erde besonders von solchen Naturereignissen betroffen sind oder aber wo sich der Klimawandel in Form von Gletscherschmelzen hauptsächlich zeigt.
Des Weiteren lässt sich auch die historische Sicht beleuchten, wenn man der Frage nachgeht, wie sich das Ausmaß der Gletscherschmelze weltweit entwickelt hat und welche Ursachen es hierfür gibt.

An diesen Gedanken zeigt sich, dass didaktische Inhalte im Sachunterricht stets mehrere, wenn nicht sogar alle Perspektiven des Perspektivrahmens für eine differenzierte Betrachtung des jeweiligen Themas zulassen.
Trotzdem stellt der Perspektivrahmen keine Didaktik dar, sondern einen \glqq Rahmen, in dem didaktische Überlegungen konkretisiert werden können. Lehrerinnen und Lehrer haben damit ein Gerüst vor sich, mit dem sie eine Themenstellung auf ihren Bildungsinhalt prüfen können\grqq{}{\cite[S.\,155]{AK08}}.
 
Das Experiment steht im Spannungsfeld zwischen \glqq dem Erleben und Deuten von Naturphänomenen und den inhaltlichen und methodischen Angeboten der Naturwissenschaften\grqq{}{\cite[S.\,7]{GPS02}}.
Nach den Aussagen des Perspektivrahmens ist unsere Umweltwahrnehmung von Naturwissenschaften mitbeeinflusst.
Daher gibt das Experiment den Schülerinnen und Schülern die Möglichkeit, die Welt im Sinne der naturwissenschaftlichen Perspektive zu erfahren und mit der Einbeziehung des Forscherheftes ihre eigenen Erfahrungen reflektieren zu können.
Auch der Aspekt des \glqq Erschließens\grqq{}, der sich in diesem Zusammenhang im Perspektivrahmen Sachunterricht anschließt, wird umgesetzt.
Die Schülerinnen und Schüler setzen sich mit der Frage auseinander, inwiefern das Schmelzen der Gletscher direkten oder auch indirekten Einfluss auf das zukünftige Leben des Menschen auf der Erde hat.
Zudem wurde ihnen die Struktur des Wassers und seine besonderen Eigenschaften im Vergleich zu anderen chemischen Verbindungen verständlich gemacht.
Die Tatsache des Schmelzvorgangs wurde anhand der bestehenden physikalischen Gesetze erläutert und veranschaulicht.

Betrachtet man das Experiment aus der historischen Perspektive des Perspektivrahmens, so rückt die Rolle des Menschen in Bezug auf das festgestellte Phänomen in den Fokus.
Auch diese Perspektive steht, wie auch die naturwissenschaftliche, in einem Spannungsfeld.
Dieses Spannungsfeld verortet sich \glqq zwischen der Erfahrung des Wandels, die Kindern zugänglich ist und den inhaltlichen und methodischen Angeboten aus der Perspektive der Geschichtswissenschaft.\grqq{}{\cite[S.\,9]{GPS02}}, das heißt einerseits auf Basis der Betrachtung des Ereignisses selbst und andererseits mit einem spezifisch historischen Blick.
Weiter heißt es, dass sich Kinder, \grqq die durch erwünschte und unerwünschte Folgen menschlichen Handelns hervorgebrachten materiellen und sozialen Bedingungen des Zusammenlebens [...] zunächst als Gegebenheiten erschließen\grqq{}{\cite[S.\,9]{GPS02}}.
In Bezug auf das vorliegende Experiment bedeutet das, dass Kinder den Klimawandel zunächst lediglich als Ereignis ansehen, das in einem weiteren Schritt durch historisches Wissen als etwas Beeinflusstes und Geschaffenes verstanden werden soll.


\appendix

\newpage
\section{Geographisches Lernen im Anfangsunterricht (\cite[S.\,159-176]{IH07})}
\begin{enumerate}
	\item{Welche Forschungsergebnisse liegen zum Umgang mit Karten in der Grundschule vor?}
	\item{Welchen Einfluss zeigen die Merkmale \glqq Geschlecht\grqq{} und \glqq Selbsteinschätzung\grqq{} im Umgang mit Karten?}
	\item{
		Welche didaktischen Schritte sollten bei der Vermittlung räumlicher Orientierungskompetenzen berücksichtigt werden?
		\begin{itemize}
			\item{Einführung ins Kartenverständnis oft mit Geographie gleichgesetzt}
			\item{beschäftigt sich jedoch einerseits mit dem Raum und andererseits mit Wechselbeziehungen zwischen Mensch und Natur}
			\item{Brückenfach zwischen Natur- und Geowissenschaften}
			\item{Einengung des Faches auf das Kartenverständnis wird der Geographie nicht gerecht, da sie bedeutende Beiträge zur Umweltbildung leistet}
			\item{räumliche Orientierung wichtiger Bestandteil des Sachunterrichts im Anfangsunterricht}
			\item{in der aktuellen Forschung wird die Geographiedidaktik stiefmütterlich behandelt}
			\item{es gibt Untersuchungen über Alltagsvorstellungen von Schülerinnen und Schülern über geowissenschaftliche Phänomene, zur Häufigkeit von Exkursionen, Einstellungen von Schülern gegenüber bestimmten Ländern und über das Umweltbewusstsein von Grundschulkindern}
			\item{zudem liegen Untersuchungen über Regionen, Heimat oder Europa vor}
			\item{Existenz grundlegender topographischer Wissensbestände bei Erst- und Zweitklässlern nicht belegt}
			\item{jedoch muss berücksichtigt werden, dass Kinder in der 1. Klasse nicht zwingend klare Vorstellungen der Zuordnung räumlicher Inklusionen und Kategorien besitzen}
			\item{allerdings können Kinder durch gezielte Übung diese Verknüpfung bereits in der 1. Klasse verstehen}
			\item{auch Drittklässler können beispielsweise Kategorien wie Kontinent, Staat, Stadt, Bundesland, etc. nicht hinreichend unterscheiden}
			\item{Erst- und Zweitklässler erzielten bei einem Test zu relativen Lagen von Flächen und Punkten gute Ergebnisse, während sie bei der Nennung von Nachbarländern Probleme hatten}
			\item{geschlechtsbezogene Differenzen im Umgang mit Karten bereits länger bekannt}
			\item{diese bestehen besonders in der Strategiekenntnis, Interesse, Vorwissen und mentaler Rotationsleistungen}
			\item{Mädchen schneiden insgesamt schlechter ab}
			\item{Mädchen sollten gezielte Erfolgserlebnisse verschafft werden, indem die Komplexität und die Materialien auf ihren Kenntnisstand abgestimmt sind}
			\item{für eigenes Selbstverständnis und Verhalten spielen die Selbsteinschätzung und das Selbstvertrauen eine entscheidene Rolle}
			\item{
				didaktisch folgt daraus:
				\begin{enumerate}
					\item{in vertrauten Räumen beginnen, Nutzung des Vorwissens in vertrauter Umgebung}
					\item{in kleineren und überschaubaren Räumen beginnen}
					\item{Schwierigkeit der Aufgaben langsam steigern}
					\item{Kartenzeichnen ab der 1. Klasse möglich}
					\item{Bestimmung der Körperrichtung ab der 1. Klasse, Himmelsrichtungen ab der 3. Klasse zu üben}
					\item{Wissen möglichst aus den Erfahungen abstrahieren, nicht von Erfahungen unabhängig machen}
					\item{Repräsentationscharakter beachten}
					\item{Transfer zwischen Repräsentation und Realraum in beide Richtungen notwendig}
					\item{Kartengestützte Orientierung im Raum im Freien üben}
				\end{enumerate}
			}
		\end{itemize}
	} 
\end{enumerate} 

\newpage 
\section{Interview mit einer Grundschülerin zum Thema \glqq Was lernst du im Sachunterricht?\grqq{}}

\begin{enumerate}
\item{		
		Bitte sprechen Sie mit einem Kind im Grundschulalter über seine Vorstellung zur Leitfrage \glqq Was lernst du im Sachunterricht?\grqq{}\par
		
		\normalfont\sffamily\textsf{Was macht ihr momentan im Sachunterricht?}\par
		\noindent{\it \glqq Im Sachunterricht beschäftigen wir uns mit dem Kompass, mit den Höhenlinien und den Himmelsrichtungen.
			Außerdem lernen wir, Karten zu zeichnen und die Legende einer Karte zu lesen.
			Eine Klassenkameradin von mir hat einen Kompass mitgebracht und die Frau Weber – unsere Lehrerin hat uns das dann gezeigt, wie das funktioniert.\grqq{}
		}\par

		\normalfont\sffamily\textsf{Könnte man das auch woanders lernen?}\par
		\noindent{\it \glqq Ja.
			Aber das wüsste ich jetzt nicht genau.
			Es passt ganz gut in den Sachunterricht und es ist gut, dass wir es dort machen.\grqq{}
		}\par

		\normalfont\sffamily\textsf{Was sollte eure Lehrerin denn unbedingt noch mit euch machen?}\par
		\noindent{\it \glqq Dieses Thema mit den vielen Tieren, die im Wald leben.
			Das haben wir zwar auch schon gemacht, aber andere Tiere, verschiedene Fische, die nicht bei uns leben, würde ich gern machen.
			Auf die verschiedenen Länder freue ich mich schon.
			Wir machen zwar schon die Städte von Rheinland-Pfalz, aber das wird bestimmt interessant.\grqq{}
		}\par

		\normalfont\sffamily\textsf{Was gefällt dir gut an dem Fach \glqq Sachunterricht\grqq{}?}\par
		\noindent{\it \glqq Am meisten macht mir das Schreiben der Tests Spaß.
			Auch die Arbeitsblätter sind spannend.
			Manchmal machen mir auch tolle Spiele im Sachunterricht.\grqq{}
		}\par

		\normalfont\sffamily\textsf{Was gefällt dir nicht?}\par
		\noindent{\it \glqq Die Noten der Tests (lacht).
			Eigentlich sind die Noten gut, aber manchmal auch nicht und dann weinen manche und das ist nicht so schön.
			Generell üben wir gut in der Klasse, aber manche auch nicht.\grqq{}
		}\par
	}
\end{enumerate}

\newpage
\section{Das Dilemma der Schülerrolle im Klassenrat (\cite{HB10})}
\begin{enumerate}
	\item{
		In welches Dilemma geraten Kinder im Klassenrat?
		Skizzieren Sie die Doppelrolle der SchülerInnen im Klassenrat und den sich daraus ergebenden Konflikt.
	}
	\item{Warum wird im Klassenratsgespräch der Konflikt zwischen Tim und Nils nicht \glqq offen\grqq{} besprochen?}
	\item{
		Welche Schlussfolgerungen können aus dem dargelegten Klassenratsgespräch für die Organisation und Durchführung des Klassenrates gezogen werden?
		\begin{itemize}
			\item{Schüler versuchen, gleichzeitig vor der Lehrperson und den Gleichaltrigen zu bestehen $\rightarrow$ Dilemma}
			\item{einerseits sind die Schüler mit dem Hintergrundwissen in der Schule, sich an gewisse Verhaltensregeln zu halten}
			\item{andererseits sind sie Gleichaltrige, zwischen denen andere Regeln gelten, um ihren eigenen Status in der Klasse zu sichern oder aber um Beziehungen zu anderen Schülern zu pflegen}
			\item{aus diesen unterschiedlichen Regeln, denen die SchülerInnen unterliegen, kann ein Dilemma entstehen}
			\item{im Klassenrat wird nur ein Teil der Ereignisse des Tages öffentlich}
			\item{Tim erzählt die Vorgeschichte des Konflikts nicht, was automatisch zu missverständlichen Deutungen führt}
			\item{Kinder sind sich bewusst, dass Rache als Argument keinen Bestand hat}
			\item{Nils wollte den Konflikt ursprünglich nicht in den Klassenrat einbringen; einerseits Angst vor Beschämung vor der Lehrerin und der Klasse, andererseits Imagewahrung als friedlicher Junge}
			\item{Kinder zeigen rollenadäquates Verhalten}
			\item{sie unterscheiden zwischen den Erwartungen der Institution Schule und denen der Gleichaltrigen}
			\item{im Klassenrat sind Kinder um soziale Anerkennung der Lehrerin und der Mitschüler bemüht}
			\item{Öffentlichkeit des Verfahrens bewirkt taktische Verhaltensweisen der Kinder}
			\item{Klassenrat kann demzufolge kaum dazu dienen, außerschulische Konflikte zu bewältigen, sondern sollte Probleme behandeln, die alle Schüler angehen}
		\end{itemize}
	}
\end{enumerate}

\newpage
\section{Der Bildungsanspruch des Sachunterrichts (\cite{GPS13})}
\begin{itemize}
	\item{Grundschule hat die Aufgabe, Schüler bei der 		Auseinenandersetzung mit der Umwelt, einem angemessenen 	Verständnis und einer Mitgestaltung dieser, bei 		systematischem und reflektiertem Lernen und bei 	der Schaffung von Vorraussetzungen zu späterem Lernen, zu 		unterstützen}
	\item{Inhalte und Verfahren müssen den Bedürfnissen der Schüler gerecht werden}
	\item{Bildung soll ermöglicht und grundgelegt werden}
	\item{Leistungsfähigkeit und -bereitschaft der Kinder entfalten}
	\item{Sachunterricht darf Grundschulkinder nicht unterfordern, daher methodisch und inhaltlich anspruchsvoll}
	\item{Kinder sollen sich mithilfe des Sachunterrichts die natürliche, soziale und technisch gestaltete Umwelt erschließen}
	\item{Grundlagen für den späteren Fachunterricht sollen gelegt werden}
	\item{seiner Aufgabe kann der Sachunterricht nur gerecht werden, indem er die Fragen, Interessen und Lernbedürfnisse der Schülerinnen und Schüler aufgreift und integriert}
\end{itemize}

\subsection{Das Kompetenzmodell des Perspektivrahmens}
\begin{itemize}
	\item{Perspektivrahmen wählt die Themen des Sachunterrichts auf der Basis von fünf Perspektiven aus: sozial- und kulturwissenschaftlich, raumbezogen, naturbezogen, technisch und historisch}
	\item{diese Perspektiven dürfte jedoch nicht einzeln betrachtet werden, sondern die Inhalte sollen miteinander verknüpft werden, um übergreifende Zusammenhänge zu erfassen}
	\item{grundlegend sind Spannungsfelder zwischen Erfahrungen der Kinder und dem fundierten Fachwissen}
	\item{beide Pole müssen zusammenspielen, um einerseits dem Entstehen von inhaltsleeren Begriffen und andererseits Beschränkungen auf Alltagswissen vorzubeugen}
\end{itemize}

\subsection{Aufbau des Perspektivrahmens}
\begin{itemize}
	\item{Teil 1: Konzeption des Perspektivrahmens}
	\item{Teil 2: Schaffung von Grundlagen für weiterführendes Lernen}
	\item{Teil 3: Beschreibung anzustrebender Kompetenzen}
	\item{Teil 4: Zusammenspiel zwischen Kompetenzen und Anwendungs- und Gestaltungsaufgaben}
	\item{Teil 5: Spezifische Bedingungen des Sachunterrichts}
\end{itemize}

\newpage
\section{Philosophieren im Unterricht (\cite{KM06})}
\begin{itemize}
	\item{im Mittelpunkt steht die Methode des gemeinsamen Gesprächs}
	\item{diese hat eine demokratische Gesprächskultur zum Ziel}
	\item{die Qualität von Gesprächen hat durch die vom Fernsehen geprägte Öffentlichkeit gelitten}
	\item{auch im Elternhaus werden zu wenige Gespräge geführt, was eine Spracharmut zur Folge hat}
	\item{Gespräche als Medien des Austauschs}
	\item{philosophische Gespräche bieten Raum und Zeit, eigene Vorstellungen und Auffassungen bewusst zu machen und zu hinterfragen}
	\item{zentrales Anliegen ist daher, dass Kinder nicht nur ihre eigenen Gedanken reflektieren und weiterdenken können, sondern auch die anderer Kinder}
	\item{philosophische Gespräche nicht planbar}
	\item{Nachdenk-Gespräche inhaltlich bestimmt durch echte Fragen und Probleme; sie haben jedoch keine inhaltlich vorherbestimmten Ergebnisse und Ziele}
	\item{nicht die Beantwortung der Frage ist das Ziel, sondern die Erschließung eines tieferen Verständnisses}
	\item{am Ende stehen keine richtigen Antworten, sondern vorläufige mögliche Antworten}
	\item{kein Defizit $\Rightarrow$ Bewusstheit der Nichtbeantwortbarkeit kann weiteres Nachdenken anregen}
	\item{Offenheit und Ergebnisoffenheit als zentrale Merkmale philosphischer Gespräche}
\end{itemize}

\newpage
\section{Aufgaben und Herausforderungen der Gesprächsleitung}
\begin{itemize}
	\item{Lehrkraft muss sich in philosophischen Gesprächen zurückhalten, um Kindern die Möglichkeit zu geben, über das zu sprechen, was sie an einem Thema interessiert und was diskussionswert erscheint}
	\item{gleichzeitig muss die Gesprächsleitung jedoch dafür sorgen, dass die inhaltliche Qualität der Beiträge gefördert wird}
	\item{Impulse für weiterführendes Denken und Nachdenken geben}
	\item{weitere Aufgabe ist es, gegebenenfalls auf die Ausgangsfrage zu verweisen}
	\item{Zusammenfassung des Gesprächsstandes}
	\item{wichtig ist für die Lehrkraft, die Balance zwischen Offenheit und Zurückhaltung einerseits, Eingreifen und Strukturieren andererseits, zu wahren}
\end{itemize}

\newpage
\section{Methode des Bilderbuches}
\begin{itemize}
	\item{gute Kinderbücher haben keine abgeschlossene Geschichte, sondern lassen Spielraum zum Nachdenken}
	\item{zudem sind duale Konzepte enthalten wie z.B. Gut und Böse, Liebe und Hass, Leben und Tod, usw.} 
	\item{durch die Auswahl eines bestimmten Buches wird von der Lehrkraft ein Thema vorgegeben}
	\item{daraus folgt, dass Kinder ihre eigenen Fragen formulieren dürfen sollten}
	\item{Frage steht im Zentrum, welche Bedeutung die Kinder dem Buch vor dem Hintergrund ihrer persönlichen Erfahrungen beimessen}
	\item{methodisches Vorgehen:
		\begin{enumerate}
		\item{gemeinsames Lesen des Buches}
		\item{Kinder formulieren Fragen zur Geschichte}
		\item{Kinder stimmen ab, welche Fragen zuerst besprochen werden sollen}
		\end{enumerate}
	}
	\item{jedes Kind darf nach dem Lesen eine oder mehrere Frage an der Tafel festhalten}
	\item{sinnvoll ist eine Reihenfolge der interessantesten Fragen zu erstellen}
\end{itemize}



\newpage
\section{Infoblatt zum Experiment}

\newpage
\bibliography{sachunterricht}
\bibliographystyle{dinat}

\newpage
\newgeometry{left=1.5cm,right=1.5cm,head=1.5cm,bottom=1.5cm}
\currentpdfbookmark{Erklärung}{pdf:declaration}
\thispagestyle{empty}
\vspace*{3cm}
\begin{center}
	\Large{\textbf{Erklärung}}
\end{center}
\vspace*{1cm}
Hiermit bestätige ich, dass die vorliegende Arbeit von mir selbstständig verfasst wurde und ich keine anderen als die angegebenen Hilfsmittel - insbesondere keine im Quellenverzeichnis nicht benannten Internet-Quellen - benutzt habe und die Arbeit von mir vorher nicht in einem anderen Prüfungsverfahren eingereicht wurde.
Die eingereichte schriftliche Fassung entspricht der auf dem elektronischen Speichermedium. (CD-ROM)
\vspace*{1cm}\\
Koblenz, den \today
\vspace*{0.75cm}\\
Martin Spoo
\restoregeometry


\end{document}

