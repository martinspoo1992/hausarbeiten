\section{Aufgaben zur Vorlesung \glqq Sachen des Sachunterrichts\grqq{}}
\begin{enumerate}
	\item{Welche Forschungsergebnisse liegen zum Umgang mit Karten in der Grundschule vor?}
	\item{Welchen Einfluss zeigen die Merkmale \glqq Geschlecht\grqq{} und \glqq Selbsteinschätzung\grqq{} im Umgang mit Karten?}
	\item{
		Welche didaktischen Schritte sollten bei der Vermittlung räumlicher Orientierungskompetenzen berücksichtigt werden?
		\begin{itemize}
			\item{Einführung ins Kartenverständnis oft mit Geographie gleichgesetzt}
			\item{sie beschäftigt sich jedoch einerseits mit dem Raum und mit Wechselbeziehungen zwischen Mensch und Natur}
			\item{Brückenfach zwischen Natur- und Geowissenschaften}
			\item{Einengung des Faches auf das Kartenverständnis wird der Geographie nicht gerecht, da sie bedeutende Beiträge zur Umweltbildung leistet}
			\item{räumliche Orientierung wichtiger Bestandteil des Sachunterrichts im Anfangsunterricht}
			\item{in der aktuellen Forschung wird die Geographiedidaktik stiefmütterlich behandelt}
			\item{es gibt Untersuchungen über Alltagsvorstellungen von Schülerinnen und Schülern über geowissenschaftliche Phänomene, zur Häufigkeit von Exkursionen, Einstellungen von Schülern gegenüber bestimmten Ländern und über das Umweltbewusstsein von Grundschulkindern}
			\item{zudem liegen Untersuchungen über Regionen, Heimat oder Europa vor}
			\item{Existenz grundlegender topographischer Wissensbestände bei Erst- und Zweitklässlern nicht belegt}
			\item{jedoch muss berücksichtigt werden, dass Kinder in der 1. Klasse nicht zwingend klare Vorstellungen der Zuordnung räumlicher Inklusionen und Kategorien besitzen}
			\item{allerdings können Kinder durch gezielte Übung diese Verknüpfung bereits in der 1. Klasse verstehen}
			\item{auch Drittklässler können beispielsweise Kategorien wie Kontinent, Staat, Stadt, Bundesland, etc. nicht hinreichend unterscheiden}
			\item{Erst- und Zweitklässler erzielten bei einem Test zu relativen Lagen von Flächen und Punkten gute Ergebnisse, während sie bei der Nennung von Nachbarländern Probleme hatten}
			\item{geschlechtsbezogene Differenzen im Umgang mit Karten bereits länger bekannt}
			\item{diese bestehen besonders in der Strategiekenntnis, Interesse, Vorwissen und mentaler Rotationsleistungen}
			\item{Mädchen schneiden insgesamt schlechter ab}
			\item{Mädchen sollten gezielte Erfolgserlebnisse verschafft werden, indem die Komplexität und die Materialien auf ihren Kenntnisstand abgestimmt sind}
			\item{für eigenes Selbstverständnis und Verhalten spielen die Selbsteinschätzung und das Selbstvertrauen eine entscheidene Rolle}
			\item{
				didaktisch folgt daraus:
				\begin{enumerate}
					\item{in vertrauten Räumen beginnen, Nutzung des Vorwissens in vertrauter Umgebung}
					\item{in kleineren und überschaubaren Räumen beginnen}
					\item{Schwierigkeit der Aufgaben langsam steigern}
					\item{Kartenzeichnen ab der 1. Klasse möglich}
					\item{Bestimmung der Körperrichtung ab der 1. Klasse, Himmelsrichtungen ab der 3. Klasse zu üben}
					\item{Wissen möglichst aus der Erfahungen abstrahieren, nicht von Erfahungen unabhängig machen}
					\item{Repräsentationscharakter beachten}
					\item{Transfer zwischen Repräsentation und Realraum in beide Richtungen notwendig}
					\item{Kartengestützte Orientierung im Raum im Freien üben}
				\end{enumerate}
			}
		\end{itemize}
	}
	\item{
		Bitte sprechen Sie mit einem Kind im Grundschulalter über seine Vorstellung zur Leitfrage \glqq Was lernst du im Sachunterricht?\grqq{}\par
		
		\normalfont\sffamily\textsf{Was macht ihr momentan im Sachunterricht?}\par
		\noindent{\it \glqq Im Sachunterricht beschäftigen wir uns mit dem Kompass, mit den Höhenlinien und den Himmelsrichtungen.
			Außerdem lernen wir, Karten zu zeichnen und die Legende einer Karte zu lesen.
			Eine Klassenkameradin von mir hat einen Kompass mitgebracht und die Frau Weber – unsere Lehrerin hat uns das dann gezeigt, wie das funktioniert.\grqq{}
		}\par

		\normalfont\sffamily\textsf{Könnte man das auch woanders lernen?}\par
		\noindent{\it \glqq Ja.
			Aber das wüsste ich jetzt nicht genau.
			Es passt ganz gut in den Sachunterricht und es ist gut, dass wir es dort machen.\grqq{}
		}\par

		\normalfont\sffamily\textsf{Was sollte eure Lehrerin denn unbedingt noch mit euch machen?}\par
		\noindent{\it \glqq Dieses Thema mit den vielen Tieren, die im Wald leben.
			Das haben wir zwar auch schon gemacht, aber andere Tiere, verschiedene Fische, die nicht bei uns leben, würde ich gern machen.
			Auf die verschiedenen Länder freue ich mich schon.
			Wir machen zwar schon die Städte von Rheinland-Pfalz, aber das wird bestimmt interessant.\grqq{}
		}\par

		\normalfont\sffamily\textsf{Was gefällt dir gut an dem Fach \glqq Sachunterricht\grqq{}?}\par
		\noindent{\it \glqq Am meisten macht mir das Schreiben der Tests Spaß.
			Auch die Arbeitsblätter sind spannend.
			Manchmal machen mir auch tolle Spiele im Sachunterricht.\grqq{}
		}\par

		\normalfont\sffamily\textsf{Was gefällt dir nicht?}\par
		\noindent{\it \glqq Die Noten der Tests (lacht).
			Eigentlich sind die Noten gut, aber manchmal auch nicht und dann weinen manche und das ist nicht so schön.
			Generell üben wir gut in der Klasse, aber manche auch nicht.\grqq{}
		}\par
	}
	\item{
		In welches Dilemma geraten Kinder im Klassenrat?
		Skizzieren Sie die Doppelrolle der SchülerInnen im Klassenrat und den sich daraus ergebenden Konflikt.
	}
	\item{Warum wird im Klassenratsgespräch der Konflikt zwischen Tim und Nils nicht \glqq offen\grqq{} besprochen?}
	\item{
		Welche Schlussfolgerungen können aus dem dargelegten Klassenratsgespräch für die Organisation und Durchführung des Klassenrates gezogen werden?
		\begin{itemize}
			\item{Schüler versuchen, gleichzeitig vor der Lehrperson und den Gleichaltrigen zu bestehen $\rightarrow$ Dilemma}
			\item{einerseits sind die Schüler mit dem Hintergrundwissen in der Schule, sich an gewisse Verhaltensregeln zu halten}
			\item{andererseits sind sie Gleichaltrige, zwischen denen andere Regeln gelten, um seinen eigenen Status in der Klasse zu sichern oder aber um Beziehungen zu anderen Schülern zu pflegen}
			\item{aus diesen unterschiedlichen Regeln, denen die SchülerInnen unterliegen, kann ein Dilemma entstehen}
			\item{im Klassenrat wird nur ein Teil der Ereignisse des Tages öffentlich}
			\item{Tim erzählt die Vorgeschichte des Konflikts nicht, was automatisch zu missverständlichen Deutungen führt}
			\item{Kinder sind sich bewusst, dass Rache als Argument keinen Bestand hat}
			\item{Nils wollte den Konflikt ursprünglich nicht in den Klassenrat einbringen; einerseits Angst vor Beschämung vor der Lehrerin und der Klasse, andererseits Imagewahrung als friedlicher Junge}
			\item{Kinder zeigen rollenadäquates Verhalten}
			\item{sie unterscheiden zwischen den Erwartungen der Institution Schule und denen der Gleichaltrigen}
			\item{im Klassenrat sind Kinder um soziale Anerkennung der Lehrerin und der Mitschüler bemüht}
			\item{Öffentlichkeit des Verfahrens bewirkt taktische Verhaltensweisen der Kinder}
			\item{Klassenrat kann demzufolge kaum dazu dienen, außerschulische Konflikte zu bewältigen, sondern sollte Probleme behandeln, die alle Schüler angehen}
		\end{itemize}
	}
\end{enumerate}