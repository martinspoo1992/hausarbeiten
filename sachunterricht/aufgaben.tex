\section{Geographisches Lernen im Anfangsunterricht}{\cite{IH07}
\begin{enumerate}
	\item{Welche Forschungsergebnisse liegen zum Umgang mit Karten in der Grundschule vor?}
	\item{Welchen Einfluss zeigen die Merkmale \glqq Geschlecht\grqq{} und \glqq Selbsteinschätzung\grqq{} im Umgang mit Karten?}
	\item{
		Welche didaktischen Schritte sollten bei der Vermittlung räumlicher Orientierungskompetenzen berücksichtigt werden?
		\begin{itemize}
			\item{Einführung ins Kartenverständnis oft mit Geographie gleichgesetzt}
			\item{sie beschäftigt sich jedoch einerseits mit dem Raum und mit Wechselbeziehungen zwischen Mensch und Natur}
			\item{Brückenfach zwischen Natur- und Geowissenschaften}
			\item{Einengung des Faches auf das Kartenverständnis wird der Geographie nicht gerecht, da sie bedeutende Beiträge zur Umweltbildung leistet}
			\item{räumliche Orientierung wichtiger Bestandteil des Sachunterrichts im Anfangsunterricht}
			\item{in der aktuellen Forschung wird die Geographiedidaktik stiefmütterlich behandelt}
			\item{es gibt Untersuchungen über Alltagsvorstellungen von Schülerinnen und Schülern über geowissenschaftliche Phänomene, zur Häufigkeit von Exkursionen, Einstellungen von Schülern gegenüber bestimmten Ländern und über das Umweltbewusstsein von Grundschulkindern}
			\item{zudem liegen Untersuchungen über Regionen, Heimat oder Europa vor}
			\item{Existenz grundlegender topographischer Wissensbestände bei Erst- und Zweitklässlern nicht belegt}
			\item{jedoch muss berücksichtigt werden, dass Kinder in der 1. Klasse nicht zwingend klare Vorstellungen der Zuordnung räumlicher Inklusionen und Kategorien besitzen}
			\item{allerdings können Kinder durch gezielte Übung diese Verknüpfung bereits in der 1. Klasse verstehen}
			\item{auch Drittklässler können beispielsweise Kategorien wie Kontinent, Staat, Stadt, Bundesland, etc. nicht hinreichend unterscheiden}
			\item{Erst- und Zweitklässler erzielten bei einem Test zu relativen Lagen von Flächen und Punkten gute Ergebnisse, während sie bei der Nennung von Nachbarländern Probleme hatten}
			\item{geschlechtsbezogene Differenzen im Umgang mit Karten bereits länger bekannt}
			\item{diese bestehen besonders in der Strategiekenntnis, Interesse, Vorwissen und mentaler Rotationsleistungen}
			\item{Mädchen schneiden insgesamt schlechter ab}
			\item{Mädchen sollten gezielte Erfolgserlebnisse verschafft werden, indem die Komplexität und die Materialien auf ihren Kenntnisstand abgestimmt sind}
			\item{für eigenes Selbstverständnis und Verhalten spielen die Selbsteinschätzung und das Selbstvertrauen eine entscheidene Rolle}
			\item{
				didaktisch folgt daraus:
				\begin{enumerate}
					\item{in vertrauten Räumen beginnen, Nutzung des Vorwissens in vertrauter Umgebung}
					\item{in kleineren und überschaubaren Räumen beginnen}
					\item{Schwierigkeit der Aufgaben langsam steigern}
					\item{Kartenzeichnen ab der 1. Klasse möglich}
					\item{Bestimmung der Körperrichtung ab der 1. Klasse, Himmelsrichtungen ab der 3. Klasse zu üben}
					\item{Wissen möglichst aus der Erfahungen abstrahieren, nicht von Erfahungen unabhängig machen}
					\item{Repräsentationscharakter beachten}
					\item{Transfer zwischen Repräsentation und Realraum in beide Richtungen notwendig}
					\item{Kartengestützte Orientierung im Raum im Freien üben}
				\end{enumerate}
			}
		\end{itemize}
	} 
	\end{enumerate} 

\newpage 

\section{Interview mit einer Grundschülerin zum Thema \glqq Was lernst du im Sachunterricht?\grqq{}}

\begin{enumerate}
\item{		
		Bitte sprechen Sie mit einem Kind im Grundschulalter über seine Vorstellung zur Leitfrage \glqq Was lernst du im Sachunterricht?\grqq{}\par
		
		\normalfont\sffamily\textsf{Was macht ihr momentan im Sachunterricht?}\par
		\noindent{\it \glqq Im Sachunterricht beschäftigen wir uns mit dem Kompass, mit den Höhenlinien und den Himmelsrichtungen.
			Außerdem lernen wir, Karten zu zeichnen und die Legende einer Karte zu lesen.
			Eine Klassenkameradin von mir hat einen Kompass mitgebracht und die Frau Weber – unsere Lehrerin hat uns das dann gezeigt, wie das funktioniert.\grqq{}
		}\par

		\normalfont\sffamily\textsf{Könnte man das auch woanders lernen?}\par
		\noindent{\it \glqq Ja.
			Aber das wüsste ich jetzt nicht genau.
			Es passt ganz gut in den Sachunterricht und es ist gut, dass wir es dort machen.\grqq{}
		}\par

		\normalfont\sffamily\textsf{Was sollte eure Lehrerin denn unbedingt noch mit euch machen?}\par
		\noindent{\it \glqq Dieses Thema mit den vielen Tieren, die im Wald leben.
			Das haben wir zwar auch schon gemacht, aber andere Tiere, verschiedene Fische, die nicht bei uns leben, würde ich gern machen.
			Auf die verschiedenen Länder freue ich mich schon.
			Wir machen zwar schon die Städte von Rheinland-Pfalz, aber das wird bestimmt interessant.\grqq{}
		}\par

		\normalfont\sffamily\textsf{Was gefällt dir gut an dem Fach \glqq Sachunterricht\grqq{}?}\par
		\noindent{\it \glqq Am meisten macht mir das Schreiben der Tests Spaß.
			Auch die Arbeitsblätter sind spannend.
			Manchmal machen mir auch tolle Spiele im Sachunterricht.\grqq{}
		}\par

		\normalfont\sffamily\textsf{Was gefällt dir nicht?}\par
		\noindent{\it \glqq Die Noten der Tests (lacht).
			Eigentlich sind die Noten gut, aber manchmal auch nicht und dann weinen manche und das ist nicht so schön.
			Generell üben wir gut in der Klasse, aber manche auch nicht.\grqq{}
		}\par
	}
	\end{enumerate}

\newpage

\section{Das Dilemma der Schülerrolle im Klassenrat} {\cite{DB08}

\begin{enumerate}
	\item{
		In welches Dilemma geraten Kinder im Klassenrat?
		Skizzieren Sie die Doppelrolle der SchülerInnen im Klassenrat und den sich daraus ergebenden Konflikt.
	}
	\item{Warum wird im Klassenratsgespräch der Konflikt zwischen Tim und Nils nicht \glqq offen\grqq{} besprochen?}
	\item{
		Welche Schlussfolgerungen können aus dem dargelegten Klassenratsgespräch für die Organisation und Durchführung des Klassenrates gezogen werden?
		\begin{itemize}
			\item{Schüler versuchen, gleichzeitig vor der Lehrperson und den Gleichaltrigen zu bestehen $\rightarrow$ Dilemma}
			\item{einerseits sind die Schüler mit dem Hintergrundwissen in der Schule, sich an gewisse Verhaltensregeln zu halten}
			\item{andererseits sind sie Gleichaltrige, zwischen denen andere Regeln gelten, um seinen eigenen Status in der Klasse zu sichern oder aber um Beziehungen zu anderen Schülern zu pflegen}
			\item{aus diesen unterschiedlichen Regeln, denen die SchülerInnen unterliegen, kann ein Dilemma entstehen}
			\item{im Klassenrat wird nur ein Teil der Ereignisse des Tages öffentlich}
			\item{Tim erzählt die Vorgeschichte des Konflikts nicht, was automatisch zu missverständlichen Deutungen führt}
			\item{Kinder sind sich bewusst, dass Rache als Argument keinen Bestand hat}
			\item{Nils wollte den Konflikt ursprünglich nicht in den Klassenrat einbringen; einerseits Angst vor Beschämung vor der Lehrerin und der Klasse, andererseits Imagewahrung als friedlicher Junge}
			\item{Kinder zeigen rollenadäquates Verhalten}
			\item{sie unterscheiden zwischen den Erwartungen der Institution Schule und denen der Gleichaltrigen}
			\item{im Klassenrat sind Kinder um soziale Anerkennung der Lehrerin und der Mitschüler bemüht}
			\item{Öffentlichkeit des Verfahrens bewirkt taktische Verhaltensweisen der Kinder}
			\item{Klassenrat kann demzufolge kaum dazu dienen, außerschulische Konflikte zu bewältigen, sondern sollte Probleme behandeln, die alle Schüler angehen}
		\end{itemize}
	}
\end{enumerate}

\newpage

\section{Der Bildungsanspruch des Sachunterrichts} {\cite{GPS13}

\begin{itemize}

	\item{Grundschule hat die Aufgabe, Schüler bei der 		Auseinenandersetzung mit der Umwelt, einem angemessenen 	Verständnis und einer Mitgestaltung dieser, bei 		systematischem und reflektiertem Lernen und bei 	der Schaffung von Vorraussetzungen zu späterem Lernen, zu 		unterstützen}
	\item{Inhalte und Verfahren müssen den Bedürfnissen der Schüler gerecht werden}
	\item{Bildung soll ermöglicht und grundgelegt werden}
	\item{Leistungsfähigkeit und -bereitschaft der Kinder entfalten}
	\item{Sachunterricht darf Grundschulkinder nicht unterfordern, daher methodisch und inhaltlich anspruchsvoll}
	\item{Kinder sollen sich mithilfe des Sachunterrichts die natürliche, soziale und technisch gestaltete Umwelt erschließen}
	\item{Grundlagen für den späteren Fachunterricht sollen gelegt werden}
	\item{seiner Aufgabe kann der Sachunterricht nur gerecht werden, indem er die Fragen, Interessen und Lernbedürfnisse aufgreift und integriert}

\end{itemize}

\subsection{Das Kompetenzmodell des Perspektivrahmens}

\begin{itemize}

	\item{Perspektivrahmen wählt die Themen des Sachunterrichts auf der Basis von fünf Perspektiven aus: sozial- und kulturwissenschaftlich, raumbezogen, naturbezogen, technisch und historisch}
	\item{diese Perspektiven dürfte jedoch nicht einzeln betrachtet werden, sondern die Inhalte sollen miteinander verknüpft werden, um übergreifende Zusammenhänge zu erfassen}
	\item{grundlegend sind Spannungsfelder zwischen Erfahrungen der Kinder und dem fundierten Fachwissen}
	\item{beide Pole müssen zusammenspielen, um einerseits dem Entstehen von inhaltsleeren Begriffen und andererseits Beschränkungen auf Alltagswissen vorzubeugen}

\end{itemize}

\subsection{Aufbau des Perspektivrahmens}

\begin{itemize}
	\item{Teil 1: Konzeption des Perspektivrahmens}
	\item{Teil 2: Schaffung von Grundlagen für weiterführendes Lernen}
	\item{Teil 3: Beschreibung anzustrebender Kompetenzen}
	\item{Teil 4: Zusammenspiel zwischen Kompetenzen und Anwendungs- und Gestaltungsaufgaben}
	\item{Teil 5: Spezifische Bedingungen des Sachunterrichts}


\end{itemize}

\newpage

\section{Philosphieren im Unterricht} {\cite{KM06}}

\begin{itemize}
	\item{im Mittelpunkt steht die Methode des gemeinsamen Gesprächs}
	\item{diese hat eine demokratische Gesprächskultur zum Ziel}
	\item{die Qualität von Gesprächen hat durch die vom Fernsehen geprägte Öffenlichkeit gelitten}
	\item{auch im Elternhaus werden zu wenige Gespräge geführt, was eine Spracharmut zur Folge hat}
	\item{Gespräche als Medien des Austauschs}
	\item{philosophische Gespräche bieten Raum und Zeit, eigene Vorstellungen und Auffassungen bewusst zu machen und zu hinterfragen}
	\item{zentrales Anliegen ist daher, dass Kinder nicht nur ihre eigenen Gedanken reflektieren und weiterdenken können, sondern auch die anderer Kinder}
	\item{philosophische Gespräche nicht planbar}
	\item{Nachdenk-Gespräche inhaltlich bestimmt durch echte Fragen und Probleme; sie haben jedoch keine inhaltlich vorherbestimmten Ergebnisse und Ziele}
	\item{nicht die Beantwortung der Frage ist das Ziel, sondern die Erschließung eines tieferen Verständnisses}
	\item{am Ende stehen keine richtigen Antworten, sondern vorläufige mögliche Antworten}
	\item{kein Defizit $\Rightarrow$ Bewusstheit der Nichtbeantwortbarkeit kann weiteres Nachdenken anregen}
	\item{Offenheit und Ergebnisoffenheit als zentrale Merkmale philosphischer Gespräche}

\end{itemize}

\section{Aufgaben und Herausforderungen der Gesprächsleitung}

\begin{itemize}

	\item{Lehrkraft muss sich in philosophischen Gesprächen zurückhalten, um Kindern die Möglichkeit zu geben, über das zu sprechen, was sie an einem Thema interessiert und was diskussionswert erscheint}
	\item{gleichzeitig muss die Gesprächsleitung jedoch dafür sorgen, dass die inhaltliche Qualität der Beiträge gefördert wird}
	\item{Impulse für weiterführendes Denken und Nachdenken geben}
	\item{weitere Aufgabe ist es, gegebenenfalls auf die Ausgangsfrage zu verweisen}
	\item{Zusammenfassung des Gesprächsstandes}
	\item{wichtig ist für die Lehrkraft, die Balance zwischen Offenheit und Zurückhaltung einerseits, Eingreifen und Strukturieren andererseits, zu wahren}

\end{itemize}

\section{Methode des Bilderbuches}

\begin{itemize}

	\item{gute Kinderbücher haben keine abgeschlossene Geschichte, sondern lassen Spielraum zum Nachdenken}
	\item{zudem sind duale Konzepte enthalten wie z.B. Gut und Böse, Liebe und Hass, Leben und Tod, usw.} 
	\item{durch die Auswahl eines bestimmten Buches wird von der Lehrkraft ein Thema vorgegeben}
	\item{daraus folgt, dass Kinder ihre eigenen Fragen formulieren dürfen sollten}
	\item{Frage steht im Zentrum, welche Bedeutung die Kinder dem Buch vor dem Hintergrund ihrer persönlichen Erfahrungen beimessen}
	\item{methodisches Vorgehen:
		\begin{enumerate}
		\item{gemeinsames Lesen des Buches}
		\item{Kinder formulieren Fragen zur Geschichte}
		\item{Kinder stimmen ab, welche Fragen zuerst besprochen werden sollen}
		\end{enumerate}
		}
	\item{jedes Kind darf nach dem Lesen eine oder mehrere Frage an der Tafel festhalten}
	\item{sinnvoll ist eine Reihenfolge der interessantesten Fragen zu erstellen}

\end{itemize}

