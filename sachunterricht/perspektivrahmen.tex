\section{Bezug des Experimentes zum Perspektivrahmen Sachunterricht}
Das beschriebene Experiment lässt sich modellhaft in die Interdisziplinarität des Perspektivrahmens einordnen.
Das Experiment an sich wird der naturwissenschaftlichen Perspektive zugeordnet, die den physikalischen Zusammenhang von Eisschmelze und Wasserspiegel darstellt.
Jedoch können wir diesen Vorgang auch mit Fokus auf die anderen Perspektiven betrachten.
Beispielsweise kann der Sachverhalt des Klimawandels, der durch das Experiment erklärt werden soll, auch aus raumbezogener Perspektive heraus beschrieben werden, indem man sich die Frage stellt, welche Gebiete der Erde besonders von solchen Naturereignissen betroffen sind oder aber wo sich der Klimawandel in Form von Gletscherschmelzen hauptsächlich zeigt.
Des Weiteren lässt sich auch die historische Sicht beleuchten, wenn man der Frage nachgeht, wie sich das Ausmaß der Gletscherschmelze weltweit entwickelt hat und welche Ursachen es hierfür gibt.

An diesen Gedanken zeigt sich, dass didaktische Inhalte im Sachunterricht stets mehrere, wenn nicht sogar alle Perspektiven des Perspektivrahmens für eine differenzierte Betrachtung des jeweiligen Themas zulassen.
Trotzdem stellt der Perspektivrahmen keine Didaktik dar, sondern einen \glqq Rahmen, in dem didaktische Überlegungen konkretisiert werden können. Lehrerinnen und Lehrer haben damit ein Gerüst vor sich, mit dem sie eine Themenstellung auf ihren Bildungsinhalt prüfen können\grqq{}{\cite[S.\,155]{AK08}}.
 
Das Experiment steht im Spannungsfeld zwischen \glqq dem Erleben und Deuten von Naturphänomenen und den inhaltlichen und methodischen Angeboten der Naturwissenschaften\grqq{}{\cite[S.\,7]{GPS02}}.
Nach den Aussagen des Perspektivrahmens ist unsere Umweltwahrnehmung von Naturwissenschaften mitbeeinflusst.
Daher gibt das Experiment den Schülerinnen und Schülern die Möglichkeit, die Welt im Sinne der naturwissenschaftlichen Perspektive zu erfahren und mit der Einbeziehung des Forscherheftes ihre eigenen Erfahrungen reflektieren zu können.
Auch der Aspekt des \glqq Erschließens\grqq{}, der sich in diesem Zusammenhang im Perspektivrahmen Sachunterricht anschließt, wird umgesetzt.
Die Schülerinnen und Schüler setzen sich mit der Frage auseinander, inwiefern das Schmelzen der Gletscher direkten oder auch indirekten Einfluss auf das zukünftige Leben des Menschen auf der Erde hat.
Zudem wurde ihnen die Struktur des Wassers und seine besonderen Eigenschaften im Vergleich zu anderen chemischen Verbindungen verständlich gemacht.
Die Tatsache des Schmelzvorgangs wurde anhand der bestehenden physikalischen Gesetze erläutert und veranschaulicht.

Betrachtet man das Experiment aus der historischen Perspektive des Perspektivrahmens, so rückt die Rolle des Menschen in Bezug auf das festgestellte Phänomen in den Fokus.
Auch diese Perspektive steht, wie auch die naturwissenschaftliche, in einem Spannungsfeld.
Dieses Spannungsfeld verortet sich \glqq zwischen der Erfahrung des Wandels, die Kindern zugänglich ist und den inhaltlichen und methodischen Angeboten aus der Perspektive der Geschichtswissenschaft.\grqq{}{\cite[S.\,9]{GPS02}}, das heißt einerseits auf Basis der Betrachtung des Ereignisses selbst und andererseits mit einem spezifisch historischen Blick.
Weiter heißt es, dass sich Kinder, \glqq die durch erwünschte und unerwünschte Folgen menschlichen Handelns hervorgebrachten materiellen und sozialen Bedingungen des Zusammenlebens [...] zunächst als Gegebenheiten erschließen\grqq{}{\cite[S.\,9]{GPS02}}.
In Bezug auf das vorliegende Experiment bedeutet das, dass Kinder den Klimawandel zunächst lediglich als Ereignis ansehen, das in einem weiteren Schritt durch historisches Wissen als etwas Beeinflusstes und Geschaffenes verstanden werden soll.
